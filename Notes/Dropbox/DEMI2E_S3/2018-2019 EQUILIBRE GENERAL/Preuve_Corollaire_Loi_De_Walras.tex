
\documentclass[11pt]{article}
%%%%%%%%%%%%%%%%%%%%%%%%%%%%%%%%%%%%%%%%%%%%%%%%%%%%%%%%%%%%%%%%%%%%%%%%%%%%%%%%%%%%%%%%%%%%%%%%%%%%%%%%%%%%%%%%%%%%%%%%%%%%%%%%%%%%%%%%%%%%%%%%%%%%%%%%%%%%%%%%%%%%%%%%%%%%%%%%%%%%%%%%%%%%%%%%%%%%%%%%%%%%%%%%%%%%%%%%%%%%%%%%%%%%%%%%%%%%%%%%%%%%%%%%%%%%
%\usepackage[applemac]{inputenc}
\usepackage{inputenc}
\usepackage[frenchb]{babel}
\usepackage{amssymb,amsfonts,amsmath}
\newcommand\R{\mathbb R}

\setcounter{MaxMatrixCols}{10}
%TCIDATA{OutputFilter=LATEX.DLL}
%TCIDATA{Version=5.00.0.2570}
%TCIDATA{<META NAME="SaveForMode" CONTENT="1">}
%TCIDATA{LastRevised=Wednesday, March 25, 2009 15:50:39}
%TCIDATA{<META NAME="GraphicsSave" CONTENT="32">}

\def \L { {\mathcal L}}
\setlength{\unitlength}{1cm} \setlength{\textwidth}{17.5cm}
\setlength{\oddsidemargin}{0cm} \setlength{\evensidemargin}{0cm}
\setlength{\topmargin}{-15pt} \setlength{\textheight}{24.5cm}
\renewcommand{\baselinestretch}{1}

%\input{tcilatex}

\begin{document}



\textbf{Corollaire de la Loi de Walras : Une preuve avec des producteurs}

Quels que soient les types de marchés considérés, la preuve s'écrit de la même manière : l'équilibre sur les premiers marchés modifie la contrainte budgétaire. Cela doit permettre de montrer qu'il est impossible d'avoir une offre ou une demande excédentaire sur le $n^e$ marché, ou encore de trouver un prix d'équilibre sur ce marché.
 
Faisons l'hypothèse que $n-1$ marchés de biens sont à l'équilibre. Nous considèrons par simplicité que les firmes utilisent toutes un seul facteur de production, le facteur travail $l$. Nous considérons que le marché du travail est à l'équilibre. Toutes les technologies sont à rendements constants. Donc, si un marché est à l'équilibre, la firme a un profit nul. 

Nous devons prouver que le $n^{e}$ marché est aussi à l'équilibre.

Le marché du travail étant à l'équilibre, on peut écrire, avec $j$ les consommateurs et $ i $ les producteurs :

\begin{equation}\tag{L} \label{L}
\sum_{j} l^j = \sum_{i=1}^{n} l_{i}
\end{equation}



Le profit étant nul sur les $n-1$ premiers marchés, on peut écrire : 

$$ \Pi_i = p_i q_i - s l_i = 0, \ \forall \ i < n $$

avec $s$ le salaire.

\ref{L} et cette dernière equation nous donne :

$$ s\sum_{j} l^j = s \sum_{i=1}^{n} l_{i} \Rightarrow s\sum_{j} l^j = \sum_{i=1}^{n-1} p_i q_{i} + s l_n  \Rightarrow  l_n =  \sum_{j} l^j - \frac{\sum_{i=1}^{n-1} p_i q_{i}}{s}  $$
 
 avec $j$ les consommateurs.
 
 On voit donc que $l_n$ est positif ou nul, mais toujours borné. Le cas où $l_n$ est infini est donc exclu.
 
 La contrainte budgétaire des consommateurs $j$ s'écrit :
 \begin{equation}\tag{CB} \label{CB}
 \sum_i p_i c_i^j = \sum_i p_i w_i^j + s l^j \ \forall \ j \Rightarrow \sum_j \sum_i p_i c_i^j = \sum_j \sum_i p_i w_i^j + s \sum_j l^j
\end{equation}  
 
 Comme les $n-1$ premiers marchés sont à l'équilibre, on peut écrire : 
 
 $$ \sum_j \sum_i^{n-1} p_i c_i^j =  \sum_j \sum_i^{n-1} p_i w_i^j + \sum_i^{n-1} p_i q_i $$
 
 Donc, la contrainte budgétaire agrégée \ref{CB} peut s'écrire : 
 
  $$ \sum_i^{n-1} p_i q_i + \sum_j p_n c_n^j =  \sum_j p_n w_n^j + s \sum_j l^j \Rightarrow  \sum_j p_n c_n^j =  \sum_j p_n w_n^j + sl_n   $$
  
  Si $l_n = 0$, on voit qu'il existe un prix $p_n$ tel que $\sum_j p_n c_n^j =  \sum_j p_n w_n^j    $, i.e. il y a équilibre (mêmes arguments que sans production).

Si $l_n>0$, en posant $p_n = \frac{s l_n}{q_n}$, on peut réécrire la contrainte budgétaire :



$$ p_n \sum_j  c_n^j = p_n \sum_j w_n^j+ sl_n \Rightarrow      \frac{s l_n}{q_n} \sum_j  c_n^j = \frac{s l_n}{q_n} \sum_j w_n^j+ sl_n \Rightarrow \sum_j c_n^j = \sum_j w_n^j+ q_n  $$

Ce qui est bien la condition d'équilibre du marché du bien $n$. Il y a donc bien un prix qui apure le marché du bien $n$ et celui-ci correspond à la condition de zéro profit de la firme produisant le bien $n$. Il y a donc bien équilibre général.




\end{document}
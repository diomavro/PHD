%\documentclass[AER]{AEA}
\documentclass[12pt]{report}
%\documentclass[12pt]{article}
%\documentclass[12pt,a4paper]{article}

\usepackage[utf8]{inputenc}


\usepackage{mathtools}
\usepackage{amsmath}
\usepackage{amssymb}
\usepackage{amsthm}

\usepackage{float}
%\usepackage[cmbold]{mathtime}
%\usepackage{mt11p}
\usepackage{placeins}
\usepackage{caption}
\usepackage{color}
\usepackage{subfigure}
\usepackage{multirow}
\usepackage{epsfig}
\usepackage{listings}
\usepackage{enumitem}
\usepackage{rotating,tabularx}
%\usepackage[graphicx]{realboxes}
\usepackage{graphicx}
\usepackage{graphics}
\usepackage{epstopdf}
\usepackage{longtable}
\usepackage[pdftex]{hyperref}
%\usepackage{breakurl}
\usepackage{epigraph}
\usepackage{xspace}
\usepackage{amsfonts}
\usepackage{eurosym}
\usepackage{ulem}

\usepackage{lmodern}
\usepackage{tikz}
\usetikzlibrary{spy}

\usepackage{verbatim}


\usepackage[framemethod=TikZ]{mdframed}
\usepackage{lipsum}
\mdfdefinestyle{MyFrame}{%
linecolor = blue, 
outerlinewidth=2pt,
roundcorner=20pt,
innertopmargin=\baselineskip,
innerbottommargin=\baselineskip,
innerrightmargin=20pt,
innerleftmargin=20pt,
backgroundcolor=gray!50!white}

\usepackage[most]{tcolorbox}


\usepackage{footmisc}
\usepackage{comment}
\usepackage{setspace}
\usepackage{geometry}
\usepackage{caption}
\usepackage{pdflscape}
\usepackage{array}
\usepackage[authoryear]{natbib}
\usepackage{booktabs}
\usepackage{dcolumn}
\usepackage{mathrsfs}
%\usepackage[justification=centering]{caption}
%\captionsetup[table]{format=plain,labelformat=simple,labelsep=period,singlelinecheck=true}%
\bibliographystyle{apalike}
%\bibliographystyle{unsrtnat}



%\bibliographystyle{aea}
\usepackage{enumitem}
\usepackage{tikz}
\usetikzlibrary{positioning}
\usetikzlibrary{arrows}
\usetikzlibrary{shapes.multipart}

\usetikzlibrary{shapes}
\def\checkmark{\tikz\fill[scale=0.4](0,.35) -- (.25,0) -- (1,.7) -- (.25,.15) -- cycle;}
%\usepackage{tikz}
%\usetikzlibrary{snakes}
%\usetikzlibrary{patterns}

%\draftSpacing{1.5}

\usepackage{xcolor}
\hypersetup{
colorlinks,
linkcolor={blue!50!black},
citecolor={blue!50!black},
urlcolor={blue!50!black}}

%\renewcommand{\familydefault}{\sfdefault}
%\usepackage{helvet}
%\setlength{\parindent}{0.4cm}
%\setlength{\parindent}{2em}
%\setlength{\parskip}{1em}

%\normalem

%\doublespacing
\onehalfspacing
%\singlespacing
%\linespread{1.5}

\newtheorem{theorem}{Theorem}
\newtheorem{corollary}[theorem]{Corollary}
\newtheorem{proposition}{Proposition}
\newtheorem{definition}{Definition}
\newtheorem{axiom}{Axiom}
\newtheorem{observation}{Observation}
\newtheorem{assumption}{Assumption}	
\newtheorem{remark}{Remark}
\newtheorem{lemma}{Lemma}
\newtheorem{result}{result}


\newcommand{\ra}[1]{\renewcommand{\arraystretch}{#1}}

\newcommand{\E}{\mathrm{E}}
\newcommand{\Var}{\mathrm{Var}}
\newcommand{\Corr}{\mathrm{Corr}}
\newcommand{\Cov}{\mathrm{Cov}}

\newcolumntype{d}[1]{D{.}{.}{#1}} % "decimal" column type
\renewcommand{\ast}{{}^{\textstyle *}} % for raised "asterisks"

\newtheorem{hyp}{Hypothesis}
\newtheorem{subhyp}{Hypothesis}[hyp]
\renewcommand{\thesubhyp}{\thehyp\alph{subhyp}}

\newcommand{\red}[1]{{\color{red} #1}}
\newcommand{\blue}[1]{{\color{blue} #1}}

%\newcommand*{\qed}{\hfill\ensuremath{\blacksquare}}%

\newcolumntype{L}[1]{>{\raggedright\let\newline\\arraybackslash\hspace{0pt}}m{#1}}
\newcolumntype{C}[1]{>{\centering\let\newline\\arraybackslash\hspace{0pt}}m{#1}}
\newcolumntype{R}[1]{>{\raggedleft\let\newline\\arraybackslash\hspace{0pt}}m{#1}}

%\geometry{left=1.5in,right=1.5in,top=1.5in,bottom=1.5in}
\geometry{left=1in,right=1in,top=1in,bottom=1in}

\epstopdfsetup{outdir=./}

\newcommand{\elabel}[1]{\label{eq:#1}}
\newcommand{\eref}[1]{Eq.~(\ref{eq:#1})}
\newcommand{\ceref}[2]{(\ref{eq:#1}#2)}
\newcommand{\Eref}[1]{Equation~(\ref{eq:#1})}
\newcommand{\erefs}[2]{Eqs.~(\ref{eq:#1}--\ref{eq:#2})}

\newcommand{\Sref}[1]{Section~\ref{sec:#1}}
\newcommand{\sref}[1]{Sec.~\ref{sec:#1}}

\newcommand{\Pref}[1]{Proposition~\ref{prop:#1}}
\newcommand{\pref}[1]{Prop.~\ref{prop:#1}}
\newcommand{\preflong}[1]{proposition~\ref{prop:#1}}

\newcommand{\Aref}[1]{Axiom~\ref{ax:#1}}

\newcommand{\clabel}[1]{\label{coro:#1}}
\newcommand{\Cref}[1]{Corollary~\ref{coro:#1}}
\newcommand{\cref}[1]{Cor.~\ref{coro:#1}}
\newcommand{\creflong}[1]{corollary~\ref{coro:#1}}

\newcommand{\etal}{{\it et~al.}\xspace}
\newcommand{\ie}{{\it i.e.}\ }
\newcommand{\eg}{{\it e.g.}\ }
\newcommand{\etc}{{\it etc.}\ }
\newcommand{\cf}{{\it c.f.}\ }
\newcommand{\ave}[1]{\left\langle#1 \right\rangle}
\newcommand{\person}[1]{{\it \sc #1}}

\newcommand{\AAA}[1]{\red{{\it AA: #1 AA}}}
\newcommand{\YB}[1]{\blue{{\it YB: #1 YB}}}

\newcommand{\flabel}[1]{\label{fig:#1}}
\newcommand{\fref}[1]{Fig.~\ref{fig:#1}}
\newcommand{\Fref}[1]{Figure~\ref{fig:#1}}

\newcommand{\tlabel}[1]{\label{tab:#1}}
\newcommand{\tref}[1]{Tab.~\ref{tab:#1}}
\newcommand{\Tref}[1]{Table~\ref{tab:#1}}

\newcommand{\be}{\begin{equation}}
\newcommand{\ee}{\end{equation}}
\newcommand{\bea}{\begin{eqnarray}}
\newcommand{\eea}{\end{eqnarray}}

\newcommand{\bi}{\begin{itemize}}
\newcommand{\ei}{\end{itemize}}

\newcommand{\Dt}{\Delta t}
\newcommand{\Dx}{\Delta x}
\newcommand{\Epsilon}{\mathcal{E}}
\newcommand{\etau}{\tau^\text{eqm}}
\newcommand{\wtau}{\widetilde{\tau}}
\newcommand{\xN}{\ave{x}_N}
\newcommand{\Sdata}{S^{\text{data}}}
\newcommand{\Smodel}{S^{\text{model}}}

\newcommand{\del}{D}
\newcommand{\hor}{H}



\setlength{\parindent}{0.0cm}
\setlength{\parskip}{0.4em}

\numberwithin{equation}{section}
\DeclareMathOperator\erf{erf}
%\let\endtitlepage\relax



% https://medium.com/@aerinykim/why-the-normal-gaussian-pdf-looks-the-way-it-does-1cbcef8faf0a

\begin{document}

\tableofcontents 

Scalia vs Dworkin, as they said vs as they intended. 

Scalia: Courts should remedy scriveners error
against strict constructionism

Dworkin: Since we can use intent for the firearm and first amendment example, Scalia must make a distinction between kinds of intent. 
Brown clearly wasn't the intent but WAS the tex. Capital punishment may be unconstitutional because they inteded for whatever is intended at the time to be bad, would be bad later on. 

More details about the church cases needed, 

Firearm case, scalia says too literal, first amendment should apply to written even if not explicit. 

the notion of lesiglative intent. 

There is a bit of an academic war occuring about constitutionalism on twitter. The war is about whether the constituion should be interpreted from the point of view of the "common good" or if it should follow something along the lines of Scalias originalism. I want to make a brief argument that I think shows that these are the same thing. 

What is the common good? It is what is good for everybody. 

What does this mean in the light of law? It means the law should allow individuals to pursue the common good. 

The common good is not static and the future is unknown. Creating and fostering the common good is a dynamic process, A needs to plan and to be able plan, A needs to know what environment she is in. An important component of the environment is it's legal environment. The problem with dynamic processes is that they are in the future, as such one has radical uncertainty about what will occur. Therefoe this uncertainty promotes the common good.

What does an individual need to make plans? An individual need to have legal uncertainty minimized. A decision that cannot be predicted, from the point of view of the individual, is arbitrary. So how can this individual have this uncertainty minimized?

The french have a related concept in law, "securité juridique". The idea is that 

A common good constitutionalism would look like textualism.

What I find baffling is that there is somehow a group of scholars who simulatenously defend home schooling AND defend common good constitutionalism. I'm going to give a VERY brief argument about why this position is untenable unless they expect home schooling to ONLY be the education of the Christian flavor. 

What the french call "securité juridique" can be more succincly be articulated, as a vital function of law is that people can predict the results of going to the judge. 

This is why the concept encompasses, numerous axes. Not having too many laws, this is obvious, people should be able to learn the laws, if there too many they won't be able to. If the laws are domain dependent, then there is an additional issue, if for instance industry X has a set of laws regulating and industry has a set Y, then there is switching cost per industry. Perhaps more generally, the less capable common men become of knowing the laws, the more the experts can extract rent from them. 

Complexity of the law is another axe, as Yaneer Bar Yam likes to point out, complexity emerges when components are dependent of one another for it to be made clear. 

Probably the most important aspect for predictability is incoherence and arbitrariness. For instance a law that says kids under 18 cannot drink when paired with ANOTHER law that says blonde haired beings can drink. This leaves the question open, what about blonde kids under 18? When a law has this characteristic, it means that suddenly there is ambiguity. A judge could favor one or the other, as such to precit the outcome, one would have to know what is in the MIND of the judge. 

Retroactive laws are also an important component, by definition, a law that is retroactive cannot be foreseen unless one can predict psychology. 

Is there a case where Investment can be asset-specific, agent-specific, or relationship-specific. 
Those whose investments are agent or relationship-specific 


 NOT having too many laws, not having a high complexity in the sense that the concepts are not too interrelated,  



Investment can be asset-specific, agent-specific, or relationship-specific. 
Those whose investments are agent or relationship-specific 


Machiavelli and the sigma algebra. There are many ways to cut up the world into categories, but is there a different category for each subject matter? Perhaps but Machiavelli does something different, the Prince and the discources discuss the same subject matter but from a different perspective, the framing is different, one is pitched at the scale of the prince, while the other is pitched at the scale of a republic as a whole. 

The Hayekian proviso. 



\bibliography{../thesisbib/bibliography}

\end{document}

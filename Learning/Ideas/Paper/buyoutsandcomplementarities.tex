%\documentclass[AER]{AEA}
\documentclass[12pt]{report}
%\documentclass[12pt]{article}
%\documentclass[12pt,a4paper]{article}

\usepackage[utf8]{inputenc}


\usepackage{mathtools}
\usepackage{amsmath}
\usepackage{amssymb}
\usepackage{amsthm}

\usepackage{float}
%\usepackage[cmbold]{mathtime}
%\usepackage{mt11p}
\usepackage{placeins}
\usepackage{caption}
\usepackage{color}
\usepackage{subfigure}
\usepackage{multirow}
\usepackage{epsfig}
\usepackage{listings}
\usepackage{enumitem}
\usepackage{rotating,tabularx}
%\usepackage[graphicx]{realboxes}
\usepackage{graphicx}
\usepackage{graphics}
\usepackage{epstopdf}
\usepackage{longtable}
\usepackage[pdftex]{hyperref}
%\usepackage{breakurl}
\usepackage{epigraph}
\usepackage{xspace}
\usepackage{amsfonts}
\usepackage{eurosym}
\usepackage{ulem}

\usepackage{tikz}
\usetikzlibrary{spy}

\usepackage{verbatim}



\usepackage{footmisc}
\usepackage{comment}
\usepackage{setspace}
\usepackage{geometry}
\usepackage{caption}
\usepackage{pdflscape}
\usepackage{array}
\usepackage[authoryear]{natbib}
\usepackage{booktabs}
\usepackage{dcolumn}
\usepackage{mathrsfs}
%\usepackage[justification=centering]{caption}
%\captionsetup[table]{format=plain,labelformat=simple,labelsep=period,singlelinecheck=true}%
\bibliographystyle{apalike}
%\bibliographystyle{unsrtnat}



%\bibliographystyle{aea}
\usepackage{enumitem}
\usepackage{tikz}
\usetikzlibrary{positioning}
\usetikzlibrary{arrows}
\usetikzlibrary{shapes.multipart}

\usetikzlibrary{shapes}
\def\checkmark{\tikz\fill[scale=0.4](0,.35) -- (.25,0) -- (1,.7) -- (.25,.15) -- cycle;}
%\usepackage{tikz}
%\usetikzlibrary{snakes}
%\usetikzlibrary{patterns}

%\draftSpacing{1.5}

\usepackage{xcolor}
\hypersetup{
colorlinks,
linkcolor={blue!50!black},
citecolor={blue!50!black},
urlcolor={blue!50!black}}

%\renewcommand{\familydefault}{\sfdefault}
%\usepackage{helvet}
%\setlength{\parindent}{0.4cm}
%\setlength{\parindent}{2em}
%\setlength{\parskip}{1em}

%\normalem

%\doublespacing
\onehalfspacing
%\singlespacing
%\linespread{1.5}

\newtheorem{theorem}{Theorem}
\newtheorem{corollary}[theorem]{Corollary}
\newtheorem{proposition}{Proposition}
\newtheorem{definition}{Definition}
\newtheorem{axiom}{Axiom}
\newtheorem{observation}{Observation}
\newtheorem{assumption}{Assumption}	
\newtheorem{remark}{Remark}
\newtheorem{lemma}{Lemma}
\newtheorem{result}{result}


\newcommand{\ra}[1]{\renewcommand{\arraystretch}{#1}}

\newcommand{\E}{\mathrm{E}}
\newcommand{\Var}{\mathrm{Var}}
\newcommand{\Corr}{\mathrm{Corr}}
\newcommand{\Cov}{\mathrm{Cov}}

\newcolumntype{d}[1]{D{.}{.}{#1}} % "decimal" column type
\renewcommand{\ast}{{}^{\textstyle *}} % for raised "asterisks"

\newtheorem{hyp}{Hypothesis}
\newtheorem{subhyp}{Hypothesis}[hyp]
\renewcommand{\thesubhyp}{\thehyp\alph{subhyp}}

\newcommand{\red}[1]{{\color{red} #1}}
\newcommand{\blue}[1]{{\color{blue} #1}}

%\newcommand*{\qed}{\hfill\ensuremath{\blacksquare}}%

\newcolumntype{L}[1]{>{\raggedright\let\newline\\arraybackslash\hspace{0pt}}m{#1}}
\newcolumntype{C}[1]{>{\centering\let\newline\\arraybackslash\hspace{0pt}}m{#1}}
\newcolumntype{R}[1]{>{\raggedleft\let\newline\\arraybackslash\hspace{0pt}}m{#1}}

%\geometry{left=1.5in,right=1.5in,top=1.5in,bottom=1.5in}
\geometry{left=1in,right=1in,top=1in,bottom=1in}

\epstopdfsetup{outdir=./}

\newcommand{\elabel}[1]{\label{eq:#1}}
\newcommand{\eref}[1]{Eq.~(\ref{eq:#1})}
\newcommand{\ceref}[2]{(\ref{eq:#1}#2)}
\newcommand{\Eref}[1]{Equation~(\ref{eq:#1})}
\newcommand{\erefs}[2]{Eqs.~(\ref{eq:#1}--\ref{eq:#2})}

\newcommand{\Sref}[1]{Section~\ref{sec:#1}}
\newcommand{\sref}[1]{Sec.~\ref{sec:#1}}

\newcommand{\Pref}[1]{Proposition~\ref{prop:#1}}
\newcommand{\pref}[1]{Prop.~\ref{prop:#1}}
\newcommand{\preflong}[1]{proposition~\ref{prop:#1}}

\newcommand{\Aref}[1]{Axiom~\ref{ax:#1}}

\newcommand{\clabel}[1]{\label{coro:#1}}
\newcommand{\Cref}[1]{Corollary~\ref{coro:#1}}
\newcommand{\cref}[1]{Cor.~\ref{coro:#1}}
\newcommand{\creflong}[1]{corollary~\ref{coro:#1}}

\newcommand{\etal}{{\it et~al.}\xspace}
\newcommand{\ie}{{\it i.e.}\ }
\newcommand{\eg}{{\it e.g.}\ }
\newcommand{\etc}{{\it etc.}\ }
\newcommand{\cf}{{\it c.f.}\ }
\newcommand{\ave}[1]{\left\langle#1 \right\rangle}
\newcommand{\person}[1]{{\it \sc #1}}

\newcommand{\AAA}[1]{\red{{\it AA: #1 AA}}}
\newcommand{\YB}[1]{\blue{{\it YB: #1 YB}}}

\newcommand{\flabel}[1]{\label{fig:#1}}
\newcommand{\fref}[1]{Fig.~\ref{fig:#1}}
\newcommand{\Fref}[1]{Figure~\ref{fig:#1}}

\newcommand{\tlabel}[1]{\label{tab:#1}}
\newcommand{\tref}[1]{Tab.~\ref{tab:#1}}
\newcommand{\Tref}[1]{Table~\ref{tab:#1}}

\newcommand{\be}{\begin{equation}}
\newcommand{\ee}{\end{equation}}
\newcommand{\bea}{\begin{eqnarray}}
\newcommand{\eea}{\end{eqnarray}}

\newcommand{\bi}{\begin{itemize}}
\newcommand{\ei}{\end{itemize}}

\newcommand{\Dt}{\Delta t}
\newcommand{\Dx}{\Delta x}
\newcommand{\Epsilon}{\mathcal{E}}
\newcommand{\etau}{\tau^\text{eqm}}
\newcommand{\wtau}{\widetilde{\tau}}
\newcommand{\xN}{\ave{x}_N}
\newcommand{\Sdata}{S^{\text{data}}}
\newcommand{\Smodel}{S^{\text{model}}}

\newcommand{\del}{D}
\newcommand{\hor}{H}



\setlength{\parindent}{0.0cm}
\setlength{\parskip}{0.4em}

\numberwithin{equation}{section}
\DeclareMathOperator\erf{erf}
%\let\endtitlepage\relax



% https://medium.com/@aerinykim/why-the-normal-gaussian-pdf-looks-the-way-it-does-1cbcef8faf0a

\begin{document}

\section{A simple theory of scale}


We present the reduced form static version of the model because it illustrates that the specific complementarity or substitutability does not matter but instead it is only the \text{relative} value of purchasing that plays a role.  We briefly present the taxonomy under the framework and discuss why each situation may occur. The model consists of two stages. First the entrepreneur receives offers from potential firms that will buyout her innovation. Stage two is simply the profits being realized.  

The market profit potential of the innovation which the entrepreneur holds is given by $\pi^e$, this profit may be earned by whoever owns the entrepreneurial project. The profit of the incumbent \textit{if the innovation is not used} is simply $\pi^i$. We denote the degree(or factor) of substitutability/complementarity \textbf{if not bought} by $\beta \in [ 0, \infty [$ and the degree of substitutability/complementarity  \textbf{if bought} by $\alpha \in [0, \infty [ $. 

\begin{proposition}
Suppose the incumbent has the option of shutting down projects. The premium the incumbent is willing to pay is $\frac{(\alpha-\beta)\pi^i}{\pi^e}$
\end{proposition}

\begin{proof}
The payoff if not bought is $ \beta \pi^i$. The payoff if bought is $\max\{ \pi^i, \pi^e + \alpha \pi^i   \}$. Therefore the willingness to pay for the product if  $\max\{ \pi^i, \pi^e + \alpha \pi^i   \} = \pi^i $ is: $(1-\beta) \pi^i$ and if $\max\{ \pi^i, \pi^e + \alpha \pi^i   \} = \pi^e + \alpha \pi^i $ the willingness to pay is: $\pi^e+ (\alpha-\beta) \pi^i$. The extra willingness to pay of the active firm is then simply: $(\alpha-\beta)\pi^i$. This form shows us that substitutability or complementarity do not matter for buyouts, instead it is only the \textit{relative} effects of buyouts which affect the premium the incumbent is willing to pay. Note that $\pi^i$ can then be seen as the \textit{scale} parameter. Or to express it another way, let $\zeta=1 + \frac{(\alpha-\beta)\pi^i}{\pi^e}$. If $\zeta$ is larger than 1, then the existing firm is willing to pay a premium and if the existing activity is of larger scale relative to the project, the incumbent is willing to pay a higher premium. We now briefly discuss the taxonomy of this framework. 
\end{proof}

\begin{corollary}
The higher the profits of the incumbent, the relatively higher is it's willingness to pay relative to an uncorrolated firm. 
\end{corollary}

The alternative is that buying out does not give the option of shutting projects down. Instead the buyout has the effect of making something open source. 

\textbf{Substitute}  if not bought and \textbf{Complementary} if bought implies: $\beta<1$ and $\alpha>1$. This case implies that the product will eat up the profits of the incumbent if allowed to compete with the current product but will expand profits if held together with the current activity. 

\textbf{Complementary} if not bought and \textbf{Complementary} if bought implies: $\beta>1$ and $\alpha>1$. This is just the case where whether the innovation is bought or not, the firm will benefit from it.

\textbf{Complementary} if not bought and \textbf{Neutral} if bought implies: $\beta>1$ and $\alpha=1$. Why would the project not be complementary if bought? If consumers have a specific aversion to buying things from one firm. 

\textbf{Neutral} if not bought and \textbf{Neutral} if bought implies: $\beta=1$ and $\alpha=1$. This is the case where the entrepreneur's project is uncorrelated to the incumbents current activity.  

\textbf{Neutral} if not bought and \textbf{Complementary} if bought implies: $\beta=1$ and $\alpha>1$. If the technology is complementary this may result either because the production process becomes more efficient or because there is a bundling effect if both goods are sold together. 

\textbf{Neutral} if not bought and \textbf{Substitute} if bought implies: $\beta=1$ and $\alpha<1$. This case would result in shutting down the project. It may be that if the firm markets some new product, the customers of this specific firm will flee to it. 

\textbf{Substitute} if not bought and \textbf{Neutral} if bought implies: $\beta<1$ and $\alpha=1$. Why would a project not be substitable if bought? Perhaps there is a certain way of selling the product that would interact with the incumbents product market but if the incumbent owns it, they can find a niche way to market it that allows it to be realized without eating away at their other products.



\bibliography{../thesisbib/bibliography}

\end{document}

\documentclass[11pt]{article}
%\documentclass[12pt]{article}
%\documentclass[12pt]{article}
%\documentclass[12pt,a4paper]{article}

\usepackage[percent]{overpic}
\usepackage{float}
\usepackage{pgfplots}
%\usepackage[cmbold]{mathtime}
%\usepackage{mt11p}
\usepackage{placeins}
\usepackage{amsmath}
\usepackage{amsthm}
\usepackage{color}
\usepackage{amssymb}
\usepackage{mathtools}
\usepackage{subfigure}
\usepackage{multirow}
\usepackage{epsfig}
\usepackage{listings}
\usepackage{enumitem}
\usepackage{rotating,tabularx}
%\usepackage[graphicx]{realboxes}
\usepackage{graphicx}
\usepackage{graphics}
\usepackage{epstopdf}
\usepackage{longtable}
\usepackage[pdftex]{hyperref}
%\usepackage{breakurl}
\usepackage{epigraph}
\usepackage{xspace}
\usepackage{amsfonts}
\usepackage{eurosym}
\usepackage{ulem}
\usepackage{footmisc}
\usepackage{comment}
\usepackage{setspace}
\usepackage{geometry}
\usepackage{caption}
\usepackage{pdflscape}
\usepackage{array}
\usepackage[round]{natbib}
\usepackage{booktabs}
\usepackage{dcolumn}
\usepackage{mathrsfs}
%\usepackage[justification=centering]{caption}
%\captionsetup[table]{format=plain,labelformat=simple,labelsep=period,singlelinecheck=true}%

%\bibliographystyle{unsrtnat}
\bibliographystyle{aea}
\usepackage{enumitem}
\usepackage{tikz}
\usetikzlibrary{decorations.pathreplacing}
%\def\checkmark{\tikz\fill[scale=0.4](0,.35) -- (.25,0) -- (1,.7) -- (.25,.15) -- cycle;}
%\usepackage{tikz}
%\usetikzlibrary{snakes}
%\usetikzlibrary{patterns}

%\draftSpacing{1.5}

\usepackage{xcolor}
\hypersetup{
colorlinks,
linkcolor={blue!50!black},
citecolor={blue!50!black},
urlcolor={blue!50!black}}

%\renewcommand{\familydefault}{\sfdefault}
%\usepackage{helvet}
%\setlength{\parindent}{0.4cm}
%\setlength{\parindent}{2em}
%\setlength{\parskip}{1em}

%\normalem

%\doublespacing
\onehalfspacing
%\singlespacing
%\linespread{1.5}

\newtheorem{theorem}{Theorem}
\newcommand{\bc}{\begin{center}}
\newcommand{\ec}{\end{center}}
\newtheorem{corollary}[theorem]{Corollary}
\newtheorem{proposition}{Proposition}
\newtheorem{definition}{Definition}
\newtheorem{axiom}{Axiom}
\newcommand{\ra}[1]{\renewcommand{\arraystretch}{#1}}

\newcommand{\E}{\mathrm{E}}
\newcommand{\Var}{\mathrm{Var}}
\newcommand{\Corr}{\mathrm{Corr}}
\newcommand{\Cov}{\mathrm{Cov}}

\newcolumntype{d}[1]{D{.}{.}{#1}} % "decimal" column type
\renewcommand{\ast}{{}^{\textstyle *}} % for raised "asterisks"

\newtheorem{hyp}{Hypothesis}
\newtheorem{subhyp}{Hypothesis}[hyp]
\renewcommand{\thesubhyp}{\thehyp\alph{subhyp}}

\newcommand{\red}[1]{{\color{red} #1}}
\newcommand{\blue}[1]{{\color{blue} #1}}

%\newcommand*{\qed}{\hfill\ensuremath{\blacksquare}}%

\newcolumntype{L}[1]{>{\raggedright\let\newline\\arraybackslash\hspace{0pt}}m{#1}}
\newcolumntype{C}[1]{>{\centering\let\newline\\arraybackslash\hspace{0pt}}m{#1}}
\newcolumntype{R}[1]{>{\raggedleft\let\newline\\arraybackslash\hspace{0pt}}m{#1}}

%\geometry{left=1.5in,right=1.5in,top=1.5in,bottom=1.5in}
\geometry{left=1in,right=1in,top=1in,bottom=1in}

\epstopdfsetup{outdir=./}

\newcommand{\elabel}[1]{\label{eq:#1}}
\newcommand{\eref}[1]{Eq.~(\ref{eq:#1})}
\newcommand{\ceref}[2]{(\ref{eq:#1}#2)}
\newcommand{\Eref}[1]{Equation~(\ref{eq:#1})}
\newcommand{\erefs}[2]{Eqs.~(\ref{eq:#1}--\ref{eq:#2})}

\newcommand{\Sref}[1]{Section~\ref{sec:#1}}
\newcommand{\sref}[1]{Sec.~\ref{sec:#1}}

\newcommand{\Pref}[1]{Proposition~\ref{prop:#1}}
\newcommand{\pref}[1]{Prop.~\ref{prop:#1}}
\newcommand{\preflong}[1]{proposition~\ref{prop:#1}}

\newcommand{\Aref}[1]{Axiom~\ref{ax:#1}}

\newcommand{\clabel}[1]{\label{coro:#1}}
\newcommand{\Cref}[1]{Corollary~\ref{coro:#1}}
\newcommand{\cref}[1]{Cor.~\ref{coro:#1}}
\newcommand{\creflong}[1]{corollary~\ref{coro:#1}}

\newcommand{\etal}{{\it et~al.}\xspace}
\newcommand{\ie}{{\it i.e.}\xspace}
\newcommand{\eg}{{\it e.g.}\xspace}
\newcommand{\etc}{{\it etc.}\xspace}
\newcommand{\cf}{{\it c.f.}\xspace}
\newcommand{\ave}[1]{\left\langle#1 \right\rangle}
\newcommand{\person}[1]{{\it \sc #1}}

\newcommand{\AAA}[1]{\red{{\it AA: #1 AA}}}
\newcommand{\YB}[1]{\blue{{\it YB: #1 YB}}}

\newcommand{\flabel}[1]{\label{fig:#1}}
\newcommand{\fref}[1]{Fig.~\ref{fig:#1}}
\newcommand{\Fref}[1]{Figure~\ref{fig:#1}}

\newcommand{\tlabel}[1]{\label{tab:#1}}
\newcommand{\tref}[1]{Tab.~\ref{tab:#1}}
\newcommand{\Tref}[1]{Table~\ref{tab:#1}}

\newcommand{\be}{\begin{equation}}
\newcommand{\ee}{\end{equation}}
\newcommand{\bea}{\begin{eqnarray}}
\newcommand{\eea}{\end{eqnarray}}

\newcommand{\bi}{\begin{itemize}}
\newcommand{\ei}{\end{itemize}}

\newcommand{\Dt}{\Delta t}
\newcommand{\Dx}{\Delta x}
\newcommand{\Epsilon}{\mathcal{E}}
\newcommand{\etau}{\tau^\text{eqm}}
\newcommand{\wtau}{\widetilde{\tau}}
\newcommand{\xN}{\ave{x}_N}
\newcommand{\Sdata}{S^{\text{data}}}
\newcommand{\Smodel}{S^{\text{model}}}

\newcommand{\del}{D}
\newcommand{\hor}{H}
\newcommand{\subhead}[1]{\mbox{}\newline\textbf{#1}\newline}

\setlength{\parindent}{0.0cm}
\setlength{\parskip}{0.5em}

\numberwithin{equation}{section}
\DeclareMathOperator\erf{erf}
%\let\endtitlepage\relax

\begin{document}

%\onehalfspacing

Let $(x ,\Delta x,t)$ represent a payment of $\Delta x $ at time $t$ with initial wealth $x$, where $x, \Delta x, t \in \mathbb{R}$ and $t \geq 0$. Let $ \{ \succsim \}^{\infty}_{t=0}$ represent the decision makers preferences over the payments at time $t$. Similarly for $\precsim_t, \sim_t, \prec_t $ and $\succ_t$ \footnote{Horizon indipendent: stationarity, no preference reversal: time consistent}



\begin{definition}
$\{ \succsim \}^{\infty}_{t=0}$ is \textit{stationary} if for all $t_a,t_b, t,\tau \in \mathbb{R}$, and $t - \tau > 0$ 
\begin{equation}
(x_a,\Delta x_a, t_a) \succsim_{t} (x_b,\Delta x_b, t_b) \leftrightarrow (x_a,\Delta x_a, t_a + \tau ) \succsim_t (x_b,\Delta x_b, t_b + \tau )
\end{equation}
\end{definition}

Stationarity implies that only the distance between the wealths $(x_a-x_b)$, the payments $(\Delta x_a-\Delta x_b)$, and the delay matter $(t_b-t_a)$. This kind of property implies that if an indifference relation is true for a time $\tau$,$ \succsim_{\tau} $, it is true for all preference relations, $\{ \succsim \}^{\infty}_{\tau=0}$. 


\begin{definition}
$\{ \succsim \}^{\infty}_{t=0}$ are time invariant if for all pairs of $\tau,\tau'$: 
\begin{equation}
(x_a,\Delta x_a, t_a+\tau) \succsim_{\tau} (x_b,\Delta x_b, t_b+ \tau) \leftrightarrow (x_a,\Delta x_a, t_a+\tau') \succsim_{\tau'} (x_b,\Delta x_b, t_b+\tau')
\end{equation}
\end{definition}

\begin{definition}
$\{ \succsim \}^{\infty}_{t=0}$ are time consistent if for all $t,t'$
\begin{equation}
(x_a,\Delta x_a, t_a) \succsim_t (x_b,\Delta x_b, t_b) \leftrightarrow (x_a,\Delta x_a, t_a) \succsim_{t'} (x_b,\Delta x_b, t_b)
\end{equation}
\end{definition}

This property entails that agents will have consistent preferences independently of when the payment is evaluated. 

@article{Halevy2015,
author = {Halevy, Yoram},
journal = {Econometrica},
number = {1},
pages = {335--352},
title = {{Time Consistency: Stationarity and Time Invariance}},
volume = {83},
year = {2015}
}

Any two of the three properties implies the other (Halevy2015), so we can drop anyone of these without loss of generality. We now make three additional definitions which are unique to our model. 

\begin{definition}
$\{ \succsim \}^{\infty}_{t=0}$ is time independent if for all $t_a,t_a',t_b,t_b', t \in \mathbb{R}$: 
\begin{equation}
(x_a,\Delta x_a, t_a) \succsim_{t} (x_b,\Delta x_b, t_b) \leftrightarrow (x_a,\Delta x_a, t_a' ) \succsim_t (x_b,\Delta x_b, t_b')
\end{equation}
\end{definition}

Here I would fit a proposition that if time independent then ALSO none of the other definitions but it might be a bit trivial. 

\begin{definition}
$\{ \succsim \}^{\infty}_{\tau =0}$ are \textit{wealth independent} if for all $\tau$, and and for all $x_a,x_b,x_a',x_b'  \in \mathbb{R}$:
\begin{equation}
(x_a,\Delta x_a, t_a) \succsim_{t} (x_b,\Delta x_b, t_b) \leftrightarrow (x_a',\Delta x_a, t_a) \succsim_{t} (x_b',\Delta x_b, t_b)
\end{equation}
\end{definition}

This assumption is implicitely made in most of the literature.
\begin{definition}
We say that $\{ \succsim \}^{\infty}_{t=0}$ are growth optimal if they can be represented by a function, $g_t: [0,x] * [\underline{t}, T]*[\underline{t}, T]*[0, \Delta x] \rightarrow R$
\begin{equation}
(x_a,\Delta x_a, t_a) \succsim_t (x_b,\Delta x_b, t_b) \leftrightarrow g_t(x_a,\Delta x_a, .) \geq g_t(x_b,\Delta x_b, .)
\end{equation}
\end{definition}

\section{The growth rate function}

We now specify what is meant by growth rate.
\begin{definition}
We say that an agents time frame is fixed if on some time interval, $t_f$, if the growth is given by $g_t(x,t_f,.)$
\end{definition}
So for any given future payment, $\Delta x_i$ paid at some time $t_i<t_f$, the growth rate on a fixed time frame is $g(x_t,t_f,\Delta x_i, t_i)$.
% \begin{definition}{Riskless Intertemporal Payment Problem.}
\begin{definition}
We say that an agents time frame is adaptive if gambles are evaluated on their own horizon $g_t(x,\Delta x_i, t_i)$
\end{definition}

This corresponds to setting $t_f=t_i$ so that the two growth rates are equal if $g(x_t,t_f,\Delta x_i, t_i)=g(x_t,\Delta x_i, t_i)$.

% A Riskless Intertemporal Payment Problem (RIPP) is a vector $\{t_0,x\left(t_0\right),t_a,\Dx_a,t_b,\Dx_b\}$. A decision maker at time $t_0$ with wealth $x\left(t_0\right)$ must choose between two future cash payments, whose amounts and payment times are known with certainty. The two options are:
% %
% \begin{enumerate}
% \item[$a$.] an earlier payment of $\Dx_a$ at time $t_a>t_0$; and
% \item[$b$.] a later payment of $\Dx_b$ at time $t_b>t_a$.
% \end{enumerate}
% %
% \end{definition}

A criterion for choosing $a$ or $b$ is required. Here we explore what happens if that criterion is maximization of the growth rate of wealth, \ie if $a$ is chosen when it corresponds to a higher growth rate of the decision maker's wealth than $b$, and \textit{vice versa}.

A growth rate is defined as the scale parameter of time in the growth function of wealth subject to dynamics. Dynamics can take different forms, each corresponding to a different form of growth rate. We treat explicitly multiplicative and additive dynamics \citep{PetersGell-Mann2016}, noting that more general dynamics can be treated similarly \citep{PetersAdamou2018a}.

\begin{definition}
We say a dynamic is \textit{multiplicative} if $x\left(t\right)$ has the process: 
\be
x\left(t\right) = x\left(t_0\right) e^{r \left(t - t_0\right)}\,,
\ee
\end{definition}
Note here that $r$ is two things. $r$ is the interest rate (or a rate of return on investment) and corresponds to investing wealth in income-generating assets, where the income is proportional to the amount invested. Howevr $r$ is also the growth rate of wealth. We can solve for r to get the growth rate: 
\be
g_m=r = \frac{\log x\left(t+\Dt\right)-\log x\left(t\right)}{\Dt}\,,
\ee

\begin{definition}
We say a dynamic is \textit{additive} if $x\left(t\right)$ has the process: 
\be
x\left(t\right) = x\left(t_0\right) + k \left(t - t_0\right)\,,
\ee
\end{definition}

In this case the exogenous parameter is $k$, which also has two functions. $k$ resembles saved labor income or, more generally, situations where investment income is negligible and wealth changes by net flows that do not depend on wealth itself. It is also clear that in the additive case $k$ is the growth rate and the scale parameter of time, so we can solve for it once again: 
%
\be
g_a=k = \frac{x\left(t+\Dt\right) - x\left(t\right)}{\Dt}\,.
\ee
%
The functional form of the growth rate then differs between the dynamics. The growth rate between time $t$ and $t+\Dt$ can be extracted from the expression for the evolution of wealth over that period. The matching of growth rate with dynamics is crucial. An additive growth rate applied to wealth following a multiplicative process would vary with time, as would a multiplicative growth rate applied to additively-growing wealth. The correct growth rate extracts a stable parameter from the dynamics, allowing processes with the same type of dynamics to be compared.

The plan for fitting into the literature would be as follows: 
For case A we are, time indepedent and wealth independent. 
For case B we are, stationary, wealth independent but not time independent.
For case C we are, wealth independent but not stationary and not time independent or not time invariant
For case D: We are none of the above


%\begin{proposition}{The Maximization of Growth is Transitive.}
%
%Under the notation of \Aref{ax1}, the Transitivity Axiom is satisfied.
%\label{prop:trans}
%\end{proposition}
%\begin{proof}
%We assume three payments, $a\equiv\left(t_a,\Dx_a\right)$, $b\equiv\left(t_b,\Dx_b\right)$ and $c\equiv\left(t_c,\Dx_c\right)$, where $t_a < t_b < t_c$. Given time $t_0< t_a$ and initial wealth $x\left(t_0\right)$, the vectors $\{t_0,x\left(t_0\right),t_a,\Dx_a,t_b,\Dx_b\}$ and $\{t_0,x\left(t_0\right),t_b,\Dx_b,t_c,\Dx_c\}$ are RIPPs. If $a \prec b$ and $b \prec c$, then $g_a < g_b$ and $g_b < g_c$. Since $t_a < t_c$, then $\{t_0,x\left(t_0\right),t_a,\Dx_a,t_c,\Dx_c\}$ is also a RIPP and $g_a < g_c$. Therefore, $a \prec c$.
%\end{proof}


\end{document}
% Options for packages loaded elsewhere
\PassOptionsToPackage{unicode}{hyperref}
\PassOptionsToPackage{hyphens}{url}
%
\documentclass[
]{article}
\usepackage{lmodern}
\usepackage{amssymb,amsmath}
\usepackage{ifxetex,ifluatex}
\ifnum 0\ifxetex 1\fi\ifluatex 1\fi=0 % if pdftex
  \usepackage[T1]{fontenc}
  \usepackage[utf8]{inputenc}
  \usepackage{textcomp} % provide euro and other symbols
\else % if luatex or xetex
  \usepackage{unicode-math}
  \defaultfontfeatures{Scale=MatchLowercase}
  \defaultfontfeatures[\rmfamily]{Ligatures=TeX,Scale=1}
\fi
% Use upquote if available, for straight quotes in verbatim environments
\IfFileExists{upquote.sty}{\usepackage{upquote}}{}
\IfFileExists{microtype.sty}{% use microtype if available
  \usepackage[]{microtype}
  \UseMicrotypeSet[protrusion]{basicmath} % disable protrusion for tt fonts
}{}
\makeatletter
\@ifundefined{KOMAClassName}{% if non-KOMA class
  \IfFileExists{parskip.sty}{%
    \usepackage{parskip}
  }{% else
    \setlength{\parindent}{0pt}
    \setlength{\parskip}{6pt plus 2pt minus 1pt}}
}{% if KOMA class
  \KOMAoptions{parskip=half}}
\makeatother
\usepackage{xcolor}
\IfFileExists{xurl.sty}{\usepackage{xurl}}{} % add URL line breaks if available
\IfFileExists{bookmark.sty}{\usepackage{bookmark}}{\usepackage{hyperref}}
\hypersetup{
  pdftitle={Forecast\_1},
  pdfauthor={Diomides\_Mavroyiannis},
  hidelinks,
  pdfcreator={LaTeX via pandoc}}
\urlstyle{same} % disable monospaced font for URLs
\usepackage[margin=1in]{geometry}
\usepackage{color}
\usepackage{fancyvrb}
\newcommand{\VerbBar}{|}
\newcommand{\VERB}{\Verb[commandchars=\\\{\}]}
\DefineVerbatimEnvironment{Highlighting}{Verbatim}{commandchars=\\\{\}}
% Add ',fontsize=\small' for more characters per line
\usepackage{framed}
\definecolor{shadecolor}{RGB}{248,248,248}
\newenvironment{Shaded}{\begin{snugshade}}{\end{snugshade}}
\newcommand{\AlertTok}[1]{\textcolor[rgb]{0.94,0.16,0.16}{#1}}
\newcommand{\AnnotationTok}[1]{\textcolor[rgb]{0.56,0.35,0.01}{\textbf{\textit{#1}}}}
\newcommand{\AttributeTok}[1]{\textcolor[rgb]{0.77,0.63,0.00}{#1}}
\newcommand{\BaseNTok}[1]{\textcolor[rgb]{0.00,0.00,0.81}{#1}}
\newcommand{\BuiltInTok}[1]{#1}
\newcommand{\CharTok}[1]{\textcolor[rgb]{0.31,0.60,0.02}{#1}}
\newcommand{\CommentTok}[1]{\textcolor[rgb]{0.56,0.35,0.01}{\textit{#1}}}
\newcommand{\CommentVarTok}[1]{\textcolor[rgb]{0.56,0.35,0.01}{\textbf{\textit{#1}}}}
\newcommand{\ConstantTok}[1]{\textcolor[rgb]{0.00,0.00,0.00}{#1}}
\newcommand{\ControlFlowTok}[1]{\textcolor[rgb]{0.13,0.29,0.53}{\textbf{#1}}}
\newcommand{\DataTypeTok}[1]{\textcolor[rgb]{0.13,0.29,0.53}{#1}}
\newcommand{\DecValTok}[1]{\textcolor[rgb]{0.00,0.00,0.81}{#1}}
\newcommand{\DocumentationTok}[1]{\textcolor[rgb]{0.56,0.35,0.01}{\textbf{\textit{#1}}}}
\newcommand{\ErrorTok}[1]{\textcolor[rgb]{0.64,0.00,0.00}{\textbf{#1}}}
\newcommand{\ExtensionTok}[1]{#1}
\newcommand{\FloatTok}[1]{\textcolor[rgb]{0.00,0.00,0.81}{#1}}
\newcommand{\FunctionTok}[1]{\textcolor[rgb]{0.00,0.00,0.00}{#1}}
\newcommand{\ImportTok}[1]{#1}
\newcommand{\InformationTok}[1]{\textcolor[rgb]{0.56,0.35,0.01}{\textbf{\textit{#1}}}}
\newcommand{\KeywordTok}[1]{\textcolor[rgb]{0.13,0.29,0.53}{\textbf{#1}}}
\newcommand{\NormalTok}[1]{#1}
\newcommand{\OperatorTok}[1]{\textcolor[rgb]{0.81,0.36,0.00}{\textbf{#1}}}
\newcommand{\OtherTok}[1]{\textcolor[rgb]{0.56,0.35,0.01}{#1}}
\newcommand{\PreprocessorTok}[1]{\textcolor[rgb]{0.56,0.35,0.01}{\textit{#1}}}
\newcommand{\RegionMarkerTok}[1]{#1}
\newcommand{\SpecialCharTok}[1]{\textcolor[rgb]{0.00,0.00,0.00}{#1}}
\newcommand{\SpecialStringTok}[1]{\textcolor[rgb]{0.31,0.60,0.02}{#1}}
\newcommand{\StringTok}[1]{\textcolor[rgb]{0.31,0.60,0.02}{#1}}
\newcommand{\VariableTok}[1]{\textcolor[rgb]{0.00,0.00,0.00}{#1}}
\newcommand{\VerbatimStringTok}[1]{\textcolor[rgb]{0.31,0.60,0.02}{#1}}
\newcommand{\WarningTok}[1]{\textcolor[rgb]{0.56,0.35,0.01}{\textbf{\textit{#1}}}}
\usepackage{graphicx,grffile}
\makeatletter
\def\maxwidth{\ifdim\Gin@nat@width>\linewidth\linewidth\else\Gin@nat@width\fi}
\def\maxheight{\ifdim\Gin@nat@height>\textheight\textheight\else\Gin@nat@height\fi}
\makeatother
% Scale images if necessary, so that they will not overflow the page
% margins by default, and it is still possible to overwrite the defaults
% using explicit options in \includegraphics[width, height, ...]{}
\setkeys{Gin}{width=\maxwidth,height=\maxheight,keepaspectratio}
% Set default figure placement to htbp
\makeatletter
\def\fps@figure{htbp}
\makeatother
\setlength{\emergencystretch}{3em} % prevent overfull lines
\providecommand{\tightlist}{%
  \setlength{\itemsep}{0pt}\setlength{\parskip}{0pt}}
\setcounter{secnumdepth}{-\maxdimen} % remove section numbering

\title{Forecast\_1}
\author{Diomides\_Mavroyiannis}
\date{25/11/2020}

\begin{document}
\maketitle

\#Loading the data

\begin{Shaded}
\begin{Highlighting}[]
\NormalTok{data2 <-}\StringTok{ }\KeywordTok{read.csv}\NormalTok{(}\StringTok{"C:/Users/Dio and Nono/Desktop/Forecasting/Rstuff/Assignment2/Second bi-weekly assignment Train data.csv"}\NormalTok{, }\DataTypeTok{sep=}\StringTok{";"}\NormalTok{)}
\NormalTok{data3 <-}\StringTok{ }\KeywordTok{read.csv}\NormalTok{(}\StringTok{"C:/Users/Dio and Nono/Desktop/Forecasting/Rstuff/Assignment2/Second bi-weekly assignment Test data.csv"}\NormalTok{, }\DataTypeTok{sep=}\StringTok{";"}\NormalTok{)}

\CommentTok{#Preparing the library}
\KeywordTok{library}\NormalTok{(forecast)}
\end{Highlighting}
\end{Shaded}

\begin{verbatim}
## Registered S3 method overwritten by 'quantmod':
##   method            from
##   as.zoo.data.frame zoo
\end{verbatim}

\begin{Shaded}
\begin{Highlighting}[]
\KeywordTok{library}\NormalTok{(Mcomp)}
\KeywordTok{library}\NormalTok{(fpp)}
\end{Highlighting}
\end{Shaded}

\begin{verbatim}
## Loading required package: fma
\end{verbatim}

\begin{verbatim}
## Loading required package: expsmooth
\end{verbatim}

\begin{verbatim}
## Loading required package: lmtest
\end{verbatim}

\begin{verbatim}
## Loading required package: zoo
\end{verbatim}

\begin{verbatim}
## 
## Attaching package: 'zoo'
\end{verbatim}

\begin{verbatim}
## The following objects are masked from 'package:base':
## 
##     as.Date, as.Date.numeric
\end{verbatim}

\begin{verbatim}
## Loading required package: tseries
\end{verbatim}

\begin{Shaded}
\begin{Highlighting}[]
\CommentTok{#Part A}
\end{Highlighting}
\end{Shaded}

\begin{Shaded}
\begin{Highlighting}[]
\CommentTok{#set to time series monthly}
\NormalTok{tspassengers <-}\StringTok{ }\KeywordTok{ts}\NormalTok{(data2}\OperatorTok{$}\NormalTok{AirPassengers , }\DataTypeTok{frequency =} \DecValTok{12}\NormalTok{)}
\NormalTok{tspassengers_test <-}\StringTok{ }\KeywordTok{ts}\NormalTok{(data3}\OperatorTok{$}\NormalTok{AirPassengers , }\DataTypeTok{frequency =} \DecValTok{12}\NormalTok{)}
\CommentTok{#drop missing}
\NormalTok{dtspassengers  <-}\StringTok{  }\KeywordTok{na.omit}\NormalTok{(tspassengers)}
\CommentTok{#decompose monthly}
\NormalTok{adec <-}\StringTok{ }\KeywordTok{decompose}\NormalTok{(dtspassengers , }\DataTypeTok{type =} \StringTok{"additive"}\NormalTok{)}
\NormalTok{mdec <-}\StringTok{ }\KeywordTok{decompose}\NormalTok{(dtspassengers , }\DataTypeTok{type =} \StringTok{"multiplicative"}\NormalTok{)}
\end{Highlighting}
\end{Shaded}

Because the sales are changing over time, the additive adjustment risks
over adjusting in years where the sales are below average and under
adjusting in years where sales are above average, as such the
multiplicative adjustment is more appropriate

\begin{Shaded}
\begin{Highlighting}[]
\KeywordTok{plot}\NormalTok{(mdec, }\DataTypeTok{type =} \StringTok{"l"}\NormalTok{, }\DataTypeTok{ylab =} \StringTok{"Index"}\NormalTok{, }\DataTypeTok{xlab =} \StringTok{"Period"}\NormalTok{)}
\end{Highlighting}
\end{Shaded}

\includegraphics{Assignment2_files/figure-latex/unnamed-chunk-3-1.pdf}

\begin{Shaded}
\begin{Highlighting}[]
\CommentTok{#Create an adjusted time series}
\NormalTok{adtspassengers <-}\StringTok{ }\NormalTok{dtspassengers }\OperatorTok{/}\StringTok{ }\NormalTok{mdec}\OperatorTok{$}\NormalTok{seasonal}
\CommentTok{#Create a seasonally adjusted and non adjusted dataset without missing values}
\NormalTok{series1 <-}\StringTok{ }\KeywordTok{na.omit}\NormalTok{(adtspassengers)}
\NormalTok{series2 <-}\StringTok{ }\KeywordTok{na.omit}\NormalTok{(dtspassengers)}
\end{Highlighting}
\end{Shaded}

\begin{Shaded}
\begin{Highlighting}[]
\KeywordTok{plot}\NormalTok{(series2, }\DataTypeTok{type =} \StringTok{"l"}\NormalTok{, }\DataTypeTok{ylab =} \StringTok{"Number of Passengers"}\NormalTok{, }\DataTypeTok{xlab =} \StringTok{"Months"}\NormalTok{ )}
\KeywordTok{lines}\NormalTok{(series1, }\DataTypeTok{col =} \StringTok{'red'}\NormalTok{)}
\KeywordTok{legend}\NormalTok{(}\StringTok{'topleft'}\NormalTok{, }\DataTypeTok{legend =} \KeywordTok{c}\NormalTok{(}\StringTok{'Original'}\NormalTok{, }\StringTok{'With adjustment'}\NormalTok{), }
       \DataTypeTok{col =} \KeywordTok{c}\NormalTok{(}\StringTok{'black'}\NormalTok{, }\StringTok{'red'}\NormalTok{), }\DataTypeTok{lty =} \DecValTok{1}\NormalTok{ )}
\end{Highlighting}
\end{Shaded}

\includegraphics{Assignment2_files/figure-latex/unnamed-chunk-5-1.pdf}

\begin{Shaded}
\begin{Highlighting}[]
\CommentTok{#forecast}
\NormalTok{fit1  <-}\StringTok{ }\KeywordTok{ses}\NormalTok{(series1, }\DataTypeTok{alpha =} \FloatTok{0.2}\NormalTok{, }\DataTypeTok{h=}\DecValTok{6}\NormalTok{)}
\NormalTok{fit2  <-}\StringTok{ }\KeywordTok{ses}\NormalTok{(series1, }\DataTypeTok{alpha =} \FloatTok{0.8}\NormalTok{, }\DataTypeTok{h=}\DecValTok{6}\NormalTok{)}
\end{Highlighting}
\end{Shaded}

Fit 2 looks better so we will use that

\begin{Shaded}
\begin{Highlighting}[]
\KeywordTok{plot}\NormalTok{(series1, }\DataTypeTok{type =} \StringTok{'l'}\NormalTok{)}
\KeywordTok{lines}\NormalTok{(fit1}\OperatorTok{$}\NormalTok{fitted, }\DataTypeTok{col=} \StringTok{'red'}\NormalTok{)}
\KeywordTok{lines}\NormalTok{(fit1}\OperatorTok{$}\NormalTok{mean, }\DataTypeTok{col=} \StringTok{'green'}\NormalTok{)}
\KeywordTok{lines}\NormalTok{(fit2}\OperatorTok{$}\NormalTok{fitted, }\DataTypeTok{col=} \StringTok{'blue'}\NormalTok{)}
\KeywordTok{lines}\NormalTok{(fit2}\OperatorTok{$}\NormalTok{mean, }\DataTypeTok{col=} \StringTok{'yellow'}\NormalTok{)}
\KeywordTok{legend}\NormalTok{(}\StringTok{'topleft'}\NormalTok{, }\DataTypeTok{legend =} \KeywordTok{c}\NormalTok{(}\StringTok{'a = 0.2'}\NormalTok{,}\StringTok{'a = 0.8'}\NormalTok{),}
       \DataTypeTok{col =} \KeywordTok{c}\NormalTok{(}\StringTok{'red'}\NormalTok{, }\StringTok{'blue'}\NormalTok{), }\DataTypeTok{lty =} \DecValTok{1}\NormalTok{, }\DataTypeTok{cex =} \FloatTok{0.8}\NormalTok{)}
\end{Highlighting}
\end{Shaded}

\includegraphics{Assignment2_files/figure-latex/unnamed-chunk-7-1.pdf}

\begin{Shaded}
\begin{Highlighting}[]
\NormalTok{model1 <-}\StringTok{ }\KeywordTok{holt}\NormalTok{(series1, }\DataTypeTok{h=}\DecValTok{3}\NormalTok{)}
\NormalTok{model2 <-}\StringTok{ }\KeywordTok{holt}\NormalTok{(series1, }\DataTypeTok{damped =} \OtherTok{TRUE}\NormalTok{,  }\DataTypeTok{h=}\DecValTok{3}\NormalTok{)}
\end{Highlighting}
\end{Shaded}

\begin{Shaded}
\begin{Highlighting}[]
\KeywordTok{plot}\NormalTok{(series1, }\DataTypeTok{type =} \StringTok{'l'}\NormalTok{)}
\KeywordTok{lines}\NormalTok{(model1}\OperatorTok{$}\NormalTok{fitted , }\DataTypeTok{col=} \StringTok{'red'}\NormalTok{)}
\KeywordTok{lines}\NormalTok{(model2 }\OperatorTok{$}\NormalTok{fitted , }\DataTypeTok{col=} \StringTok{'green'}\NormalTok{)}
\KeywordTok{legend}\NormalTok{(}\StringTok{'topleft'}\NormalTok{, }\DataTypeTok{legend =} \KeywordTok{c}\NormalTok{(}\StringTok{'holt'}\NormalTok{,}\StringTok{'dampened'}\NormalTok{),}
       \DataTypeTok{col =} \KeywordTok{c}\NormalTok{(}\StringTok{'red'}\NormalTok{, }\StringTok{'blue'}\NormalTok{), }\DataTypeTok{lty =} \DecValTok{1}\NormalTok{, }\DataTypeTok{cex =} \FloatTok{0.8}\NormalTok{)}
\end{Highlighting}
\end{Shaded}

\includegraphics{Assignment2_files/figure-latex/unnamed-chunk-9-1.pdf}

To compare we split our series1 into train and test

\begin{Shaded}
\begin{Highlighting}[]
\NormalTok{train <-}\StringTok{ }\NormalTok{series1}
\NormalTok{test <-}\StringTok{ }\KeywordTok{na.omit}\NormalTok{(tspassengers_test)}

\NormalTok{trainsimple <-}\StringTok{ }\KeywordTok{ses}\NormalTok{(train, }\DataTypeTok{alpha =} \FloatTok{0.8}\NormalTok{, }\DataTypeTok{h=}\DecValTok{12}\NormalTok{)}
\NormalTok{trainholt <-}\StringTok{ }\KeywordTok{holt}\NormalTok{(train, }\DataTypeTok{h=}\DecValTok{12}\NormalTok{)}
\NormalTok{traindamped <-}\StringTok{ }\KeywordTok{holt}\NormalTok{(train, }\DataTypeTok{damped =} \OtherTok{TRUE}\NormalTok{,  }\DataTypeTok{h=}\DecValTok{12}\NormalTok{)}

\CommentTok{#Compute the in sample accuracy}
\NormalTok{mse_in_simple  <-}\StringTok{ }\KeywordTok{mean}\NormalTok{( (train }\OperatorTok{-}\StringTok{ }\NormalTok{trainsimple}\OperatorTok{$}\NormalTok{fitted )}\OperatorTok{^}\DecValTok{2}\NormalTok{ )}
\NormalTok{mse_in_holt  <-}\StringTok{ }\KeywordTok{mean}\NormalTok{( (train }\OperatorTok{-}\StringTok{ }\NormalTok{trainholt}\OperatorTok{$}\NormalTok{fitted )}\OperatorTok{^}\DecValTok{2}\NormalTok{ )}
\NormalTok{mse_in_damped  <-}\StringTok{ }\KeywordTok{mean}\NormalTok{( (train }\OperatorTok{-}\StringTok{ }\NormalTok{traindamped}\OperatorTok{$}\NormalTok{fitted )}\OperatorTok{^}\DecValTok{2}\NormalTok{ )}

\CommentTok{#Compute the out sample accuracy}
\NormalTok{mse_out_simple  <-}\StringTok{ }\KeywordTok{mean}\NormalTok{( (test }\OperatorTok{-}\StringTok{ }\NormalTok{trainsimple}\OperatorTok{$}\NormalTok{fitted )}\OperatorTok{^}\DecValTok{2}\NormalTok{ )}
\NormalTok{mse_out_holt  <-}\StringTok{ }\KeywordTok{mean}\NormalTok{( (test }\OperatorTok{-}\StringTok{ }\NormalTok{trainholt}\OperatorTok{$}\NormalTok{fitted )}\OperatorTok{^}\DecValTok{2}\NormalTok{ )}
\NormalTok{mse_out_damped  <-}\StringTok{ }\KeywordTok{mean}\NormalTok{( (test }\OperatorTok{-}\StringTok{ }\NormalTok{traindamped}\OperatorTok{$}\NormalTok{fitted )}\OperatorTok{^}\DecValTok{2}\NormalTok{ )}

\KeywordTok{c}\NormalTok{(mse_in_simple, mse_out_simple)}
\end{Highlighting}
\end{Shaded}

\begin{verbatim}
## [1]     86.01671 128188.90171
\end{verbatim}

\begin{Shaded}
\begin{Highlighting}[]
\KeywordTok{c}\NormalTok{(mse_in_holt,mse_out_holt )}
\end{Highlighting}
\end{Shaded}

\begin{verbatim}
## [1]     76.50818 126044.72363
\end{verbatim}

\begin{Shaded}
\begin{Highlighting}[]
\KeywordTok{c}\NormalTok{(mse_in_damped, mse_out_damped)}
\end{Highlighting}
\end{Shaded}

\begin{verbatim}
## [1]     80.18093 127851.95272
\end{verbatim}

\begin{Shaded}
\begin{Highlighting}[]
\CommentTok{#So the lowest MSE error is the Holt}

\NormalTok{model1 <-}\StringTok{ }\KeywordTok{ses}\NormalTok{(series1, }\DataTypeTok{alpha =} \FloatTok{0.2}\NormalTok{, }\DataTypeTok{h=}\DecValTok{12}\NormalTok{)}
\NormalTok{model2 <-}\StringTok{ }\KeywordTok{ses}\NormalTok{(series1, }\DataTypeTok{alpha =} \FloatTok{0.8}\NormalTok{, }\DataTypeTok{h=}\DecValTok{12}\NormalTok{)}
\NormalTok{model3 <-}\StringTok{ }\KeywordTok{holt}\NormalTok{(series1, }\DataTypeTok{h=}\DecValTok{12}\NormalTok{)}
\NormalTok{model4 <-}\StringTok{ }\KeywordTok{holt}\NormalTok{(series1, }\DataTypeTok{damped =} \OtherTok{TRUE}\NormalTok{, }\DataTypeTok{h=}\DecValTok{12}\NormalTok{)}
\end{Highlighting}
\end{Shaded}

\begin{Shaded}
\begin{Highlighting}[]
\KeywordTok{par}\NormalTok{(}\DataTypeTok{mfrow =} \KeywordTok{c}\NormalTok{(}\DecValTok{2}\NormalTok{,}\DecValTok{2}\NormalTok{))}
\KeywordTok{plot}\NormalTok{(model1);}\KeywordTok{plot}\NormalTok{(model2)}
\KeywordTok{plot}\NormalTok{(model3);}\KeywordTok{plot}\NormalTok{(model4)}
\end{Highlighting}
\end{Shaded}

\includegraphics{Assignment2_files/figure-latex/unnamed-chunk-11-1.pdf}

Exercise B

\begin{Shaded}
\begin{Highlighting}[]
\CommentTok{#Make dataset without the first column the data}
\NormalTok{trainset <-}\StringTok{ }\NormalTok{data2[}\OperatorTok{-}\DecValTok{1}\NormalTok{]}
\NormalTok{testset <-}\StringTok{ }\NormalTok{data3[}\OperatorTok{-}\DecValTok{1}\NormalTok{]}

\CommentTok{#Fit a regression}
\NormalTok{mlr <-}\StringTok{ }\KeywordTok{lm}\NormalTok{(power }\OperatorTok{~}\StringTok{ }\NormalTok{speed, }\DataTypeTok{data=}\NormalTok{trainset)}
\NormalTok{frc_mlr <-}\StringTok{ }\KeywordTok{as.numeric}\NormalTok{(}\KeywordTok{predict}\NormalTok{(mlr, testset))}
\end{Highlighting}
\end{Shaded}

\begin{Shaded}
\begin{Highlighting}[]
\KeywordTok{plot}\NormalTok{(data2}\OperatorTok{$}\NormalTok{power , }\DataTypeTok{type =} \StringTok{"l"}\NormalTok{, }\DataTypeTok{ylab =} \StringTok{"power"}\NormalTok{, }\DataTypeTok{xlab =} \StringTok{"Hours"}\NormalTok{ )}
\end{Highlighting}
\end{Shaded}

\includegraphics{Assignment2_files/figure-latex/unnamed-chunk-13-1.pdf}

\begin{Shaded}
\begin{Highlighting}[]
\KeywordTok{plot}\NormalTok{(}\DataTypeTok{x =}\NormalTok{data2}\OperatorTok{$}\NormalTok{power, }\DataTypeTok{y =}\NormalTok{ data2}\OperatorTok{$}\NormalTok{speed, }\DataTypeTok{xlab =} \StringTok{'Power'}\NormalTok{, }\DataTypeTok{ylab =} \StringTok{'Speed'}\NormalTok{ )}
\end{Highlighting}
\end{Shaded}

\includegraphics{Assignment2_files/figure-latex/unnamed-chunk-13-2.pdf}

\begin{Shaded}
\begin{Highlighting}[]
\KeywordTok{cor}\NormalTok{(data2}\OperatorTok{$}\NormalTok{power,data2}\OperatorTok{$}\NormalTok{speed)}
\end{Highlighting}
\end{Shaded}

\begin{verbatim}
## [1] 0.7326155
\end{verbatim}

This is a pretty high level of correlation! Now on to the neural
network!

\begin{Shaded}
\begin{Highlighting}[]
\CommentTok{#Rescale the data}
\NormalTok{trainset_s <-}\StringTok{ }\NormalTok{trainset}
\NormalTok{trainset_s}\OperatorTok{$}\NormalTok{speed <-}\StringTok{ }\NormalTok{(trainset_s}\OperatorTok{$}\NormalTok{speed}\OperatorTok{-}\KeywordTok{min}\NormalTok{(trainset_s}\OperatorTok{$}\NormalTok{speed))}\OperatorTok{/}\NormalTok{(}\KeywordTok{max}\NormalTok{(trainset_s}\OperatorTok{$}\NormalTok{speed)}\OperatorTok{-}\KeywordTok{min}\NormalTok{(trainset_s}\OperatorTok{$}\NormalTok{speed))}
\NormalTok{trainset_s}\OperatorTok{$}\NormalTok{power <-}\StringTok{ }\NormalTok{(trainset_s}\OperatorTok{$}\NormalTok{power}\OperatorTok{-}\KeywordTok{min}\NormalTok{(trainset_s}\OperatorTok{$}\NormalTok{power))}\OperatorTok{/}\NormalTok{(}\KeywordTok{max}\NormalTok{(trainset_s}\OperatorTok{$}\NormalTok{power)}\OperatorTok{-}\KeywordTok{min}\NormalTok{(trainset_s}\OperatorTok{$}\NormalTok{power))}

\CommentTok{#Rescale the data}
\NormalTok{testset_s <-}\StringTok{ }\NormalTok{testset}
\NormalTok{testset_s}\OperatorTok{$}\NormalTok{speed <-}\StringTok{ }\NormalTok{(testset_s}\OperatorTok{$}\NormalTok{speed}\OperatorTok{-}\KeywordTok{min}\NormalTok{(testset_s}\OperatorTok{$}\NormalTok{speed))}\OperatorTok{/}\NormalTok{(}\KeywordTok{max}\NormalTok{(testset_s}\OperatorTok{$}\NormalTok{speed)}\OperatorTok{-}\KeywordTok{min}\NormalTok{(testset_s}\OperatorTok{$}\NormalTok{speed))}
\NormalTok{testset_s}\OperatorTok{$}\NormalTok{power <-}\StringTok{ }\NormalTok{(testset_s}\OperatorTok{$}\NormalTok{power}\OperatorTok{-}\KeywordTok{min}\NormalTok{(testset_s}\OperatorTok{$}\NormalTok{power))}\OperatorTok{/}\NormalTok{(}\KeywordTok{max}\NormalTok{(testset_s}\OperatorTok{$}\NormalTok{power)}\OperatorTok{-}\KeywordTok{min}\NormalTok{(testset_s}\OperatorTok{$}\NormalTok{power))}

\KeywordTok{library}\NormalTok{(neuralnet)}
\KeywordTok{set.seed}\NormalTok{(}\DecValTok{512}\NormalTok{)}
\NormalTok{mlp1 <-}\StringTok{ }\KeywordTok{neuralnet}\NormalTok{(power }\OperatorTok{~}\StringTok{ }\NormalTok{speed, }\DataTypeTok{data=}\NormalTok{trainset_s,}
                  \DataTypeTok{hidden =} \DecValTok{1}\NormalTok{, }\DataTypeTok{act.fct =} \StringTok{"logistic"}\NormalTok{, }\DataTypeTok{linear.output =} \OtherTok{FALSE}\NormalTok{)}
\CommentTok{#This is our forecast but scaled}
\NormalTok{frc_mlp_s <-}\StringTok{ }\KeywordTok{as.numeric}\NormalTok{(}\KeywordTok{predict}\NormalTok{(mlp1, testset_s))}


\CommentTok{#Now we unscale it}
\NormalTok{frc_mlp <-}\StringTok{ }\NormalTok{frc_mlp_s}\OperatorTok{*}\NormalTok{(}\KeywordTok{max}\NormalTok{(trainset}\OperatorTok{$}\NormalTok{power)}\OperatorTok{-}\KeywordTok{min}\NormalTok{(trainset}\OperatorTok{$}\NormalTok{power))}\OperatorTok{+}\KeywordTok{min}\NormalTok{(trainset}\OperatorTok{$}\NormalTok{power)}
\KeywordTok{data.frame}\NormalTok{(frc_mlp,testset}\OperatorTok{$}\NormalTok{power)}
\end{Highlighting}
\end{Shaded}

\begin{verbatim}
##      frc_mlp testset.power
## 1  6.4465076           2.4
## 2  5.9771731           3.0
## 3  4.7681263           2.1
## 4  6.1319886           1.9
## 5  0.2635972           0.0
## 6  0.2669094           0.0
## 7  2.9209639           0.0
## 8  5.1799698           1.5
## 9  6.5992996           7.4
## 10 3.5208850           0.8
## 11 6.4223166           4.6
## 12 3.7174968           1.2
## 13 6.5992499           9.0
## 14 1.3439708           1.2
## 15 0.2964732           0.0
## 16 6.5297536           4.4
## 17 0.4097138           0.0
## 18 6.3110250           3.1
## 19 6.3943474           6.9
## 20 6.5994690           8.6
## 21 5.1169709           3.1
## 22 6.5994711           8.9
## 23 6.0335454           2.2
## 24 6.5661749           7.6
\end{verbatim}

\begin{Shaded}
\begin{Highlighting}[]
\CommentTok{#So this is our prediction  the data}
\end{Highlighting}
\end{Shaded}

Finally, we compare the actual data with our predictions.

\begin{Shaded}
\begin{Highlighting}[]
\CommentTok{#RMSE of regression and neural network}
\KeywordTok{sqrt}\NormalTok{(}\KeywordTok{mean}\NormalTok{((testset}\OperatorTok{$}\NormalTok{power}\OperatorTok{-}\NormalTok{frc_mlr)}\OperatorTok{^}\DecValTok{2}\NormalTok{))}
\end{Highlighting}
\end{Shaded}

\begin{verbatim}
## [1] 1.387682
\end{verbatim}

\begin{Shaded}
\begin{Highlighting}[]
\KeywordTok{sqrt}\NormalTok{(}\KeywordTok{mean}\NormalTok{((testset}\OperatorTok{$}\NormalTok{power}\OperatorTok{-}\NormalTok{frc_mlp)}\OperatorTok{^}\DecValTok{2}\NormalTok{))}
\end{Highlighting}
\end{Shaded}

\begin{verbatim}
## [1] 2.419144
\end{verbatim}

\end{document}

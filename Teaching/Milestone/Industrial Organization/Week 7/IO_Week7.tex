%\documentclass[10pt,aspectratio=43,t,l]{beamer}
\documentclass[10pt,aspectratio=169,t,l,fleqn,mathsanserif,sanserif]{beamer}
%\documentclass[10pt]{beamer}

%\setbeamertemplate{footline}[page number]{}

\usepackage{framed}
\usepackage{tcolorbox}
\colorlet{shadecolor}{blue!15}
\usepackage{color,amsmath,xmpmulti,textpos,comment,eurosym,bm,amsthm,tabularx,cancel}
\usepackage{epsfig}
\usepackage{nicefrac}
\usepackage{listings}
%\usepackage{enumitem}
\usepackage{graphicx}    
\usepackage{graphics}
\usepackage{epstopdf}
\usepackage[normalem]{ulem}
\usepackage{float}
%\usepackage[cmbold]{mathtime}
%\usepackage{mt11p}
\usepackage{placeins}
\usepackage{amsmath}
\usepackage{pifont}
\usepackage{color}
\usepackage{amssymb}
\usepackage{mathtools}
\usepackage{subfigure}
\usepackage{multirow}
\usepackage{epsfig}
\usepackage{listings}
%\usepackage{enumitem}
\usepackage{rotating,tabularx}
%\usepackage[graphicx]{realboxes}
\usepackage{graphicx}
\usepackage{graphics}
\usepackage{epstopdf}
\usepackage{longtable}
%\usepackage[pdftex]{hyperref}
\usepackage{breakurl}
\usepackage{epigraph}
\usepackage{xspace}
\usepackage{amsfonts}
\usepackage{eurosym}
\usepackage{ulem}
\usepackage{footmisc}
\usepackage{comment}
\usepackage{setspace}
\usepackage{geometry}
\usepackage{caption}
\usepackage{pdflscape}
\usepackage{array}
\usepackage[round]{natbib}
\usepackage{booktabs}
\usepackage{dcolumn}
\usepackage{mathrsfs}
\usepackage{tikz}
\usetikzlibrary{decorations.pathreplacing}
\usepackage{sansmathaccent}
\pdfmapfile{+sansmathaccent.map}
\usetikzlibrary{shapes.geometric, arrows,chains}
\tikzset{
  startstop/.style={
    rectangle, 
    rounded corners,
    minimum width=3cm, 
    minimum height=1cm,
    align=center, 
    draw=black, 
    fill=red!30
    },
  startsleft/.style={
    rectangle, 
    rounded corners,
    minimum width=3cm, 
    minimum height=1cm,
    align=left, 
    draw=black, 
    fill=red!30
    },
  startsright/.style={
    rectangle, 
    rounded corners,
    minimum width=3cm, 
    minimum height=1cm,
    align=right, 
    draw=black, 
    fill=red!30
    },
  process/.style={
    rectangle, 
    minimum width=3cm, 
    minimum height=1cm, 
    align=center, 
    draw=black, 
    fill=blue!30
    },
  decision/.style={
    rectangle, 
    minimum width=3cm, 
    minimum height=1cm, align=center, 
    draw=black, 
    fill=green!30
    },
  arrow/.style={thick,->,>=stealth},
  dec/.style={
    ellipse, 
    align=center, 
    draw=black, 
    fill=green!30
    },
  font={\fontsize{9pt}{12}\selectfont}
}
%\renewcommand{\labelitemi}{$\blacktriangleright$}

\epstopdfsetup{outdir=./}

\newcolumntype{Y}{>{\centering\arraybackslash}X}
\def\Put(#1,#2)#3{\leavevmode\makebox(0,0){\put(#1,#2){#3}}}

\newcommand{\subhead}[1]{\mbox{}\newline\textbf{#1}\newline}
\newcommand{\ave}[1]{\left\langle #1 \right \rangle}
\newcommand{\eg}{{\it e.g.}}
\newcommand{\ie}{{\it i.e.}}
\newcommand{\cf}{{\it c.f.}}
\newcommand{\etc}{{\it etc.}}
\newcommand{\etal}{{\it et al.}}
%\newcommand{\btVFill}{\vskip0pt plus 1filll}

\newcommand{\del}{D}
\newcommand{\hor}{H}

\newcommand{\threepartdef}[6]
{
  \left\{
    \begin{array}{lll}
      #1 & \mbox{if } #2 \\
      #3 & \mbox{if } #4 \\
      #5 & \mbox{if } #6
    \end{array}
  \right.
}


\newcommand{\Ito}{It\^{o}}
\newcommand{\SP}{S{\&}P500}
\newcommand{\lopt}{\ell_{\text{opt}}}
\newcommand{\gest}{g_{\text{N,T}}}
\newcommand{\elabel}[1]{\label{eq:#1}}
\newcommand{\eref}[1]{Eq.~(\ref{eq:#1})}
\newcommand{\Eref}[1]{Equation~(\ref{eq:#1})}

\newcommand{\flabel}[1]{\label{fig:#1}}
\newcommand{\fref}[1]{Fig.~\ref{fig:#1}}
\newcommand{\Fref}[1]{Figure~\ref{fig:#1}}
\newcommand{\person}[1]{{#1}}
\newcommand{\ra}[1]{\renewcommand{\arraystretch}{#1}}
\newcommand{\vs}[1]{\vspace{.#1cm}}
\newcommand{\vf}{\vspace{.25cm}}
\newcommand{\vff}{\vspace{.6cm}}
\newcommand{\np}{\\ \vf}
\newcommand{\npp}{\\ \vff}
\newcommand{\be}{\begin{equation*}}
\newcommand{\ee}{\end{equation*}}
\newcommand{\bea}{\begin{eqnarray*}}
\newcommand{\eea}{\end{eqnarray*}}
\newcommand{\bc}{\begin{center}}
\newcommand{\ec}{\end{center}}
\newcommand{\bie}{\begin{enumerate}}
\newcommand{\eie}{\end{enumerate}}
\newcommand{\bi}{\begin{itemize}}
\newcommand{\ei}{\end{itemize}}
\newcommand{\toinf}{\rightarrow\infty}
\newcommand{\D}{{\Delta}}
\newcommand{\Dx}{{\Delta x}}
\newcommand{\Dy}{{\Delta y}}
\newcommand{\Du}{{\Delta u}}
\newcommand{\DW}{{\Delta W}}
\newcommand{\DU}{{\Delta U}}
\newcommand{\du}{{\delta u}}
\newcommand{\Dv}{{\Delta v}}
\newcommand{\dt}{{\delta t}}
\newcommand{\gens}{g_{\ave{\,}}}
\newcommand{\ft}[1]{\frametitle{#1}}
\newcommand{\bq}{\begin{quote}}
\newcommand{\eq}{\end{quote}}
\newcommand{\ww}[1]{\bq{\small\rm#1\\}\eq}
\newcommand{\E}{\mathrm{E}}
\newcommand{\Var}{\mathrm{Var}}
\newcommand{\Cov}{\mathrm{Cov}}
\newcommand{\sgn}{\mathrm{sgn}}
\newcommand{\prob}[1]{\mathcal{P}\left(#1\right)}
\newcommand{\lra}{\longrightarrow}
\newcommand{\eps}{\varepsilon}
\newcommand{\ga}{g_\text{ave}}
\newcommand{\gt}{g_\text{typ}}
\newcommand{\gbar}{\bar{g}}
\newcommand{\mbar}{\bar{m}}
\newcommand{\red}[1]{\textcolor{red}{#1}}
\newcommand{\xf}{{x_F}}
\newcommand{\xb}{{x_B}}
\newcommand{\muf}{{\mu_F}}
\newcommand{\mub}{{\mu_B}}
\newcommand{\sigf}{{\sigma_F}}
\newcommand{\sigb}{{\sigma_B}}
\newcommand{\gf}{{\gbar_F}}
\newcommand{\gb}{{\gbar_B}}
\newcommand{\pa}{\textit{pa}}
\newcommand{\taus}{{\tau_\text{s}}}
\newcommand{\Dt}{\Delta t}
\newcommand{\etau}{\tau^\text{eqm}}
\newcommand{\taue}{\tau^\text{EGBM}}
\newcommand{\wtau}{\widetilde{\tau}}
\newcommand{\xN}{\ave{x}_N}
\newcommand{\Sdata}{S^{\text{data}}}
\newcommand{\Smodel}{S^{\text{model}}}
\beamertemplatenavigationsymbolsempty

\newcommand{\tlabel}[1]{\label{tab:#1}}
\newcommand{\tref}[1]{Tab.~\ref{tab:#1}}
\newcommand{\Tref}[1]{Table~\ref{tab:#1}}

\newenvironment{myindentpar}[1]%
{\begin{list}{}%
    {\setlength{\leftmargin}{#1}}%
  \item[]%
}
{\end{list}}

%\usetheme[width=1.8cm,hideothersubsections]{Frankfurt}
\usetheme{Frankfurt}

\newcommand\BackgroundPicture[1]{
\setbeamertemplate{background}{
\parbox[c][\paperheight]{\paperwidth}{
\vfill \hfill
\includegraphics[width=1\paperwidth,height=1\paperheight]{#1}
\hfill \vfill
}}}


\definecolor{lmlblue}{RGB}{0,77,123}
\definecolor{deepblue}{RGB}{35,33,169}
\definecolor{lmllb}{RGB}{237,244,255}
\definecolor{lmlred}{RGB}{155,29,29}
\definecolor{lmlgrey}{RGB}{142,142,142}
\definecolor{lmlgrey2}{RGB}{82,82,82}
\definecolor{grey}{RGB}{210,210,210}
\xdefinecolor{lightblue}{rgb}{0,200,255}
\setbeamercolor{important}{bg=lightblue,fg=red}
\AtBeginEnvironment{definition}{%
  \setbeamercolor{block body}{fg=black,bg=white}
  \setbeamercolor{block title}{bg=lmllb,fg=black}
}

\AtBeginEnvironment{theorem}{%
  \setbeamercolor{block body}{fg=black,bg=white}
  \setbeamercolor{block title}{bg=lmllb,fg=black}
}

%\newcommand{\propnumber}{} % initialize
%\newtheorem*{prop}{Proposition \propnumber}
%\newenvironment{propc}[1]
%  {\renewcommand{\propnumber}{#1}%
%   \begin{shaded}\begin{prop}}
%  {\end{prop}\end{shaded}}
%\AtBeginEnvironment{propc}{%
%  \setbeamercolor{block body}{fg=black,bg=white}
%  \setbeamercolor{block title}{bg=lmllb,fg=black}
%}

\setbeamercolor{fine separation line}{fg=lmllb}

\setbeamercolor{item projected}{fg=white, bg=black}

\setbeamercolor{frametitle}{bg=lmllb, fg=black}
\setbeamertemplate{frametitle}[default][left,colsep=-4bp,rounded=false,shadow=false]

\setbeamercolor{structure}{bg=white, fg=black}
%structure changes color of title in sidebar.

%\setbeamertemplate{frametitle}[default][colsep=-4bp,rounded=false,shadow=false]
\setbeamercolor{section in head/foot}{fg=lmlgrey2, bg=lmllb}
\setbeamercolor{normal text}{fg=black}
\setbeamercolor{title}{bg=white,fg=deepblue}

\hypersetup{colorlinks,linkcolor=,urlcolor=deepblue}

\setbeamerfont{title in sidebar}{size=\fontsize{9}{9}\selectfont}
\setbeamerfont{section in sidebar}{size=\fontsize{7}{7}\selectfont}

% add frame numbers to navigation bar
% (for RSS 2015 conference)
\addtobeamertemplate{navigation symbols}{}{
    \usebeamerfont{footline}
    \usebeamercolor[fg]{footline}
    \hspace{1em}
    \scriptsize
%    \insertframenumber/\inserttotalframenumber
}
\setbeamertemplate{navigation symbols}{} %gets rid of navigation symbols
\setbeamercovered{transparent=20}
%\setbeamertemplate{footline}[frame number] % to show overlay page numbers type: page number
%\setbeamersize{text margin left=0.65cm, text margin right=0.65cm}
%\setbeamercolor{item}{fg=black!70!black} % red bullets
%\setbeamertemplate{itemize subitem}[triangle] % triangle sub-bullets
%\setbeamerfont{itemize/enumerate subbody}{size=\normalsize}


\title[\begin{flushleft} \color{white} {\tiny ???} \end{flushleft}]{\bf {Industrial Organization, Week 7 \\ 
Price Discrimnation}}
\author[shortname]{ Dio Mavroyiannis \inst{\dag}}
\institute[\begin{flushleft} \color{white} {\large ???} \end{flushleft}]{Milestone Institute}
%\institute[shortinst]{\inst{*} London Mathematical Laboratory \and \inst{\dag} Universit\'{e} Paris-Dauphine \and \inst{\ddag} London Mathematical Laboratory and Santa Fe Institute}
\date{17 March 2021}

\AtBeginSection[]{
\frame{
\ft{Agenda}
\tableofcontents[currentsection,hideallsubsections]
}
}

\makeatletter
\setbeamertemplate{headline}{%
  \pgfuseshading{beamer@barshade}%
  \ifbeamer@sb@subsection%
    \vskip-9.75ex%
  \else%
    \vskip-7ex%
  \fi%
    \begin{beamercolorbox}[ht=3.5ex,dp=2.125ex]{section in head/foot}
     \insertsectionnavigationhorizontal{\textwidth}{}{}%
  \end{beamercolorbox}%
  \ifbeamer@sb@subsection%
    \begin{beamercolorbox}[ignorebg,ht=2.125ex,dp=1.125ex,%
      leftskip=.3cm,rightskip=.3cm plus1fil]{subsection in head/foot}
      \usebeamerfont{subsection in head/foot}\insertsubsectionhead
    \end{beamercolorbox}%
  \fi%
}%

\makeatother

\setbeamertemplate{mini frames}{}
\setbeamertemplate{itemize items}[triangle]

\begin{document}
\frame{\titlepage\insertlogo}

\section{Big picture}
%%%%%%%%%%%%%%%%%%%%%%%%%%%%%%%%%%%%%%%%%%%%%
\frame{\ft{Price discrimination plan}
\setbeamertemplate{itemize items}[triangle]

\bi
\item Plan: Price discrimination
\item First we will look at what each kind of discrimination is
\item How do you price discrminate?
\item We look at some examples of price discrmination
\ei

}


\section{Definition}
%%%%%%%%%%%%%%%%%%%%%%%%%%%%%%%%%%%%%%%%%%%%%
\frame{\ft{Definition}
\setbeamertemplate{itemize items}[triangle]

\bi
\item Price Discrmination: The pricing of the same or similar goods at different levels
\item Requires: No arbitrage or resale
\item No resale requires either: 1) Non transferability, 2) high transaction cost, 3) Resale Illegal
\ei

}


\section{What is price discrimination?}

\frame{\ft{First degree price discrimination}
\setbeamertemplate{itemize items}[triangle]

Feautures:

\bi
\item Every consumer charged their highest willingness to pay
\item No consumer surplus
\item Efficient but rare
\item Linguistically: perfect price discrmination or perfect appropriation
\ei

}



\frame{\ft{Third degree price discrimination?}
\setbeamertemplate{itemize items}[triangle]

Feautures:

\bi
\item Selection by indicator(age, sex, etc)
\item Different price for each type
\item Movie tickets for young or old
\ei

}


\frame{\ft{Second degree price discrimination?}
\setbeamertemplate{itemize items}[triangle]

Feautures:

\bi
\item Self-selection by consumers
\item Consumer type unknown to producer
\item Example: Mobile telephone, subscription services
\ei

}

\frame{\ft{Types of price discrimination?}
\setbeamertemplate{itemize items}[triangle]

\bi
\item If goods are homogenous: menu pricing over quantity
\item Vertically differentated goods 
\item degree of price discrimination does not neccesarily entail the surplus taken
\ei

}

\section{First degree price discrmination}

\frame{\ft{Look into the demand}
\setbeamertemplate{itemize items}[triangle]
A downward sloping demand curve function can be generated by three different processes

\bie
\item A single individual with continous demand
\item Many identical individuals with continous
\item Heterogenous individuals
\eie

}

\frame{\ft{Consumer preferences}
\setbeamertemplate{itemize items}[triangle]
A single consumer with continous demand interpretation:

\begin{equation}
U_i = v_iq_i-\frac{q_i^2}{2}-t(q_i)
\end{equation}

In first degree price discrmination, this equation will always equal 0 in equilibrium

}


\frame{\ft{The two part tarrif}
\setbeamertemplate{itemize items}[triangle]

\bi
\item A two part tarrif can always lead to the monopolist extracting all surplus
\item Requires a two part tarrif for every type of consumer, then all producer surplus.
\item $t(q)=T+pq$
\item The oprimal price is the welfare maximizing quantity
\item The optimal subscription fee is the individual rationality equation
\ei

}

\frame{\ft{Two part tarrif}
\setbeamertemplate{itemize items}[triangle]

\bi
\item The monopolist chooses the welfare maximizing quantity or price
\item The monopolist charges each type of consumer their exact surplus as a fixed fee
\item This reasoning works when employed in other industries, lump sum taxes are efficient.
\ei

}


\section{Third degree price discrmination}

\frame{\ft{Look into the demand}
\setbeamertemplate{itemize items}[triangle]

\bi
\item The producer can distinguish between different kinds of demands. 
\item Example: Children, senior citizens, etc. 
\item Compared to monopoly: monopolist better off, some consumers worse off(not always)
\ei
}


\frame{\ft{Example:}
\setbeamertemplate{itemize items}[triangle]
A mini example to gain some intuition

\bi
\item Suppose consumers have a WTP between $0-1$
\item Costs of production are $0$. 
\item Profit whilst blind: $p(1-p)$
\item Profit whilst distinguishing between upper and lower half: $p_1(1-p_1)$ for upper half and $p_2(\frac{1}{2}-p_2)$ for lower half 
\ei

}


\section{Second degree price discrmination}

\frame{\ft{Second degree price discrimination}
\setbeamertemplate{itemize items}[triangle]
More uncertainty 

\bi
\item Consumers have heterogenous demands
\item Monopolist cannot differentiate between them
\item The return of incentive compatibility
\item Perfect discrimination impossible
\item '3 for 2', 60p for 1m 1.20 for 3. 
\ei

}

\frame{\ft{Example of second degree price discrimination}
\setbeamertemplate{itemize items}[triangle]

Suppose we have two types of consumers in equal proportions

\bi
\item Utility function: $U_i = \frac{(A_i-p_i)^2}{2}-T_i$
\item Suppose that $q_l = 2-p;q_h = 3-p$
\item Firm sets two packages, $(p_1, T_1),(p_2, T_2) $
\item Individual Rationality L: $\frac{(2-p_1)^2}{2}-T_1>0$
\item Individual Rationality H: $\frac{(3-p_2)^2}{2}-T_2>0$
\item Incentive compatibility of low(ICL): $\frac{(2-p_1)^2}{2}-T_1>\frac{(2-p_2)^2}{2}-T_2$
\item Incentive compatibility of high(ICH): $\frac{(3-p_2)^2}{2}-T_2>\frac{(3-p_1)^2}{2}-T_1$
\ei

}

\frame{\ft{The profit function}
\setbeamertemplate{itemize items}[triangle]

It turns out that we only need IRL and ICH

\begin{align}
T_1 &= \frac{(2-p_1)^2}{2} \\
T_2 &= \frac{(3-p_2)^2}{2}+\frac{(2-p_1)^2}{2}-\frac{(3-p_1)^2}{2} \\
\pi &= \frac{1}{2} 
( 
(2-p_1)p_1 + (3-p_2)p_2 + T_1+ T_2 ) \\
\rightarrow & (p_1,T_1) = (1,0.5) \\
\rightarrow & (p_2,T_2) = (0,3)
\end{align}

}

\frame{\ft{Result}
\setbeamertemplate{itemize items}[triangle]

\bi
\item Option 1, low fixed cost, high per unit price
\item Option 2, High fixed cost, low unit price
\ei

}



\frame{\ft{Conclusion}
\setbeamertemplate{itemize items}[triangle]

\bi
\item Price discrmination increases welfare
\item From first, to third, to second, the firm loses capacity to discriminate. 
\ei
}

\end{document}
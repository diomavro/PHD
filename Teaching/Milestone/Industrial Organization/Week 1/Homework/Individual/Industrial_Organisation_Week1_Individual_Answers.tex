%\documentclass[AER]{AEA}
\documentclass[12pt]{report}
%\documentclass[12pt]{article}
%\documentclass[12pt,a4paper]{article}

\usepackage[utf8]{inputenc}


\usepackage{mathtools}
\usepackage{amsmath}
\usepackage{amssymb}
\usepackage{amsthm}

\usepackage{float}
%\usepackage[cmbold]{mathtime}
%\usepackage{mt11p}
\usepackage{placeins}
\usepackage{caption}
\usepackage{color}
\usepackage{subfigure}
\usepackage{multirow}
\usepackage{epsfig}
\usepackage{listings}
\usepackage{enumitem}
\usepackage{rotating,tabularx}
%\usepackage[graphicx]{realboxes}
\usepackage{graphicx}
\usepackage{graphics}
\usepackage{epstopdf}
\usepackage{longtable}
\usepackage[pdftex]{hyperref}
%\usepackage{breakurl}
\usepackage{epigraph}
\usepackage{xspace}
\usepackage{amsfonts}
\usepackage{eurosym}
\usepackage{ulem}

\usepackage{tikz}
\usetikzlibrary{spy}

\usepackage{verbatim}



\usepackage{footmisc}
\usepackage{comment}
\usepackage{setspace}
\usepackage{geometry}
\usepackage{caption}
\usepackage{pdflscape}
\usepackage{array}
\usepackage[authoryear]{natbib}
\usepackage{booktabs}
\usepackage{dcolumn}
\usepackage{mathrsfs}
%\usepackage[justification=centering]{caption}
%\captionsetup[table]{format=plain,labelformat=simple,labelsep=period,singlelinecheck=true}%
\bibliographystyle{apalike}
%\bibliographystyle{unsrtnat}



%\bibliographystyle{aea}
\usepackage{enumitem}
\usepackage{tikz}
\usetikzlibrary{positioning}
\usetikzlibrary{arrows}
\usetikzlibrary{shapes.multipart}

\usetikzlibrary{shapes}
\def\checkmark{\tikz\fill[scale=0.4](0,.35) -- (.25,0) -- (1,.7) -- (.25,.15) -- cycle;}
%\usepackage{tikz}
%\usetikzlibrary{snakes}
%\usetikzlibrary{patterns}

%\draftSpacing{1.5}

\usepackage{xcolor}
\hypersetup{
colorlinks,
linkcolor={blue!50!black},
citecolor={blue!50!black},
urlcolor={blue!50!black}}

%\renewcommand{\familydefault}{\sfdefault}
%\usepackage{helvet}
%\setlength{\parindent}{0.4cm}
%\setlength{\parindent}{2em}
%\setlength{\parskip}{1em}

%\normalem

%\doublespacing
\onehalfspacing
%\singlespacing
%\linespread{1.5}

\newtheorem{theorem}{Theorem}
\newtheorem{corollary}[theorem]{Corollary}
\newtheorem{proposition}{Proposition}
\newtheorem{definition}{Definition}
\newtheorem{axiom}{Axiom}
\newtheorem{observation}{Observation}
\newtheorem{assumption}{Assumption}	
\newtheorem{remark}{Remark}
\newtheorem{lemma}{Lemma}
\newtheorem{result}{result}


\newcommand{\ra}[1]{\renewcommand{\arraystretch}{#1}}

\newcommand{\E}{\mathrm{E}}
\newcommand{\Var}{\mathrm{Var}}
\newcommand{\Corr}{\mathrm{Corr}}
\newcommand{\Cov}{\mathrm{Cov}}

\newcolumntype{d}[1]{D{.}{.}{#1}} % "decimal" column type
\renewcommand{\ast}{{}^{\textstyle *}} % for raised "asterisks"

\newtheorem{hyp}{Hypothesis}
\newtheorem{subhyp}{Hypothesis}[hyp]
\renewcommand{\thesubhyp}{\thehyp\alph{subhyp}}

\newcommand{\red}[1]{{\color{red} #1}}
\newcommand{\blue}[1]{{\color{blue} #1}}

%\newcommand*{\qed}{\hfill\ensuremath{\blacksquare}}%

\newcolumntype{L}[1]{>{\raggedright\let\newline\\arraybackslash\hspace{0pt}}m{#1}}
\newcolumntype{C}[1]{>{\centering\let\newline\\arraybackslash\hspace{0pt}}m{#1}}
\newcolumntype{R}[1]{>{\raggedleft\let\newline\\arraybackslash\hspace{0pt}}m{#1}}

%\geometry{left=1.5in,right=1.5in,top=1.5in,bottom=1.5in}
\geometry{left=1in,right=1in,top=1in,bottom=1in}

\epstopdfsetup{outdir=./}

\newcommand{\elabel}[1]{\label{eq:#1}}
\newcommand{\eref}[1]{Eq.~(\ref{eq:#1})}
\newcommand{\ceref}[2]{(\ref{eq:#1}#2)}
\newcommand{\Eref}[1]{Equation~(\ref{eq:#1})}
\newcommand{\erefs}[2]{Eqs.~(\ref{eq:#1}--\ref{eq:#2})}

\newcommand{\Sref}[1]{Section~\ref{sec:#1}}
\newcommand{\sref}[1]{Sec.~\ref{sec:#1}}

\newcommand{\Pref}[1]{Proposition~\ref{prop:#1}}
\newcommand{\pref}[1]{Prop.~\ref{prop:#1}}
\newcommand{\preflong}[1]{proposition~\ref{prop:#1}}

\newcommand{\Aref}[1]{Axiom~\ref{ax:#1}}

\newcommand{\clabel}[1]{\label{coro:#1}}
\newcommand{\Cref}[1]{Corollary~\ref{coro:#1}}
\newcommand{\cref}[1]{Cor.~\ref{coro:#1}}
\newcommand{\creflong}[1]{corollary~\ref{coro:#1}}

\newcommand{\etal}{{\it et~al.}\xspace}
\newcommand{\ie}{{\it i.e.}\ }
\newcommand{\eg}{{\it e.g.}\ }
\newcommand{\etc}{{\it etc.}\ }
\newcommand{\cf}{{\it c.f.}\ }
\newcommand{\ave}[1]{\left\langle#1 \right\rangle}
\newcommand{\person}[1]{{\it \sc #1}}

\newcommand{\AAA}[1]{\red{{\it AA: #1 AA}}}
\newcommand{\YB}[1]{\blue{{\it YB: #1 YB}}}

\newcommand{\flabel}[1]{\label{fig:#1}}
\newcommand{\fref}[1]{Fig.~\ref{fig:#1}}
\newcommand{\Fref}[1]{Figure~\ref{fig:#1}}

\newcommand{\tlabel}[1]{\label{tab:#1}}
\newcommand{\tref}[1]{Tab.~\ref{tab:#1}}
\newcommand{\Tref}[1]{Table~\ref{tab:#1}}

\newcommand{\be}{\begin{equation}}
\newcommand{\ee}{\end{equation}}
\newcommand{\bea}{\begin{eqnarray}}
\newcommand{\eea}{\end{eqnarray}}

\newcommand{\bi}{\begin{itemize}}
\newcommand{\ei}{\end{itemize}}

\newcommand{\Dt}{\Delta t}
\newcommand{\Dx}{\Delta x}
\newcommand{\Epsilon}{\mathcal{E}}
\newcommand{\etau}{\tau^\text{eqm}}
\newcommand{\wtau}{\widetilde{\tau}}
\newcommand{\xN}{\ave{x}_N}
\newcommand{\Sdata}{S^{\text{data}}}
\newcommand{\Smodel}{S^{\text{model}}}

\newcommand{\del}{D}
\newcommand{\hor}{H}



\setlength{\parindent}{0.0cm}
\setlength{\parskip}{0.4em}

\numberwithin{equation}{section}
\DeclareMathOperator\erf{erf}
%\let\endtitlepage\relax



% https://medium.com/@aerinykim/why-the-normal-gaussian-pdf-looks-the-way-it-does-1cbcef8faf0a

\begin{document}
\section{Industrial Organization, Week 1 Answers}

\begin{align*}
Q_s = 400p-100&; ~~Q_d = 1100-200(p+t) \\
&\rightarrow 
p_s = \frac{Q_o}{400} + \frac{1}{4} ; ~~p_d = \frac{1100-Q_d}{200}-t  \\
Q_s = Q_d 
&\leftrightarrow p^* = 2 - \frac{1}{3}t; ~~ Q^*=700 - \frac{400}{3}t \\ 
\end{align*}

Surplus consumer(geometry way):
\begin{align*}
S_c = Base*Height/2 =& (5.5-p^*)*Q^*/2 =\frac{1}{2}(5.5-t-2 + \frac{1}{3}t)*(700 - \frac{400}{3}t) \\
=& \frac{1}{2}(3.5 - \frac{2}{3}t)*(700 - \frac{400}{3}t) \\
=& \frac{1}{2}(3.5 \frac{200}{200} - \frac{2*200}{3*200}t)*(700 - \frac{400}{3}t) \\
=& \frac{1}{400}(700 - \frac{200}{3}t)*(700 - \frac{400}{3}t) \\
=& \frac{1}{2}(700 - \frac{400}{3}t)^2 \\
\end{align*}




Surplus producer(geometry):
\begin{align*}
S_c = Base*Height/2 =& (p^*-1/4)*Q^*/2 = \frac{1}{2}( 2 - \frac{1}{3}t -1/4)*(700 - \frac{400}{3}t) \\
=& \frac{1}{2}( \frac{7}{4} - \frac{1}{3}t )*(700 - \frac{400}{3}t) \\
=& \frac{1}{2}( \frac{700}{400} - \frac{400}{3*400}t )*(700 - \frac{400}{3}t)  \\
=& \frac{1}{800}( 700 - \frac{400}{3}t )^2
\end{align*}

In the special cases:
\begin{align*}
t = 0 &\rightarrow p^* = 2; ~~ Q^*= 700;  ~~S_c = 1225; ~~ S_p = 612.5\\
t = 1 &\rightarrow p^* = \frac{5}{3}; ~~ Q^*= \frac{1700}{3}= 566.66...;  ~~ S_c = 802.8;  ~~  S_p = 401.4 \\
DWL &= 1225+612.5-802.8-401.4-566.6=66.7
\end{align*}

In case some of you want to dig deeper, when the functions are not linear you are sometimes forced to into calculus, so this is the more general approach: 
Surplus consumer(calculus way):
\begin{align*}
S_c = &\int^{Q^*}_{0} (p_d - p^*)dq \\
= &\int^{Q^*}_{0} \left(\left(\frac{1100-q}{200}-t \right)-\left(2 - \frac{1}{3}t \right) \right) dq \\
= &\int^{Q^*}_{0} \left(\frac{1100-q}{200}-\frac{2}{3}t -2 \right)  dq \\
= &\int^{Q^*}_{0} \left(\frac{700-q}{200}-\frac{2}{3}t \right)  dq \\
= &\left| \left(\frac{700q}{200}-\frac{q^2}{400}-q\frac{2}{3}t \right) \right|^{Q^*}_0  \\
= &\left| q \left(\frac{700}{200}-\frac{q}{400}-\frac{2}{3}t \right) \right|^{Q^*}_0  \\
= & \left(700 - \frac{400}{3}t \right) \left(\frac{700}{200}-\frac{700 - \frac{400}{3}t }{400}-\frac{2}{3}t \right)  \\
= & \left(700 - \frac{400}{3}t \right) \left(\frac{700}{400}-\frac{ t }{3} \right)  \\
= &\frac{1}{400} \left(700 - \frac{400}{3}t \right)^2   \\
% & \text{si } t = 0 \rightarrow S_c = 1225
\end{align*}

Surplus producer(calculus way):
\begin{align*}
S_p = &\int^{Q^*}_{0} (p^* - p_o)dq \\
= &\int^{Q^*}_{0} \left(\left(2 - \frac{1}{3}t\right) - \left(\frac{q}{400} + \frac{1}{4} \right) \right) dq \\
= &\int^{Q^*}_{0} \left( \frac{7}{4} - \frac{1}{3}t - \frac{q}{400}  \right) dq \\
= &\left|  \left( \frac{7q}{4} - \frac{q}{3}t - \frac{q^2}{800}  \right) \right|^{Q^*}_0\\
= &\left| q\left( \frac{7}{4} - \frac{1}{3}t - \frac{q}{800}  \right) \right|^{Q^*}_0\\
= & \left( 700 - \frac{400}{3}t \right) \left( \frac{7}{4} - \frac{1}{3}t - \frac{700 - \frac{400}{3}t}{800}  \right) \\
= & \left( 700 - \frac{400}{3}t \right) \left( \frac{700}{800} - \frac{1}{6}t  \right) \\
= & \frac{1}{800} \left( 700 - \frac{400}{3}t \right)^2 \\
% & \text{si } t = 0 \rightarrow S_p = 612.5
\end{align*}

\bibliography{../thesisbib/bibliography}

\end{document}

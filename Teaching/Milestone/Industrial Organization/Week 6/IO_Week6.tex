%\documentclass[10pt,aspectratio=43,t,l]{beamer}
\documentclass[10pt,aspectratio=169,t,l,fleqn,mathsanserif,sanserif]{beamer}
%\documentclass[10pt]{beamer}

%\setbeamertemplate{footline}[page number]{}

\usepackage{framed}
\usepackage{tcolorbox}
\colorlet{shadecolor}{blue!15}
\usepackage{color,amsmath,xmpmulti,textpos,comment,eurosym,bm,amsthm,tabularx,cancel}
\usepackage{epsfig}
\usepackage{nicefrac}
\usepackage{listings}
%\usepackage{enumitem}
\usepackage{graphicx}    
\usepackage{graphics}
\usepackage{epstopdf}
\usepackage[normalem]{ulem}
\usepackage{float}
%\usepackage[cmbold]{mathtime}
%\usepackage{mt11p}
\usepackage{placeins}
\usepackage{amsmath}
\usepackage{pifont}
\usepackage{color}
\usepackage{amssymb}
\usepackage{mathtools}
\usepackage{subfigure}
\usepackage{multirow}
\usepackage{epsfig}
\usepackage{listings}
%\usepackage{enumitem}
\usepackage{rotating,tabularx}
%\usepackage[graphicx]{realboxes}
\usepackage{graphicx}
\usepackage{graphics}
\usepackage{epstopdf}
\usepackage{longtable}
%\usepackage[pdftex]{hyperref}
\usepackage{breakurl}
\usepackage{epigraph}
\usepackage{xspace}
\usepackage{amsfonts}
\usepackage{eurosym}
\usepackage{ulem}
\usepackage{footmisc}
\usepackage{comment}
\usepackage{setspace}
\usepackage{geometry}
\usepackage{caption}
\usepackage{pdflscape}
\usepackage{array}
\usepackage[round]{natbib}
\usepackage{booktabs}
\usepackage{dcolumn}
\usepackage{mathrsfs}
\usepackage{tikz}
\usetikzlibrary{decorations.pathreplacing}
\usepackage{sansmathaccent}
\pdfmapfile{+sansmathaccent.map}
\usetikzlibrary{shapes.geometric, arrows,chains}
\tikzset{
  startstop/.style={
    rectangle, 
    rounded corners,
    minimum width=3cm, 
    minimum height=1cm,
    align=center, 
    draw=black, 
    fill=red!30
    },
  startsleft/.style={
    rectangle, 
    rounded corners,
    minimum width=3cm, 
    minimum height=1cm,
    align=left, 
    draw=black, 
    fill=red!30
    },
  startsright/.style={
    rectangle, 
    rounded corners,
    minimum width=3cm, 
    minimum height=1cm,
    align=right, 
    draw=black, 
    fill=red!30
    },
  process/.style={
    rectangle, 
    minimum width=3cm, 
    minimum height=1cm, 
    align=center, 
    draw=black, 
    fill=blue!30
    },
  decision/.style={
    rectangle, 
    minimum width=3cm, 
    minimum height=1cm, align=center, 
    draw=black, 
    fill=green!30
    },
  arrow/.style={thick,->,>=stealth},
  dec/.style={
    ellipse, 
    align=center, 
    draw=black, 
    fill=green!30
    },
  font={\fontsize{9pt}{12}\selectfont}
}
%\renewcommand{\labelitemi}{$\blacktriangleright$}

\epstopdfsetup{outdir=./}

\newcolumntype{Y}{>{\centering\arraybackslash}X}
\def\Put(#1,#2)#3{\leavevmode\makebox(0,0){\put(#1,#2){#3}}}

\newcommand{\subhead}[1]{\mbox{}\newline\textbf{#1}\newline}
\newcommand{\ave}[1]{\left\langle #1 \right \rangle}
\newcommand{\eg}{{\it e.g.}}
\newcommand{\ie}{{\it i.e.}}
\newcommand{\cf}{{\it c.f.}}
\newcommand{\etc}{{\it etc.}}
\newcommand{\etal}{{\it et al.}}
%\newcommand{\btVFill}{\vskip0pt plus 1filll}

\newcommand{\del}{D}
\newcommand{\hor}{H}

\newcommand{\threepartdef}[6]
{
  \left\{
    \begin{array}{lll}
      #1 & \mbox{if } #2 \\
      #3 & \mbox{if } #4 \\
      #5 & \mbox{if } #6
    \end{array}
  \right.
}


\newcommand{\Ito}{It\^{o}}
\newcommand{\SP}{S{\&}P500}
\newcommand{\lopt}{\ell_{\text{opt}}}
\newcommand{\gest}{g_{\text{N,T}}}
\newcommand{\elabel}[1]{\label{eq:#1}}
\newcommand{\eref}[1]{Eq.~(\ref{eq:#1})}
\newcommand{\Eref}[1]{Equation~(\ref{eq:#1})}

\newcommand{\flabel}[1]{\label{fig:#1}}
\newcommand{\fref}[1]{Fig.~\ref{fig:#1}}
\newcommand{\Fref}[1]{Figure~\ref{fig:#1}}
\newcommand{\person}[1]{{#1}}
\newcommand{\ra}[1]{\renewcommand{\arraystretch}{#1}}
\newcommand{\vs}[1]{\vspace{.#1cm}}
\newcommand{\vf}{\vspace{.25cm}}
\newcommand{\vff}{\vspace{.6cm}}
\newcommand{\np}{\\ \vf}
\newcommand{\npp}{\\ \vff}
\newcommand{\be}{\begin{equation*}}
\newcommand{\ee}{\end{equation*}}
\newcommand{\bea}{\begin{eqnarray*}}
\newcommand{\eea}{\end{eqnarray*}}
\newcommand{\bc}{\begin{center}}
\newcommand{\ec}{\end{center}}
\newcommand{\bie}{\begin{enumerate}}
\newcommand{\eie}{\end{enumerate}}
\newcommand{\bi}{\begin{itemize}}
\newcommand{\ei}{\end{itemize}}
\newcommand{\toinf}{\rightarrow\infty}
\newcommand{\D}{{\Delta}}
\newcommand{\Dx}{{\Delta x}}
\newcommand{\Dy}{{\Delta y}}
\newcommand{\Du}{{\Delta u}}
\newcommand{\DW}{{\Delta W}}
\newcommand{\DU}{{\Delta U}}
\newcommand{\du}{{\delta u}}
\newcommand{\Dv}{{\Delta v}}
\newcommand{\dt}{{\delta t}}
\newcommand{\gens}{g_{\ave{\,}}}
\newcommand{\ft}[1]{\frametitle{#1}}
\newcommand{\bq}{\begin{quote}}
\newcommand{\eq}{\end{quote}}
\newcommand{\ww}[1]{\bq{\small\rm#1\\}\eq}
\newcommand{\E}{\mathrm{E}}
\newcommand{\Var}{\mathrm{Var}}
\newcommand{\Cov}{\mathrm{Cov}}
\newcommand{\sgn}{\mathrm{sgn}}
\newcommand{\prob}[1]{\mathcal{P}\left(#1\right)}
\newcommand{\lra}{\longrightarrow}
\newcommand{\eps}{\varepsilon}
\newcommand{\ga}{g_\text{ave}}
\newcommand{\gt}{g_\text{typ}}
\newcommand{\gbar}{\bar{g}}
\newcommand{\mbar}{\bar{m}}
\newcommand{\red}[1]{\textcolor{red}{#1}}
\newcommand{\xf}{{x_F}}
\newcommand{\xb}{{x_B}}
\newcommand{\muf}{{\mu_F}}
\newcommand{\mub}{{\mu_B}}
\newcommand{\sigf}{{\sigma_F}}
\newcommand{\sigb}{{\sigma_B}}
\newcommand{\gf}{{\gbar_F}}
\newcommand{\gb}{{\gbar_B}}
\newcommand{\pa}{\textit{pa}}
\newcommand{\taus}{{\tau_\text{s}}}
\newcommand{\Dt}{\Delta t}
\newcommand{\etau}{\tau^\text{eqm}}
\newcommand{\taue}{\tau^\text{EGBM}}
\newcommand{\wtau}{\widetilde{\tau}}
\newcommand{\xN}{\ave{x}_N}
\newcommand{\Sdata}{S^{\text{data}}}
\newcommand{\Smodel}{S^{\text{model}}}
\beamertemplatenavigationsymbolsempty

\newcommand{\tlabel}[1]{\label{tab:#1}}
\newcommand{\tref}[1]{Tab.~\ref{tab:#1}}
\newcommand{\Tref}[1]{Table~\ref{tab:#1}}

\newenvironment{myindentpar}[1]%
{\begin{list}{}%
    {\setlength{\leftmargin}{#1}}%
  \item[]%
}
{\end{list}}

%\usetheme[width=1.8cm,hideothersubsections]{Frankfurt}
\usetheme{Frankfurt}

\newcommand\BackgroundPicture[1]{
\setbeamertemplate{background}{
\parbox[c][\paperheight]{\paperwidth}{
\vfill \hfill
\includegraphics[width=1\paperwidth,height=1\paperheight]{#1}
\hfill \vfill
}}}


\definecolor{lmlblue}{RGB}{0,77,123}
\definecolor{deepblue}{RGB}{35,33,169}
\definecolor{lmllb}{RGB}{237,244,255}
\definecolor{lmlred}{RGB}{155,29,29}
\definecolor{lmlgrey}{RGB}{142,142,142}
\definecolor{lmlgrey2}{RGB}{82,82,82}
\definecolor{grey}{RGB}{210,210,210}
\xdefinecolor{lightblue}{rgb}{0,200,255}
\setbeamercolor{important}{bg=lightblue,fg=red}
\AtBeginEnvironment{definition}{%
  \setbeamercolor{block body}{fg=black,bg=white}
  \setbeamercolor{block title}{bg=lmllb,fg=black}
}

\AtBeginEnvironment{theorem}{%
  \setbeamercolor{block body}{fg=black,bg=white}
  \setbeamercolor{block title}{bg=lmllb,fg=black}
}

%\newcommand{\propnumber}{} % initialize
%\newtheorem*{prop}{Proposition \propnumber}
%\newenvironment{propc}[1]
%  {\renewcommand{\propnumber}{#1}%
%   \begin{shaded}\begin{prop}}
%  {\end{prop}\end{shaded}}
%\AtBeginEnvironment{propc}{%
%  \setbeamercolor{block body}{fg=black,bg=white}
%  \setbeamercolor{block title}{bg=lmllb,fg=black}
%}

\setbeamercolor{fine separation line}{fg=lmllb}

\setbeamercolor{item projected}{fg=white, bg=black}

\setbeamercolor{frametitle}{bg=lmllb, fg=black}
\setbeamertemplate{frametitle}[default][left,colsep=-4bp,rounded=false,shadow=false]

\setbeamercolor{structure}{bg=white, fg=black}
%structure changes color of title in sidebar.

%\setbeamertemplate{frametitle}[default][colsep=-4bp,rounded=false,shadow=false]
\setbeamercolor{section in head/foot}{fg=lmlgrey2, bg=lmllb}
\setbeamercolor{normal text}{fg=black}
\setbeamercolor{title}{bg=white,fg=deepblue}

\hypersetup{colorlinks,linkcolor=,urlcolor=deepblue}

\setbeamerfont{title in sidebar}{size=\fontsize{9}{9}\selectfont}
\setbeamerfont{section in sidebar}{size=\fontsize{7}{7}\selectfont}

% add frame numbers to navigation bar
% (for RSS 2015 conference)
\addtobeamertemplate{navigation symbols}{}{
    \usebeamerfont{footline}
    \usebeamercolor[fg]{footline}
    \hspace{1em}
    \scriptsize
%    \insertframenumber/\inserttotalframenumber
}
\setbeamertemplate{navigation symbols}{} %gets rid of navigation symbols
\setbeamercovered{transparent=20}
%\setbeamertemplate{footline}[frame number] % to show overlay page numbers type: page number
%\setbeamersize{text margin left=0.65cm, text margin right=0.65cm}
%\setbeamercolor{item}{fg=black!70!black} % red bullets
%\setbeamertemplate{itemize subitem}[triangle] % triangle sub-bullets
%\setbeamerfont{itemize/enumerate subbody}{size=\normalsize}


\title[\begin{flushleft} \color{white} {\tiny ???} \end{flushleft}]{\bf {Industrial Organization, Week 6 \\ 
Advertising}}
\author[shortname]{ Dio Mavroyiannis \inst{\dag}}
\institute[\begin{flushleft} \color{white} {\large ???} \end{flushleft}]{Milestone Institute}
%\institute[shortinst]{\inst{*} London Mathematical Laboratory \and \inst{\dag} Universit\'{e} Paris-Dauphine \and \inst{\ddag} London Mathematical Laboratory and Santa Fe Institute}
\date{10 March 2021}

\AtBeginSection[]{
\frame{
\ft{Agenda}
\tableofcontents[currentsection,hideallsubsections]
}
}

\makeatletter
\setbeamertemplate{headline}{%
  \pgfuseshading{beamer@barshade}%
  \ifbeamer@sb@subsection%
    \vskip-9.75ex%
  \else%
    \vskip-7ex%
  \fi%
    \begin{beamercolorbox}[ht=3.5ex,dp=2.125ex]{section in head/foot}
     \insertsectionnavigationhorizontal{\textwidth}{}{}%
  \end{beamercolorbox}%
  \ifbeamer@sb@subsection%
    \begin{beamercolorbox}[ignorebg,ht=2.125ex,dp=1.125ex,%
      leftskip=.3cm,rightskip=.3cm plus1fil]{subsection in head/foot}
      \usebeamerfont{subsection in head/foot}\insertsubsectionhead
    \end{beamercolorbox}%
  \fi%
}%

\makeatother

\setbeamertemplate{mini frames}{}
\setbeamertemplate{itemize items}[triangle]

\begin{document}
\frame{\titlepage\insertlogo}

\section{Big picture}
%%%%%%%%%%%%%%%%%%%%%%%%%%%%%%%%%%%%%%%%%%%%%
\frame{\ft{Advertising plan}
\setbeamertemplate{itemize items}[triangle]

\bi
\item Plan: We look at advertising today
\item This is more of applied IO, we have the main theory under our belts
\item Persuasive model and informative model under monopoly
\item Competition with advertising
\ei

}

\frame{\ft{How do we model advertising?}
\setbeamertemplate{itemize items}[triangle]

\begin{equation}
Q_p \equiv \frac{\partial Q}{\partial p}<0; Q_p \equiv \frac{\partial Q}{\partial A}>0
\end{equation}

\begin{align}
\Pi (p,A) &= pQ(p,Q) - C(Q(p,A))-A \\
\frac{\partial \Pi}{\partial p} &= (p-C')Q_p = 0 \leftrightarrow \frac{p-C'}{p} = -\frac{Q}{p Q_p} = \frac{1}{\eta_{Q,p}} \\
\frac{\partial \Pi}{\partial A} &= (p-C')Q_A-1 = 0 \leftrightarrow \frac{p-C'}{p} = \frac{1}{ Q_A}\frac{1}{P} = \frac{Q}{A Q_A}\frac{A}{P Q} = \frac{1}{\eta_{Q,A}} \frac{A}{pQ} \\
\frac{1}{\eta_{Q,p}} &= \frac{1}{\eta_{Q,A}} \frac{A}{pQ} \leftrightarrow \frac{A}{pQ}= \frac{\eta_{Q,A}}{\eta_{Q,p}} 
\end{align}

So the monopolist sets their advertising expenditure as a function of the ratio advertising elasticity of demand and price elasticity of demand. 

}

\section{Monopoly}

\frame{\ft{Interpreting Advertising: Persuasive}
\setbeamertemplate{itemize items}[triangle]


\bi
\item We can use hotelling to represent different valuations. Consumers on the hotelling map with the firm at 1. 
\item We can boost everyones valuation by advertising, let the willingness to pay be: $g(A) x$
\item $Q(p,A) = 1 - \frac{p}{g(A)} \leftrightarrow \eta_{Q,p} = \frac{p}{g(A)-p}$
\item Can also be interpreted as complementary
\ei


}

% I am already proud of you all for keeping up, so the exam will be mostly on the first 6 weeks. I will select a topic for week 7, information, but it it will not be on the exam. Week 8 might not even have a problem set so if you all want a specific topic we can do that. 


\frame{\ft{Interpreting Advertising: Informative}
\setbeamertemplate{itemize items}[triangle]


\bi
\item The firm randomly sends advertising to N consumers hoping to inform them.  
\item The firm sends A messages, the consumers who have not received an ad is $(1-\frac{1}{N})^A \approx e^{-\frac{A}{N}}$
\item $Q(p,Q) = N(1-e^{-\frac{A}{N}})d(p) \equiv G(A)d(p)$
\item First derivative is positive, second derivative is negative.
\item The price is not affected in this view. 
\ei

}

\frame{\ft{Interpreting Advertising: Informative 2}
\setbeamertemplate{itemize items}[triangle]


\bi
\item Information does not have to be about the product, it can be about the firm
\item Signalling theory tells us it can also just be about costs. 
\item If the good product and bad product look identical, then a firm may advertise to differentiate. 
\ei

}



\frame{\ft{The welfare effects are difficult}
\setbeamertemplate{itemize items}[triangle]


\bi
\item Result 1: If price is not increasing in advertising, too little advertising is supplied by monopolist. 
\item Result 2: If the price is increasing in advertising,  ambigous welfare effects. 
\ei

}

\section{Oligopoly}

\frame{\ft{Hotelling with three types of consumers}
\setbeamertemplate{itemize items}[triangle]


\bi
\item Fully informed consumers $\lambda_1 \lambda_2$. Informed only about firm 1 $\lambda_1 (1-\lambda_2)$
\item $Q_1(p_1,p_2,\lambda_1, \lambda_2 ) = \lambda_1[(1-\lambda_2)+\lambda_2 \overline{x}(p_1,p_2) ] $
\item If the good product and bad product look identical, then a firm may advertise to differentiate. 
\item To simplify assume that the advertising cost function is: $A(\lambda_i ) = a \frac{\lambda_i^{2}}{2}$
\item with $a> \frac{\tau}{2}$ to ensure not everybody is informed at equilibrium
\ei

}

\frame{\ft{Hotelling with three types of consumers 2}
\setbeamertemplate{itemize items}[triangle]

\begin{align}
\Pi =& (p_1-c)Q_1(p_1,p_2,\lambda_1, \lambda_2 ) - A(\lambda_1 ) \\
\rightarrow& \lambda_1 = \frac{1}{a}(p_1-c)[1-\lambda_2 + \lambda_2 \frac{1}{2 \tau }(p_2-p_1 +\tau ) ] \\
\rightarrow& p_1 = \frac{p_2+c+\tau}{2}+\frac{1- \lambda_2}{\lambda_2}\tau \\
\rightarrow& p = c + \frac{2-\lambda^*}{\lambda^*} \tau \\
\rightarrow& \lambda^* = \frac{2}{1+\sqrt{\frac{2a}{\tau}}} \\
\rightarrow& \pi_1 = \frac{2a}{(1+\sqrt{\frac{2a}{\tau}})^2}
\end{align}

}

\frame{\ft{Hotelling with three types of consumers 3}
\setbeamertemplate{itemize items}[triangle]

\bi
\item There is a zero sum aspect to advertising competition.
\item Higher advertising costs lead more market power
\item Firms prefer there to be higher advertising costs
\ei

}

\frame{\ft{Conclusion}
\setbeamertemplate{itemize items}[triangle]


\bi
\item The effect of advertising is ambigous
\item It is difficult to know if advertising informative or persuasive
\ei

}
\end{document}
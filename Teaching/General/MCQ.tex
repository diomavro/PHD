%\documentclass[AER]{AEA}
\documentclass[11pt]{article}
%\documentclass[12pt]{article}
%\documentclass[12pt,a4paper]{article}
\usepackage[utf8x]{inputenc}
\usepackage{float}
%\usepackage[cmbold]{mathtime}
%\usepackage{mt11p}
\usepackage{placeins}
\usepackage{amsmath}
\usepackage{color}
\usepackage{amssymb}
\usepackage{mathtools}
\usepackage{subfigure}
\usepackage{multirow}
\usepackage{epsfig}
\usepackage{listings}
\usepackage{enumitem}
\usepackage{rotating,tabularx}
%\usepackage[graphicx]{realboxes}
\usepackage{graphicx}
\usepackage{graphics}
\usepackage{epstopdf}
\usepackage{longtable}
\usepackage[pdftex]{hyperref}
%\usepackage{breakurl}
\usepackage{epigraph}
\usepackage{xspace}
\usepackage{amsfonts}
\usepackage{eurosym}
\usepackage{ulem}
\usepackage{footmisc}
\usepackage{comment}
\usepackage{setspace}
\usepackage{geometry}
\usepackage{caption}
\usepackage{pdflscape}
\usepackage{array}
\usepackage[round]{natbib}
\usepackage{booktabs}
\usepackage{dcolumn}
\usepackage{mathrsfs}
%\usepackage[justification=centering]{caption}
%\captionsetup[table]{format=plain,labelformat=simple,labelsep=period,singlelinecheck=true}%

%\bibliographystyle{unsrtnat}
\bibliographystyle{aea}
\usepackage{enumitem}
\usepackage{tikz}
\def\checkmark{\tikz\fill[scale=0.4](0,.35) -- (.25,0) -- (1,.7) -- (.25,.15) -- cycle;}
%\usepackage{tikz}
%\usetikzlibrary{snakes}
%\usetikzlibrary{patterns}

%\draftSpacing{1.5}

\usepackage{xcolor}
\hypersetup{
colorlinks,
linkcolor={blue!50!black},
citecolor={blue!50!black},
urlcolor={blue!50!black}}

%\renewcommand{\familydefault}{\sfdefault}
%\usepackage{helvet}
%\setlength{\parindent}{0.4cm}
%\setlength{\parindent}{2em}
%\setlength{\parskip}{1em}

%\normalem

%\doublespacing
\onehalfspacing
%\singlespacing
%\linespread{1.5}

\newtheorem{theorem}{Theorem}
\newtheorem{corollary}[theorem]{Corollary}
\newtheorem{proposition}{Proposition}
\newcommand{\ra}[1]{\renewcommand{\arraystretch}{#1}}

\newcommand{\E}{\mathrm{E}}
\newcommand{\Var}{\mathrm{Var}}
\newcommand{\Corr}{\mathrm{Corr}}
\newcommand{\Cov}{\mathrm{Cov}}

\newcolumntype{d}[1]{D{.}{.}{#1}} % "decimal" column type
\renewcommand{\ast}{{}^{\textstyle *}} % for raised "asterisks"

\newtheorem{hyp}{Hypothesis}
\newtheorem{subhyp}{Hypothesis}[hyp]
\renewcommand{\thesubhyp}{\thehyp\alph{subhyp}}

\newcommand{\red}[1]{{\color{red} #1}}
\newcommand{\blue}[1]{{\color{blue} #1}}

\newcommand*{\qed}{\hfill\ensuremath{\blacksquare}}%

\newcolumntype{L}[1]{>{\raggedright\let\newline\\arraybackslash\hspace{0pt}}m{#1}}
\newcolumntype{C}[1]{>{\centering\let\newline\\arraybackslash\hspace{0pt}}m{#1}}
\newcolumntype{R}[1]{>{\raggedleft\let\newline\\arraybackslash\hspace{0pt}}m{#1}}

\geometry{left=1.5in,right=1.5in,top=1.5in,bottom=1.5in}
%\geometry{left=1in,right=1in,top=1in,bottom=1in}

\epstopdfsetup{outdir=./}

\newcommand{\elabel}[1]{\label{eq:#1}}
\newcommand{\eref}[1]{Eq.~(\ref{eq:#1})}
\newcommand{\ceref}[2]{(\ref{eq:#1}#2)}
\newcommand{\Eref}[1]{Equation~(\ref{eq:#1})}
\newcommand{\erefs}[2]{Eqs.~(\ref{eq:#1}--\ref{eq:#2})}

\newcommand{\Sref}[1]{Section~\ref{sec:#1}}
\newcommand{\sref}[1]{Sec.~\ref{sec:#1}}

\newcommand{\Pref}[1]{Proposition~\ref{prop:#1}}
\newcommand{\pref}[1]{Prop.~\ref{prop:#1}}
\newcommand{\preflong}[1]{proposition~\ref{prop:#1}}

\newcommand{\clabel}[1]{\label{coro:#1}}
\newcommand{\Cref}[1]{Corollary~\ref{coro:#1}}
\newcommand{\cref}[1]{Cor.~\ref{coro:#1}}
\newcommand{\creflong}[1]{corollary~\ref{coro:#1}}

\newcommand{\etal}{{\it et~al.}\xspace}
\newcommand{\ie}{{\it i.e.}\ }
\newcommand{\eg}{{\it e.g.}\ }
\newcommand{\etc}{{\it etc.}\ }
\newcommand{\cf}{{\it c.f.}\ }
\newcommand{\ave}[1]{\left\langle#1 \right\rangle}
\newcommand{\person}[1]{{\it \sc #1}}

\newcommand{\AAA}[1]{\red{{\it AA: #1 AA}}}
\newcommand{\YB}[1]{\blue{{\it YB: #1 YB}}}

\newcommand{\flabel}[1]{\label{fig:#1}}
\newcommand{\fref}[1]{Fig.~\ref{fig:#1}}
\newcommand{\Fref}[1]{Figure~\ref{fig:#1}}

\newcommand{\tlabel}[1]{\label{tab:#1}}
\newcommand{\tref}[1]{Tab.~\ref{tab:#1}}
\newcommand{\Tref}[1]{Table~\ref{tab:#1}}

\newcommand{\be}{\begin{equation}}
\newcommand{\ee}{\end{equation}}
\newcommand{\bea}{\begin{eqnarray}}
\newcommand{\eea}{\end{eqnarray}}

\newcommand{\bi}{\begin{itemize}}
\newcommand{\ei}{\end{itemize}}

\newcommand{\Dt}{\Delta t}
\newcommand{\Dx}{\Delta x}
\newcommand{\Epsilon}{\mathcal{E}}
\newcommand{\etau}{\tau^\text{eqm}}
\newcommand{\wtau}{\widetilde{\tau}}
\newcommand{\xN}{\ave{x}_N}
\newcommand{\Sdata}{S^{\text{data}}}
\newcommand{\Smodel}{S^{\text{model}}}

\setlength{\parindent}{0.0cm}
\setlength{\parskip}{0.6em}

\numberwithin{equation}{section}
\DeclareMathOperator\erf{erf}
%\let\endtitlepage\relax

\begin{document}

Instructions: 
Vous devez répondre à au moins 5 questions. Votre note sera donné par: $n = 10 + 2 x \% $, où x est le nombre net de réponses correctes, ceci peut prendre des valeurs négatives.

\textbf{Question 1:} La demande est donné par $ 120-2 p = Q_d$ et l'offre par $4p = Q_s$. On introduit une tax une tax sur la consommation a $30$ euros. Calculé le surplus des consommateurs avant et aprés la tax. 

\begin{align*}
&1) 3500,3000 && 2) 4000, 1600 \\
&3) 4000,2000 && 4) \text{autre}
\end{align*}


\textbf{Question 2:}
On suppose que la fonction d'utilité d'un agent est donné par $u=xy$. On suppose que les prix sont $p_x=1$ et $p_y=5$. Le revenu de l'agent est $100$. Quel est sa consomation d'equilibre? (x,y)

\begin{align*}
&1) (30,50) && 2) (50,10) \\
&3) (40,12) && 4) \text{autre}
\end{align*}

\textbf{Question 3:}
On suppose que la fonction d'utilité d'un agent est donné par $u=min\{x,2y\}$. On suppose que les prix sont $p_x=1$ et $p_y=4$. Le revenu de l'agent est $100$. Quel est sa consomation d'equilibre? (x,y)

\begin{align*}
&1) (60,10) && 2) (20,20) \\
&3) (40,12) && 4) \text{autre}
\end{align*}

\newpage

\textbf{Question 4:}
Les entreprises A et B produisent chacune 50 Iphones. Pour chaque Iphone qu'ils produisent, ils polluent 50 tonnes. L'entreprise A peut créer le sans polluer en augmentant ses coûts par iphone de 20 euros, et l'entreprise B peut produire des Iphones sans polluer en augmentant ses coûts par Iphone de 50 euros.

Scénario 1. Le gouvernement a décidé de forcer toutes les entreprises à forcer toutes les entreprises à réduire leur pollution de 50\%.

Scénario 2: Le gouvernement a décidé d'accorder à chaque entreprise un permis échangeable pour polluer 25 tonnes.

La production au scénario 2 est ...
La pollution au scénario 2 est ...

\begin{align*}
&1) \textit{faiblement supérieure, inférieure} && 2) \textit{faiblement inférieure, supérieure} \\
&3) \textit{faiblement supérieure, identique} && 4) \textit{autre}
\end{align*}

\textbf{Question 5:}
A) À court terme, le coût total par unité doit être égal au coût marginal.

B) Les coûts à long terme sont plus importants que les coûts à court terme.

\begin{align*}
&1) A: vrai; B: faux  && 2) A: vrai; B: vrai \\
&3) A: faux; B: faux && 4) A: faux; B: vrai
\end{align*}

\end{document}

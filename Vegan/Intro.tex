
\chapter{Prelude}
In south central park, on 60th street, I went to go meet my ex Coach, Prince Gomez, who I had not seen in a decade, we had a lot to catch up on so we met at the square, I was in New York for wedding of a friend who payed 50k for flowers for a marriage that lasted less than 6 months. Anyway it turned out my ex coach was obsessed with Pokemon Go and every 10 minutes or so a large mass of people would randomy run somewhere in central park to catch a rare pokemon. I never played this game, didn't seem very interesting to me but I tagged along with my coach anyway. As I ran around I noticed a small second hand bookstore on the side of central park, I had lived here for years but never noticed it before, always passed by, always had seen it, but too distracted, my mind on other things. It was here that I first ran into Peter Singer's book \cite{singer1995animal}. I went back home and finished the book in a day, it was a fun read, and moved on with my life. 

Somehow I started following philosophers on twitter and noticed that Singer was a big name to them. I was a bit perplexed at this popularity. What is happening? So I went back to re-read the book to see if I had missed some potentially life changing experience. Admitedly I had a hard to understanding what people enjoyed in the book. 

I have written this book because I have not found a consise statement defending the morality of eating animals. I find this especially puzzling since philosophers are known to leave no stone left unturned, almost every position you can imagine has been defended \footnote{List some Nazi phislophers, Schmitt, (that guy on youtube), Frege etc}. 

The answer to this odd phenomenon is philosophers are attached to a sort of deductive method of reasoning. It so happens that grand principles often rely on measurable quality. That is, most philosophers refuse to cede ground to intuition or if they do speak of intuition in a sort of formal way\footnote{Huemer, ethical intuitinism}. 

With no intent to being polemical, it seems fairly clear that most philosphers are in fact heavily left-wing. The left-wing are known to work in "movements", that is a popular trend catches on and the left think they have found a new truth that must be striven for. It is perhaps no exageration to say that 99\% of those movements fail to achieve what they aim for, indeed most such movements are forgotten after few hundred years but it is in the credo of this group to cherry pick their succeses and adverise them as if they were always ahead of the curve. 

My intend for this book to be eminenly readable by everyone. I am articulating what I think is simply common sense because the common man has no interest in articulating his common sense, and to the uncharitable academics this is often interpreted as not having an argument. 

Academics forget that we invent our intellectual ideas to aid us in our everyday endeavours. Philosophers are often tempted to change this reasoning, they think that our everyday endeavours are there to aid our intellectual ideas. Ideas such as equality/liberty/freedom/independence etc, represent such a backward reasoning. Philosophers should aim to show how these ideas capure what we are trying to do, for instance it may be that in our everyday endeavours, the heuristics of equality/liberty/freedom/independence make our task easier to analyze and our endeavour less costly, but the way philosophers usually use these concepts is that these are the ultimate ends in themselves. 


It must be remembered that Philosophy is the generator of all knowledge, all modern disciplines were once under the wing of philosophers and as they grew beyond their infancy they became their own disciplines, evolving independely of philosophical trends\footnote{Mathematics, Physics, Anthropology, Psychology, Economics(Adam Smith), etc}. 


Each chapter in this book will be independent of the others, so there is not much need to read it linearly. The specific kind of use I expect of this book is as a sort of reference book of arguments. 

Much of the problem in philosophy is being parsimious, it how many ways should one use to divide up the world? I considered when writing the book to demarcate between consequentialist methods of ethics and non-consequentialit methods. Of course the problem with attacking or defending from the categories of consequences is that it is not clear what counts as a consequence. 

The first part of the book will be analyzing existing arguments made my various vegetarians and vegans to explain why these are false. My first chapter will be the most important and most popular argument that is mainted today, that is a minimizing-harm argument. 

After defending the idea that formulating arguments is superfluous for the practicioners, I will try to formulate what I think is a charitable interpreation of their arguments. That is I will present a series of argument FOR meat eating. 

Finally in the last section I will present what I think are some weak ways of formulating pro-eating arguments. 


\chapter{Introduction}

Philosophy used to be a topic whose main ethical branch was virtue ethics. Skipping unnecesary details about virtue ethics, the main thing to know about it is that it does not prescribe to ordinary people what to do. Instead it assumes that professional reasoners don't really have a superior ethical framework, instead they simply think about things clearly and help the non-professional to articulate what it is they really want to achieve and to articulate superior and inferior ways of achieving the things they have already decided. In other words, ethics didn't use to mean to prescribe to others what to do, it was simply to describe to others how to achieve what they already want to achieve. 

This is perhaps the most change that has occured from the enlightnement to today. A sort of scientism, the idea that morality must be discovered by those who are experts at reasoning and then imposed on the non-experts. 

Notice that though this enlighnment approach often tries to take credit for the ban of slavery, this is trivially untrue, the origins of the ban on slavery historically had distinctly christian origins with (Jacobins), being explicltey funneled through the natural law tradition. Indeed, William Wilberforce held no special expert knowledge, instead he was a religious zealot who held that his task was given to him by god. In other words, in the christian conception, the moral solution can strike any individual, the expert is in no better position to judge what is correct. 

Memetic points: 
Virtue theory: experts dont make morality, they have no expertise on what is right, they only have expertise on achieving what is right. 

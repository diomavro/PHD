
\chapter{The bad arguments for eating animals}

\section{It is natural}

\begin{align}
\text{Eating Meat is natural} \\
\text{Doing what is natural is good} \\
\text{eating meat is good.}
\end{align}

Failed arguments;


They sometimes dispute the naturalness but most evolutionary biologists agree that we have evolved because we have used less energy on digestion and more on brains.  

\section{We are superior}

\section{Empirical differences leaning}
Empirical facts, suppose you lean on an empirical difference between. 

Information based arguments are rather weak. For instance slavery, one could simply say that in the past, society was filled with racial prejudices and did not realize that those who are slaves are ALSO capable of higher pleasures. 


https://twitter.com/SteveCooke/status/1182015830349041665

https://twitter.com/diomavro/status/1182013623906062343

https://twitter.com/diomavro/status/1182013623906062343

https://twitter.com/SteveCooke/status/1182015130034429959?s=20

\subsection{WHy drop additivity}

It is true that non-utilitarian/consequentialist methods have the disadvantage of not being able to weight the different lives of agents. So upon being given a counter-example like the one above how does the non-utilitarian revise his theory? Well a wholist would simply deduce things from those impossibilities. If for instance you are given the choice between saving 10 or 5 people using different means, then the focus will either be on the means or to make sure the scenario would never emerge. If for instance I reply to the trolley problem that I would not make a decision, and hence kill the 5 instead of the 1, the inference should be that society should be structured in such a way so that this kind of choice becomes impossible. Why should one drop the rule "never murder an innocent" instead of dropping the rule "drop additivity?" 


\subsection{Lifestyle preferences}

It is often assumed that good is attributable to specific actions. However there is a complete failure to articulate the discontinuities of life. Archtypes or cutlural goods are non reducible. It is NOT true that you can remove the olive oil from the greek diet and still have the greek diet. Some things are just fundamental. 

How people measure the good and bad is with lifestyles, not with individual deeds. In other words, one may in theory be content with their daughter sleeping around with a different guy every night. However the repugnance to this may in fact be with the lifestyle that is associated, the alienation from feelings, the frequenting of night clubs. When someone tells you they want 
 x, this is not necessarily because they want x in itself. Indeed it may be that they simply want things that are associated with x. 

For animals specifically the reason is easy enough to see. I want a reason to live with animals. Or I want a reason for animals to exist or to exist in greater number. And by wanting x, I am creating that reason. 

One may try to envision different ways that we can live with animals, but it is ultimately an empirical question. One such way we could imagine is to worship the animals, perhaps between the gaps we have at work, we could go to the chicken altars and worship them. One may imagine simply that these animals are publicly financed to live in the streets, were they are fed or their poop is cleaned at public expense. This all sounds reasonable, and it is likely a few meat eaters would be convinced if this was shown to work in practice.

Of course if the animal rights activists position becomes less extreme the answer is quite natural for a few animals. Perhaps cows sheep goats and chickens could all give us ample reason to live with them. But still there remain animals which don't have this property, such as pigs. 

It is wrong to want to have kids for your own pleasure. You should want to have kids because your gut tells you to. You can't articulate WHY you should have kids, but there is this tendency in you to have kids. It is not that you think kids will make you happier, but that having kids itself is what life is about, it is is simply the expression of who you are, just like a flute gains no pleasure from being played, but it is its purpose. Indeed it is it's very reason for existing, it is the cause of it existing. 

Dogs have found their place with us and most of us are thankful. Perhaps a significant difference between dogs and other animals. 

It is perhaps a typically modern tendency to try and calculate the costs and benefits of all structures.  The search for deductive beauty is often a homogenizing force. However inductive beauty gives very different results. Deductively we may have a positive impression of an idillic community where everyone has the same life. But inductively there is something disgusting in knowing that everything is the same everywhere. Diversity is a value in itself, we want everyone to plant different plants in their gardens. We can imagine that the diversity is such that everyone plants the same different plants. 


\subsection{Formulation vs induction}
It seems like philosophers have taken up a task that the ancients never dared. That is to assume that if an argument cannot be formulated, that the position is incoherent. It is hard to imagine a more arrogant position than the position that if something cannot be defended, it should be abandoned. 

In reality even though you cannot defend an action as such, you can imagine reasons why it emerged. And without whosing that those reasons are no longer neccesry, it is incoherent to throw it away. 

\subsection{Against false principles}

How do we know if a principle is good? If it accords with our intuitions. Indeed almost all tragedies in the human race follow this same pattern. "My principle is good, therefore x is justified". in reality people should just see if a principle holds true to their intuition and then go with it. The animal rights activists are a failure because they go by the principles of minimize unnecesary suffering, or some other weird principle. In reality the reasoning we should be following is the opposite, "eating animals is okay" therefore the principle of minimizin suffering is false. 

There is this class of arguments which intuitively many people have which sound immoral. For instance many people will fall back into the notion of "what if you have to?", clearly morality takes into account neccesity. If something is immoral, your obligation does not somehow change because of your need. The the clear formulation of this argument would be:

\begin{align}
1) \text{It was immoral to kill animals then.} \\
2) \text{If our ancestors didn't kill animals they would not have evolved as they did} \\
1&2 \rightarrow \text{We are buit on immorality}
\end{align}

This line of argument implies the world would be more moral if humans had never existed. 
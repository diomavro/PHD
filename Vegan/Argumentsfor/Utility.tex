
\section{Utilitarianism} 


Today there are two main versions of utilitarianism, the first is the benthamite version which does assume that utility is additive. The second is the John stuart Mill version, which though utilitarian in its own right, assumes that utility between being cannot be compared. This second one is the one usually employed by economists but sometimes we have economists which ignore the conditions neccesary for such comparability laid out by Harayani and simply proceed to compare these utilities anyway. Of course 


\begin{tcolorbox}[enhanced,%fit to height=10cm,
  colback=green!25!black!10!white,colframe=green!75!black,title=Principle of utility (10cm),
  drop fuzzy shadow,watermark color=white,watermark text=Fit]
 %\lipsum[1-4]
“Nature has placed mankind under the governance of two sovereign masters, pain and pleasure. It is for them alone to point out what we ought to do, as well as to determine what we shall do. On the one hand the standard of right and wrong, on the other the chain of causes and effects, are fastened to their throne. They govern us in all we do, in all we say, in all we think: every effort we can make to throw off our subjection, will serve but to demonstrate and confirm it. In words a man may pretend to abjure their empire: but in reality he will remain subject to it all the while. The principle of utility recognizes this subjection, and assumes it for the foundation of that system, the object of which is to rear the fabric of felicity by the hands of reason and of law. Systems which attempt to question it, deal in sounds instead of sense, in caprice instead of reason, in darkness instead of light.”
― Jeremy Bentham, The Principles of Morals and Legislation
\end{tcolorbox}

\begin{tcolorbox}[enhanced,%fit to height=10cm,
  colback=green!25!black!10!white,colframe=green!75!black,title=Principle of utility (10cm),
  drop fuzzy shadow,watermark color=white,watermark text=Fit]
 %\lipsum[1-4]
"The day may come when the rest of the animal creation may acquire those
rights which never could have been witholden from them but by the hand of
tyranny. The French have already discovered that the blackness of the skin
is no reason why a human being should be abandoned without redress to the
caprice of a tormentor. It may one day come to be recognized that the
number of the legs, the villosity of the skin, or the termination of the os
sacrum, are reasons equally insufficient for abandoning a sensitive being to
the same fate. What else is it that should trace the insuperable line? Is it the
faculty of reason, or perhaps the faculty of discourse? But a full-grown
horse or dog is beyond comparison a more rational, as well as a more
conversable animal, than an infant of a day, or a week, or even a month,
old. But suppose they were otherwise, what would it avail? The question is
not, Can they reason? nor Can they talk? but, Can they suffer?"
― Jeremy Bentham, The Principles of Morals and Legislation
\end{tcolorbox}

\begin{tcolorbox}[enhanced,%fit to height=5cm,
  colback=green!25!black!10!white,colframe=green!75!black,title=Fit box (5cm),
  drop fuzzy shadow,watermark color=white,watermark text=Fit]
\begin{align*}
1)& \text{We ought to maximize total utility} \\
2)& \text{Eating animals does not maximize total utility} \\
\rightarrow& \text{Therefore we ought not to eat animals}
\end{align*}
\end{tcolorbox}

Bentham was the founder of modern utilitarianism. As seen in the above quote, he even predicted that a consistent application of his ethical framework would be to extend rights to animals. Peter Singer is the most influential philosopher who has clearly and succincly brought Bentham's views into modern philosophy, the Singer formulation goes something like this: 
%Bentham predictd it

\begin{tcolorbox}[enhanced,%fit to height=5cm,
  colback=green!25!black!10!white,colframe=green!75!black,title=Fit box (5cm),
  drop fuzzy shadow,watermark color=white,watermark text=Fit]
\begin{align*}
1)& \text{The utility humans get from tasting animals is $\epsilon$} \\
2)& \text{The disutility animals get from being eaten is >$\epsilon$ utility from eating animals is trivially small} \\
\rightarrow& \text{Therefore eating animals does not maximize utility}
\end{align*}
\end{tcolorbox}


How does the theory work in practice? A being has numerous disutility and utility dimensions. Some disuutility dimensions could be \{ability to feel physical pain, ability to have dreams ruined, suffering from seeing others suffer, etc \}. Similarly there are a number of dimensions that can cause utility \{ direct pleasure, comfort, dreaming, love, family, etc \}. Let the different elements of utility be X and the different element in disutility be Y. 

Now the question is, how do these dimensions interact? If for instance we have an individible unit of physical pain, $x_1$ to distribute who should we give it to? Without loss of generality, let us imagine that Amy is a tough girl, and feels less pain of type $x_1$ than Bob. Does this mean we should give the unit of pain to Amy? Not quite. Utilitarianism requires us to look at the interaction between the variables. Let us take an example to demonstrate. 

1) Private Ryan is a war veteran who is very sensitive to pain, he dreams of becoming a professional surfer in California. 

2) Athena is a lawyer working in Nicosia, she is very resistant to pain and generally very in touch with her feelings, she dreams of writing a book one day.

Suppose we have one indivisible unit of pain to distribute to one of these two. If we give it to Ryan he will feel more direct physical pain than Athena. However if we give it to Athena she will feel the pain less but be traumatized by it for 5 years while Ryan won't feel the trauma. So if we used only the first dimension we would give it to Athena, if we used only the second, we would give it to Ryan. If we take both of these dimensions it could go either way. 

What occurs if the unit of pain also affects the capacity to attain their dreams? Perhaps the unit of pain is in fact the cutting off of a leg, in this case, it seems clear that Ryan will be affected more favorably because without a leg he won't be able to be a sufer, while Athena's positive item of writing a book will be less negatively affected(perhaps even positively affected). 

Of course if Ryan has numerous dreams, and some of the dreams are attainable even with the pain, then a utilitarian would ignore the fact that Ryan's choice set is reduced. We start here to see where utilitarianism may start to have difficulties, specifically, it values consequences, not the fact that the agent has a choice. Indeed this assumption is regularly made in the economic of general equilibrium, where the social planners utility maximizing plan always has at least as much total utility as the individuals utilities. 

Now let us conjure a third being: 

3) Hachiko is a loyal dog who can feel pain and can also be traumatized. 

Once again we can distribute the unit of pain to Hachiko or to Ryan and Athena. To claim that we should give Hachiko equal consideration merely means that we will weigh his pain and trauma to the same extent as we weigh the other beings. Of course we can immediatly notice that since Hachiko has no aspirations these cannot be adversely affected, so if the unit of pain has the property destroying aspirations, it may be optimally given to Hachiko since he will be less adversely affected. 

In other words, utilitarianism does entail that we accept a higher degree of pain on non-humans than on humans, simply because humans are more complex creatures and the same pain may have different effects. Indeed one could argue that chopping the leg of a human is not only inhibiting their aspirations but their social status and hence their ability to have a family etc. 

All this implies that there is a set of objections which fail. Some are found below:

1) You think an animal life is worth as much as a human? 

This is clearly false, as they can simply point to the fact that utility optimization entails that human lives are generaly worth more. 

\section{Problems}

Utilitarianism is a philosophy which has numerous issues. 
1) There is a direct measurement issue, that is, the categories the utilitarians want to use, may themselves not be well measured. 

2) There is a scale issue which is that by attempting to create addititivity the utilitarian ignores the nexus of decision. That is a decision taken by the individual has no difference from a decision taken by a leader. 

3) Preference, there is no way to aggregate preferences that doesn't cause issues. 

The arguments against the utilitarian view can heuristically be classified into two types. Arguments of the measurability type, and arguments of the scale type. 

\section{Teleology}
Suppose you are given the option to either free a slave or kill him. Perhaps there is harmless way of castrating them. A pill perhaps? 

Indeed since Chickens are in fact evolutionarily bread, to be eaten, there is in fact nothing 

Suppose a plant developed mobility or that it simply was more quick to react to its environment. 

The incredible thing is that they have even re-written words to create an inability to express things in the non-utilitarian manner. For instance, suppose I want to cleanse the planet of some type of specie. 


\section{Utilitarianism and preferences}

Utilitarianism and preferences are an odd mix. There are questions first about preferences, what exactly do people have preferences over? One common answer is over goods. A second common answer is how far are consumers aware of these preferences? Another question is are these preferences fixed? That is, can we meaningfully talk of comparing worlds when things aren't kept constant? We can imagine that if agents have preferences, that the social planner will simply find the world that best fits these preferences? 

If however we preferences are adaptive to the environment and not the xogenous thin that is pretended then the very notion of maximizing utility is meaningless without imagining possible worlds, indeed in this situation one must first have the vision of the world one wants to go to, and the preferences will simply follow. But this is perhaps exactly the kind of world that is sustainable, a specific vision. 

In economic jargon this might be termed endogenous preferences. Or adaptive behavior, the best work on this has been developed by Ole Peters. Indeed, this then asks us to imagine WHICH preferences can be satisifed best. For instance a world where aspirations are low might best satisfy preferences and reach higher utility, a world where one preferences are shaped by their family can give higher total utility than another one. 

The enlightment project should be revised as taking into account the importance of consent. Indeed the teleological vision of Aristotle is fundamentally correct but WHO creates habits that foster the virtues is of vital importance. Indeed, the idea is to make the environment the least dependent possible on external help, to make sure that the virtues can be cultivated without the fragility of a state, the only way to allow for this adaptivity is to make sure that the components foster those virtues on their own. 

Though many philosophers try to reconcile the ancients with scale, this in fact fails. The Ancients absolutely did fail to notice the link between consent and scale. A culture of consent is what allows for stability in this kind of thing. 

Here we have the famous Jordan peterson Lobster, whose very brain structure adjusts based on ones place in the dominance hierarchy. 

The problem we have here is the usual almost ridiculous objection, what if someone enjoys being a slave? What if a slave is happy being a slave? He has gotten used to the idea. Ester 1982. The question is, does the concept of slave actually have any meaningful content if we control for consent? No of course not. Tge bituib if slave entails the notion of lack of consent. If somebody lives on a farm and wants to listen to their masters orders, there is no meaningful way this is a slave. 


Notice that since nothing is good in itself but only things that satisfy preferences, there is no way around it. This problem, indeed 



Or what if some preferences are harmful? Kymlicha 2002

Of course they reply that we should ONLY apply utilitarianism to things which are universally desired(Goodin 1995)

Sidwick(1907) Utilitarian elite Rawls 1980. 

Dworkin 1931 and 1977, Personal preferences and external preferences, Harsayani (1976)


\subsection{On measurability and epistemic arrogance}
Imagine that we have the utilitarian framework. What next? How this is wholly insufficient, how do we measure these utils? Indeed, there is no sense in talking about optimizing when the proposed measure is not observable. 

Epistemic arrogance is baked into utilitarianism. There is the assumption that reasoning by adding and subtracting can work precisely because humans have all the knowledge that is needed to do this calculation. Indeed if one is deciding what action to partake in because of the information that is avaialble to him implicitely this implies the information one has is sufficient to change behavior. 

Utilitarians pretend like this is a completely harmless assumption but in fact these point of view is denial of the founders of our civilization. Socrates and Jesus. Imagine that the default strategy is X, but some information cuases us to believe that the best strategy is Y. If this was a mistake, cultural evolution would attempt to correct against this arrogance, I argue that it in fact DOES fight back against this arrogrance. The adam and Eve story seems to emerge naturall from exactly this kind of ridiculous history. 

Before assigning probabilities to events there should first be a clear appraisal of the kinds of events that can occur. Indeed, it is virtually impossible to take a calculated risk if one does not go all the way in evaluating the higher order effects of every change possible. 

Indeed there exists a certain kind of arrogance to believe that the good can be infered empirically. There are a multitude of ways to frame ethics alternatively. Indeed one can argue that they KNOW what the good is before any empirical foray. 

Change is inevitable, it occurs without any top down imposition, merely by people exploring what is around them, getting to know others, travelling etc. No old structure can survive without significant resliency built into it. Change often becomes neccesary for adaptation, the vegetarian option is obviously not one of those neccesary options. What I find stupefying is how much of the absolutely preposterous literature of the sort "if we don't go green we won't be sustainable"  literature actually comes out. Even the diet literature seems corrupted beyond repair due to the vegetarians absolutely insisting on comparing correlations. Funnily enough the same vegeterians who accept these correlations, deny them for other measures such as IQ. 

Though arguments about measurability may not neccesarily be fatal to utilitarian calculuation, they do in fact break its practical application. Indeed if we posit that the dimensions which increase utility the most cannot be measured, then somebody who is trying to maximize utility would in fact refrain from doing utilitarian calculation. 

Many utilitarian may reply to this that they can simply be utilitarian by adopting "rules". While rules can be useful for choices which are sometimes measurable or reversible, in practice these things cannot be done in this way. That is, perhaps what gives utility simply cannot be measured in any meaningful way. Or we cannot predict which actions or rules will cause utility. 

%articulate

There are two kinds of things we are interested in measuring. The things that give us utility themselves, and the things that allow us to make decisions that that cause utility. For instance if we imagine that going to the Trodos mountains gives us utility, though maybe we can't recognize the trodos mountains themselves, perhaps we can recognize the signs that point to the mountaints. 

Three things could cause an issue here. First we could not know that going to the mountains causes us utility. Second we could not recognize that we are in the mountains. Third, we could not recognize the signs that lead us to the mountains. 

Notice that if any one of these things is true utilitarianism cannot be useful as a guide to ethical decision making. Of course the type of ignorance we have can take on a number of forms. 

Rule utilitarians in fact think that we can measure the criteria we need to make the right choices even if the choices don't lead to a utilitarian maximizing outcome every time. But even these rules are meaningless, this is a more general problem with consequentialism, the problem of cluelessness.

Suppose you are in a small german town in 100ad. You run into a woman and you decide not to kill her. In fact, killing this woman would have prevented Hitler from being born. So what kind of rule can we truly draw out of this? Can we learn the rule "always kill women from german villages in situation X?" In fact the information we would need to create the rule is even more demanding the information we need to make the individual decision, this is for the simple reason that to take action we need to know the effect of the action in this instance, while to create the rule we need to know all the possible effects that action could have in all possible situations. 

This is the problem of cluelessness\cite{Lenman2000} and it is a more general problem for all consequentialist philosophies. If the action isn't defined as good in itself, or perhaps the motivation that leads to action as being good, then there is no way one can get past this problem of uncertainty. 


\subsection{Network vision of utilitarianism}

There is this implied argument in the short story "The ones who walk away from Omelas", which describes a society which is thriving by torturing a little girl. Though the argument can work, a utilitarian can just tweak with the cardinality of the values to make utility at lower levels much higher. That is, aggravate the diminishing marginal utility rule. 

However a similar story can get us a conclusion with very similar characteristics. For instance suppose that one of the utility dimensions is the health of those you love. That is, if person A is in good health then that increases the utility of his whole family. But now suppose that we have to distribute a unit of pain in an economy where 100 people are a family and their utilities are interelated, as opposed to giving a unit of pain to an orphan who has no family. Here utility theory is unequivocal, the unit of pain must be distributed to the orphan. 

Though utilitarians almost never articulate their philosophy in this way they are perhaps aware of this shortcoming. Indeed many utilitarians take positions against family ties and inheritance for exactly this reason. Indeed theere are numerous other things which occur. 
 

% In reality if we imagine that society is a network of individuals who have different links to each other, utlitarianism defines the agents as the nodes and assigns value only to the individual agents. But perhaps this misses another kind of value, value outside 

\subsection{Scleable vs non scaleable}

Utilitarians in trying to make a theory with the additive property have tried to make a theory that is universalizeable at all scales. Whether the decision maker is a policy maker whose choices affect millions, or whether the decision maker is an old lady, utilitarianism has a prescription for both of them. Of course since utilitarianism looks at the total utility, whenever one tries to optimize locally, they will neccesarily not get a result that is as good as optimizing at the global level. 

In other words, utilitarianism has baked into it a centralizing tendency, the central planner with perfect information can achieve everything in the best way possible. This scale invariance or at the very least efficiency increasing as a function of scale has some properties which most people would find repelling. 

Arguments of the scale type are about how utilitarianism ignores the nexus of decision making. Indeed, being a consequentialist theory, as long as the results are the same it doesn't matter how the results are arrived at. 

Utilitarianism because of its universal nature, can never work, any flaw found in utilitarianism would hold at all scales. 
On the other hand other moral codes, though not scaleable, can work just fine in smaller societies. 

\section{Incest}

"Julie and Mark are brother and sister. They are traveling together in France on summer vacation from college. One night, they are staying alone in a cabin near the beach. They decide that it would be interesting and fun if they tried making love. At the very least, it would be a new experience for each of them. Julie was already taking birth control pills, but Mark uses a condom, too, just to be safe. They both enjoy making love, but they decide never to do it again. They keep that night as a special secret, which makes them feel even closer to each other. What do you think about that? Was it okay for them to make love?"

\section{After birth abortion}

The killing of baby feautus. It seems clear to me that if there is no moral rule but simply a legal rule and that people have no problem obeying such a rule this is a recipee for the kind of people 

https://jme.bmj.com/content/39/5/261

\section{Consent}

Utilitarianism, being a consequentialist theory has no particular interest in consent. However a utilitarian would deny that creating a policy that forces everybody to give their kidney should the need arise optimizes welfare, they might invoke psychological pain. Nevetheless they have no objection to a scenario of the following kind:

The CIA randomly sneaks into people's homes at night and steals their kidneys and then uses them to save one of the thousands of people on the waiting list. In other words, the people whose kidney is being stolen are not aware that their kidney was stolen. 

Perhaps a more extreme example, suppose one wants to have sex with a certain woman but she is not interested, everyday she sleeps from midnight to 8am. If someone were for example to use chlorofoam to knock her out while she slept and then silently rape her. It is not clear what the disutility that would occur is here. 

Indeed Singer himself uses the following example to try to convince people that their attachment to non-utilitarianism is purely irrational: 

Only a rights based approach can object to such a scenario, the rights based approach will simply separate the world into physical objects where people have rights and obligations with respect to those objects. Ones kidney is someones property and they have the right to choose how to allocate it. 


\section{Information}

How does this all work when there is differential information? Suppose that there are two agents, A and B. A believes the best way to increase utility is by doing action X, whilst B believes it is by doing action Y. How does utility theory overcome this diffential information? Perhaps it refines the set of actions we take to the set of actions we all agree increase utility. This is a particularly odd position, clearly if we include the whole world this set will be very small. 

How can they get around this objection? Perhaps they will want to apply their utilitarianism by making sure to group people who agree. 

With this we could potentially have a theory of property rights

\section{Capacity}


The problem with any system that relies on optimization is that it assumes that what is known is more important than what isn't. For instance a utilitarian calculus might get you that building lots of infrastructure will be good for increasing consumption utility. But as we now know, at the time when these things were being invested we did not know that an increase in infrastructure would cause the environmental damage that it does, through, the destruction of habitat and by encouraging people to travel more. 

A utilitarian would optimize 

UNcertainty arguments against utilitarianism, what if measures? 


% If we give this unit to 

% One effect of giving the unit of pain to Amy is that this pain will also cause another kind of pain. Perhaps Amy is more resistant to physical pain than Bob but if she is inflicted with physical pain, she has nightmares about($x_1$) it for 5 years, while Bob only has nightmares for 1 year. 

% Another possible effect of inflicting the unit of pain to Amy is that it affects her positive utility dimensions more adversely than Bobs. Suppose the unit of pain implies a lower physical capacity. Suppose that Amy's positive consumption good it being an athlete, but Bobs consumption good is being a writer. In this case if we give the unit of pain to Bob, we will have a greater total utility because even though he feels the pain more, his positive aspiration is unaffected, while if we had given it to Amy, her positive aspiration WOULD be affected. 

% \begin{align*}
% Amy = U(\{ x_1, x_2, y_1 \}) \\
% Bob = U(\{ x_1, x_2, y_2 \})
% \end{align*}

% Note that this kind of reasoning works with any asset, we simply take the positive and negative it would bring to each person in total and then give it to them with an eye on the maximum. The equal weight consideration we are meant to be giving to animals is exactly this same argument. The non-human can be argued to simply have less dimensions than the human, for instance, perhaps inflicting physical pain will not affect the animals aspirations and beliefs. But for the aspects that ARE comparable, for instance the ability to feel physical pain, should be given equal weight, if the animal feels more or less pain than a human is a matter of exact calculation. 

%An example:

So suppose that person A has some disease which can't make him feel direct physical Pain, and animal B has that capacity. Those kinds of pains should be given equal weight in the calculus. Of course, Singer ackowledges that the pains that a human can have are deeper and of different kinds. Perhaps there a pain of type C that humans have and animals don't have, this kind of pain will obviously give moral TOTAL weight to the human, but per unit of pain felt, both being would be equal. 

Suppose for instance that we have 10 units of pain to distribute. How should we distribute these units? Since utilitarianism aims at minimizing the impact of these units of pain there are a couple of solutions. Suppose the human and animal will both feel these units of pain equally and at a constant rate, then utilitarianism is in fact indifferent to how we distribute these units of pain. 

Now suppose that units of pain are complementary, that is, two units of pain cause more damage than two times the units of pain absorbed, in this scenario, the prescription is that the units of pain should be spread (5,5) between the human and the animal. 

Now suppose that the 10 units of pain are substitutable, that is, if absorbing an extra unit of pain after already having absorbed some pain is smaller, then the philosophy says that we should give it all to one of them, the philosophy being indifferent to which one of the two. 

Of course the philosophy implicitely treats all resources in this way, units of pain being like labor, which gives disutility, and the question is what is the optimal usage of labor. We can imagine that pain is the cost of all activities, and pleasure is the gain. 

Utilitarian has the advantage of getting over the basic flaw in the formulation of neccesity. Whether we will have 10 units of pain or 20 units of pain depends on the total utility of the pain. 

\section{Measurability and additivty}

Perhaps the main postulate of utilitarianism is that there is this concept, utility which can MEASURE everything that matters. This is a claim in and of itself. Indeed, it seems very odd to someone intuitively that he would be able to measure such a thing as suffering and pain. Indeed, a person may give an ad hoc answer to "is it worth it" but this is hardly proof of the measurability of things. I missed my childs first word but I was there for her first walk... which is better? It seems ridiculous as postule, hard to imagine that a whole class of philosophers have based. 

Perhaps more importantly, the assumption that whatever it is these measures are, THEY ARE ALL comparable by one grand MEASURE. As absurd as the above sounds, this is a whole OTHER level of absurd. It is a GOD of sorts, the ultimate measure. 

\section{Mother child and human life}

Suppose that in the mountaints some mother has a child that dies very young. When asked, the mother still says she prefers that the child be born than not. 

\section{The role of agency}

Suppose that in the mountaints some mother has a child that dies very young. When asked, the mother still says she prefers that the child be born than not. 




%Is killing a chicken worse than killing a person who does not feel pain
%Is preventing a cat from killing rats bad?
%Nozick experience machine




Though the concept of utility sounds rather abstract, essentially the founders of the doctrine meant simply, to mazimize happiness defined as the sum of pleasure and pain. Though the doctrine as initially thought up does not neccesarily imply vegeterianism, when combined with other intuitive ideas it quickly becomes evident why vegetarians take this route. 

Utilitarianism is a philosophy which isn't really too bugged about the details of what the good is. It merely states whatever that ultimate good is, actions should maximize it. It is designed in such a way that was occured in the past cannot be a reason to do something in the future.  To a utilitarian the family interaction is about pleasure, if family A switched children with family B and total utility was increased this would be a good change. 

The question, is what if one persons happiness comes at the expensve of anothers? Utilitarianism has an answer, whatever action maximizes the \textit{total} happiness should be taken. A funny thing about utilitarianism is that stated brutally, it is an ethical system which is indifferent to the distribution. Which is why, to make it adhere to their intuitions, most philosophers complement it with a second rule. 

Most adherents of this also make a second assumption, "the law of diminishing marginal utility", that is, the enjoyment one person gets for every marginal unit is diminishing. Or the second banana gives me less happiness than the second banana. \footnote{This assumption is questioned in Frankfurts book, equality}. This framework is especially appealing to some left leaning authors, it allows them to justify animal right and income/wealth re-distribution with a single framework. 

How do these two premises, maximize utility, diminishing marginal utility help the vegetarian make his case? The argument is simple enough, animals have a higher utility from livng than humans have from eating animals. This is of course, simply an assertion, it is difficult to counter-argue such a position because it burries all the complexities behind the concept of "total utility". 

The most natural question to ask is, "whose utility?"?, this is an odd. Who do we include? Naturally the vegetarian will ask that we include humans, mammals, and perhaps non-mammalian species, plants, or even the planet? A common idea for who to include is to simply only include those who can feel pain. It is unclear what kind of analysis a utilitarianism will accept for concluding his theory is false or absurd or counter-intuitive. If Andreas cannot feel pain perhaps through some biological disorder, it seems the utilitarian would exclude him, or at least were we to choose if we should whip Andreas or a dog, the utilitarian would say we should whip the Andreas.  They may re-work their criteria for inclusion by making up some other criterion and it is always possible to do this. 

Another question to ask is, how much weight should we put on these utilities? Is the happiness of a fish the same as the happiness of a human? For instance, suppose we have one drug to distribute, if there are two agents who are sick, but one has built up a stoic character such that they feel less pain, does this entail that we ought to give the drug to the agent who feels pain less? Utilitarianism is clear about this, we give it to the person who feels pain more. There is this reverse natural selection at work implicit in utilitarianism. There is also a special of this kind of weighting in utilitarianism that is particularly popular, the Rawlsian case, where we take into account only the utility of the worse off person. 

Mill on the other hand revises Bentham with a distinction about lower and higher pleasures. Specifically he says one would rather be socrates unsatisfied over a fool who is satisfied. Now this kind of caveat, either just means that higher pleasures have more weight OR that there is a lexicographic priority, where higher pleasures win. The question is of course, is the appreciation of good meat, a higher or a lower pleasure? He is very clear, find a being who has experience of both pleasures 

Quote "If I am asked, what I mean by difference of quality in pleasures, or what
makes one pleasure more valuable than another, merely as a pleasure, except its
being greater in amount, there is but one possible answer. Of two pleasures, if
there be one to which all or almost all who have experience of both give a
decided preference, irrespective of any feeling of moral obligation to prefer it,
that is the more desirable pleasure. If one of the two is, by those who are
competently acquainted with both, placed so far above the other that they prefer
it, even though knowing it to be attended with a greater amount of discontent,
and would not resign it for any quantity of the other pleasure which their nature
is capable of, we are justified in ascribing to the preferred enjoyment a
superiority in quality, so far out-weighing quantity as to render it, in comparison,
of small account...
From this verdict of the only competent judges, I apprehend there can be no
appeal. On a question which is the best worth having of two pleasures, or which
of two modes of existence is the most grateful to the feelings, apart from its
moral attributes and from its consequences, the judgment of those who are
qualified by knowledge of both, or, if they differ, that of the majority among
them, must be admitted as final. And there needs be the less hesitation to accept
this judgment respecting the quality of pleasures, since there is no other tribunal
to be referred to even on the question of quantity. What means are there of
determining which is the acutest of two pains, or the intensest of two pleasurable
sensations, except the general suffrage of those who are familiar with both?
Neither pains nor pleasures are homogeneous, and pain is always heterogeneous
with pleasure. What is there to decide whether a particular pleasure is worth
purchasing at the cost of a particular pain, except the feelings and judgment of
the experienced? When, therefore, those feelings and judgment declare the
pleasures derived from the higher faculties to be preferable in kind, apart from
the question of intensity, to those of which the animal nature, disjoined from the
higher faculties, is susceptible, they are entitled on this subject to the same
regard"

What IS perhaps an odd thing is that it is clear that when it comes to taste, this can be considered a higher pleasure of humans. Indeed we need only note that like every taste, eating can be developed as a taste. For instance would these people be open to saying that music is a higher taste to be cultivated. 



\section{Utility monster}

%Is killing a chicken worse than killing a person who does not feel pain

The utility monster is an individual who has a lot of pleasure. In fact the individual has so much pleasure that maximizing utility entails giving this person all the goods. THis seems like a weak objection but how can they deny the existance of such an individual? 


\section{Experience machine}

There is in principle nothing in Utilitarianism which can prevent itself to the experience machine objection. 

\section{Robot replacement?}


A peculiar feature of utilitarianism is that it anonymizes the decision maker. It says that as long as the consequences of an action are identical the identity of the agent don't matter. This is not to say that it prescribes that all agents should do the same thing, but as long as we condition on the agents preferences and information, their actions should be identical. 

Suppose that every time a moral choice is to be made, the all knowing BOT is given the lever and it decides what the right thing to do is based on utilitarian reasoning. Would this work? 

Indeed the problem is that the correct action is person specific. Suppose that a mother is in the hospital and it is burning down, suppose that in the baby ward, there are 10 other babies, the mother can only save one, should she save hers or some other baby(their mothers are all outside). Indeed the robot would simply randomize, or perhaps get the healthiest baby out of there. 

This isn't really that far fetched, in fact there is evidence that brain damage causes utilitarinism. 

\section{Utilitarianism and existance}

Even Rawls suffers from the same problem as the labor theory of value, that is, why should the labor animals not contribute? Or indeed why should the veil of ignorance not apply to animals? Behind the the veil we could be born as a tiger or a human. 

Parfit repugnant conclusion

These two questions, "whose" and "how much", combine to give another problem, the problem of existence. How should we value a being who we can make exist? This is more important than it seems, we could for instance argue that our future selves do not yet exist, let's overlook this detail and pretend that it poses no problem. Let's ask a related question, how should we weigh future generations utility? Economists often discount future utility using a discount rate, but it is unclear how to weigh beings that do not yet exist. A known result of Nordhaus's economic climate model is that it is impossible to justify even moderate measures for mitigation if we do not give close to equal weight to the future generation. Note that even in these models the assumption is that the existence of a being does not depend on our actions. In other words, utilitarianism cannot answer questions about whether to bring a person into existence or not.

The future existence of being is one blind spot for utilitarianism, but so is the past existence of a being. The framework given, gives zero weight to past generations. For instance if a shrine wishes for his son to inherit a shrine and take care of it in solitude, the utilitarian would simply weigh the utility of a single person using it against the utility of turning into a touristic spot. The will of a the dead is only to be given weight as far it increases the weight of the current generation. 

This could be interpreted as "time discrimination". It seems odd that at time t, we give full weight to the utility of an agent, and and time t+1, the agents choices simply don't matter. Indeed it is unclear what the discount rate should be. 

If we imagine that choices are reversible, and each person existing at a time can simply switch the button on or off, and there is no then there is no need for present agents 


\section{Utilitarianism and and trolley}

% https://jemh.ca/issues/v2n1/documents/JEMH_V2N1_Article1_UtilitarianismAsAnEthicalTheory.pdf

% FOOT and 1967, ALSO in short story about Omelas

Double effect, Aquinas (Cavanagh, 1997)

In medical ethics, this issue has been discussed primarily in
terms of the intentions of the moral agent, and the proportionality of the harm in relation to the good (Boyle, 1991).

to this principle, rather than
tolerating completely impersonal considerations of the positive
and negative effects of actions (Nagel, 1986).

\subsection{Utilitarianism more closely}

Is it true that it results in a conclusion that we OUGHT not to eat animals?  


\subsection{Utilitarianism and integrity}
% https://ocw.mit.edu/courses/linguistics-and-philosophy/24-231-ethics-fall-2009/lecture-notes/MIT24_231F09_lec14.pdf

(1) George, who has just taken his Ph.D. in chemistry, finds it extremely difficult
to get a job. He is not very robust in health, which cuts down the number of jobs
he might be able to do satisfactorily. His wife has to go out to work to keep
them, which itself causes a great deal of strain, since they have small children
and there are severe problems about looking after them. The results of this,
especially on the children, are damaging. An older chemist, who knows about
this situation, says that he can get George a decently paid job in a certain
laboratory, which pursues research into chemical and biological warfare. George
says that he cannot accept this, since he is opposed to chemical and biological
warfare. The older man replies that he is not too keen on it himself, come to that,
but after all George’s refusal is not going to make the job or the laboratory go away;
what is more, he happens to know that if George refuses the job, it will certainly
go to a contemporary of George’s who is not inhibited by any such scruples and
is likely if appointed to push along the research with greater zeal than George
would. Indeed, it is not merely concern for George and his family, but (to speak
frankly and in confidence) some alarm about this other man’s excess of zeal,
which has led the older man to offer to use his influence to get George the job…
George’s wife, to whom he is deeply attached, has views (the details of which
need not concern us) from which it follows that at least there is nothing
particularly wrong with research into CBW. What should he do?

(2) Jim finds himself in the central square of a small South American town.
Tied up against the wall are a row of twenty Indians, most terrified, a few
defiant, in front of them several armed men in uniform. A heavy man in a sweatstained khaki shirt turns out to be the captain in charge and, after a good deal of
questioning of Jim which establishes that he got there by accident while on a
botanical expedition, explains that the Indians are a random group of the
inhabitants who, after recent acts of protest against the government, are just about
to be killed to remind other possible protestors of the advantages of not
protesting. However, since Jim is an honoured visitor from another land, the
captain is happy to offer him a guest’s privilege of killing one of the Indians
himself. If Jim accepts, then as a special mark of the occasion, the other Indians
will be let off. Of course, if Jim refuses, then there is no special occasion, and
Pedro here will do what he was about to do when Jim arrived, and kill them all. Jim,
with some desperate recollection of schoolboy fiction, wonders whether if he got
hold of a gun, he could hold the captain, Pedro and the rest of the soldiers to
threat, but it is quite clear from the set-up that nothing of that kind is going to work:
any attempt at that sort of thing will mean that all the Indians will be killed, and
A CRITIQUE OF UTILITARIANISM 93
himself. The men against the wall, and the other villagers, understand the
situation, and are obviously begging him to accept. What should he do?

Utilitarianism as a doctrine is known to be false because it cannot take into account good character. Bernard Williams in a "critique of consequentialism" gives us the general formulation of the counter-argument. If doing action X gives us consequence A, and action Y gives us consequence B, and we prefer A>B, then we ought to do X. But consider now being a guard in a North Korean labor camp, you are asked to kill somebody, since you want to be compassionate, you will aim to kill with the least amount of suffering possible. If you don't do it you know Greg loves killing, he even likes to torture them a little bit. There is nothing in consequentialism that tells you that you should not kill them. 

Standard FAT man and the trolley problem. 

Now suppose that if you don't push the fat guy, some other guy will push him and the delay from pushing the fat guy will result in only saving 4 people instead of five. 


% \begin{table}[]
% \begin{tabular}{ll|l|l|l|}
% \cline{3-5}
%                                                     &                          & \multicolumn{3}{c|}{\textbf{Andy}}                                     \\ \cline{3-5} 
% \multicolumn{1}{c}{}                                &                          & \textit{Shoot(headshot)}   & \multicolumn{2}{l|}{\textit{Don't shoot}} \\ \hline
% \multicolumn{1}{|l|}{\multirow{2}{*}{\textbf{Bob}}} & \textit{Shoot(gut shot)} & X                          & \multicolumn{2}{l|}{Y}                    \\ \cline{2-5} 
% \multicolumn{1}{|l|}{}                              & \textit{Don't Shoot}     & Z                          & \multicolumn{2}{l|}{W}                    \\ \hline
% \end{tabular}
% \end{table}
\subsection{Finishing comments}

It is interesting to note how stale utilitarianism truly has been, indeed most philosophers, in the last 200 years have been utilitarians. One must wonder how it is that in an atmosphere where the dominant norm is utitarian, factory farming can rise to such an extent, and then the utitarians can turn around and say that animals are incompatible with utitarianism. Indeed it is an inconvenient truth to utilitarians that slavery was not banned by the brandishing of their own ideology but by steadfast application of the christian doctrine. Neverthless they insist that they should take credit for it, indeed utilitarianism is so weak and unituitive that we cannot blame a large portion of the early 20th century intellectuals of having embraced eugenics. There is nothing in principle in Eugenics which would make the utilitarian opposed, indeed, it is simply the use of Eugenics by the third Reich that has reduced its popularity, I would expect that as the memory of the second world war fades, the popularity will rise once again. 

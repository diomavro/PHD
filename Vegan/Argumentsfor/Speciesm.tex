\section{Speciesm}


\begin{tcolorbox}[enhanced,%fit to height=5cm,
  colback=green!25!black!10!white,colframe=green!75!black,title=Fit box (5cm),
  drop fuzzy shadow,watermark color=white,watermark text=Fit]
\begin{align*}
1)& \text{If X be the set of characteristics which ALL humans have} \\
2)& \text{At least one animal has a characteristic in X} \\
\rightarrow& \text{Therefore X cannot be used to demarcate between animals and humans}
\end{align*}
\end{tcolorbox}


This is an argument whose weight comes from trying to draw paralels between species and races. In other words, this argument is a bit of a trap. Those who are arguing are hoping that you will fall into the trap so that you will re-consider your views. The trap is made up of two elements: backward compatibility, and present bias. 

Backward compatitibility means that the counter-argument the meat-eater must conjure up in this case must not be useable in the case of slavery. For instance many arguments that were used, had to do with notions of "nature" or notions of "intelligence".

The present bias is simply that any criterion that is used to construct the demarcation between okay to eat and not okay to eat must not be found to be ridiculous later on. That is, they will accuse us of being ideological and just making arguments to justify our ideology. 

In trying to defend against this charge we must also take into account the disney factor. It seems almost obvious but the caricatural perspective of most people in developed countries, has an intuitive reaction that if the animals in the disney movie were real, it would not be acceptable to kill them. This seems to me to be a vision shared by meat eaters and non meat eaters alike. That is, if it were true that an ape of the kind found in Tarzan, that can exhibit moral agency then we would obviously extend our moral code to them. Similarly, if a Superman(an alien from another planet) came on the planet, and he exhibited all the same behavioral pattern that humans have, then we would also extend our morality to him. 

So it seems clear that the argument, fails, or at least if it did not fail, the argument would apply to specific animals and expand our moral circle. 

\section{Is it moral to kill your dog?}

I'm not really sure how to frame their argument here. 


Here we have a vision of killing an animal in front of its owner. This seems to be the peak vision which frames our understanding. It is intuitive that killing an animal and causing suffering to its owner is prima facie worse than 


That is, extra caution has to be taken

The arbitrary nature of speciesm, but false since superman would have our moral compas, as well as a moral ape or animal. 

\section{Cost-benefit}

\subsection{calculus}

Another argument vegetarian may make has more intuitive appeal:
If the costs exceed the benefits, it is bad
The suffering from animals exceeds the benefits.
Therefore you ought not to eat animals. 

This may seem like a striking argument, the obvious question to ask is "how do you know?" More specifically:

How come 1) we can measure the benefit and the harm? What exactly makes these categories measurable? It is perhaps intuitive that every human can measure their own suffering, it is less clear that they can measure their own happiness or joy. But even if they could, this is different than actually measuring the happiness of another. 

2) the measures we come up with are comparable? Did we assert that these are of the same kind? How do we know it isn't like comparing temperature to distance? Even if we suppose that the two are measurable, how come they also also comparable?  

3) The benefit is lower than the harm? How come the benefit is lower than the harm? It is obvious here that the vegetarian must attempt to define what the benefit is. 

\section{Equal consideration}

\section{Ecological:Plants}

Vegetarians will often use empirical arguments to attack this kind of reasoning, that is, the sheet magnitude of plants needed to make an animal live. 


\chapter{The good arguments for eating animals}

\section{The extreme case: What if you have to?}

\section{Function and teleology}

The ethics of teleology are understudied and for this reason many people are not open to such dimensions. Begin with an ordinary claim:

I want to have kids so they can take care of me when I am older.
I want want to have kids because it reduces my tax burden
I want to have kids becaue it makes me happy. 

Indeed if there is some reason you want to have kids OTHER than the kids themselves, and this reason is the deciding reason, then it is unethical to have kids. The function of the child must not be pre-defined. 

This makes sense of course. 

The argument is as follows: If
1) ALL human beings have a set of characteristics, X and
2) At least one animal has one characteristic that is also in X
Therefore X cannot be used to demarcate between animals and humans.
X here can be used to represent characteristic. For example, we may say that X is the “ability to feel pain”. In that case we can say that Dolphins also feel pain, therefore pain cannot be used as a criteria to discriminate between animals and humans. This can also work with two characteristics, say the set could be {ability to feel pain, swim}. 
It can also work in the reverse manner. Say that X is the ability to reason, one could try and claim that chimps may be able to reason but they cannot reason as well as humans. But then the proponent of the argument can argue,  “the best reasoning chimp can reason better than the worst reasoning human” and then claim that using X implies we should treat some chimps better than some humans. 
Of course many people will bite the bullet on this accusation. They will say,” well… if that chimp has it and that human does not, then that chimp should not be treated differently than humans”. An alternative method may be that they will select X in such a way as to exclude X chimps, so if the best chimp can reason at 5/10 and the best human at a 4/10, they will set the standard at 6/10. This leads to a somewhat absurd conclusion that which humans get to be treated as humans depends on the performance of animals. 
Let us give another example: Suppose that some woman is trying to decide who to choose as her husband. She could pick factors like kindness, clever, etc. I think those are less controversial, but suppose she chose “kinder than average”. In other words, the standard does not depend on the intrinsic characteristics of the person but on his ranking in the population. 
This may not appear to be common but quite a few standards would implicitly have ranking criteria. Of course one could argue that the criteria “has a yacht” isn’t inherently a ranking criterion, maybe there is just a strong preference for Yachts, but it seems to me that most people understand that Yacht is the new “car”, and standards evolving in the way they do, imply a ranking. 
Most people have an intuition that having a standard which is set FOR the purpose of discriminating is deeply unjust, especially to the humans who happen not to meet it. Some people want to claim that the reason we have this intuition is because we are against hierarchy of any form. Others want to argue that we seek equality and a hierarchical standard causes people on the lower end to be harmed. 
So if we can’t set the standard for that purpose what is the standard for? One plausible answer is that we have already decided how to discriminate without the standard, and the standard is merely an attempt at an articulation of why we have discriminated. 
The Disney factor
Going beyond the accusations that the argument makes, the argument itself is flimsy for the reason that it begs the question that it is a specific characteristic which demarcates between species. Just because dolphins have characteristic A and apes have characteristic B, humans can still be the only beings humans who have characteristic A and B. Of course if that isn’t enough, it probably is quite easy to come up with a characteristic that only humans have, for example contemplating God or contemplating morality or even just the ability to imagine or tell a story. In other words we can appeal to “personhood”, it just so happens that the only beings who are persons are humans.
We can see the intuitive appeal of personhood; we can call it the Disney factor. Many westerners grew up watching animations of different species that are designed to create empathy from the viewer. It is natural that fans of such animation would intuitively create a correspondence between those animals and animals in the real world. Where they are right is that IF the animals in Disney movie were real, it would not be acceptable to kill them. This seems to me to be a vision shared by meat eaters and non-meat eaters alike. That is, if it were true that an ape of the kind found in Tarzan, that can exhibit moral agency then we would obviously extend our moral code to them. Similarly, if a Superman (an alien from another planet) came on the planet, and he exhibited all the same contemplative capacities as humans have, then we would also extend our morality to him.
So it seems clear that the argument, fails, or at least if it did not fail, the argument would apply to specific non-human species (such as the apes in Tarzan or Superman) and expand our moral circle.










The arrogance of the philosopher. 
This is the general problem of philosophers, they seek to always articulate and if they cannot articulate they assume there is no reason. 
But this is absurd on its face, many people have various routines which they adopt and act upon and cannot articulate WHY it is that they have those routines. Perhaps they used to know but forgotten, perhaps they learned it from someone else. Perhaps nobody ever knew the reason but it just gradually emerged, as a habit in an evolutionary way. 


I think most people have an intuitive aversion to this kind of discrimination. In other words, selection should not be done by truncating a distribution. It should be done simply on objective criteria. Of course, there
Choosing a husband
Of course this impression we have that having a standard for the purpose of discriminating could be wrong. Let us try and get another example and see if our intuition holds. 
I think most people have an intuitive aversion to this kind of discrimination. This seems to be the portion where Christian Ethics that comes into play, “the meek will inherit the earth”. The Christian ethics, in other words, selection should not be done by truncating a distribution. It should be done simply on objective criteria. 
How does this work? Simply, we look at the function for which we are selecting and see what characteristics it requires. For instance a good husband may require loyalty, hard work, pedagogical, etc.  In practice this is sufficient to make the choice, but it the academic case where two people meet all those criteria equally; one can use their OWN feelings, including trust to make the choice. 
To clarify, the inputs that will be used to make the choice must be found exclusively in the objects of choice and the person making the decision. 
One may object, “What about standardized tests? They are designed so as to truncate”. I think that here the answer is that clearly standardized tests fail the “meek will inherit the earth” test. Indeed I would say that standardized tests are unethical. If a specific task can be equally achieved by two people and the one making the choice is indifferent between them, it does seem rather arbitrary that we select based on a test which arbitrarily demarcates. 
So it seems our intuition applies in this case too, but perhaps it is a distinctly Christian ethic. I am not specialized enough to know of the roots of a specific ethical code. We could try to come up with arguments for this, but it seems unnecessary, this simply IS the ethic we have inherited. 
Do we need to consciously?
Of course the idea that we NEED to have a set X is misguided. Perhaps even having the set X causes discrimination to be intuitively unjust. Most people would not have trouble telling a human apart from an animal intuitively. This clearly means that there ARE standards by which we could perfectly discriminate. So it seems odd that the criterion for discrimination has to be articulated before we discriminate. Why is there a need to articulate the standard when everybody already knows how to discriminate? 
The Disney factor
Going beyond the accusations that the argument makes, the argument itself is flimsy for the reason that it begs the question that it is a specific characteristic which demarcates between species. Just because dolphins have characteristic A and apes have characteristic B, humans can still be the only beings humans who have characteristic A and B. Of course if that isn’t enough, it probably is quite easy to come up with a characteristic that only humans have, for example contemplating God or contemplating morality or even just the ability to imagine or tell a story. In other words we can appeal to “personhood”, it just so happens that the only beings who are persons are humans.
We can see the intuitive appeal of personhood; we can call it the Disney factor. Many westerners grew up watching animations of different species that are designed to create empathy from the viewer. It is natural that fans of such animation would intuitively create a correspondence between those animals and animals in the real world. Where they are right is that IF the animals in Disney movie were real, it would not be acceptable to kill them. This seems to me to be a vision shared by meat eaters and non-meat eaters alike. That is, if it were true that an ape of the kind found in Tarzan, that can exhibit moral agency then we would obviously extend our moral code to them. Similarly, if a Superman (an alien from another planet) came on the planet, and he exhibited all the same contemplative capacities as humans have, then we would also extend our morality to him.
So it seems clear that the argument, fails, or at least if it did not fail, the argument would apply to specific non-human species (such as the apes in Tarzan or Superman) and expand our moral circle.

Why does this argument seem so strong?

The apparent strength of this argument is that it tries to draw parallel between modern animal right causes and slavery. By asking for criteria X, it hopes to homogenize treatment as far as possible. Of course the implicit argument being made is:
Things should be treated differently if it is justified to treat them differently. 
In that it wants the treatment of different creatures to be as homogenous as possible. This argument is a bit of a trap, those who use it are hoping that you will trigger it and that you will be compelled to re-consider your views.
To form the trap they want to claim that those who would give-arguments believe in discrimination and as such would also be the same people justifying slavery. If we stopped trying to find X, we could not justify slavery and it would have ended earlier. 
It cannot be doubted that there is an element of truth to this line of thinking. People who want to discriminate on a group basis infamously had (and have) an odd attachment to analyzing differences between the groups they want to discriminate. Measuring skulls was historically a popular way of doing this.

Overzealousness
However, activists are too ambitious in their inference from this argument. In fact, the argument is too general to be applied systematically. I think the most charitable interpretation is to say that “we should not look for differences for the purpose of discriminating”
But this is an odd statement. Does that mean it is okay to look for differences if it is NOT for the purpose of discriminating and we subsequently decide that what we discovered is worth discriminating over? 
Indeed what is the purpose of looking for differences if it isn’t so that it can affect our action? Perhaps somebody has allergies, and you want to know who they are so that you can serve them the right food. 
But what exactly is the purpose of discriminating? 

 It seems to be saying that that pointing to differences between groups for the purpose of discriminating is always bad. The problem is of course that they confuse causality. They assume that we look for differences in order to discriminate, when in reality, we discriminate and then look for differences. 
Even if this wasn’t the case, it is not true that looking for differences for the purpose of discriminating is bad. Indeed, it could be that I want to look at the difference pigs and hens because if I don’t and group them together, the pig may end up eating the hens. 
Activists may try to counter this point by saying “Only look for differences among groups if you have already decided to treat differently” but again, this begs the question, what if we don’t know if we should treat them differently UNTIL we find that characteristic. 
However, activists may claim that we should be trying to only draw differences among groups who we already aim to treat differently. This answer does not suffice as it begs the question of how to establish which members we are to treat differently without knowing what their differences are.
It must be highlighted that noticing differences does not necessarily imply superiority. In the same way that noticing that an animal that is higher up on the food chain does not make of it a superior animal. Indeed nobody tries to say that lions are superior to gazelles merely because they eat them. When one discusses the hierarchy in the food chain, the goal is not to point to a superiority of a specific species but to an interaction between species. Much like the fact that termites eat humans who are buried does not imply that termites are superior.
The trap is made up of two elements: backward compatibility, and present bias. We will start with these two implications and then tackle the argument head on.
Backward compatibility means that the counter-argument the meat-eater must conjure up in this case must not be useable in the case of slavery. For instance many arguments that were used had to do with notions of "nature" or notions of "intelligence".
The present bias is simply that any criterion that is used to construct the demarcation between “okay to eat” and “not okay to eat” must not be found to be ridiculous later on. That is, they will accuse us of being ideological and just making arguments to justify our ideology.

Backward compatibility
Aristotle is famous for saying that it is in the nature of some people to be slaves. Though many modern philosophers who cite Aristotle often feel compelled to object to this line of thinking, it can be said that in some manner he was right. It is in indeed in the nature of some people to be slaves. But as Aristotle himself pointed out, this nature isn’t exclusive to any one group of people every group has members with characteristics making them more prone to becoming slaves.
In Aristotle’s view, slavery was not a legal condition but a mindset, “he who cannot decide on his own, he who seeks orders”. It is obvious to me that this represents a large portion of the modern workforce today as people seldom aim for independence and creation. Many prefer to be slaves and have their free time to themselves. Even the cursory interest that does exist in participation in planning is usually for the ends of a higher consumption. 
Accordingly, it must be underlined that it is not possible to physically notice who is a slave or not. Being a slave is a behavior, it may even be a temporary behavior, but it is a behavior and such we can notice it when it takes place but cannot predict who will have this behavior.
To summarize backward compatibility is easy to meet. One can even be tolerant of slavery without endorsing the slave legal orders of the past.
Present bias
The rather intellectual error proponents of this argument make is to put forth arguments themselves on a pedestal. Just because it is that this argument was used by people at the time, does not mean that the argument itself is the cause of their behavior. Indeed people can find all sorts of reasons to justify their behavior. The behavior occurs first, then the theory.
Once we realize this, we can understand that the charge of present bias cannot possibly stick because arguments in of themselves are not the cause.
It is not true that acknowledging differences is the cause of injustice. Instead, injustice often causes an attachment to some differences, and usually when the causality is this way around, injustice seeks measurable (physical differences).


 

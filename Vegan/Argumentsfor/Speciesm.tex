\section{Speciesm}


\begin{tcolorbox}[enhanced,%fit to height=5cm,
  colback=green!25!black!10!white,colframe=green!75!black,title=Fit box (5cm),
  drop fuzzy shadow,watermark color=white,watermark text=Fit]
\begin{align*}
1)& \text{If X be the set of characteristics which ALL humans have} \\
2)& \text{At least one animal has a characteristic in X} \\
\rightarrow& \text{Therefore X cannot be used to demarcate between animals and humans}
\end{align*}
\end{tcolorbox}


This is an argument whose weight comes from trying to draw paralels between species and races. In other words, this argument is a bit of a trap. Those who are arguing are hoping that you will fall into the trap so that you will re-consider your views. The trap is made up of two elements: backward compatibility, and present bias. 

Backward compatitibility means that the counter-argument the meat-eater must conjure up in this case must not be useable in the case of slavery. For instance many arguments that were used, had to do with notions of "nature" or notions of "intelligence".

The present bias is simply that any criterion that is used to construct the demarcation between okay to eat and not okay to eat must not be found to be ridiculous later on. That is, they will accuse us of being ideological and just making arguments to justify our ideology. 

In trying to defend against this charge we must also take into account the disney factor. It seems almost obvious but the caricatural perspective of most people in developed countries, has an intuitive reaction that if the animals in the disney movie were real, it would not be acceptable to kill them. This seems to me to be a vision shared by meat eaters and non meat eaters alike. That is, if it were true that an ape of the kind found in Tarzan, that can exhibit moral agency then we would obviously extend our moral code to them. Similarly, if a Superman(an alien from another planet) came on the planet, and he exhibited all the same behavioral pattern that humans have, then we would also extend our morality to him. 

So it seems clear that the argument, fails, or at least if it did not fail, the argument would apply to specific animals and expand our moral circle. 

\section{Is it moral to kill your dog?}

I'm not really sure how to frame their argument here. 

\begin{mdframed}[style=MyFrame]
\begin{align*}
1)& \text{} \\
2)& \text{} \\
\rightarrow& \text{}
\end{align*}
\end{mdframed}

Here we have a vision of killing an animal in front of its owner. This seems to be the peak vision which frames our understanding. It is intuitive that killing an animal and causing suffering to its owner is prima facie worse than 


That is, extra caution has to be taken

The arbitrary nature of speciesm, but false since superman would have our moral compas, as well as a moral ape or animal. 

\section{Cost-benefit}

\subsection{calculus}

Another argument vegetarian may make has more intuitive appeal:
If the costs exceed the benefits, it is bad
The suffering from animals exceeds the benefits.
Therefore you ought not to eat animals. 

This may seem like a striking argument, the obvious question to ask is "how do you know?" More specifically:

How come 1) we can measure the benefit and the harm? What exactly makes these categories measurable? It is perhaps intuitive that every human can measure their own suffering, it is less clear that they can measure their own happiness or joy. But even if they could, this is different than actually measuring the happiness of another. 

2) the measures we come up with are comparable? Did we assert that these are of the same kind? How do we know it isn't like comparing temperature to distance? Even if we suppose that the two are measurable, how come they also also comparable?  

3) The benefit is lower than the harm? How come the benefit is lower than the harm? It is obvious here that the vegetarian must attempt to define what the benefit is. 

\section{Equal consideration}

\section{Ecological:Plants}

Vegetarians will often use empirical arguments to attack this kind of reasoning, that is, the sheet magnitude of plants needed to make an animal live. 


\chapter{The good arguments for eating animals}

\section{The extreme case: What if you have to?}

\section{Function and teleology}

The ethics of teleology are understudied and for this reason many people are not open to such dimensions. Begin with an ordinary claim:

I want to have kids so they can take care of me when I am older.
I want want to have kids because it reduces my tax burden
I want to have kids becaue it makes me happy. 

Indeed if there is some reason you want to have kids OTHER than the kids themselves, and this reason is the deciding reason, then it is unethical to have kids. The function of the child must not be pre-defined. 

This makes sense of course. 

This harm argument is a short chapter mostly because the view that I put forward here does not seem to be held by any philosophers but it is a common everyday rherhotical view.

\section{The argument}


\begin{tcolorbox}[enhanced,%fit to height=5cm,
  colback=green!25!black!10!white,colframe=green!75!black,title=Fit box (5cm),
  drop fuzzy shadow,watermark color=white,watermark text=Fit]
\begin{align*}
1)& \text{Causing unnecesary harm is bad} \\
2)& \text{Eating animals causes unnecesary harm} \\
\rightarrow& \text{Therefore eating animals is bad}
\end{align*}
\end{tcolorbox}


% \begin{tcolorbox}[enhanced,%fit to height=10cm,
%   colback=green!25!black!10!white,colframe=green!75!black,title=Principle of utility (10cm),
%   drop fuzzy shadow,watermark color=white,watermark text=Fit]
%  %\lipsum[1-4]
% “the only purpose for which power can be rightfully exercised over any member of a civi- lised community, against his will, is to prevent harm to others”
% ― John Stuart Mill, On Liberty
% \end{tcolorbox}


% Though Mill is perhaps the most known proponent of the harm principle, to Mill this was not a sufficient principle for morality but merely a heuristic for legal principles. 

Though the argument is simple enough, it is nevertheless worth clarifying some terminology. \textbf{Bad} is used in a strong sense of "we should not do what is bad". In other words bad is supposed to have moral force, if somebody accepts that something has this property they should be compelled to not take that action. Any one act may have numerous positive or negative effects but the argument can still have some meaning. The simplest way these effects can be aggregated is through additivity, if one has reason to believe these effects can be added up then one can claim the action is bad in total. 

A related example is if the act of helping causes harm that is unnecsary for the consequence one wants. For instance perhaps I want to push you out of the way so that a car won't run you over, but as a consequence you lose your hat. It is true that you losing your hat is not neccesary for being saved. Nevertheless we can judge a posteriori that it was neccesary for the action taken. 
%total bad

The argument can also hold if there are cases where the actions can be positive. There may be a circmustance where the action in question is good and some circmustances where the action is bad(notice that this can only occur in a consequentialist context). In this case our argument can be revised to be only made for the cases where it is bad. Alternatively the argument can again rely on a property of additivitiy to claim that in total the bad that is caused is greater than the good that is caused in the totality of cases. 

\textbf{Harm} is a better term than the alternatives because of its generality. The argument would also work if we used the words "pain" or "suffering". However the use of alternative words might exclude the concept of "killing". Using those words would then make the argument more open to objections via empirical methods. For instance one could just point to some painless way of killing and the argument would instantly fail. On the other hand, the harm version can survive such an attack. On the other hand if one attacks a pain formulation that would probably also apply to the harm formulation. This generalization also corresponds to what most vegetarians actually believe. If a very ethical farmer showed up that filmed the painless killing of the animal, it is doubtful that many vegetarians would change their mind and eat this specific animal. 

The problem of "harm" is that it may be general enough to encompass non-consious agents, such as killing a plant or tree. The argument as presented makes reference only to animals but somebody might object: "why only animals and not plants?". In this case, the vegetarian may return to the previous standards of "suffering" or "harm". Alternatively they could commit "organicism", that is, arbitrarility discriminate between organic beings based on their categorization. However I suspect the most likely position they will take is that it is not a matter of category but a matter of degree. That is, they will agree that harming plants is bad, but not sufficiently bad given the benefits. That is, one may think that the value of plants is high but the value of humans living is higher. I believe this kind of position automatically locks you into additivity.  

Note that this argument of harm 


% http://www.veganfuturenow.com/answering-the-objections-to-veganism#do-you-want-animals-to-have-the-right-to-get-married-and-vote

% \section{Where does the argument apply? }

% The harm principle as articulated by Mill was the main rherhorical tool used for legalizing homosexuality and is often used to argue for 


\section{Problems}

\subsection{Necesity}

The most obvious problem with the argument is the notion of neccesity. A slightly circular definition of the term is: X is neccesay if the presence of X is required for a certain other thing to occur. It makes little linguistic sense to talk of neccesity without a cause, neccesity is a constraint and there must be some objective for the contraint to work on. For instance if I want to make a cake it is necessary that I use the ingredients necessary to make the cake. The sentence "flour is neccesary to make the flour cake" makes sense. The sentence "flour is neccesary" does not make sense. So then it is clear that vegetarians are assuming that there is some goal(cake), which can be achieved through a variety of means. 

What is the "cake" of the harm done to animals? Suppose an agent is trying to get the best "taste" possible, the omega taste. If the omega taste does not require eating animals then the argument works, this would be equivalent to saying "don't eat animals because there are better tastes out there". If on the other hand the "omega taste" must include animal flesh, then the argument instantly fails. That is, if I am trying to have the best taste I can, then it IS neccesary that I eat animals. 

I suspect that the herbivores then have a rather different meaning. They are instead re-directing us to look for another cake, the question is of course, who are they to try and tell us what our goals should be? Perhaps they feel they have found a truth that we are ignorant to ackowledge. To me it seems like knowledge can either be a priori or aposteriori.

 The former can be classified as a reasoning knowledge, if they are in fact better reasoners than us, then perhaps this knowledge can be shared and it explained to us why we should change our functions. Perhaps here we might have a Kantian argument, but Kant famously excluded animals from his categorical imperative. 

If on the other hand they have better experience than us, then they need only try and help us experience those same things. In fact it seems rather the opposite, those with very little experience of animal life are usually vegan. 

We should not be trying to get this taste at all but then this would be a different argument entirely. 




Do people eat meat because of the taste? Though I suspect many people do consiously believe they eat meat because they enjoy the taste, evolutionary reasoning actually works backwards: They like the taste because they meat. In other words, the argument should not be taken at face value, people are comfortable eating meat but the reason they eat meat is not it's taste. For instance when the first humans started eating meat, it is likely they did not have the neccesary genetic make-up to enjoy it's taste, nevertheless those groups that did eat meat survived better and overtime developed the taste buds. 

So notice what is happening here, a group does something even if it goes against their taste buds, in other words, they put up with it, and overtime this either turns out to have been a good choice or not. This is in contrast to somebody merely satisfying their desires, indeed merely satisfying ones desires implies the lack of change. To repeat, if an organism attempts simply to satisfy it's existing desires and not to stick to things that go against its desires, by definition the organism can only evolve through one avenue, natural selection among desires. 

This kind of prescription seems to me to be animal like, advocating that humans should refuse non to evolve in a non-desire dimension just seems to be exactly anti-human. Humanity is about cultural evolution, it is about norms that overule our desire. From the evolutionary point of view this allows for quicker adaptation, but from a christian point of view this explains the culture of helping the needy. It refuses to allow for natural selection among desires but insists on cultural selection. 

This is a more general problem with utilitarianism, it insists that our preferences must be met, it denies the very human stuborness which is the cause of much of our adaptation. 

This is not to say that taste plays no role. Indeed once we have this evolutionary reasoning clear in our heads we can interpret taste as an attractor, that is, to make experimenting more likely to lead to a meat diet. As such taste can play an important role for habit formation, an invaluable role for new generations.

Suppose there exists an agent who only changes his behavior if he can find reasons to change behavior. For instance suppose that there are green frogs and red frogs, the red frogs are poison but not the green frogs. This might naturally create an instinct of disgust when seeing red frogs, which might partly carry over to green frogs. Depending on the environment the fine tuning of the disgust instinct may change, for instance if the only thing that can be eaten is green frogs then more evolution will take place to fine tune the instinct to discriminate, but if there are 

In other words, somebody who reasons about his ethics by definition can't learn about ethics empirically. They can only learn how best to achieve their existing ethics. 

Imagine that there is a brain that ignores culture and always aims to deduce what it should do. 

The evolutionary view put the weight on unconcious learning. Indeed an organism that did overrule it's instincts or who reasoned its behavior would not be able to evolve. 

 that concious learning is less important than unconcious learning. 

 In other words agent's are not optimizing creatures, they just have a set of habits, there is no sense in speaking of constraints. 

Is this obvious truth, that people have habits and don't analyze their actions and simply do things, a deathblow to philosophers trying to analyze them? A philosopher may be interested in two distinct things, trying to explain the behavior of the the agents, and trying to convince the agents to change their mind. 

If the goal of the philosopher is simply to explain the behavior, he is in fact indifferent to how the agent's decide, instead he is interested in studying their behavior and will simply try to re-formulate his theory to say that agents are acting AS IF they maximize their pleasure. This approach is most interesting for those who with a scientific inclination, it can be used to try and predict the behavior. 

If on the other hand the philosopher is interested in changing the agents mind, then he will try to stop the agent from doing thing unconsiously. This may provoke anger or dismissable from the agent, understandably, a bit like socrates who was known for being really annoying by trying to have people articulate everything. Of course this may be a more dangerous exercise than it seems because making an agent less reliant on one habit may make them doubt their other habits. Of course philosophers may think this is a desirable state of affairs to the desirable, but they are open to Chesterton's fence criticism. 

But let's play the philosophers game and assume for a moment that people are eating meat because they are maximizing some underlying variable. What is their goal if it is not taste? What else can be the optimand of people? Perhaps people are trying simply to optimize their pleasure, but that would simply result in a similar argument to the "taste" argument. Perhaps they are trying to lead a good life, in which case the vegetarian would have to appeal defining the "good life", something many vegetarians don't wish to do because it makes retaining a subjectivist position difficult. If they are willing to empbrace non-subjectivist positions then they would have to fall to objective standards. Nevertheless the most likely turn of vegetarians after reflecting is to use happiness as the standard. 

In other words, the vegetarians will simply try to use the utilitarian standard and attempt to convert meat-eaters to adopt this standard. 

% First: https://twitter.com/JoshHochschild/status/1246061518665506816?s=20

% Second: https://twitter.com/diomavro/status/1246497670757302272?s=20



% Third: https://twitter.com/diomavro/status/1246858198688116739?s=20

% https://twitter.com/diomavro/status/1246859165013889024?s=20
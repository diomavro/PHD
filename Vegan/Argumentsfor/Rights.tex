
\section{Right approach to animals} 
This view has been critisized by \cite{Regan2020}

The language of rights is very odd to apply to non-rational agents. This is because rights exist in relation to objects and persons\cite{Midgley1983}. To give rights is to ackowledge sovereignty. It is clear on the other hand that animals cannot have duties, indeed whether they will respect the right or not is purely a matter of luck. 

Nevertheless perhaps we could interpret rights to mean something else, perhaps that we could claim that a ROCK has a duty to me, in the sense that I have the right to physically remove it. 

What is very interesting about the rights approach is that the most philosophically developed non utilitarian theory of rights is the natural law approach. Unfortuntaly, most natural law theorists deny that animals should have rights. 

As such this is a purely political contest. 

\section{Psychoanalysis of veggies}

Many philosophers try to pretend like arguments are important, of course, this would be like phycists saying the material is most important, or a biologist saying organic, economist saying trade is important. 

Of course, while the latter categories may change their minds if presented with a good argument, the philosopher is unlikely to change his because his position is about arguments themselves. When should somebody change their mind? When they hear a good argument! That's their criterion, of course, the problem is that these philosophers are usually attracted by elegant views. However there is no argument about why reality would fits simple arguments better than complicated arguments(Huemer). Of course science can claim parsimony is important because predictive capacity plays a vital role. But philosophy aims at ethics, and there is no reason ethics needs this kind of parsimony. 

There is no reason some philosophical system which treats every situtuation differently is better or worse than one which has specific criterion which is evaluated universally. Indeed, if in situation A use criteria X, if in situation B use criterion Y, can simply work. 


%%%%%%%%%%%%%%%%%%%%%%%%%%%%%%%%%%%%%%%%%%%%%%%%%%%%%%%%%%%%%%%%%%%%%%%%%%
%%%%%%%%%%%%%%%%%%%%%%%%%%%%%%%%%%%%%%%%%%%%%%%%%%%%%%%%%%%%%%%%%%%%%%%%%%
%%%%%%%%%%%%%%%%%%%%%%%%%%%%%%%%%%%%%%%%%%%%%%%%%%%%%%%%%%%%%%%%%%%%%%%%%%
%%%%%%%%%%%%%%%%%%%%%%%%%%%%%%%%%%%%%%%%%%%%%%%%%%%%%%%%%%%%%%%%%%%%%%%%%%
%%%%%%%%%%%%%%%%%%%%%%%%%%%%%%%%%%%%%%%%%%%%%%%%%%%%%%%%%%%%%%%%%%%%%%%%%%
%%%%%%%%%%%%%%%%%%%%%%%%%%%%%%%%%%%%%%%%%%%%%%%%%%%%%%%%%%%%%%%%%%%%%%%%%%
The choice is self evident if the production method is less costly, but less so if it is more costly. 


\section{An alternative picture of humans}

It almost seems caricatural, to try and talk of people in this way, that is people don't have objectives they are trying to optimize. Instead they have goals they wish to achieve, this may seem like just a linguistic difference but from the analytical point of view it flips it all around. There is a list of things a person hopes and desires to have, is meat neccesary for the achievement of any of those goals? If it isn't neccesary maybe they only consume meat because it makes it "easier", to be more precise, perhaps eating meat allows them to meet more of their goals. Once again we are pulled into the empirical world, a massive can of worms is opened, perhaps there is a vegan rich person who will give you money and help you achieve more of your goals if you don't eat meat. 

% So with this new understanding of the word neccesary. We can go back to the animal example and ask, what is the "cake" of the suffering of animals? Typically, vegetarians will go after the "taste" argument.

% The taste argument is related to the empirical world in a peculiar way. If it is true that eating animals is the only way to produce that taste, then the argument is clearly false, because CLEARLY it is neccesary to cause harm to create the taste.

% If we CAN emulate the taste then that means we can create that taste without suffering. 

% http://www.veganfuturenow.com/answering-the-objections-to-veganism#do-you-want-animals-to-have-the-right-to-get-married-and-vote

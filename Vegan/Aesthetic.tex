\section{Dynamics of belief}

Though some vegetarians do believe their own arguments, I posit that to most of them, the arguments are secondary. Instead, on a purely descriptive level, they have some intuition which stems entirely from an aesthetic reactions. These intuitions then lead the agents to look for articulations of the arguments which cohere with their intuitions. 

In this respect they are no different to other people, a psychologist might call this "motivated reasoning". Indeed even non-vegetarians can have similar aesthetic reactions to vegetarians, for instance, factory like killing gives us a feeling of disgust, one I share. 

The difference is then about the inference that is done after one has this intuition. Vegetarians often make positive assertions, that animals should be treated a certain way. On the other hand, others with more humility, draw less general implications, they simply say that this situation is wrong. 

The problem with simply saying something is wrong is that it does not offer an alternative, but this is by design, they are inviting people to experiment with other worlds. Perhaps we are having this reaction because the animal isn't killed by somebody it doesn't know, and who in turn has no familiarity with the animal. This de-personalization causes revulsion in humans, in a similar way that many would be for un-plugging a man in a coma if the person deciding knows him but not if they do not. Alternatively, one could try to claim that the killing would be justified if the environment of slaughter was different, if there was some ritual showing respect, or if the animal wasn't away from its natural habitat. 

People who enjoy eating meat often have a fundamental intuition about a case where killing an animal is acceptable. The fundamental vision of a farmer raising an animal on his farm and killing it with his own hands and sharing that meat with his community, is still fundamentally sound. Of course, living by this vision probably implies a lower intake of meat than we currently consume. Such a vision is positive, almost platonic, does imply that activism should be directed to return the production process to the farm, get rid of the regulations that force farmers to take them to the slaughterhouse.\footnote{find sources on the EU here}. 

Of course one might object that there are reasons for regulating the slaughter of animals, disease, quality etc. These are indeed legitimate reasons but the the logic can be reversed, instead of regulating so that those things are better controlled, we should be structuring thing such that those things can't do much damage, a farmer's product being eaten by him and his own community is only the begining of such an accountability process. 

Aesthetic reasons dominate, this is not to say that people don't change their minds, but fundamentally they will only change them when presented with alternative aesthetic visions. However there exists a class of philosophers who formalize and think in objects and this class of philosophers is immune to evidence, this class of philosophers are immune to other kinds of arguments.
See this: https://twitter.com/JoshHochschild/status/1242527953101246467

Indeed this is often obvious nobody is informed of an animal being killed on a farm and suddenly becomes a vegetarian.

Much of the attempt in this chapter should read like a philosohical journey vegetarians go through. That is, many of them will stay on the first argument presented, others, will have started here and evolved in the same way I describe here. Many would have skipped this argument altogether and gone further down the chain. 
% !TEX TS-program = pdflatex
% !TEX encoding = UTF-8 Unicode

% This is a simple template for a LaTeX document using the "article" class.
% See "book", "report", "letter" for other types of document.

\documentclass[11pt]{article} % use larger type; default would be 10pt

\usepackage[utf8]{inputenc} % set input encoding (not needed with XeLaTeX)

%%% Examples of Article customizations
% These packages are optional, depending whether you want the features they provide.
% See the LaTeX Companion or other references for full information.

%%% PAGE DIMENSIONS
\usepackage{geometry} % to change the page dimensions
\geometry{hmargin=2cm,vmargin=0.7cm}
\geometry{a4paper} % or letterpaper (US) or a5paper or....
% \geometry{margin=2in} % for example, change the margins to 2 inches all round
% \geometry{landscape} % set up the page for landscape
%   read geometry.pdf for detailed page layout information

\usepackage{graphicx} % support the \includegraphics command and options

% \usepackage[parfill]{parskip} % Activate to begin paragraphs with an empty line rather than an indent

%%% PACKAGES
\usepackage{booktabs} % for much better looking tables
\usepackage{array} % for better arrays (eg matrices) in maths
\usepackage{paralist} % very flexible & customisable lists (eg. enumerate/itemize, etc.)
\usepackage{verbatim} % adds environment for commenting out blocks of text & for better verbatim
\usepackage{subfig} % make it possible to include more than one captioned figure/table in a single float
% These packages are all incorporated in the memoir class to one degree or another...
\usepackage{amsmath, amsfonts,amsthm, amssymb,mathrsfs}
%%% HEADERS & FOOTERS
\usepackage{fancyhdr} % This should be set AFTER setting up the page geometry
\pagestyle{fancy} % options: empty , plain , fancy
\renewcommand{\headrulewidth}{0pt} % customise the layout...
\lhead{}\chead{}\rhead{}
\lfoot{}\cfoot{\thepage}\rfoot{}

%%% SECTION TITLE APPEARANCE
\usepackage{sectsty}
\allsectionsfont{\sffamily\mdseries\upshape} % (See the fntguide.pdf for font help)
% (This matches ConTeXt defaults)

%%% ToC (table of contents) APPEARANCE
\usepackage[nottoc,notlof,notlot]{tocbibind} % Put the bibliography in the ToC
\usepackage[titles,subfigure]{tocloft} % Alter the style of the Table of Contents
\renewcommand{\cftsecfont}{\rmfamily\mdseries\upshape}
\renewcommand{\cftsecpagefont}{\rmfamily\mdseries\upshape} % No bold!

%%% END Article customizations

%%% The "real" document content comes below...

\title{Microeconomie 2 / Examen Final}
%\author{The Author}

\date{Mai 2016} % Activate to display a given date or no date (if empty),
         % otherwise the current date is printed 

\begin{document}

\maketitle


\section*{Exercise : Exchange and Production Economy }
It is advised to do the first part before the second part.\\

\subsection*{Part I {\small (4.5 points, difficulty **)}}
Assume an economy with two consumers $i = A, B$, and two goods $l = 1, 2$. The individual endowments of $A$ and $B$ are $\omega^A = \omega^B = (\frac{1}{2}, \frac{1}{2})$. Good 2 is the numeraire good (i.e. $p_2 = 1$). We note $p_1 = p$. The preferences of the consumer are represented by the utility functions :
\begin{equation*}
u^A(x_1^A, x_2^A) = ln(x_1^A) + ln(x_2^A) \quad  \quad
u^B(x_1^B, x_2^B) = (x_1^B)^{\frac{1}{4}}(x_2^B)^{\frac{3}{4}} 
\end{equation*}


\begin{enumerate}
\item Determine the Walrasian equilibrium (find $p = \frac{3}{5}$ and allocations $((\frac{2}{3},\frac{2}{5}) ; (\frac{1}{3},\frac{3}{5})$). (2.5 points)
\item Check if the Walrasian equilibrium is Pareto-optimal. Which computations should be made to check that the equilibrium is in the core ? (2 points)

\end{enumerate}

\subsection*{Part II {\small (7 points, difficulty ** and ***)}}
We carry on working in the same framework with the same consumers (same preferences and endowments). A firm is created by the consumer B to produce good 2 using good 1 as input. The production function is $y_2 = \sqrt{y_1}$. We note $\pi$ the firm's profit. In the following questions, the firm maximises its profit independently of the consumer B's preferences. The profit is then added to the consumer B's budget.

\begin{enumerate}

\item Determine the demand for good 1 of the firm and the consumers. Prove the price $p$ is equal to $p =\frac{3+\sqrt{59}}{3}$. (2 points) 
\item The production function becomes $y_2 = y_1$. Determine the demand for good 1 of the firm and the consumers by distinguish 3 cases with respect to the value of $p$. (2.5 points)
\item The production function becomes $y_2 = \frac{y_1}{c}$ (with $c>0$). Determine the values of c such that the the firm is active at equilibrium (i.e. $y_1 > 0$) and the values of $c$ such that the firm is not active (Hint : Show that, for some values of $c$, there is
an excess demand of good 1 when the firm is active). Compare the equilibrium of question I.1 with the equilibrium with the non active firm. (2.5 points)

\end{enumerate}


 
\end{document}
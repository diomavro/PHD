This assumption in more restrictive than it needs to be but it simplifies the proof, neccesary assumption for the results in this paper will also be presented. 

The increase in the total profits in an industry is lower when that industry has a merger. 

This is equivalent to an assumption that the difference in merger profits is lower than the differences between firms 

The difference in profits between the monopoly with the advanced the technology and the monopoly with the less advanced technology is lower than the sum of competitive profits with and without technology. 

\begin{assumption}{Decreasing monopoly profits}\label{as2}
\begin{equation*}
\pi_{i}(c_{i2},c_{e}) -\pi_{i}(c_{i},c_{e}) \leq (\pi_{i}(c_{i},c_{e2}) + \pi_{e}(c_{i},c_{e2}))-(\pi_{i}(c_{i},c_{e1}) + \pi_{e}(c_{i},c_{e1}))
\end{equation*}
\end{assumption}

\begin{proof}
We need to set: $\Delta B < \Delta \Pi$
\begin{align*}
\omega \left(  \pi_{i}(c_{i2},c_{e})- \pi_{i}(c_{i},c_{e2})-\pi_{e}(c_{i},c_{e2})  \left( t_2- \frac{1}{q} \left( 1-(1-q)^{T} \right) \right) \right. \\
%%%%%%%%%%%%%%%%%%%%%%%%%%%%%%%%%%%%%
-\left.\pi_i(c_i,c_{e})-\pi_i(c_i,c_{e1})-\pi_e(c_i,c_{e1}) \right)  <0 \\
\end{align*}
\end{proof}

\begin{align*}
\pi_{i}(c_{i2},c_{e})-(\pi_{i}(c_{i},c_{e2})+\pi_{e}(c_{i},c_{e2})) \leq \pi_i(c_i,c_{e})-\pi_i(c_i,c_{e1})-\pi_e(c_i,c_{e1}) \\
%%%%%%%%%%%%%%%%%%%%%%%%%%%%%%%%%%%%%
\pi_{i}(c_{i2},c_{e})-\pi_{i}(c_{i},c_{e2})-\pi_{e}(c_{i},c_{e2})\left( t_2- \frac{1}{q} \left( 1-(1-q)^{T} \right)+\pi_{e}(c_{i},c_{e2})\left( t_2-1- \frac{1}{q} \left( 1-(1-q)^{T} \right) \leq \pi_i(c_i,c_{e})-\pi_i(c_i,c_{e1})-\pi_e(c_i,c_{e1}) \\
%%%%%%%%%%%%%%%%%%%%%%%%%%%%%%%%%%%%%
\pi_{i}(c_{i2},c_{e})-\pi_{i}(c_{i},c_{e2})-\pi_{e}(c_{i},c_{e2})\left( t_2- \frac{1}{q} \left( 1-(1-q)^{T} \right)
-\pi_i(c_i,c_{e})-\pi_i(c_i,c_{e1})-\pi_e(c_i,c_{e1})\leq -2(\pi_i(c_i,c_{e1})-\pi_e(c_i,c_{e1}))+\pi_{e}(c_{i},c_{e2})\left( t_2-1- \frac{1}{q} \left( 1-(1-q)^{T} \right) 
\end{align*}
\documentclass{article}
\usepackage{tikz,pgfplots,calc,graphicx}
\usepackage{preview}	
\usepackage{mathtools}
\usepackage{amsmath}
\usepackage{amssymb}
\usepackage{amsthm}
\usepackage[english]{babel}
\usepackage[utf8]{inputenc}
\usepackage[english]{babel}	
\usepackage{natbib}
\usepackage{color}
\usepackage{chronology}
\usepackage[a4paper,top=3cm,bottom=3cm, right=2cm, left=2cm]{geometry}
\usepackage[normalem]{ulem}
\usetikzlibrary{math}
\bibliographystyle{agsm}
\usepackage{blindtext}	
\usepackage{hyperref}

\begin{document}
\title{Discrimination and Network effects}

If we imagine a firm faced with a simple linear demand curve, it is trivial the profit maximizing level of profits is just to set the price at exactly half the population. Now if there is an additional way to discriminate within the non-consumers and not affecting the first tranche, then the firm will evaluate what that will be worth and try to sub discriminate. This is with the standard assumptions, that is a firm will follow an algorithm of sub discrimination until the next tranche isn't worth it. 

If however increasing the tranche, also increases the height of the initial demand function then the firm will be even more likely to want to sell to the sub group. However the information requirements will be quite different. 

\end{document}
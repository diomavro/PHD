\documentclass{article}
\usepackage{graphicx}
\usepackage{tikz,pgfplots}
\usepackage{preview}	
\usepackage{mathtools}
\usepackage{amsmath}
\usepackage{amssymb}
\usepackage{amsthm}
\usepackage[english]{babel}
\usepackage[utf8]{inputenc}
\usepackage[english]{babel}	
\usepackage{natbib}
\usepackage{color}
\usepackage[a4paper,top=3cm,bottom=3cm, right=2cm, left=2cm]{geometry}
\usepackage[normalem]{ulem}
\usetikzlibrary{math}



\bibliographystyle{agsm}
 
\usepackage{blindtext}						

\newtheorem{theorem}{Theorem}	
\newtheorem{corollary}{Corollary}
\newtheorem{proposition}{Proposition}
\newtheorem{observation}{Observation}

\begin{document}

Let the general triangle distribution be 
1-F(x,z)=G(x)
The slope of G'(x) is positive initially but decreasing in x and at some critical level of x, c it turns negative. The point c which strictly increasing in z. 

\section{$\tilde{x}$ }
This is strictly decreasing in z but the degree at which $\tilde{x}$ is influenced decreases with higher values of z.

Is strictly decreasing in a. However this the effect of a is quickly decreasing. 

Is increasing in p. Constant effect 

\section{$\check{x}$ }

Is increasing in r. 

Is decreasing in z. 

\section{$\hat{x}$ }

remember, decreasing $\hat{x}$ increases profits

If p = 0 then $\hat{x}$ = 0

Is strictly increasing in p. 

Note: I should compare the effects of increasing p on profits vs the decrease in $\hat{x}$

z is strictly decreasing $\hat{x}$. Note:(this increase is relatively more severe in hat than check? )

alpha decreases $\hat{x}$

\section{$general$ }

There exists a critical level of r for all parameter values where the people who are willing to pay collapses. 

The lower the z parameter, the more resistant to r they are, but if too low there is no equilibrium at all. 

\section{Who has the bigger mass}

Note that the area of the intermediate triangle is $(z-\check{x})f(\check{x})\frac{1}{2}$

The area of the smaller triangle is $(z-\hat{x})f(\hat{x})\frac{1}{2}$

The area between the x's is therefore. 

$(z-\check{x})f(\check{x})\frac{1}{2}-(z-\hat{x})f(\hat{x})\frac{1}{2}$

There are more pirates than non-pirates iff:

$(z-\check{x})f(\check{x})\frac{1}{2}-(z-\hat{x})f(\hat{x}) > 0$

$\rightarrow 2\hat{x}(1+\frac{1}{z}(z-\hat{x}))-\check{x}(1+\frac{2}{z}(z-\check{x}))-z>0$



\end{document}


\documentclass{article}
\usepackage[utf8]{inputenc}
\usepackage{enumerate}
\usepackage{amsmath}
\DeclareMathOperator*{\argmax}{argmax}
\DeclareMathOperator*{\argmin}{arg\,min}
\usepackage{amsfonts}
\usepackage{dsfont}
\usepackage{bbm}
\usepackage{graphicx}
\usepackage{lmodern}
\usepackage{asymptote}
\usepackage[font=small,skip=0pt]{caption}
\captionsetup[figure]{font=small,skip=0pt}
\usepackage{pstricks}
\usepackage{pst-plot}
\usepackage{pst-plot,pst-math,pstricks-add}
\usepackage{graphicx}
\usepackage{amsmath}
\usepackage{arydshln}
\usepackage{breqn}
\usepackage{amssymb}
\usepackage{amsthm}

\usepackage{natbib}
\usepackage{color}
\bibliographystyle{agsm}

\usepackage[colorlinks=true, allcolors=blue]{hyperref}

\usepackage{cleveref}

\usepackage{geometry}
\usepackage{titlesec}
\usepackage{nth}
\usepackage{enumerate}
%\usepackage{enuitem}
\usepackage{pgfplots}
\usepackage{graphicx}
\usepackage{enumitem}
\usepackage{tikz}
\usetikzlibrary{matrix,arrows,decorations.pathmorphing}
\usepgflibrary{arrows}
\usepackage{float}
\pgfplotsset{compat=1.12}

\newtheorem{theorem}{Theorem}	
\newtheorem{corollary}{Corollary}
\newtheorem{proposition}{Proposition}
\newtheorem{observation}{Observation}
\newtheorem{assumption}{Assumption}	
\newtheorem{definition}{Definition}
\newtheorem{remark}{Remark}
\newtheorem{lemma}{Lemma}
\newtheorem{result}{result}

\begin{document}

\section{Introduction}

Royalty stacking is the phenomenon that when a firm enters a market it must pay numerous royalties because its product builds upon numerous previous innovations. This occurs because there is no legal obligation that the total royalty fees must remain below some threshold(such as a fixed monetary amount or a proportion of the cost of the product). Royalty stacking is similar to the the term "Patent stacking" except that the latter implies a single owner whilst a former is indifferent to the ownership structure of the stack. 

The empirical evidence, though limited is that the cost of royalty stacking can be quite significant. In smartphones, the cost of royalties has been said to be higher than the cost of components. Estimates show that the cost of patent stacking is about a fourth of of the sale price. \cite{Armstrong2014}

Royalties come to represent such a significant portion because of innovation is sequential. When each innovation builds on the previous innovations it leads to firms a chain of innovation. If strong intellectual property rights are available then this is equivalent to saying that every predecessor in the chain must consent to the new product being built. One interpretation of patent length is how far back in the chain one must pay royalties. 

The hold up problem is not the only kind of issue that can arise. In the usual analysis, the more fragmented is the ownership structure of the chain, the more it will be hard to create a new product. However the hold up problem only a sort of worse case scenario. In practice even if it is assumed that royalties are fixed and there is no hold up problem, there are structural problems that can arise due to royalty stacking. 

Courts are said to recognize that royalty fees must reward with respect to the value added of the innovation and not using the entire value of the product, nevertheless, in practice this does not occur. However in practice the value added of a given royalty is not an observable quantity. In practice, what is especially difficult to discern is the value added of non-patented innovations relative to patented innovations. For a full discussion about why royalties end up in practice being a larger share than their value added, \cite{Elhauge2008}

The main result we have in this paper is to show that in Cournot competition, if there are n firms entering sequentially, for the n firms to form an chain of length n, it must be that the last firm to enter has a higher profit than the first firm to enter. 

\section{The model}

\indent Consider a set of firms that each decides sequentially to enter the market of a given good. Let $i$ be any typical firm in the former set, let $N$ be the set of firms that decide to enter the market and let $n=|N|$ the total number of firms which \textbf{operate} on the market. If $i\in N$, then $i$ pays a fixed cost $0\leq F< \infty$. This fixed cost is paid by all in $N$, irrespective of their production decisions. If $i\notin N$, then $i$'s payoff is zero. A firm is characterized by its production technology. This technology is simply the firm's marginal cost of production. We assume that every firm is endowed with the same marginal cost, denoted as $c_1=c_2=...=c$ ex ante. 

\indent Firms can improve their production costs. A firm $i$ that enters the market can upgrade its original technology $c$ by improving upon some of the technologies already used in the industry, that is $c_1,\ldots, c_{i-1}$. For this, we assume that a firm $i$ may form directed links with some of its predecessors $j$ on the market. A link will capture the transmission of $j$'s technology to $i$. This process of accessing the technologies in the industry (via a firm's investment in links) and innovating upon them enables the firms to reduce their marginal cost of production. 

\indent The specific ways a firm can innovate on its production process depends on where in the production network it is placed. After taking the decision to enter the market, any firm $i\in N$ chooses with which of its predecessors' technologies it wants its technology to build upon. By build upon, we mean improvement upon some of the technologies that already existed in the market prior to $i$'s entry. For example, if $i$ chooses to improve upon firm $j$'s technology, for $j\in \{1,...,(i-1)\}$, we say that $i$'s technology is building on that of $j$'s with probability one. We then say that $i$ has a link with $j$, and this link is represented graphically as $i~\rightarrow ~j$ and denoted as $ij$ throughout. Any link is always directed ; it allows the transfer of $j$'s technology to its successor $i$ but not vice-versa. A strategy of link formation $s_i$ for any firm $i\in N$ is an element of the class of all subsets of $\{1,\ldots , i-1\}$.  The set of all strategies of $i$ is $\mathcal{S}_i$; it is the set of all firms with whom can possibly form links\footnote{A firm cannot form a link with any of its successors. Therefore $i$ can only choose from the set of its predecessors for forming links.}, and it follows that $|\mathcal{S}_i|=2^{i-1}$. All of the firms' decisions in link formation map a technological network. Generally, a network \text{g}$=(V,E)$ is defined on its set of vertices $V$ (i.e. the nodes of the network, thus here $V=N$) and its set of edges (or links). Thence if $j\in s_i$ and $i$ is a vertex of some network \text{g}, $i$'s technology is building upon with that of $j$, and the link $i\rightarrow j$ exists in the technological network \text{g}. 

\indent We make the strong assumption that the firms deterministically infringe upon predecessors of the technology which they select. That is, we consider that a firm may choose to innovate upon some existing technologies - orienting purposefully its R\&D efforts on improving the technologies of the firms the former has links with. Assume that some firm $i$ belongs to some technological network \text{g}. The set $T_i$ refers to the set of firms that have their technologies infringed on by $i$'s. As we mentioned in the precious paragraph, a link $ij$ implies that $i$'s technology infringes on that of $j$. Thus $s_i\subseteq T_i$. But note that if $j$'s technology is building upon the technology of some firm $k\in \{1,\ldots,j-1\}$ then $i$'s technology also infringes on that of $k$'s. Belong to $T_i$ all firms $j\neq i$ such that there exists a directed path from $i$ to $j$ in \text{g}. The set $T_i$ is the set of all predecessors of $i$ such that $i$ has its technology building upon all of their technology. Here, a path from $i$ to $j$ is a sequence of links which connect a sequence of vertices (firms). Note that if $j\in T_i$ and $i$ does not have a link with $j$, then there must exist $k\in T_i$ such that $j\in s_k$, for all distinct firms $j,k\in T_i$. Now, the longest path that starts from firm $i$ (denoted just after as $i_1$) in \text{g} is the longest sequence of directed links $ i_1i_2,\ldots , i_mj$ in \text{g} which first link is a link formed by $i$. The length of this longest path will be referred to as $\rho_i$ (for the previous sequence $\rho_i=m$). Given the network \text{g}, firm $i$'s ex-post technology $\tilde{c}_i$ is given by the expression:
\begin{equation}
\tilde{c}_i=\beta^{\rho_i}c
\end{equation}
for $\beta \in [0,1]$. \\
\indent It follows that the existence of a link entitles the firm which initiates it to some technological benefit, i.e. a reduction in its marginal cost of production. The discrepancy $\beta^{\rho_i}$ in the marginal cost of firm $i$ depends on the length of the longest path that starts at firm $i$, that is on the largest number of different technologies $i$'s technology infringes on.\\
\indent However improving upon some technologies does not come at no cost for the firms. We assume that all firms that do produce must pay royalties when their technology infringes on some other technologies. For our typical firm $i$, her technology $\tilde{c}_i$ costs her $rt_i$, for $r$ the fixed royalty paid to any firm in $T_i$ and $t_i$ the cardinal of the later set. The links between $i$'s technology and all the technologies of the firms in $T_i$ are graphically captured by paths from $i$ to any firm in $T_i$. We consider that a firm's expenditure in royalties is linear in the number of the firms to which it is connected in the technological network. By the same principle, firm $i$ may receive royalties paid by some of its successors which technologies infringes on $i$'s (there exists a path from some firm $j>i$ to firm $i$ in \text{g}, i.e. $i\in T_j$). Let $M_i$ be the set of these firms $j$ which technology infringes on that of $i$ in some network \text{g}, and let $m_i$ the cardinal of this set.\\

\indent Once the technological network is mapped and the infringements between the technologies of the firms revealed, the firms play a classic Cournot game. We take a linear inverse demand function of the form $P(q_1,...,q_n)=\alpha - Q$ for $Q$ the total output supplied on the market. For the sake of clarity, let $q$ be the (row) vector of all of the firms' outputs such that $q\times 1_{n,1}=Q$. Note that a firm that did decide to enter the market in the first stage may not necessarily produce in the second stage of the game. Such a firm does not pay royalties at all, but it can receive some royalties revenue (as long as at least one firm in the corresponding set $M$ does produce a strictly positive amount of output). Therefore, firm $i$'s total expenditure in link formation is given by the expression $r t_i$ if and only if $i$ does offer a strictly positive quantity on the market ; otherwise, $i$ does not pay anything at all. Also, $i$ receives a royalty revenue from some firm $j>i$ if and only if $j$ chooses to produce.\\

The payoff of any firm $i$ that entered the market in the first stage and which produces is given by the following expression: 

\begin{equation}
\pi_i(q, (s_i,s_{-i}),r) = \Bigg(\alpha - \sum_{j=1}^n q_j - \tilde{c}_i \Bigg)q_i + r\Bigg( \sum_{j\in M_i}\mathbbm{1}_{q_j>0} - \mathbbm{1}_{q_i>0}~ t_i\Bigg)  - F \label{payoff},
\end{equation}  
where the first term into bracket is $i$'s net revenue on every unit of good that it sells on the market, and the second term into bracket is $i$'s net revenue from royalties. \\

\indent The next section of the paper is devoted to solving for the subgame perfect Nash equilibria of the game. We will use backward induction in order first to get the best-response functions for the firms that decide to enter the market, then the optimal strategies in link formation, and finally solve for the entrance decision. We will consider that the firms are foresighted, which entails that they do consider in their expected profit the revenue from royalties that might be generated by the arrival of successor firms on the market. 

\section{SPNEs of the game }

\indent We solve for the SPNE of this sequential game using backward induction. We first clarify the stages of the game, and the actions taken by the representative firm $i$ in each and every of these stages. \\

\textit{First stage: entry and link formation }\\

\indent In the first stage, a firm $i\in \{1,2,...\}$ decides whether or not it enters the market. If $i$ decides not to enter the market, the game ends for this firm and its payoff is zero. If $i$ enters instead, therefore expecting a positive profit from doing so, it chooses its innovation level $\tilde{c}_i$ that depends on $i$'s strategy of link formation $s_i$. This strategy of link formation determines which technologies $i$'s production process will infringe upon. Note that the decision to enter and the choice of the links that the firms maintain can be broken down into two stages. \\

\textit{Second stage: Cournot game }\\

\indent All of the firms in $N$ choose a quantity to offer on the market. By the beginning of stage two, the technological network \text{g} is realized. If $i$ chooses not to produce, then this firm is free not to pay any royalty - since $i$ does not use its technology. However, $i$ may receive some royalty revenue from some firms in $M_i$. If a firm chooses not to produce, then this firm is referred to as a \textit{rent seeker}. If $i$ does produce, then we refer to $i$ as a \textit{producer}. Note that $i$'s net revenue from royalties never depends on the quantity it produces when this quantity is strictly positive. A producer $i$ is always more efficient at producing than any producer in $T_i$ (i.e. $\tilde{c}_i<\tilde{c}_j$ with $j\in T_i$), and always less efficient at producing than any producer in $M_i$ (i.e. $\tilde{c}_i>\tilde{c}_j$ with $j\in M_i$).  \\

\textit{Third stage: the payoffs are realized}\\

See equation \ref{payoff}. \\

\indent We solve for the second stage of the game. Assume that the technological network \text{g} has been mapped after the first stage. Now all firms in $N$ must decide whether they want to produce, and if so which quantity. Let $q_{-i}$ be the total production supplied on the market when $i$ does not produce. The production strategy $q_i$ of firm $i$ is said to be a best-response to $q_{-i}$ if given some fixed value $r$ for the royalty : 
\begin{equation}
\pi_i((q_i,q_{-i}), (s_i,s_{-i}),r)\geq \pi_i((q'_i,q_{-i}), (s_i,s_{-i}), r),~~\text{for all}~~q_i\in \mathbb{R}~~\text{and}~~q_{-i}\in \mathbb{R}^{n-1}.
\end{equation}

The set of all of firm $i$'s best-responses (in production) is denoted $BR_i(q_{-i})$. Furthermore, the total output $Q$ supplied on the market is a SPNE if $q_i\in BR_i(q_{-i})$ is the strategy played by each firm $i\in N$. We now present the analytic form of the best-responses of all firms in this second stage.\\

\textbf{Proposition 1.} \textit{Let $i$ be any firm in the set $N$. Let $\Omega\subseteq N$ be the set of firms which produce on the market, and let $\omega$ the number of these firms. The best-response of firm $i$ to the vector $q_{-i}$ of its competitors' strategies is given by the expression: }
\begin{equation}
BR_i(q_{-i}) = \dfrac{1}{\omega+1}\Big(\alpha - (\omega+1)\tilde{c}_i + \sum \limits_{ j\in \Omega}\tilde{c}_j \Big), \label{BR}
\end{equation}
\textit{for any $i \in \{1,...,n\} $. For some $i\in N$, $i$'s ex-post payoff is :}
\begin{multline} 
\pi_i(q, (s_i,s_{-i}),r)=\\
\max 
\begin{cases}
\dfrac{1}{(\omega+1)^2}\Big(\alpha - (\omega+1)\tilde{c}_i + \sum \limits_{ j\in \Omega}\tilde{c}_j \Big) ^2 + r\Bigg( \sum_{j\in M_i}\mathbbm{1}_{q_j>0} - \mathbbm{1}_{q_i>0}~ t_i\Bigg)  - F 
 \\
 r  \sum_{j\in M_i}\mathbbm{1}_{q_j>0}-F
\end{cases}\\
\end{multline}

\indent Although a firm's best-response may prescribe to produce a strictly positive amount of output, it might be more profitable for this firm to adopt a rent seeking behavior. This is true whenever the Cournot profit of a firm is less than its expenditure in links formation (or royalties expenditure). \\

\indent We turn to the equilibrium strategies of the firms for the first stage of the game. First, we make explicit the optimal strategies in terms of link formation of the firms that decide to enter the market. We show in what follows that forming a single link in the network strictly dominates any strategy that consists of maintaining strictly more than one link, and this for all the firms in $N$. \\

\textbf{Proposition 2.} \textit{If $r>0$ then any firm $i\in \Omega$ always forms either one link or none in equilibrium. }\\

\indent \textbf{Proof.} This is a direct proof. Take any firm $i\in \Omega$. Let $s_i\in \mathcal{S}_i$ be any strategy for firm $i$ that consists of maintaining at least two links in the industry. Let $j$ be the firm that has the lowest marginal cost in the set $s_i$. Now consider the alternate strategy $s_i'\subset s_i$  for firm $i$ that consists of forming one single link to this firm $j$ just defined. We show that $s'_i$ always strictly dominates $s_i$. First the technological benefits of $s_i$ and $s'_i$, respectively. By hypothesis, only the longest path that starts at vertex $i$ matters for gauging the technological benefit from $i$'s connections. Thus $\tilde{c}_i=\tilde{c}'_i$, for $\tilde{c}_i$ the cost achieved by $i$ when it plays $s_i$ and $\tilde{c}'_i$ when $i$ plays $s'_i$. And $T'_i\subset T_i$ since $s'_i\subset s_i$, for $T'_i$ the set of all firms that have their technology building $i$'s technology when the later plays $s'_i$, and $T_i$ the same set but when $i$ plays $s_i$. Let $t'_i$ the cardinal of $T'_i$ and $t_i$ the cardinal of $T_i$. Since $i\in \Omega$, then $i$ produces. Its expenditures in links is $rt_i$ if $i$ plays $s_i$, and $rt'_i$ if $i$ plays $s'_i$. By the previous point, $rt_i>rt'_i$. Note that $i$ playing $s_i$ over $s'_i$ has no influence on the number of the successors of $i$ which will eventually connect to $i$ (since $i$'s technology is the same for both strategies). The result follows. $ \qed$\\  

\textbf{Remark 1.} For all $i\in \Omega$ we have that $\rho_i=t_i$ in equilibrium, and the path from $i$ to any $j\in T_i$ is unique in \text{g} if \text{g} is a Nash network. \\

\indent \textbf{Proof.} This result is implied by proposition 2. To see this, consider the set $s_i$ for any firm $i\in \Omega$. The statement needs only be proven when $s_i\neq \emptyset$. Therefore $|s_i|=1$ by proposition 2. Consider all paths that start at vertex $i$ in the technological network $i$ belongs to. The path from $i$ to a firm $j$ that is a predecessor of $i$ and that has a worse technology than $\tilde{c}_i$ exists if and only if $j\in T_i$. And this path is unique since all of $i$'s predecessors $j$ with $\tilde{c}_j>\tilde{c}_i$ have either no link or one link. Thus if there is $\tilde{c}_k=\tilde{c}_j>\tilde{c}_i$ then if a path from $i$ to $j$ exists then no path from $i$ to $k$ exist by proposition 2. Therefore : $j\in T_i~~\Leftrightarrow$ a unique path from $i$ to $j$ exists. It follows that if $i\in \Omega$, $i$ must pay royalties to all firms in $T_i$ provided that all of these firms belong to the longest path that starts at $i$. This path is unique and is the path that connects $i$ to the firm in $s_i$ that has the worst technology. Hence proved. $\qed$\\

\textbf{Remark 2.} \textit{The last firm which enters the market always produces in equilibrium.}\\

\indent \textbf{Proof.} If $n$ is the last firm to enter the market, then $M_n=\emptyset$. Thus if $n$ does not produce, its payoff is always $-F$. But then no entry would have been preferable over entry and not producing for $n$. Therefore a contradiction. \\
\indent \textit{One then notes that the Cournot profit of the last firm that enters must exceed the sum of its royalties expenditures and the fixed cost. }

\section{Case study: the chain network with all firms producing}

Consider the following network: all $n$ firms which entered the market formed a chain network. A chain is a sequence of links $i\rightarrow j$ such that $j=i+1$ and $j= s_i$, for all firms $i\in N\setminus \{1\}$. Assume further that all firms in the chain produce in equilibrium: $\Omega = N$. The payoff of a typical firm indexed $i$ is given by: 
\begin{equation*}
    \pi_i((q_i,q_{-i}), (s_i,s_{-i}), r) = \dfrac{1}{(n+1)^2}\Big[A+c\Big(\dfrac{1-\beta^n}{1-\beta}-(n+1)\beta^{i-1}\Big)\Big]^2 + (n-2i+1)r-F.
\end{equation*}
The payoff of any firm $i$ in a chain where all firms produce is denoted $\pi_i$ for the sake of clarity. \\
We are interested in the relation between a firm's payoff and its position in the chain. Here, $i$'s position is simply the value $i$ of its index. Therefore, the larger $i$, the more efficient the later at producing compared to the rest of the firms; however, the larger its royalty expenditures. We study the sign of the first derivative of $i$'s profit with respect to its index: 
\begin{equation}
    \dfrac{\partial \pi_i}{\partial i}= -2 c ~q_i (i-1)ln(\beta)\beta^{i-1}-2r,
\end{equation}
for $q_i$ the quantity supplied by $i$. The first term is monotonic and decreasing in $i$; and it is always strictly positive. The second one is always negatively valued since $r\geq 0$ by assumption. Thence, if the derivative changes sign, it is first positive then negative (single-crossing from above). We may have then one of the three cases: 
\begin{enumerate}
    \item the royalty $r$ is small enough so that the payoffs are increasing in the chain: $\pi_i\geq \pi_j$ for any $i$ and $j<i$; 
    \item the royalty $r$ is large enough so that the payoffs are decreasing in the chain: $\pi_i\leq \pi_j$ for any $i$ and $j<i$;
    \item the royalty revenue $r$ takes on a intermediate value for which the derivative is single-crossing. I.e. there exists $1<k<n$ such that: (i) $\pi_i\geq \pi_j$ for all $i\leq k$ and $j<i$, and (ii) $\pi_i\leq \pi_j$ for all $j\leq k$ and $j<i$. 
\end{enumerate}

\begin{proposition}
Consider the chain network where all firms produce in equilibrium. If $\pi_1> \pi_2$, then profits are decreasing along the chain. That is: for 
\begin{equation*}
    r>c(1-\beta)\Big(p-\dfrac{c(1+\beta)}{2}\Big)
\end{equation*}
then the relation $\pi_i\geq \pi_j$ is verified, for all $i\in N\setminus\{n\}$ and $j>i$. 
\end{proposition} 
\begin{proof}
    Go back to the study of the profit of firm $i$ with respect to its index. The derivative is single-crossing from above if we consider that $i\in \mathbb{R}_{+}$. If $\pi_1>\pi_2$, we are in the negative part of the derivative. Therefore the result. 
\end{proof}

\begin{proposition}
Consider the same chain network where all firms produce. If $\pi_{n-1}<\pi_n$, then profits are increasing along the chain. That is: for
\begin{equation*}
    r< c\beta^{n-2}(1-\beta)\Big(p-\dfrac{c\beta^{n-2}(1+\beta)}{2}\Big)
\end{equation*}
then the relation $\pi_i\leq \pi_j$ is verified, for all $i\in N\setminus\{n\}$ and $j>i$. 
\end{proposition} 
\begin{proof}
    Again, the derivative of $i$'s profit with respect to $i$ would be single-crossing from below if $i\in \mathbb{R}_+$. If $\pi_{n-1}<\pi_n$, then we are on the positive part of the derivative. The result follows.  
\end{proof}

\begin{corollary}
In a chain network where all firms produce in equilibrium, the fixed cost must verify: 
\begin{equation}
    F\leq \min \{\pi_1,\pi_n\}. 
\end{equation}
\end{corollary} 
\begin{proof}
Payoffs along the chain when all firms produce are either (i) increasing along the chain, (ii) decreasing along the chain, or (iii) first increasing from $1$ to some $<1k<n$, then decreasing till $n$. Let us consider case (i). Then firm $1$ is the firm with the lowest payoff in the network. For it to enter, then $F\leq \pi_1$. The two statements imply that all of the $n$ firms enter the market. For case (ii), the reasoning is the same, except that the it is $n$ which gets the lowest payoff among all firms. Finally, for case (iii), payoffs are increasing up to some firm $k$, then they decrease. Thence $\min_{i\in N}\pi_i \in \{\pi_1,\pi_n\}$. 
\end{proof}

We now provide an upper bound for the royalty cost that guarantees that $n$'s best-response is indeed to form a link to $n-1$ in the chain. We prove this result in two steps. The first part of the demonstration consists of deriving the maximal value of $r$ for which $n$ is indifferent between linking to $n-1$ and linking to any firm in the chain that is not $(n-1)$. Then, we show that this bound on $r$ is sufficient to prove that connecting to $n-1$ is a best-response for $n$. \\
\indent Consider the deviation $s'_n$ for $n$ that consists of forming a link to $1\leq k\leq n-2$ (and $s'_n=k\neq n-1$). Let $q'_n$ be the (best-response) quantities supplied by $n$ after the later played the strategy $s'_n$. Note that $n$ always produces in equilibrium as it is the last firm to enter the market. Let $q'$ be the vector $(q'_1, \ldots, q'_n)$ of all firms' equilibrium outputs. We assumed in this section that all firms produce in the chain of length $n$. Thence if $n$ changes strategy by choosing a less efficient technology, then all firms in the resulting network should be all producing. Assume that $n$'s change in strategy does not affect $(n+1)$'s entry decision. That is, although $n$ is less efficient at producing, no other firm find it profitable to enter the market. The pair of alternate strategies $(s'_n,q'_n)$ gives $n$ (at most) the following payoff:   
\begin{equation}
    \pi((s'_n,s_{-n}),(q'_n,q'_{-n}), r)\leq \dfrac{1}{(n+1)^2}\Big[ A +c \Big( \dfrac{1-\beta^{n-1}}{1-\beta}-n\beta^{k}\Big)\Big]^2-rkc-F\equiv \tilde{\pi}_n(k) 
\end{equation}

\begin{definition}
Let $r^*$ be the the value of the royalty cost such that, absent any deterrence and any production structure considerations, all r below this $r^*$ would lead the nth firm to prefer to expand the chain. Defined as: 
\begin{equation*}
    r=r^* \Rightarrow \pi_n=\max_{1\leq k\leq n-2}\pi_n(k) 
\end{equation*}
\end{definition}

\begin{proposition}
If $r<r^*$, then the best-response in link formation of the last firm that enters the market is $s_n=n-1$.  
\begin{proof}
An alternate strategy for $n$ is $s'_n=k$ with $1\leq k\leq n-1 $. The payoff associated with any deviation $s'_n=k$ is always less than $\tilde{\pi}_n(k)$. Let $s'_n=k^*$ the most profitable deviation for $n$, for $1\leq k^*\leq n$. Since $s_n=n-1$ is a best-response for $n$, then forming a link to $k^*$ gives $n$ a lower payoff than $\pi_n$. Also, this payoff is lower than $\pi_n(k^*)$. Thus, the loss in $n$'s payoff from this move would be less than $\pi_n-\pi_n(k^*)$. The result follows.   
\end{proof}
\end{proposition}













\section{The three firm case}

The discussion until now has been quite abstract, so we take a simple 3 firm example and show what kind of equilibria can be ruled out. We proceed by backward induction. 


\subsection{3rd firm}
When the third firm enters it can only see one of two scenarios. It can either observe two singletons, or it can observe a two firm chain. \footnote{We rule out the case where the second firm did not enter because it would imply the third firm also does not enter}.  If the third firm observes two singletons, then the the relevant payoff vector is given by:

\begin{align*}
& \text{max} 
\{ No~ entry,
No~ Attachment, 
Attach~ to~1
\}& \\
& \text{max} 
\{ 0,
\left(\frac{A-c}{4}\right)^2-F, 
 \left( \frac{1}{4}(A+c(2-3b))\right)^2-r-F
\} &
\end{align*}

The royalty cost for which the firm will not attach itself is given by:

\begin{align*}
& \left(\frac{A-c}{4}\right)^2> \left(\frac{1}{4}(A+c(2-3b))\right)^2-r & \\
\rightarrow 
& r> r_{sym} = \frac{3c}{16}(1-b)(2A+c-3bc) &
\end{align*}

If instead, the entrant observes a chain, the relevant payoff vector will be

\begin{align*}
& \text{max} \{No~ entry, No~ attachment, Attach ~to ~1, Attach~ to~ 2 \} &\\
& \text{max} \{
0,
\frac{1}{4}(A+(b-2)c), \frac{1}{4} (A+c(1-2b))-r, \frac{1}{4}(A+c(1+b-3b^2))-2r \} &
\end{align*}

This means that there are now three conditions to establish the firms preferences over attaching. The third firm Will prefer no attachment to attaching to the 1st firm iff:

\begin{equation*}
r> \frac{3}{16} c \left(2 A (1-b)-c(1-b^2 ) \right) = r_1
\end{equation*}

Will prefer no attachment to attachment 2nd firm iff:

\begin{equation*}
r > \frac{3}{32} \left(1-b^2\right) c \left(2 A-c(1-2b+3 b^2) \right) = r_2
\end{equation*}


Will prefer attach to 1st over 2nd iff:

\begin{equation*}
r > \frac{3}{16} (1 - b) b c (2 A + (2 - b - 3 b^2) c) = r_3
\end{equation*}


The third firm can only attach itself to firm 1 if:

\begin{align*}
\frac{3}{16} (1 - b) b c (2 A + (2 - b - 3 b^2) c)<\frac{3}{16} c \left(2 A (1-b)+c(b^2 -1) \right) \\ 
\Leftrightarrow 
2A>c(1+b(4+3b))
\end{align*}

We will now create a strategy space for the 3rd firm. The strategy space will simplify the analysis by separating the subcases for the analysis from the point of view of the second firm. The strategy space will be given by a two element response vector, where the first element represents the reaction if there are two singletons and the second element the reaction if there is a two firm chain. We label each of the six cases. 

\begin{align*}
Luddite: ~&S_{lud} =  
\{ No~ Attachment, 
No~ Attachment
\}& \\
 Copycat:~& S_{cop} =  
\{ No~ Attachment, 
Attach~to~1
\}& \\
Double~or~nothing:~ & S_{don} = 
\{ No~ Attachment, 
Attach~to~2
\}& \\ 
Ego:~& S_{ego} =  
\{ Attach~to~1, 
No~ Attachment
\}& \\
 Moderate~Innovator:~ & S_{mod} =  
\{  Attach~to~1, 
Attach~to~1
\}& \\
 Unconditional~Expansion:~ & S_{unc} = 
\{  Attach~to~1, 
Attach~to~2
\}& \\ 
\end{align*}

\begin{proposition}
The copycat strategy never exists
\end{proposition}

\begin{proof}
The definition of copycat implies that when the 2nd firm does not attach then 3 also does not attach(symmetric outcome). $r > \frac{3 c(1-b)}{16}(2A+c-3bc)=r_{sym}$. When firm 2 does attach to the first firm, then the third firm also attaches to the first firm. $r< \frac{3}{16} c \left(2 A (1-b)-c(1-b^2 ) \right)=r_{1}$ and $r > \frac{3}{16} (1 - b) b c (2 A + (2 - b - 3 b^2) c)=r_{3} $ Therefore the copycat strategy is Nash iff $r \in [max\{r_{sym}, r_{3} \}, r_{1}]$. We need only note that $r_{sym}$ is strictly greater than $r_{1}$, therefore the set is empty and the result follows
\end{proof}

\subsection{2nd firm}

Will be using the strategy space specified for the third firm to discuss the strategies of the 2nd firm. 

\subsubsection{Luddite}

If the third firm will always choose to be a singleton the second firm has the choice of either enabling the three way Cournot competition or to be downstream firm against two upstream firms. The relevant payoff vector is then:

\begin{align*}
&\text{max} \{No~entry, No~attachment, Attach~to~1 \} &\\
&\text{max} \{ 
0, \frac{A-c}{4}, \frac{1}{4}(A+c(2-3b))-r
\}&
\end{align*}

The second firm will only prefer no attachment if r is higher than a certain threshold. This is intuitive because if r is 0, then it can get an advantage on the other two firms at no cost. The relevant r that will determine this decision(as lon as at least one of the payoff is greater 0) will be given by: 

\begin{equation*}
r> r_{lud} = \frac{3}{16} (1-b) c (2 A-3 b c+c)
\end{equation*}

\begin{proposition} \label{symislud}
If the third firm is a luddite $\Rightarrow$ the second firm does not attach
\end{proposition}

\begin{proof}
Need only note that the r for making the decision is the same in both cases. ($r_{sym}=r_{lud}$)
\end{proof}

\subsubsection{Double or nothing}

If it is the case that the third firm will either expand the two person chain or be a singleton then this creates a clear cut scenario for the second firm. The relevant payoff vector is: 

\begin{align*}
&\text{max} \{No~entry, No~attachment, Attach~to~1 \} &\\
& \text{max} \{ 
0, \left(\frac{A-C}{4} \right)^2, \left(\frac{1}{4}(A+c(1-3b+b^2)) \right)^2
\} &
\end{align*}

Notice that no matter the outcome, the second firm will never have to pay any royalties. This occurs because if it does not attach itself then it will have the symmetric competition outcome and no royalties will be paid. If the firm does attach itself to the first firm, then the third firm want to make a chain which means that the second firm will be paying a royalty to the first firm and receiving a royalty from the third firm, therefore it will be royalty neutral. 

\begin{proposition}
If the third firm plays Double or nothing $\Rightarrow$ the second firm always attaches to 1
\end{proposition}

\begin{proof}
To see this we need only note that  $No~ attachment<Attach~to~1 \rightarrow 3b-2<b^2$ which is always verified for $b \in [0,1]$
\end{proof}

\subsubsection{Entrepreneur with an ego}

If the third firm is playing entrepreneur with an ego then we have that it wants to be the first firm to innovate, if the second firm innovates first, then the third firm will simply be a singleton. The relevant payoff vector of the second firm is then: 

\begin{equation*}
\text{max} \{ 
0, \left(\frac{1}{4}(A+ c(b-2)) \right)^2+\frac{1}{2}r, \left( \frac{1}{4} (A+c(2-3b) )\right)^2-r
\}
\end{equation*}

Note that here in the no attachment case, the royalty is divided by two because if the second firm decides to be a singleton, it has probability $\frac{1}{2}$ of being selected by the entrepreneur. The relevant cutoff point for the firm to prefer no attachment to attachment is given by: 

\begin{equation*}
r> r_{ego} = \frac{1}{3} (1-b) c (A-b c)
\end{equation*}

\begin{proposition}\label{egolud}
$r_{ego}<r_{lud}$. This implies that if 2 attaches when 3 plays luddite, then it always attaches when 3 plays ego.   
\end{proposition}

\begin{proof}
trivial
\end{proof}

\subsubsection{Moderate Innovator} 
If the third firm expands to the second tier and never to the third tier. Then the second firm has the choice of either being one of two downstream firms paying royalties to the upstream firm, or it can be one of two upstream firms with some probability of receiving royalties from one downstream firm. 

\begin{equation*}
\text{max} \{ 
0, \left(\frac{1}{4}(A+ c(b-2)) \right)^2+\frac{1}{2}r, \left( \frac{1}{4}(A+c(1-2b)) \right)^2 -r
\}
\end{equation*}

The relevant cutoff point for the second firm to prefer no attachment to attachment is then given by:

\begin{equation*}
r>r_{mod}= \frac{3}{8} c \left(2 a (1-b)-c(1-b^2)\right)    
\end{equation*}

\begin{proposition}\label{egomod}
$r_{ego}<r_{mod}$. This implies that if 2 attaches when 3 plays mod, then it always attaches when 3 plays ego.   
\end{proposition}

\begin{proof}
trivial
\end{proof}

\subsubsection{Unconditional expansion } Finally, we look at the case where r is low enough than no matter what the scenario, the third firm firm will always expand the chain. From the point of view of the second firm, we have that either it will be optimal for the second firm to play the rent seeking game and be a singleton with some probability of the third firm paying it, or its best to be the firm in the middle chain. The relevant vector the second firm will be considering is: 

\begin{equation*}
\text{max} \{ 
0, \left(\frac{1}{4}(A+ c(b-2)) \right)^2+\frac{1}{2}r, \left(\frac{1}{4}(A+c(1-3b+b^2)) \right)^2
\}
\end{equation*}

Which gives the following cutoff point. 

\begin{equation*}
r>r_{unc}=\frac{1}{8} \left(b^2-4 b+3\right) c \left(2 a+\left(b^2-2 b-1\right) c\right)
\end{equation*}

\subsubsection{Lowest r results}

\begin{proposition}\label{uncmod}
$r_{unc}<r_{mod}$. This implies that if 2 attaches when 3 played mod, then 2 always attaches when 3 plays unc. 
\end{proposition}


\begin{proof}
trivial
\end{proof}

\begin{proposition}\label{uncego}
$r_{unc}<r_{ego}$. This implies that if 2 attaches when 3 plays unconditional, then it always attaches when 3 plays ego.  
\end{proposition}

\begin{proof}
trivial
\end{proof}

\begin{proposition}\label{minr}
The lowest cutoff point is given by: 
$r_{unc}$. 
\end{proposition}

\begin{proof}
The corrolary is given by transitivity. Since by proposition \ref{egolud}, we have that $r_{ego}<r_{lud}$ and by proposition \ref{egomod} that $r_{ego}<r_{lud}$, and by proposition \ref{uncmod}, we also have that $r_{unc}<r_{mod}$, it follows from $r_{unc}<r_{ego}$ that $r_{unc}$ is the lowest cutoff point. 
\end{proof}



\subsubsection{Upper r results}

\begin{proposition}
If  $r<  r_{unc}$ then a Nash equilibrium is either a chain or a tree.
\end{proposition}

\begin{proof}
If $r$ is lower than $   r_{unc}, r_{copy} $ then by \ref{minr} firm 2 always attaches to firm 1. Now, firm 3 prefers to attach to firm 1 over not getting attach at all if $r<r_1$. We need only note that the relation $r_1<r_{unc}$ is always verified.  
\end{proof}


\begin{corollary}
If $r<  r_{unc}$ then 3 plays either unc expansion, OR moderate innovator in equilibriuim.
\end{corollary}

\begin{corollary}
If $r>$ max$\{ r_{mod},r_{luddite} \}$ then firm 3 only plays either luddite or entrepreneur with ego in equilibrium. 
\end{corollary}

\begin{proposition} \label{Symmetric}
If $r>$ max$\{ r_{mod},r_{luddite} \}$, then the only Nash equilibrium is the symmetric outcome. 
\end{proposition}

\begin{proof}
First note that max$\{ r_{mod},r_{luddite} \}$ is the maximum over all the bounds. So if r is larger than these two, it is larger than all the bounds. Which implies that firm 2 never attaches itself, regardless of 3's strategy. 

Firm 3 will only attach to either firm 1 or 2 iff $r \leq r_{sym}$. But by proposition \ref{symislud} $r_{sym}=r_{lud}$; and by hypothesis $r > r_{lud}$. Thus firm 3 never attaches. 
\end{proof}


\section{Bertrand}

In the setup of our model, only the firms that are producing have to pay royalties. If the firms compete a la Bertrand, all firms' decisions in link formation would form a chain, which length $\rho=n$ is such that $n+1$'s profit would be strictly negative if he were to extend the chain. First, we show that Bertrand competition between the firms which produce implies that the network is a chain. Then, we demonstrate that only the last firm produces in equilibrium. And this firm, since the network is a chain, is the firm that is the most efficient at producing.  

\begin{proposition}
If $F=0$, then only the first firm enters
\end{proposition}

\begin{proof}
In Bertrand, cournot profit is 0 for producers. 
\end{proof}

\begin{proposition}
If we are in Bertrand competition then there is never a tree as long as $F>0$
\end{proposition}

\begin{proof}
To be made
\end{proof}



\begin{proposition}
Suppose the maximal profit possible is $\overline{\pi}$, then the innovation is expanded to its maximum iff: $\overline{\pi}-n r-F>0$ or $\frac{\overline{\pi}-F}{n}$
\end{proposition}

\begin{proof}
To be done
\end{proof}


\begin{corollary}
If the maximum is not pursued then, the optimum number of n is given by:
\begin{equation*}
max_n\{\overline{\pi}(n)-n r-F  \}
\end{equation*}
\end{corollary}

\subsection{Reduced form notes}

Let the market profit of firm i in an n firm chain be given by: $\pi(i,n)$. 

If this is strictly increasing in i, then we trivially have something that is Nash stable. What can we say about profit as a function of n? 

If it is strictly decreasing, it may still be stable, depending on how profitable it is to be a singleton. 

We may be able to combine these two insights for the single crossing notes. 

\bibliography{royalty.bib}


\end{document}
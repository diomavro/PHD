%\documentclass[AER]{AEA}
\documentclass[12pt]{report}
%\documentclass[12pt]{article}
%\documentclass[12pt,a4paper]{article}

\usepackage[utf8]{inputenc}


\usepackage{mathtools}
\usepackage{amsmath}
\usepackage{amssymb}
\usepackage{amsthm}

\usepackage{float}
%\usepackage[cmbold]{mathtime}
%\usepackage{mt11p}
\usepackage{placeins}
\usepackage{caption}
\usepackage{color}
\usepackage{subfigure}
\usepackage{multirow}
\usepackage{epsfig}
\usepackage{listings}
\usepackage{enumitem}
\usepackage{rotating,tabularx}
%\usepackage[graphicx]{realboxes}
\usepackage{graphicx}
\usepackage{graphics}
\usepackage{epstopdf}
\usepackage{longtable}
\usepackage[pdftex]{hyperref}
%\usepackage{breakurl}
\usepackage{epigraph}
\usepackage{xspace}
\usepackage{amsfonts}
\usepackage{eurosym}
\usepackage{ulem}

\usepackage{tikz}
\usetikzlibrary{spy}

\usepackage{verbatim}



\usepackage{footmisc}
\usepackage{comment}
\usepackage{setspace}
\usepackage{geometry}
\usepackage{caption}
\usepackage{pdflscape}
\usepackage{array}
\usepackage[authoryear]{natbib}
\usepackage{booktabs}
\usepackage{dcolumn}
\usepackage{mathrsfs}
%\usepackage[justification=centering]{caption}
%\captionsetup[table]{format=plain,labelformat=simple,labelsep=period,singlelinecheck=true}%
\bibliographystyle{apalike}
%\bibliographystyle{unsrtnat}



%\bibliographystyle{aea}
\usepackage{enumitem}
\usepackage{tikz}
\usetikzlibrary{positioning}
\usetikzlibrary{arrows}
\usetikzlibrary{shapes.multipart}

\usetikzlibrary{shapes}
\def\checkmark{\tikz\fill[scale=0.4](0,.35) -- (.25,0) -- (1,.7) -- (.25,.15) -- cycle;}
%\usepackage{tikz}
%\usetikzlibrary{snakes}
%\usetikzlibrary{patterns}

%\draftSpacing{1.5}

\usepackage{xcolor}
\hypersetup{
colorlinks,
linkcolor={blue!50!black},
citecolor={blue!50!black},
urlcolor={blue!50!black}}

%\renewcommand{\familydefault}{\sfdefault}
%\usepackage{helvet}
%\setlength{\parindent}{0.4cm}
%\setlength{\parindent}{2em}
%\setlength{\parskip}{1em}

%\normalem

%\doublespacing
\onehalfspacing
%\singlespacing
%\linespread{1.5}

\newtheorem{theorem}{Theorem}
\newtheorem{corollary}[theorem]{Corollary}
\newtheorem{proposition}{Proposition}
\newtheorem{definition}{Definition}
\newtheorem{axiom}{Axiom}
\newtheorem{observation}{Observation}
\newtheorem{assumption}{Assumption}	
\newtheorem{remark}{Remark}
\newtheorem{lemma}{Lemma}
\newtheorem{result}{result}


\newcommand{\ra}[1]{\renewcommand{\arraystretch}{#1}}

\newcommand{\E}{\mathrm{E}}
\newcommand{\Var}{\mathrm{Var}}
\newcommand{\Corr}{\mathrm{Corr}}
\newcommand{\Cov}{\mathrm{Cov}}

\newcolumntype{d}[1]{D{.}{.}{#1}} % "decimal" column type
\renewcommand{\ast}{{}^{\textstyle *}} % for raised "asterisks"

\newtheorem{hyp}{Hypothesis}
\newtheorem{subhyp}{Hypothesis}[hyp]
\renewcommand{\thesubhyp}{\thehyp\alph{subhyp}}

\newcommand{\red}[1]{{\color{red} #1}}
\newcommand{\blue}[1]{{\color{blue} #1}}

%\newcommand*{\qed}{\hfill\ensuremath{\blacksquare}}%

\newcolumntype{L}[1]{>{\raggedright\let\newline\\arraybackslash\hspace{0pt}}m{#1}}
\newcolumntype{C}[1]{>{\centering\let\newline\\arraybackslash\hspace{0pt}}m{#1}}
\newcolumntype{R}[1]{>{\raggedleft\let\newline\\arraybackslash\hspace{0pt}}m{#1}}

%\geometry{left=1.5in,right=1.5in,top=1.5in,bottom=1.5in}
\geometry{left=1in,right=1in,top=1in,bottom=1in}

\epstopdfsetup{outdir=./}

\newcommand{\elabel}[1]{\label{eq:#1}}
\newcommand{\eref}[1]{Eq.~(\ref{eq:#1})}
\newcommand{\ceref}[2]{(\ref{eq:#1}#2)}
\newcommand{\Eref}[1]{Equation~(\ref{eq:#1})}
\newcommand{\erefs}[2]{Eqs.~(\ref{eq:#1}--\ref{eq:#2})}

\newcommand{\Sref}[1]{Section~\ref{sec:#1}}
\newcommand{\sref}[1]{Sec.~\ref{sec:#1}}

\newcommand{\Pref}[1]{Proposition~\ref{prop:#1}}
\newcommand{\pref}[1]{Prop.~\ref{prop:#1}}
\newcommand{\preflong}[1]{proposition~\ref{prop:#1}}

\newcommand{\Aref}[1]{Axiom~\ref{ax:#1}}

\newcommand{\clabel}[1]{\label{coro:#1}}
\newcommand{\Cref}[1]{Corollary~\ref{coro:#1}}
\newcommand{\cref}[1]{Cor.~\ref{coro:#1}}
\newcommand{\creflong}[1]{corollary~\ref{coro:#1}}

\newcommand{\etal}{{\it et~al.}\xspace}
\newcommand{\ie}{{\it i.e.}\ }
\newcommand{\eg}{{\it e.g.}\ }
\newcommand{\etc}{{\it etc.}\ }
\newcommand{\cf}{{\it c.f.}\ }
\newcommand{\ave}[1]{\left\langle#1 \right\rangle}
\newcommand{\person}[1]{{\it \sc #1}}

\newcommand{\AAA}[1]{\red{{\it AA: #1 AA}}}
\newcommand{\YB}[1]{\blue{{\it YB: #1 YB}}}

\newcommand{\flabel}[1]{\label{fig:#1}}
\newcommand{\fref}[1]{Fig.~\ref{fig:#1}}
\newcommand{\Fref}[1]{Figure~\ref{fig:#1}}

\newcommand{\tlabel}[1]{\label{tab:#1}}
\newcommand{\tref}[1]{Tab.~\ref{tab:#1}}
\newcommand{\Tref}[1]{Table~\ref{tab:#1}}

\newcommand{\be}{\begin{equation}}
\newcommand{\ee}{\end{equation}}
\newcommand{\bea}{\begin{eqnarray}}
\newcommand{\eea}{\end{eqnarray}}

\newcommand{\bi}{\begin{itemize}}
\newcommand{\ei}{\end{itemize}}

\newcommand{\Dt}{\Delta t}
\newcommand{\Dx}{\Delta x}
\newcommand{\Epsilon}{\mathcal{E}}
\newcommand{\etau}{\tau^\text{eqm}}
\newcommand{\wtau}{\widetilde{\tau}}
\newcommand{\xN}{\ave{x}_N}
\newcommand{\Sdata}{S^{\text{data}}}
\newcommand{\Smodel}{S^{\text{model}}}

\newcommand{\del}{D}
\newcommand{\hor}{H}



\setlength{\parindent}{0.0cm}
\setlength{\parskip}{0.4em}

\numberwithin{equation}{section}
\DeclareMathOperator\erf{erf}
%\let\endtitlepage\relax



% https://medium.com/@aerinykim/why-the-normal-gaussian-pdf-looks-the-way-it-does-1cbcef8faf0a

\begin{document}

\tableofcontents 

\newpage

\chapter{Prelude}
I have written this book because I have not found a consise statement defending the morality of eating animals. I find this especially puzzling since philosophers are known to leave no stone left unturned, almost every position you can imagine has been defended \footnote{List some Nazi phislophers, Schmitt, (that guy on youtube), Frege etc}. 

The answer to this odd phenomenon is philosophers are attached to a sort of deductive method of reasoning. It so happens that grand principles often rely on measurable quality. That is, most philosophers refuse to cede ground to intuition or if they do speak of intuition in a sort of formal way\footnote{Huemer, ethical intuitinism}. 

With no intent to being polemical, it seems fairly clear that most philosphers are in fact heavily left-wing. The left-wing are known to work in "movements", that is a popular trend catches on and the left think they have found a new truth that must be striven for. It is perhaps no exageration to say that 99\% of those movements fail to achieve what they aim for, indeed most such movements are forgotten after few hundred years but it is in the credo of this group to cherry pick their succeses and adverise them as if they were always ahead of the curve. 

My intend for this book to be eminenly readable by everyone. I am articulating what I think is simply common sense because the common man has no interest in articulating his common sense, and to the uncharitable academics this is often interpreted as not having an argument. 

Academics forget that we invent our intellectual ideas to aid us in our everyday endeavours. Philosophers are often tempted to change this reasoning, they think that our everyday endeavours are there to aid our intellectual ideas. Ideas such as equality/liberty/freedom/independence etc, represent such a backward reasoning. Philosophers should aim to show how these ideas capure what we are trying to do, for instance it may be that in our everyday endeavours, the heuristics of equality/liberty/freedom/independence make our task easier to analyze and our endeavour less costly, but the way philosophers usually use these concepts is that these are the ultimate ends in themselves. 


It must be remembered that Philosophy is the generator of all knowledge, all modern disciplines were once under the wing of philosophers and as they grew beyond their infancy they became their own disciplines, evolving independely of philosophical trends\footnote{Mathematics, Physics, Anthropology, Psychology, Economics(Adam Smith), etc}. 


\chapter{Introduction}


Each chapter in this book will be independent of the others, so there is not much need to read it linearly. The specific kind of use I expect of this book is as a sort of reference book of arguments. 

Much of the problem in philosophy is being parsimious, it how many ways should one use to divide up the world? I considered when writing the book to demarcate between consequentialist methods of ethics and non-consequentialit methods. Of course the problem with attacking or defending from the categories of consequences is that it is not clear what counts as a consequence. 

The first part of the book will be analyzing existing arguments made my various vegetarians and vegans to explain why these are false. My first chapter will be the most important and most popular argument that is mainted today, that is a minimizing-harm argument. 

After defending the idea that formulating arguments is superfluous for the practicioners, I will try to formulate what I think is a charitable interpreation of their arguments. That is I will present a series of argument FOR meat eating. 

Finally in the last section I will present what I think are some weak ways of formulating pro-eating arguments. 

\chapter{The arguments for not eating animals}

\section{Dynamics of belief}

Though some vegetarians do believe their own arguments, I posit that to most of them, the arguments are secondary. Instead, on a purely descriptive leve, they have some intuition which is closely related to aesthetic reactions. These intuitions then lead the agents to look for articulations of the arguments which cohere with their intuitions. 

In this respect they are no different to other people, a psychologist might call this "motivated reasoning". Indeed even non-vegetarians can have similar aesthetic reactions to vegetarians, for instance, factory like killing gives us a feeling of disgust, one I share. 

The difference is then about the inference that is done after one has this intuition. Vegetarians often make positive assertions, that animals should be treated a certain way. On the other hand, others with more humility, draw less general implications, they simply say that this situation is wrong. 

The problem with simply saying something is wrong is that it does not offer an alternative, but this is by design, they are inviting people to experiment with other worlds. Perhaps we are having this reaction because the animal isn't killed by somebody it doesn't know, and who in turn has no familiarity with the animal. This de-personalization causes revulsion in humans, in a similar way that many would be for un-plugging a man in a coma if the person deciding knows him but not if they do not. Alternatively, one could try to claim that the killing would be justified if the environment of slaughter was different, if there was some ritual showing respect, or if the animal wasn't away from its natural habitat. 

People who enjoy eating meat often have a fundamental intuition about a case where killing an animal is acceptable. The fundamental vision of a farmer raising an animal on his farm and killing it with his own hands and sharing that meat with his community, is still fundamentally sound. Of course, living by this vision probably implies a lower intake of meat than we currently consume. Such a vision is positive, almost platonic, does imply that activism should be directed to return the production process to the farm, get rid of the regulations that force farmers to take them to the slaughterhouse.\footnote{find sources on the EU here}. 

Of course one might object that there are reasons for regulating the slaughter of animals, disease, quality etc. These are indeed legitimate reasons but the the logic can be reversed, instead of regulating so that those things are better controlled, we should be structuring thing such that those things can't do much damage, a farmer's product being eaten by him and his own community is only the begining of such an accountability process. 

Aesthetic reasons dominate, this is not to say that people don't change their minds, but fundamentally they will only change them when presented with alternative aesthetic visions. However there exists a class of philosophers who formalize and think in objects and this class of philosophers is immune to evidence, this class of philosophers are immune to other kinds of arguments.
See this: https://twitter.com/JoshHochschild/status/1242527953101246467

Indeed this is often obvious nobody is informed of an animal being killed on a farm and suddenly becomes a vegetarian.

Much of the attempt in this chapter should read like a philosohical journey vegetarians go through. That is, many of them will stay on the first argument presented, others, will have started here and evolved in the same way I describe here. Many would have skipped this argument altogether and gone further down the chain. 

\section{Minimizing Harm}

\begin{align*}
1) \text{Causing unnecesary harm is bad}
2) \text{Eating animals causes unnecesary harm}
\rightarrow Therefore eating animals is bad
\end{align*}

\subsection{Why these words?}

Though the argument should be obvious, for the pedantic in the room some quick clarifications are in order. \textbf{Bad} is used in a strong sense of "we should not do what is bad". If the person making the argument has a system of ethics that is additive, then presumably they can add or subtract this good to other bads. In such a case, their formulation of the argument would be that it is "bad" in TOTAL.  

\textbf{Harm} was chosen over other words because of it's generality. The argument would also work if we used the words "pain" or "suffering". However the use of alternative words might exclude the concept of "killing". Using those words would then make the argument more open to objections via empirical methods, for instance they could just point to some painless way of killing and the argument would instantly fail. On the other hand, the harm version can survive such an attack, while all attacks on the harm formulation would also work if less general notions were used. This generalization also corresponds to what most vegetarians actually believe. If a very ethical farmer showed up that filmed the painless killing of the animal, it is doubtful that many vegetarians would change their mind and eat this specific animal. 

The problem of "harm" is that it may be general enough to encompass non-consious agents, such as killing a plant or tree. The argument as presented makes reference only to animals but somebody might object: "why only animals and not plants?". In this case, the vegetarian may return to the previous standards of "suffering" or "harm". Alternatively they could commit "organicism", that is arbitrarility discriminate between organic beings based on their categorization. However I suspect the most likely position they will take is that it is not a matter of category but a matter of degree. That is, they will agree that harming plants is bad, but not sufficiently bad given the benefits. That is, one may think that the value of plants is high but the value of humans living is higher. I believe this kind of position automatically locks you into a sort of  "additivity" or comparability of values. 


% http://www.veganfuturenow.com/answering-the-objections-to-veganism#do-you-want-animals-to-have-the-right-to-get-married-and-vote

\subsection{Necesity}

The most obvious problem with the argument is the word, "unnecesary". What does this word mean? It is perhaps best to work with it's negation, necessity. Neccesity is a fairly rare word in that the common use and the philosophical use are identical: Something that must be present for a certain other thing to occur. It makes little linguistic sense to talk of neccesity without linking it to something, there must be a second part to the use of the word, neccesity is a constraint and there must be some objective for the contraint to work on. For instance if I want to make a cake it is necessary that I use the ingredients necessary to make the cake. The sentence "flour is neccesary to make the flour cake" makes sense. The sentence "flour is neccesary" does not make sense. So then it is clear that vegetarians are assuming that there is some goal(cake), which can be achieved through a variety of means. 

What is the "cake" of the suffering of animals? Suppose an agent is trying to get the best "taste" possible, the omega taste. If the omega taste does not require eating animals\footnote{Note here that this is more plausible than it appears, since there are specific labs that aim to create vegetables that emulate the taste of meat} then the argument works, this would be equivalent to saying "don't eat animals because there are better tastes out there". If on the other hand the "omega taste" must include animal flesh, then the argument instantly fails. That is, if I am trying to have the best taste I can, then it IS neccesary that I eat animals. 

Do people eat meat because of the taste? Though I suspect many people do consiously believe they eat meat because they enjoy the taste, evolutionary reasoning actually works backwards: They like the taste because they meat. In other words, the argument should not be taken at face value, people are comfortable eating meat but the reason they eat is not because of the taste. Nevertheless taste plays an important role for habit formation of new generations. In other words agent's are not optimizing creature, they just have a set of habits, there is no sense in speaking of constraints. 

Is this obvious truth, that people have habits and don't analyze their actions and simply do things, a deathblow to philosophers trying to analyze them? A philosopher may be interested in two distinct things, trying to explain the behavior of the the agents, and trying to convince the agents to change their mind. 

If the goal of the philosopher is simply to explain the behavior, he is in fact indifferent to how the agent's decide, instead he is interested in studying their behavior and will simply try to re-formulate his theory to say that agents are acting AS IF they maximize their pleasure. This approach is most interesting for those who with a scientific inclination, it can be used to try and predict the behavior. 

If on the other hand the philosopher is interested in changing the agents mind, then he will try to stop the agent from doing thing unconsiously. This may provoke anger or dismissable from the agent, understandably, a bit like socrates who was known for being really annoying by trying to have people articulate everything. Of course this may be a more dangerous exercise than it seems because making an agent less reliant on one habit may make them doubt their other habits. Of course philosophers may think this is a desirable state of affairs to the desirable, but they are open to Chesterton's fence criticism. 

But let's play the philosophers game and assume for a moment that people are eating meat because they are maximizing some underlying variable. What is their goal if it is not taste? What else can be the optimand of people? Perhaps people are trying simply to optimize their pleasure, but that would simply result in a similar argument to the "taste" argument. Perhaps they are trying to lead a good life, in which case the vegetarian would have to appeal defining the "good life", something many vegetarians don't wish to do because it makes retaining a subjectivist position difficult. If they are willing to empbrace non-subjectivist positions then they would have to fall to objective standards. Nevertheless the most likely turn of vegetarians after reflecting is to use happiness as the standard. 

In other words, the vegetarians will simply try to use the utilitarian standard and attempt to convert meat-eaters to adopt this standard. 

\subsection{Utilitarianism} 

\begin{align*}
1) \text{We ought to maximize total utility}
2) \text{Eating animals does not maximize total utility}
\rightarrow Therefore we ought not to eat animals
\end{align*}

Though the concept of utility sounds rather abstract, essentially the founders of the doctrine meant simply, to mazimize happiness \footnote{Bentham and Mill}. 

Though the doctrine as initially thought up does not neccesarily imply one action over another, when combined with other intuitive ideas it quickly becomes evident why vegetarians take this route. 

The question, is what if one persons happiness comes at the expensve of anothers? Utilitarianism has an answer, whatever action maximizes the \textit{total} happiness should be taken. A funny thing about utilitarianism is that stated brutally, it is an ethical system which is indifferent to the distribution. Which is why, to make it adhere to their intuitions, most philosophers complement it with a second rule. 

Most adherents of this also make a second assumption, "the law of diminishing marginal utility", that is, the enjoyment one person gets for every marginal unit is diminishing. Or the second banana gives me less happiness than the second banana. \footnote{This assumption is questioned in Frankfurts book, equality}. This framework is especially appealing to some left leaning authors, it allows them to justify animal right and income/wealth re-distribution with a single framework. 

How do these two premises, maximize utility, diminishing marginal utility help the vegetarian make his case? The argument is simple enough, animals have a higher utility from livng than humans have from eating animals. This is of course, simply an assertion, it is difficult to counter-argue such a position because it burries all the complexities behind the concept of "total utility". 

The most natural question to ask is, "whose utility?"?, this is an odd. Who do we include? Naturally the vegetarian will ask that we include humans, mammals, and perhaps non-mammalian species, plants, or even the planet? A common idea for who to include is to simply only include those who can feel pain. It is unclear what kind of analysis a utilitarianism will accept for concluding his theory is false or absurd or counter-intuitive. If Andreas cannot feel pain perhaps through some biological disorder, it seems the utilitarian would exclude him, or at least were we to choose if we should whip Andreas or a dog, the utilitarian would say we should whip the Andreas.  They may re-work their criteria for inclusion by making up some other criterion and it is always possible to do this. 

Another question to ask is, how much weight should we put on these utilities? Is the happiness of a fish the same as the happiness of a human? For instance, suppose we have one drug to distribute, if there are two agents who are sick, but one has built up a stoic character such that they feel less pain, does this entail that we ought to give the drug to the agent who feels pain less? Utilitarianism is clear about this, we give it to the person who feels pain more. There is this reverse natural selection at work implicit in utilitarianism. There is also a special of this kind of weighting in utilitarianism that is particularly popular, the Rawlsian case, where we take into account only the utility of the worse off person. 

These two questions, "whose" and "how much", combine to give another problem, the problem of existence. How should we value a being who we can make exist? This is more important than it seems, we could for instance argue that our future selves do not yet exist, let's overlook this detail and pretend that it poses no problem. Let's ask a related question, how should we weigh future generations utility? Economists often discount future utility using a discount rate, but it is unclear how to weigh beings that do not yet exist. A known result of Nordhaus's economic climate model is that it is impossible to justify even moderate measures for mitigation if we do not give close to equal weight to the future generation. Note that even in these models the assumption is that the existence of a being does not depend on our actions. In other words, utilitarianism cannot answer questions about whether to bring a person into existence or not.

The future existence of being is one blind spot for utilitarianism, but so is the past existence of a being. The framework given, gives zero weight to past generations. For instance if a shrine wishes for his son to inherit a shrine and take care of it in solitude, the utilitarian would simply weigh the utility of a single person using it against the utility of turning into a touristic spot. The will of a the dead is only to be given weight as far it increases the weight of the current generation. 

This could be interpreted as "time discrimination". It seems odd that at time t, we give full weight to the utility of an agent, and and time t+1, the agents choices simply don't matter. Indeed it is unclear what the discount rate should be. 

If we imagine that choices are reversible, and each person existing at a time can simply switch the button on or off, and there is no then there is no need for present agents 


\section{Psychoanalysis of veggies}

Many philosophers try to pretend like arguments are important, of course, this would be like phycists saying the material is most important, or a biologist saying organic, economist saying trade is important. 

Of course, while the latter categories may change their minds if presented with a good argument, the philosopher is unlikely to change his because his position is about arguments themselves. When should somebody change their mind? When they hear a good argument! That's their criterion, of course, the problem is that these philosophers are usually attracted by elegant views. However there is no argument about why reality would fits simple arguments better than complicated arguments(Huemer). Of course science can claim parsimony is important because predictive capacity plays a vital role. But philosophy aims at ethics, and there is no reason ethics needs this kind of parsimony. 

There is no reason some philosophical system which treats every situtuation differently is better or worse than one which has specific criterion which is evaluated universally. Indeed, if in situation A use criteria X, if in situation B use criterion Y, can simply work. 


%%%%%%%%%%%%%%%%%%%%%%%%%%%%%%%%%%%%%%%%%%%%%%%%%%%%%%%%%%%%%%%%%%%%%%%%%%
%%%%%%%%%%%%%%%%%%%%%%%%%%%%%%%%%%%%%%%%%%%%%%%%%%%%%%%%%%%%%%%%%%%%%%%%%%
%%%%%%%%%%%%%%%%%%%%%%%%%%%%%%%%%%%%%%%%%%%%%%%%%%%%%%%%%%%%%%%%%%%%%%%%%%
%%%%%%%%%%%%%%%%%%%%%%%%%%%%%%%%%%%%%%%%%%%%%%%%%%%%%%%%%%%%%%%%%%%%%%%%%%
%%%%%%%%%%%%%%%%%%%%%%%%%%%%%%%%%%%%%%%%%%%%%%%%%%%%%%%%%%%%%%%%%%%%%%%%%%
%%%%%%%%%%%%%%%%%%%%%%%%%%%%%%%%%%%%%%%%%%%%%%%%%%%%%%%%%%%%%%%%%%%%%%%%%%
The choice is self evident if the production method is less costly, but less so if it is more costly. 


\section{An alternative picture of humans}

It almost seems caricatural, to try and talk of people in this way, that is people don't have objectives they are trying to optimize. Instead they have goals they wish to achieve, this may seem like just a linguistic difference but from the analytical point of view it flips it all around. There is a list of things a person hopes and desires to have, is meat neccesary for the achievement of any of those goals? If it isn't neccesary maybe they only consume meat because it makes it "easier", to be more precise, perhaps eating meat allows them to meet more of their goals. Once again we are pulled into the empirical world, a massive can of worms is opened, perhaps there is a vegan rich person who will give you money and help you achieve more of your goals if you don't eat meat. 

% So with this new understanding of the word neccesary. We can go back to the animal example and ask, what is the "cake" of the suffering of animals? Typically, vegetarians will go after the "taste" argument.

% The taste argument is related to the empirical world in a peculiar way. If it is true that eating animals is the only way to produce that taste, then the argument is clearly false, because CLEARLY it is neccesary to cause harm to create the taste.

% If we CAN emulate the taste then that means we can create that taste without suffering. 
\section{Speciesm}

The arbitrary nature of speciesm, but false since superman would have our moral compas, as well as a moral ape or animal. 

\section{Cost-benefit}

\subsection{calculus}

Another argument vegetarian may make has more intuitive appeal:
If the costs exceed the benefits, it is bad
The suffering from animals exceeds the benefits.
Therefore you ought not to eat animals. 

This may seem like a striking argument, the obvious question to ask is "how do you know?" More specifically:

How come 1) we can measure the benefit and the harm? What exactly makes these categories measurable? It is perhaps intuitive that every human can measure their own suffering, it is less clear that they can measure their own happiness or joy. But even if they could, this is different than actually measuring the happiness of another. 

2) the measures we come up with are comparable? Did we assert that these are of the same kind? How do we know it isn't like comparing temperature to distance? Even if we suppose that the two are measurable, how come they also also comparable?  

3) The benefit is lower than the harm? How come the benefit is lower than the harm? It is obvious here that the vegetarian must attempt to define what the benefit is. 

\section{Equal consideration}

\section{Ecological:Plants}

Vegetarians will often use empirical arguments to attack this kind of reasoning, that is, the sheet magnitude of plants needed to make an animal live. 


\chapter{The good arguments for eating animals}

\section{The extreme case: What if you have to?}

\chapter{The bad arguments for eating animals}

\section{It is natural}

\begin{align}
\text{Eating Meat is natural} \\
\text{Doing what is natural is good} \\
\text{eating meat is good.}
\end{align}

Failed arguments;


They sometimes dispute the naturalness but most evolutionary biologists agree that we have evolved because we have used less energy on digestion and more on brains.  

\section{We are superior}




\section{Harm and necceccity}





\subsection{Lifestyle preferences}

It is often assumed that good is attributable to specific actions. However there is a complete failure to articulate the discontinuities of life. Archtypes or cutlural goods are non reducible. It is NOT true that you can remove the olive oil from the greek diet and still have the greek diet. Some things are just fundamental. 

How people measure the good and bad is with lifestyles, not with individual deeds. In other words, one may in theory be content with their daughter sleeping around with a different guy every night. However the repugnance to this may in fact be with the lifestyle that is associated, the alienation from feelings, the frequenting of night clubs. When someone tells you they want 
 x, this is not necessarily because they want x in itself. Indeed it may be that they simply want things that are associated with x. 

For animals specifically the reason is easy enough to see. I want a reason to live with animals. Or I want a reason for animals to exist or to exist in greater number. And by wanting x, I am creating that reason. 

One may try to envision different ways that we can live with animals, but it is ultimately an empirical question. One such way we could imagine is to worship the animals, perhaps between the gaps we have at work, we could go to the chicken altars and worship them. One may imagine simply that these animals are publicly financed to live in the streets, were they are fed or their poop is cleaned at public expense. This all sounds reasonable, and it is likely a few meat eaters would be convinced if this was shown to work in practice.

Of course if the animal rights activists position becomes less extreme the answer is quite natural for a few animals. Perhaps cows sheep goats and chickens could all give us ample reason to live with them. But still there remain animals which don't have this property, such as pigs. 

It is wrong to want to have kids for your own pleasure. You should want to have kids because your gut tells you to. You can't articulate WHY you should have kids, but there is this tendency in you to have kids. It is not that you think kids will make you happier, but that having kids itself is what life is about, it is is simply the expression of who you are, just like a flute gains no pleasure from being played, but it is its purpose. Indeed it is it's very reason for existing, it is the cause of it existing. 

Dogs have found their place with us and most of us are thankful. Perhaps a significant difference between dogs and other animals. 

It is perhaps a typically modern tendency to try and calculate the costs and benefits of all structures.  The search for deductive beauty is often a homogenizing force. However inductive beauty gives very different results. Deductively we may have a positive impression of an idillic community where everyone has the same life. But inductively there is something disgusting in knowing that everything is the same everywhere. Diversity is a value in itself, we want everyone to plant different plants in their gardens. We can imagine that the diversity is such that everyone plants the same different plants. 


\subsection{Formulation vs induction}
It seems like philosophers have taken up a task that the ancients never dared. That is to assume that if an argument cannot be formulated, that the position is incoherent. It is hard to imagine a more arrogant position than the position that if something cannot be defended, it should be abandoned. 

In reality even though you cannot defend an action as such, you can imagine reasons why it emerged. And without whosing that those reasons are no longer neccesry, it is incoherent to throw it away. 

\subsection{Against false principles}

How do we know if a principle is good? If it accords with our intuitions. Indeed almost all tragedies in the human race follow this same pattern. "My principle is good, therefore x is justified". in reality people should just see if a principle holds true to their intuition and then go with it. The animal rights activists are a failure because they go by the principles of minimize unnecesary suffering, or some other weird principle. In reality the reasoning we should be following is the opposite, "eating animals is okay" therefore the principle of minimizin suffering is false. 

There is this class of arguments which intuitively many people have which sound immoral. For instance many people will fall back into the notion of "what if you have to?", clearly morality takes into account neccesity. If something is immoral, your obligation does not somehow change because of your need. The the clear formulation of this argument would be:

\begin{align}
1) \text{It was immoral to kill animals then.} \\
2) \text{If our ancestors didn't kill animals they would not have evolved as they did} \\
1&2 \rightarrow \text{We are buit on immorality}
\end{align}

This line of argument implies the world would be more moral if humans had never existed. 




 

\bibliography{../thesisbib/bibliography}

\end{document}

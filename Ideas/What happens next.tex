%\documentclass[AER]{AEA}
\documentclass[12pt]{report}
%\documentclass[12pt]{article}
%\documentclass[12pt,a4paper]{article}

\usepackage[utf8]{inputenc}


\usepackage{mathtools}
\usepackage{amsmath}
\usepackage{amssymb}
\usepackage{amsthm}

\usepackage{float}
%\usepackage[cmbold]{mathtime}
%\usepackage{mt11p}
\usepackage{placeins}
\usepackage{caption}
\usepackage{color}
\usepackage{subfigure}
\usepackage{multirow}
\usepackage{epsfig}
\usepackage{listings}
\usepackage{enumitem}
\usepackage{rotating,tabularx}
%\usepackage[graphicx]{realboxes}
\usepackage{graphicx}
\usepackage{graphics}
\usepackage{epstopdf}
\usepackage{longtable}
\usepackage[pdftex]{hyperref}
%\usepackage{breakurl}
\usepackage{epigraph}
\usepackage{xspace}
\usepackage{amsfonts}
\usepackage{eurosym}
\usepackage{ulem}

\usepackage{tikz}
\usetikzlibrary{spy}

\usepackage{verbatim}



\usepackage{footmisc}
\usepackage{comment}
\usepackage{setspace}
\usepackage{geometry}
\usepackage{caption}
\usepackage{pdflscape}
\usepackage{array}
\usepackage[authoryear]{natbib}
\usepackage{booktabs}
\usepackage{dcolumn}
\usepackage{mathrsfs}
%\usepackage[justification=centering]{caption}
%\captionsetup[table]{format=plain,labelformat=simple,labelsep=period,singlelinecheck=true}%
\bibliographystyle{apalike}
%\bibliographystyle{unsrtnat}



%\bibliographystyle{aea}
\usepackage{enumitem}
\usepackage{tikz}
\usetikzlibrary{positioning}
\usetikzlibrary{arrows}
\usetikzlibrary{shapes.multipart}

\usetikzlibrary{shapes}
\def\checkmark{\tikz\fill[scale=0.4](0,.35) -- (.25,0) -- (1,.7) -- (.25,.15) -- cycle;}
%\usepackage{tikz}
%\usetikzlibrary{snakes}
%\usetikzlibrary{patterns}

%\draftSpacing{1.5}

\usepackage{xcolor}
\hypersetup{
colorlinks,
linkcolor={blue!50!black},
citecolor={blue!50!black},
urlcolor={blue!50!black}}

%\renewcommand{\familydefault}{\sfdefault}
%\usepackage{helvet}
%\setlength{\parindent}{0.4cm}
%\setlength{\parindent}{2em}
%\setlength{\parskip}{1em}

%\normalem

%\doublespacing
\onehalfspacing
%\singlespacing
%\linespread{1.5}

\newtheorem{theorem}{Theorem}
\newtheorem{corollary}[theorem]{Corollary}
\newtheorem{proposition}{Proposition}
\newtheorem{definition}{Definition}
\newtheorem{axiom}{Axiom}
\newtheorem{observation}{Observation}
\newtheorem{assumption}{Assumption}	
\newtheorem{remark}{Remark}
\newtheorem{lemma}{Lemma}
\newtheorem{result}{result}


\newcommand{\ra}[1]{\renewcommand{\arraystretch}{#1}}

\newcommand{\E}{\mathrm{E}}
\newcommand{\Var}{\mathrm{Var}}
\newcommand{\Corr}{\mathrm{Corr}}
\newcommand{\Cov}{\mathrm{Cov}}

\newcolumntype{d}[1]{D{.}{.}{#1}} % "decimal" column type
\renewcommand{\ast}{{}^{\textstyle *}} % for raised "asterisks"

\newtheorem{hyp}{Hypothesis}
\newtheorem{subhyp}{Hypothesis}[hyp]
\renewcommand{\thesubhyp}{\thehyp\alph{subhyp}}

\newcommand{\red}[1]{{\color{red} #1}}
\newcommand{\blue}[1]{{\color{blue} #1}}

%\newcommand*{\qed}{\hfill\ensuremath{\blacksquare}}%

\newcolumntype{L}[1]{>{\raggedright\let\newline\\arraybackslash\hspace{0pt}}m{#1}}
\newcolumntype{C}[1]{>{\centering\let\newline\\arraybackslash\hspace{0pt}}m{#1}}
\newcolumntype{R}[1]{>{\raggedleft\let\newline\\arraybackslash\hspace{0pt}}m{#1}}

%\geometry{left=1.5in,right=1.5in,top=1.5in,bottom=1.5in}
\geometry{left=1in,right=1in,top=1in,bottom=1in}

\epstopdfsetup{outdir=./}

\newcommand{\elabel}[1]{\label{eq:#1}}
\newcommand{\eref}[1]{Eq.~(\ref{eq:#1})}
\newcommand{\ceref}[2]{(\ref{eq:#1}#2)}
\newcommand{\Eref}[1]{Equation~(\ref{eq:#1})}
\newcommand{\erefs}[2]{Eqs.~(\ref{eq:#1}--\ref{eq:#2})}

\newcommand{\Sref}[1]{Section~\ref{sec:#1}}
\newcommand{\sref}[1]{Sec.~\ref{sec:#1}}

\newcommand{\Pref}[1]{Proposition~\ref{prop:#1}}
\newcommand{\pref}[1]{Prop.~\ref{prop:#1}}
\newcommand{\preflong}[1]{proposition~\ref{prop:#1}}

\newcommand{\Aref}[1]{Axiom~\ref{ax:#1}}

\newcommand{\clabel}[1]{\label{coro:#1}}
\newcommand{\Cref}[1]{Corollary~\ref{coro:#1}}
\newcommand{\cref}[1]{Cor.~\ref{coro:#1}}
\newcommand{\creflong}[1]{corollary~\ref{coro:#1}}

\newcommand{\etal}{{\it et~al.}\xspace}
\newcommand{\ie}{{\it i.e.}\ }
\newcommand{\eg}{{\it e.g.}\ }
\newcommand{\etc}{{\it etc.}\ }
\newcommand{\cf}{{\it c.f.}\ }
\newcommand{\ave}[1]{\left\langle#1 \right\rangle}
\newcommand{\person}[1]{{\it \sc #1}}

\newcommand{\AAA}[1]{\red{{\it AA: #1 AA}}}
\newcommand{\YB}[1]{\blue{{\it YB: #1 YB}}}

\newcommand{\flabel}[1]{\label{fig:#1}}
\newcommand{\fref}[1]{Fig.~\ref{fig:#1}}
\newcommand{\Fref}[1]{Figure~\ref{fig:#1}}

\newcommand{\tlabel}[1]{\label{tab:#1}}
\newcommand{\tref}[1]{Tab.~\ref{tab:#1}}
\newcommand{\Tref}[1]{Table~\ref{tab:#1}}

\newcommand{\be}{\begin{equation}}
\newcommand{\ee}{\end{equation}}
\newcommand{\bea}{\begin{eqnarray}}
\newcommand{\eea}{\end{eqnarray}}

\newcommand{\bi}{\begin{itemize}}
\newcommand{\ei}{\end{itemize}}

\newcommand{\Dt}{\Delta t}
\newcommand{\Dx}{\Delta x}
\newcommand{\Epsilon}{\mathcal{E}}
\newcommand{\etau}{\tau^\text{eqm}}
\newcommand{\wtau}{\widetilde{\tau}}
\newcommand{\xN}{\ave{x}_N}
\newcommand{\Sdata}{S^{\text{data}}}
\newcommand{\Smodel}{S^{\text{model}}}

\newcommand{\del}{D}
\newcommand{\hor}{H}



\setlength{\parindent}{0.0cm}
\setlength{\parskip}{0.4em}

\numberwithin{equation}{section}
\DeclareMathOperator\erf{erf}
%\let\endtitlepage\relax
\DeclarePairedDelimiter\floor{\lfloor}{\rfloor}


% https://medium.com/@aerinykim/why-the-normal-gaussian-pdf-looks-the-way-it-does-1cbcef8faf0a

\begin{document}

\section{Introduction}

Suppose we have an agent who is going to make some choice at some specific point in time. How should the agent think of this choice? One way would be to assume that after this choice is made, the choice will result in some outcome, and the agent will simply live with that outcome for the rest of his life. This seems like it would give the same result as if the agent would die after the choice is given.

Another possible option is to assume that the agents choice will lead to some outcome but this is not the end. If it is not the end but other choices are to happen after this choice then we need to make an assumption about what happens next, wanting to make the fewest possible assumptions we say, let's just assume the same choice will be presented to you next time you choose. 

What does "same" mean? How can the choice be the same? Same choice can mean either that the objective choice set is the same, OR it can mean that the effect of the choice set is the same. An example of the former, would be if the initial choice was between an orange and an apple, and the future choice was also between an orange and an apple. In the second notion of "same" we can interpret that the effect of the choices on the agent are identical. 

It is not obviously clear if one of these notions is more realistic than the other. Their respective realism depends on detailing the problem. It may indeed be that if one asks someone if they prefer an apple or an orange, they will prefer an apple but if we ask the same question 5 times in quick succession, they may at some point choose an orange. However if we ask them once a year for five years, we might expect the choice to change. 

If we assume that the agent has some state the interpretation is clearer. We can say that either the choice will affect the agents state or it will not. If it does not affect the state, then the notion of "same" of the future choice is clear, it is simply the same set of goods that will be presented to the agent. If on the other hand the first choice affects the state of the agent, then the notion of having the same objective choice set does not represent the thing. The same choice set can either be more or less important to the agent than before. The parsimonious assumption to make about the an agent with a state variable is that all choices affect the state variable in the same way. 

If we take the state of the agent to be the agents wealth, then there the two kinds of assumptions start becoming clearer. Suppose we ask the agent if he would rather receive 10 dollars or have a 10\% chance of receiving 100 dollars, call it set A. What would the agents "next" choice be? In one interpretation it would be that the future choice is also set A. Or it could be that the future choice set is some function of the state variable. 





\section{Numerical Example:}

It is often the case in economics that when we analyze choice sets throughout time we assume that an individual choice does not affect the set of future choices. Incorporating this feauture into a time frame is very simple. We will illustrate this through a detailed example: 

There is a bureau of architects which has n architects. This bureau receives projects at regular time intervals, say a month and can assign a single architect to it. The bureau can accept either a 10 month project yielding 100\$ or a 2 month project yielding 50\$. Each project takes up a single architect. How should the bureau decide which to accept? The answer is that it depends on the number of architects. For simplicity, assume the firm does not have access to an interest rate and that the project offer occurs $\epsilon$ time after existing projects offers end. The firm can accept two projects at the same time. 

Suppose that every future choice is also between a 10 month project and a two month project. How many architects does the firm need to accept the 10 month project every time? Since projects arrive every month and an architect working on a 10 month project will only be freed after 10 months, then the bureau needs 10 architects to be able to always choose the 10 month project. In fact in this case, the firm that wishes to maximize long term growth will simply take the highest payoff on every choice, 100\$. This leads us to the first lesson:

\begin{remark}
If future projects are identical to present projects, a profit maximizing firm does not take time into account. 
\end{remark}

Suppose now that the firm does not have 10 architects but 5. How should the firm evaluate projects? The problem here is slightly more complicated. If it puts all 5 on long term projects, consider what occurs. Every 10 months, there will be 5 months where the firm will receive 100 per month. So it's growth rate will be 500 every 15 months, $500/15months=33.3$. Consider now what happens if the firm only takes the short term projects of two months. In this case, every $15*50/15 months=50$. However, since it has 5 architects and projects only take $2 months$ and arrive every month, this implies that there will always be 3 architects which will be idle. 

The firm can mix between long term and short term projects. Specifically, the optimal mix of long term to short term projects will be $3$ long and $2$ short. This will mean that every 15 months, the firm will earn, $15*50+300/15=88.88$, however there will also be some years where the firm will wish to be able to earn more than this, due to the time structure of the projects, for instance if we complete 3 long term projects in the first 3 months, then the growth rate that year will be $15*50+600/15$. 

The firm then must then reason depending on how its current choice affects its current choices. If the firm has more capacity than neccesary for future choices, then it can always just take the highest project but as soon as the current choice affect the capacity to undertake future projects, then it must reason in an elastic way. 

\begin{remark}
The firm will simply use the highest payout and ignore the time structure so long as it does not affect the ability to choose future projects. 
\end{remark}

To see this we need only note that since three people are always idle, then the choice only takes a different structure when the three people are already working and will be working for at least another four months. If we are in this scenario where the firm is at the capacity constraint, then the firm will be comparing the growth rate per unit of time, that is, it will look at the \textit{within} project growth rate. For instance in this case the firm will simply compare $50/2=25$ to $100/10=10$. 

So a firm has to use both kinds of criteria for decision making. So when would a firm be indifferent between these two payments. When they give the same growth rate. This implies that the firm is indifferent between the time structure of projects if they give the same payout. In fact this is exactly the same thing as the definition of a discount factor, that is, a discount factor is what you need to multiply the future payment so that you are indifferent between the future and present payment. $50= \delta 250$, which means the firm will effectively be discounting by: $\delta = .2$, however we can also recover the discount factor by analyzing the time structure in this case it would be $\frac{1}{1+\frac{8}{2}}=\frac{1}{1+\frac{8}{2}}=.2$

In other words, once we know whether the firm is capacity constrained or not, we should be able to deduce a hyperbolic discount rate from its behavior. This reasoning is not unique to firms, indeed this reasoning could apply with consumers choosing or even animals. The optimality of hyperbolic discounting depends only on whether agents will conisder options that result in the same after choices, if the options do not give the same after choices, then the agents will end up hyperbolically discounting. 

\section{Theory}

It is often the case in economics that when we analyze choice sets throughout time we assume that an individual choice does not affect the set of future choices. Incorporating this feauture into a time frame is very simple. 

\newpage

To simplify we assume there are two kinds of projects, a long term project denoted by $l$ and a short term project $s$. Suppose the agents capacity is denoted by $n$ (this can be money or employees etc). The firm receives take it or leave it offers for projects it can undertake at regular time intervals, $\tau$. The number of offers it receives are $m_l$ and $m_s$ and can assign capacity to each project. The bureau can accept either a $t_l$ month project yielding $x_l$\$ or a $t_s$ project yielding $x_s$\$. The projects occupy agents, $n_l$ and $n_s$. 

% Consider from $t_0$ to $T$:
% \begin{align*}
% & 0 && \text{ if: } \tau > t_i \text{  and: } n_i>n \\
% &x_i\floor*{\frac{T}{\tau}} \floor*{\frac{n}{n_i}} \text{ or } x_i\left(\floor*{\frac{T}{\tau}}-1\right) \floor*{\frac{n}{n_i}} && \text{ if: } \tau > t_i \text{  and: } n_i<n<n_i m_i \\
% &x_i\floor*{\frac{T}{\tau}} m_i \text{ or } x_i\left(\floor*{\frac{T}{\tau}}-1\right) m_i  && \text{ if: } \tau > t_i \text{  and: } n_i<n_im_i<n \\
% &\frac{x_i t_i}{\tau} \frac{n}{n_i} && \text{ if: } \tau < t_i \text{  and: } \frac{t_i}{\tau} n_i m_i<n \\
% % &\frac{x_i t_i}{\tau}m_i && \text{ if: } \tau < t_i \text{  and: } \frac{t_i}{\tau} n_i m_i<n \\
% &\frac{x_i }{\tau} \frac{n}{n_i} m_i && \text{ if: } \tau < t_i \text{  and: } \frac{t_i}{\tau} n_i m_i>n
% \end{align*}


The maximum growth rate of the firm specializing is given by:
\begin{align*}
& 0 && \text{ if: } \tau > t_i \text{  and: } n_i>n \\
& \frac{x_i}{\tau} \floor*{\frac{n}{n_i}} && \text{ if: } \tau > t_i \text{  and: } n_i<n<n_i m_i \\
&\frac{x_i}{\tau} m_i && \text{ if: } \tau > t_i \text{  and: } n_i<n_im_i<n \\
&\frac{x_i }{\tau} m_i && \text{ if: } \tau < t_i \text{  and: } \frac{t_i}{\tau} n_i m_i<n \\
% &\frac{x_i t_i}{\tau}m_i && \text{ if: } \tau < t_i \text{  and: } \frac{t_i}{\tau} n_i m_i<n \\
&\frac{x_i }{t_i} \floor*{\frac{n}{n_i}} && \text{ if: } \tau < t_i \text{  and: } \frac{t_i}{\tau} n_i m_i>n
\end{align*}

The average number of idle capacity is then:  

\begin{align*}
& n && \text{ if: } \tau > t_i \text{  and: } n_i>n \\
& \frac{ \floor{ \frac{n_i m_i}{n}} n_i(\tau - t_i)+(n-\floor{ \frac{n_i m_i}{n}}n_i) \tau }{\tau} = n - \frac{n_i \floor{ \frac{n_i m_i}{n}} t_i}{\tau} && \text{ if: } \tau > t_i \text{  and: } n_i<n<n_i m_i \\
& \frac{(n_i m_i)(\tau - t_i)+n-n_i m_i}{\tau}= n - \frac{n_i m_i t_i}{\tau} && \text{ if: } \tau > t_i \text{  and: } n_i<n_im_i<n \\
& n-\frac{n_i m_i t_i}{\tau} && \text{ if: } \tau < t_i \text{  and: } \frac{t_i}{\tau} n_i m_i<n \\
% &\frac{x_i t_i}{\tau}m_i && \text{ if: } \tau < t_i \text{  and: } \frac{t_i}{\tau} n_i m_i<n \\
&n-\floor{\frac{n}{n_im_i}}n_im_i + \frac{\tau n_i m_i}{t_i}  && \text{ if: } \tau < t_i \text{  and: } \frac{t_i}{\tau} n_i m_i>n
\end{align*}

Two projects:
\begin{align*}
& \text{maxarg}_k \{ x_l\frac{n-k}{n_l \tau}+x_s\frac{k}{n_s \tau} \}  && \text{ if: } \tau > t_l>t_s \\
&x_i \frac{1}{\tau} && \text{ if: } t_l> \tau > t_s \\
&x_i \frac{t_i}{\tau} && \text{ if: } \tau < t_i \text{  and: } \frac{t_s}{\tau}<\frac{t_l}{\tau}<\frac{t_s}{\tau}+\frac{t_l}{\tau}<n \\
&x_i \frac{n}{\tau} && \text{ if: } \tau < t_i \text{  and: } \frac{t_i}{\tau}>n
\end{align*}


\bibliography{../thesisbib/bibliography}

\end{document}

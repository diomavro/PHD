%\documentclass[AER]{AEA}
\documentclass[12pt]{report}
%\documentclass[12pt]{article}
%\documentclass[12pt,a4paper]{article}

\usepackage[utf8]{inputenc}


\usepackage{mathtools}
\usepackage{amsmath}
\usepackage{amssymb}
\usepackage{amsthm}

\usepackage{float}
%\usepackage[cmbold]{mathtime}
%\usepackage{mt11p}
\usepackage{placeins}
\usepackage{caption}
\usepackage{color}
\usepackage{subfigure}
\usepackage{multirow}
\usepackage{epsfig}
\usepackage{listings}
\usepackage{enumitem}
\usepackage{rotating,tabularx}
%\usepackage[graphicx]{realboxes}
\usepackage{graphicx}
\usepackage{graphics}
\usepackage{epstopdf}
\usepackage{longtable}
\usepackage[pdftex]{hyperref}
%\usepackage{breakurl}
\usepackage{epigraph}
\usepackage{xspace}
\usepackage{amsfonts}
\usepackage{eurosym}
\usepackage{ulem}

\usepackage{tikz}
\usetikzlibrary{spy}

\usepackage{verbatim}



\usepackage{footmisc}
\usepackage{comment}
\usepackage{setspace}
\usepackage{geometry}
\usepackage{caption}
\usepackage{pdflscape}
\usepackage{array}
\usepackage[authoryear]{natbib}
\usepackage{booktabs}
\usepackage{dcolumn}
\usepackage{mathrsfs}
%\usepackage[justification=centering]{caption}
%\captionsetup[table]{format=plain,labelformat=simple,labelsep=period,singlelinecheck=true}%
\bibliographystyle{apalike}
%\bibliographystyle{unsrtnat}



%\bibliographystyle{aea}
\usepackage{enumitem}
\usepackage{tikz}
\usetikzlibrary{positioning}
\usetikzlibrary{arrows}
\usetikzlibrary{shapes.multipart}

\usetikzlibrary{shapes}
\def\checkmark{\tikz\fill[scale=0.4](0,.35) -- (.25,0) -- (1,.7) -- (.25,.15) -- cycle;}
%\usepackage{tikz}
%\usetikzlibrary{snakes}
%\usetikzlibrary{patterns}

%\draftSpacing{1.5}

\usepackage{xcolor}
\hypersetup{
colorlinks,
linkcolor={blue!50!black},
citecolor={blue!50!black},
urlcolor={blue!50!black}}

%\renewcommand{\familydefault}{\sfdefault}
%\usepackage{helvet}
%\setlength{\parindent}{0.4cm}
%\setlength{\parindent}{2em}
%\setlength{\parskip}{1em}

%\normalem

%\doublespacing
\onehalfspacing
%\singlespacing
%\linespread{1.5}

\newtheorem{theorem}{Theorem}
\newtheorem{corollary}[theorem]{Corollary}
\newtheorem{proposition}{Proposition}
\newtheorem{definition}{Definition}
\newtheorem{axiom}{Axiom}
\newtheorem{observation}{Observation}
\newtheorem{assumption}{Assumption}	
\newtheorem{remark}{Remark}
\newtheorem{lemma}{Lemma}
\newtheorem{result}{result}


\newcommand{\ra}[1]{\renewcommand{\arraystretch}{#1}}

\newcommand{\E}{\mathrm{E}}
\newcommand{\Var}{\mathrm{Var}}
\newcommand{\Corr}{\mathrm{Corr}}
\newcommand{\Cov}{\mathrm{Cov}}

\newcolumntype{d}[1]{D{.}{.}{#1}} % "decimal" column type
\renewcommand{\ast}{{}^{\textstyle *}} % for raised "asterisks"

\newtheorem{hyp}{Hypothesis}
\newtheorem{subhyp}{Hypothesis}[hyp]
\renewcommand{\thesubhyp}{\thehyp\alph{subhyp}}

\newcommand{\red}[1]{{\color{red} #1}}
\newcommand{\blue}[1]{{\color{blue} #1}}

%\newcommand*{\qed}{\hfill\ensuremath{\blacksquare}}%

\newcolumntype{L}[1]{>{\raggedright\let\newline\\arraybackslash\hspace{0pt}}m{#1}}
\newcolumntype{C}[1]{>{\centering\let\newline\\arraybackslash\hspace{0pt}}m{#1}}
\newcolumntype{R}[1]{>{\raggedleft\let\newline\\arraybackslash\hspace{0pt}}m{#1}}

%\geometry{left=1.5in,right=1.5in,top=1.5in,bottom=1.5in}
\geometry{left=1in,right=1in,top=1in,bottom=1in}

\epstopdfsetup{outdir=./}

\newcommand{\elabel}[1]{\label{eq:#1}}
\newcommand{\eref}[1]{Eq.~(\ref{eq:#1})}
\newcommand{\ceref}[2]{(\ref{eq:#1}#2)}
\newcommand{\Eref}[1]{Equation~(\ref{eq:#1})}
\newcommand{\erefs}[2]{Eqs.~(\ref{eq:#1}--\ref{eq:#2})}

\newcommand{\Sref}[1]{Section~\ref{sec:#1}}
\newcommand{\sref}[1]{Sec.~\ref{sec:#1}}

\newcommand{\Pref}[1]{Proposition~\ref{prop:#1}}
\newcommand{\pref}[1]{Prop.~\ref{prop:#1}}
\newcommand{\preflong}[1]{proposition~\ref{prop:#1}}

\newcommand{\Aref}[1]{Axiom~\ref{ax:#1}}

\newcommand{\clabel}[1]{\label{coro:#1}}
\newcommand{\Cref}[1]{Corollary~\ref{coro:#1}}
\newcommand{\cref}[1]{Cor.~\ref{coro:#1}}
\newcommand{\creflong}[1]{corollary~\ref{coro:#1}}

\newcommand{\etal}{{\it et~al.}\xspace}
\newcommand{\ie}{{\it i.e.}\ }
\newcommand{\eg}{{\it e.g.}\ }
\newcommand{\etc}{{\it etc.}\ }
\newcommand{\cf}{{\it c.f.}\ }
\newcommand{\ave}[1]{\left\langle#1 \right\rangle}
\newcommand{\person}[1]{{\it \sc #1}}

\newcommand{\AAA}[1]{\red{{\it AA: #1 AA}}}
\newcommand{\YB}[1]{\blue{{\it YB: #1 YB}}}

\newcommand{\flabel}[1]{\label{fig:#1}}
\newcommand{\fref}[1]{Fig.~\ref{fig:#1}}
\newcommand{\Fref}[1]{Figure~\ref{fig:#1}}

\newcommand{\tlabel}[1]{\label{tab:#1}}
\newcommand{\tref}[1]{Tab.~\ref{tab:#1}}
\newcommand{\Tref}[1]{Table~\ref{tab:#1}}

\newcommand{\be}{\begin{equation}}
\newcommand{\ee}{\end{equation}}
\newcommand{\bea}{\begin{eqnarray}}
\newcommand{\eea}{\end{eqnarray}}

\newcommand{\bi}{\begin{itemize}}
\newcommand{\ei}{\end{itemize}}

\newcommand{\Dt}{\Delta t}
\newcommand{\Dx}{\Delta x}
\newcommand{\Epsilon}{\mathcal{E}}
\newcommand{\etau}{\tau^\text{eqm}}
\newcommand{\wtau}{\widetilde{\tau}}
\newcommand{\xN}{\ave{x}_N}
\newcommand{\Sdata}{S^{\text{data}}}
\newcommand{\Smodel}{S^{\text{model}}}

\newcommand{\del}{D}
\newcommand{\hor}{H}



\setlength{\parindent}{0.0cm}
\setlength{\parskip}{0.4em}

\numberwithin{equation}{section}
\DeclareMathOperator\erf{erf}
%\let\endtitlepage\relax



% https://medium.com/@aerinykim/why-the-normal-gaussian-pdf-looks-the-way-it-does-1cbcef8faf0a

\begin{document}

There is this axiom in decision theory called the axiom of irrelevant alternatives. It is simple enough, if I choose A when i'm choosing between A, B and C, then I should also choose A when choosing between A and B. To see why this is intuitive let's take a simple example: 
Nurse: Doctor do you think he is sick or healthy?
Doctor: Ugh, healthy.
Nurse: You know you can also say that it's not determined yet, maybe you need further testing?
Doctor: Oh in that case he is sick. 

This may seem like it would never occur in real life but actually it can occur all the time. It will occur whenever people have two decision criteria. For instance if somebody goes to the supermarket with something like "I will buy either the cheapest halloumi or the best halloumi". The problem is, that unless you have a specific measurement that allows quality and price comparison it will be very difficult. This is because unless the cheapest is ALSO the best, I will need to have a secondary criteria. If am holding the best halloumi on the left hand and the cheapest halloumi in my right hand, I need to have a secondary criteria, maybe the prettiest Halloumi. Well whenver I am applying a secondary criteria which is a refinement on my primary criteria, then the problem can emerge. 

More literary:
I went to go buy halloumi the other day. I could drive to the super-market that's close and get Christie's, or go little further and get Kesses, or I could go for the real stuff and just go to Pitsilia. I have two primary criteria, closeness and quality. If I used closeness, I would get the Christies, if was using quality I would go to Pitsilia. So i'm either going to choose Pitsilia or Christies, but how do I choose? I need a secondary criteria, i'm just gonna get the prettiest looking one, so I will get Christies. 

The next week, I had the same craving for a halloumi and was going to use the same criteria. Except that they had blocked the roads to pitsilia. So the highest quality was Kesses, it also happens to be the prettiest one so it passes round two. So when the road to Pitsilia is closed I get a Kesses from the further super-market and when the road is open I get Christies. 

Why we should non-overlapping primary goals.

Πήγα να πάω να αγοράσω χολούμι την άλλη μέρα. Θα μπορούσα να οδηγήσω στη σούπερ-αγορά που είναι κοντά και να πάρω τη Christie's, ή να πάω λίγο περισσότερο(further) και να πάρω Kesses, ή θα μπορούσα να πάω για τα πραγματικά πράγματα και απλά να πάω στην Pitsilia. Έχω δύο βασικά κριτήρια, εγγύτητα και ποιότητα. Αν χρησιμοποιούσα την εγγύτητα, θα έκανα τα Christies, αν χρησιμοποιούσα ποιότητα θα πήγαινα στην Πιτσιλιά. Επομένως, είμαι είτε πρόκειται να επιλέξω Pitsilia ή Christies, αλλά πώς να επιλέξω; Χρειάζομαι ένα δευτερεύον κριτήριο, θα πάρω μόνο τον πιο ωραίο, έτσι θα πάρω τις Christies.

Την επόμενη εβδομάδα, είχα την ίδια επιθυμία για ένα halloumi και επρόκειτο να χρησιμοποιήσω τα ίδια κριτήρια. Εκτός από το ότι είχαν μπλοκάρει τους δρόμους προς την πιτσιλιά. Έτσι, η υψηλότερη ποιότητα ήταν Kesses, είναι επίσης η πιο όμορφη, έτσι περνάει γύρω από δύο. Έτσι όταν ο δρόμος προς την Πιτσιλιά είναι κλειστός, παίρνω ένα Kesses από την περαιτέρω σούπερ-αγορά και όταν ο δρόμος είναι ανοιχτός, παίρνω τις Christies.


Γιατί δεν πρέπει να επικαλύπτουμε τους πρωταρχικούς μας στόχους 

Πρίν από μερικές μέρες βγήκα  να αγοράσω χαλούμι. Θα μπορούσα να παω  στην  υπεραγορά  κοντά στο σπίτι μου και να πάρω χαλούμι "Κρίστης ", ευπρεπές βιομηχανικό προϊόν,  ή να οδηγήσω και να πάω σε μια πιο μακρινή υπεραγορά και να βρώ ώριμο χαλούμι "Κεσές" , ευπρεπές παραδοσιακού τύπου βιοτεχνικό προϊόν. Αν όμως ήθελα να βρώ το αυθεντικό παραδοσιακό σπιτικό αιγινό χαλούμι, θα έπρεπε  να κάνω όλη την διαδρομή μέχρι την περιοχή Πιτσιλιάς (ή της Πάφου). Έχω δύο βασικά κριτήρια, εγγύτητα και ποιότητα. Με βάση την εγγύτητα, θα έπαιρνα "Κριστης",  με βάση την ποιότητα θα ανέβαινα στην Πιτσιλιά.  Πώς όμως θα επιλέξω ανάμεσα στα δύο; Τελικά καταφευγω σε ένα Δευτερεύων κριτήριο( που εφαρμόζω μόνο μετά την εφαρμογή του πρώτου), την εμφάνιση, και παίρνω το "Κρίστης". 

Μια εβδομάδα αργότερα είχα την ίδια επιθυμία για χαλλουμι και  επρόκειτο να χρησιμοποιήσω τα ίδια κριτήρια. Δεν ήταν ομως ακριβώς εφικτό γιατί είχαν μπλοκάρει τους δρόμους προς την Πιτσιλιά. Έτσι, η υψηλότερη διαθέσιμη ποιότητα ήταν το χαλούμι "Κεσές", που έχει επίσης την καλύτερη εμφάνιση, γι'αυτο επικρατεί στην τελικη μου αποφαση. 
 Έτσι όταν ο δρόμος προς την Πιτσιλιά είναι κλειστός, παίρνω χαλούμι "Κεσές"απο την μακρινοτερη υπεραγορά και όταν ο δρόμος είναι ανοιχτός, καταλήγω να παίρνω χαλούμι "Κρίστης"




If I wanted the best, I would go to pitsilia, or if I wanted the closest I would have gone to go buy the a Christie's next door or I could go a little bit

 my left hand I held the cheapest one, Christie's, on my right hand I held the best one from pitsilia. I sat there wondering which one of the two to buy. I decided to just get the one that was sold in the largest quantity, it was my favorite. 

I went back a week later, I had the same needs. 

 

\bibliography{../thesisbib/bibliography}

\end{document}

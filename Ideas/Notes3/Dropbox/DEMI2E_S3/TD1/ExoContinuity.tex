% !TEX TS-program = pdflatex
% !TEX encoding = UTF-8 Unicode

% This is a simple template for a LaTeX document using the "article" class.
% See "book", "report", "letter" for other types of document.

\documentclass[11pt]{article} % use larger type; default would be 10pt

\usepackage[utf8]{inputenc} % set input encoding (not needed with XeLaTeX)

%%% Examples of Article customizations
% These packages are optional, depending whether you want the features they provide.
% See the LaTeX Companion or other references for full information.

%%% PAGE DIMENSIONS
\usepackage{geometry} % to change the page dimensions
\geometry{hmargin=2cm,vmargin=0.7cm}
\geometry{a4paper} % or letterpaper (US) or a5paper or....
% \geometry{margin=2in} % for example, change the margins to 2 inches all round
% \geometry{landscape} % set up the page for landscape
%   read geometry.pdf for detailed page layout information

\usepackage{graphicx} % support the \includegraphics command and options

% \usepackage[parfill]{parskip} % Activate to begin paragraphs with an empty line rather than an indent

%%% PACKAGES
\usepackage{booktabs} % for much better looking tables
\usepackage{array} % for better arrays (eg matrices) in maths
\usepackage{paralist} % very flexible & customisable lists (eg. enumerate/itemize, etc.)
\usepackage{verbatim} % adds environment for commenting out blocks of text & for better verbatim
\usepackage{subfig} % make it possible to include more than one captioned figure/table in a single float
% These packages are all incorporated in the memoir class to one degree or another...
\usepackage{amsmath, amsfonts, amsthm, amssymb, mathrsfs}
%\usepackage{bbold}
%%% HEADERS & FOOTERS
\usepackage{fancyhdr} % This should be set AFTER setting up the page geometry
\pagestyle{fancy} % options: empty , plain , fancy
\renewcommand{\headrulewidth}{0pt} % customise the layout...
\lhead{}\chead{}\rhead{}
\lfoot{}\cfoot{\thepage}\rfoot{}

%%% SECTION TITLE APPEARANCE
\usepackage{sectsty}
\allsectionsfont{\sffamily\mdseries\upshape} % (See the fntguide.pdf for font help)
% (This matches ConTeXt defaults)

%%% ToC (table of contents) APPEARANCE
\usepackage[nottoc,notlof,notlot]{tocbibind} % Put the bibliography in the ToC
\usepackage[titles,subfigure]{tocloft} % Alter the style of the Table of Contents
%\renewcommand{\cftsecfont}{\rmfamily\mdseries\upshape}
%\renewcommand{\cftsecpagefont}{\rmfamily\mdseries\upshape} % No bold!

\usepackage{parskip}
%%% END Article customizations

%%% The "real" document content comes below...

\title{Microeconomics }
%\author{The Author}


\begin{document}

\maketitle


\section*{Exercise: Continuous preferences }

\textit{If $u : X \rightarrow R$ a continuous function represents $\succeq$, show that $\succeq$ is rational and continuous. }

$u$ is defined on $X$, so all the elements of X are comparable and $\succeq$ is complete.

Suppose $ x,y$ and $z$ elements of $X$ such that $u(x) \geq u(y) (\Leftrightarrow x \succeq y) $ and $u(y) \geq u(z) \ (\Leftrightarrow y \succeq z) $. It implies directly that $\succeq$ is transitive since $u(x) \geq u(z) \ \Leftrightarrow x \succeq z $.

If $u$ is continuous, we know by the characterization of the continuity by the limits that: assuming two sequences $(x_n) \rightarrow x $ and $(y_n) \rightarrow y $ implies $u(x_n) \rightarrow u(x) $ and $u(y_n) \rightarrow u(y) $. Now suppose that the two sequences are such that $\forall n, \  (x_n) \succeq (y_n) \ (\Leftrightarrow u(x_n) \geq u(y_n))$.



Here, there are two cases : either the two sequences are such that $\exists N \in \mathbb{N}, \ \mathrm{s.t.}, \forall n \geq N, u(x_n) = u(y_n) $ and the result is immediate. Either  $\exists N \in \mathbb{N}, \ \mathrm{s.t.}, \forall n \geq N, u(x_n) > u(y_n) + \eta $. We know $\forall \epsilon > 0, \ \exists N', \ \forall n >  N',  \ \vert u(x_n) - u(x) \vert < \epsilon,  \ \vert u(y_n) - u(y) \vert < \epsilon $.
Thus it is still true for: 
\[
\overline{ \epsilon } =  \dfrac{\eta}{ 2}
\]

Then we can write that: $$ u(x) \geq u(x_n) - \overline{ \epsilon } \geq u(y_n) + \overline{ \epsilon } \geq u(y)  $$
It means that $u(x) \geq u(y) $ and finally implies by the definition of $u$ that $x \succeq y $, the expected result.\\


\textit{May a continuous preference be represented by a discontinuous utility function?}

Yes, an example where we set the function $f : \mathbb{R} \rightarrow \mathbb{R} $: 
$f(x) = \begin{cases}
x&\mbox{ if } x \in [0, \dfrac{1}{2}) \\
x+1&\mbox{ if }x \in [\dfrac{1}{2}, 1]
\end{cases} $ 

%\begin{table}[]
%\centering
%\caption{My caption}
%\label{my-label}
%\begin{tabular}{llll}
%$f(x)$ & $=$ & $x $  & if $x \in [0, \dfrac{1}{2}) $  \\
 %    &   & $x+1$ & if $x \in [\dfrac{1}{2}, 1]$   
%\end{tabular}
%\end{table}

Then, for any continuous function $u$ which takes values in $[0,1] $ (with respect to the Debreu's Theorem) that represents the preferences relation, the  $f \circ u $ still represents the preferences (in fact, any composition of a utilty function representing the preferences by a strictly monotonic function  will preserves representation of the preferences). Indeed, suppose $x, y \in X $, if $u(x), u(y) < \dfrac{1}{2} $, it is immediate that $f \circ u $ still represents the preferences. If  $u(x), \ u(y) > \dfrac{1}{2} $, it is straightforward that $u(x) \geq u(y) \ (\Leftrightarrow x \succeq y)  \Leftrightarrow u(x)+1 \geq u(y)+1 $. The conclusion is the same when $x$ and $y$ are not in the same subset with respect to $u$. \\

\textit{Show that when the alternatives set is $\mathbb{R} $, the preference represented by the floor function is not continuous}

We set two sequences in $\mathbb{R} $ such that with $\epsilon < 0.01 $, and $ n \in \mathbb{N^*}$:

$ \begin{cases}
x_n = 1 - \epsilon^n \rightarrow x \equiv 1 \\
y_n = 0.99 + \epsilon^n \rightarrow y \equiv 0.99 
\end{cases} $ 



Since the preferences are represented by the floor function, we get by $u(y) \geq u(x) \Leftrightarrow y \succeq x  $ that $ \forall n, \ y_n \succeq x_n$ since $  \lfloor x_n \rfloor = \lfloor y_n \rfloor = 0$.

However we notice that $  \lfloor x \rfloor > \lfloor y \rfloor $. Then $x \succ y$ and finally $\urcorner  (y \succeq x)$. By definition, the preferences are not continuous.  



 
\end{document}
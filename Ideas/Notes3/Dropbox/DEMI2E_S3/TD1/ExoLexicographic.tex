% !TEX TS-program = pdflatex
% !TEX encoding = UTF-8 Unicode

% This is a simple template for a LaTeX document using the "article" class.
% See "book", "report", "letter" for other types of document.

\documentclass[11pt]{article} % use larger type; default would be 10pt

\usepackage[utf8]{inputenc} % set input encoding (not needed with XeLaTeX)

%%% Examples of Article customizations
% These packages are optional, depending whether you want the features they provide.
% See the LaTeX Companion or other references for full information.

%%% PAGE DIMENSIONS
\usepackage{geometry} % to change the page dimensions
\geometry{hmargin=2cm,vmargin=0.7cm}
\geometry{a4paper} % or letterpaper (US) or a5paper or....
% \geometry{margin=2in} % for example, change the margins to 2 inches all round
% \geometry{landscape} % set up the page for landscape
%   read geometry.pdf for detailed page layout information

\usepackage{graphicx} % support the \includegraphics command and options

% \usepackage[parfill]{parskip} % Activate to begin paragraphs with an empty line rather than an indent

%%% PACKAGES
\usepackage{booktabs} % for much better looking tables
\usepackage{array} % for better arrays (eg matrices) in maths
\usepackage{paralist} % very flexible & customisable lists (eg. enumerate/itemize, etc.)
\usepackage{verbatim} % adds environment for commenting out blocks of text & for better verbatim
\usepackage{subfig} % make it possible to include more than one captioned figure/table in a single float
% These packages are all incorporated in the memoir class to one degree or another...
\usepackage{amsmath, amsfonts, amsthm, amssymb,mathrsfs}
%\usepackage{bbold}
%%% HEADERS & FOOTERS
\usepackage{fancyhdr} % This should be set AFTER setting up the page geometry
\pagestyle{fancy} % options: empty , plain , fancy
\renewcommand{\headrulewidth}{0pt} % customise the layout...
\lhead{}\chead{}\rhead{}
\lfoot{}\cfoot{\thepage}\rfoot{}

%%% SECTION TITLE APPEARANCE
\usepackage{sectsty}
\allsectionsfont{\sffamily\mdseries\upshape} % (See the fntguide.pdf for font help)
% (This matches ConTeXt defaults)

%%% ToC (table of contents) APPEARANCE
\usepackage[nottoc,notlof,notlot]{tocbibind} % Put the bibliography in the ToC
%\usepackage[titles,subfigure]{tocloft} % Alter the style of the Table of Contents
%\renewcommand{\cftsecfont}{\rmfamily\mdseries\upshape}
%\renewcommand{\cftsecpagefont}{\rmfamily\mdseries\upshape} % No bold!

\usepackage{parskip}
%%% END Article customizations

%%% The "real" document content comes below...

\title{Microeconomics }
%\author{The Author}


\begin{document}

\maketitle


\section*{Exercise: Lexicographic preferences }
\textit{The preferences' rationality:}

The completeness is immediate since $\geq $ is a total order and if two elements of $ \mathbb{R} $ are different, one is greater than the other.

If $ x \succeq y $ and  $ y \succeq z $, then we have to deal with several cases. If $ x_1 > y_1 > z_1$, the result is immediate. If $ x_1 = y_1 > z_1$ (or $ x_1 > y_1 = z_1$), then $ x_1 > z_1$ and we get the result. If the three terms are equal, by hypothesis we know that $ x_2 \geq y_2 \geq z_2$ by hypothesis and we can conclude that the relation is transitive. 


\textit{Proof of the non-continuity of the function:}

Assume a function $u : \mathbb{R}^2_+ \rightarrow \mathbb{R} $ represents the lexicographic preferences.

Then, $ u(x + \dfrac{1}{n}, y-1) > u(x , y) > u(x - \dfrac{1}{n}, y+1) $

The sequences in $\mathbb{R}$  $(x + \dfrac{1}{n})_n$ and $(x - \dfrac{1}{n})_n$ are convergent and both have the same limit x. 

We notice that $ u(x , y-1)< u(x , y) < u(x , y+1) $.

We know that a function is continuous if and only if it preserves the limits of sequences. It is immediate that it is not the case here.\\

\textit{Proof of the non-existence of the function:}

We want to show that a function does not exist, i.e. the function is not well-defined. In words, a function is a way to associate the elements of two sets. In order to show that a function is not well-defined is to emphasize an incoherence between the function properties and the sets properties. Here we use an associate function $q$ to show that $u$ is not well-defined.

Assume a function  $u : \mathbb{R}^2_+ \rightarrow \mathbb{R} $ represents the lexicographic preferences.

For any $a \in [0,1]$, $ (a, 1) \succ (a, 0)$ since $ (a, 1) \succeq (a, 0)$ but $ (a, 0) \succ (a, 1)$ is not true. Thus, if such a function $ u$ existed, we would have $ u(a, 1) > u(a, 0)$ what implies that the interval $I_a = (u(a, 0), u(a, 1))$ is non-empty.

Assume a function $q$ from $[0, 1] $ into the set of the rational numbers $ \mathbb{Q} $ such that for any $b \in [0,1]$, $q(b)  \in  I_b = (u(b, 0), u(b, 1))$. Such a function exists since $ \mathbb{Q}$ is dense in  $\mathbb{R}$. 

By definition, we have that $ I_a \cap I_b = \emptyset$, so $q(a) \neq q(b) $. Then we get that $ q$ is a one-to-one (injective) function, and that: $$\forall a \neq b, \quad q(a) \neq q(b). $$

In words, there are as many $q(a) $ as $ a$. However, $\vert \mathbb{Q} \vert < \vert \mathbb{R} \vert $, what contradicts the fact that $q$ is a one-to-one function.\\




 
\end{document}
% !TEX TS-program = pdflatex
% !TEX encoding = UTF-8 Unicode

% This is a simple template for a LaTeX document using the "article" class.
% See "book", "report", "letter" for other types of document.

\documentclass[11pt]{article} % use larger type; default would be 10pt

\usepackage[utf8]{inputenc} % set input encoding (not needed with XeLaTeX)

%%% Examples of Article customizations
% These packages are optional, depending whether you want the features they provide.
% See the LaTeX Companion or other references for full information.

%%% PAGE DIMENSIONS
\usepackage{geometry} % to change the page dimensions
\geometry{hmargin=2cm,vmargin=0.7cm}
\geometry{a4paper} % or letterpaper (US) or a5paper or....
% \geometry{margin=2in} % for example, change the margins to 2 inches all round
% \geometry{landscape} % set up the page for landscape
%   read geometry.pdf for detailed page layout information

\usepackage{graphicx} % support the \includegraphics command and options

% \usepackage[parfill]{parskip} % Activate to begin paragraphs with an empty line rather than an indent

%%% PACKAGES
\usepackage{booktabs} % for much better looking tables
\usepackage{array} % for better arrays (eg matrices) in maths
\usepackage{paralist} % very flexible & customisable lists (eg. enumerate/itemize, etc.)
\usepackage{verbatim} % adds environment for commenting out blocks of text & for better verbatim
\usepackage{subfig} % make it possible to include more than one captioned figure/table in a single float
% These packages are all incorporated in the memoir class to one degree or another...
\usepackage{amsmath, amsfonts, amsthm, amssymb, mathrsfs}
%\usepackage{bbold}
%%% HEADERS & FOOTERS
\usepackage{fancyhdr} % This should be set AFTER setting up the page geometry
\pagestyle{fancy} % options: empty , plain , fancy
\renewcommand{\headrulewidth}{0pt} % customise the layout...
\lhead{}\chead{}\rhead{}
\lfoot{}\cfoot{\thepage}\rfoot{}

%%% SECTION TITLE APPEARANCE
\usepackage{sectsty}
\allsectionsfont{\sffamily\mdseries\upshape} % (See the fntguide.pdf for font help)
% (This matches ConTeXt defaults)

%%% ToC (table of contents) APPEARANCE
\usepackage[nottoc,notlof,notlot]{tocbibind} % Put the bibliography in the ToC
\usepackage[titles,subfigure]{tocloft} % Alter the style of the Table of Contents
%\renewcommand{\cftsecfont}{\rmfamily\mdseries\upshape}
%\renewcommand{\cftsecpagefont}{\rmfamily\mdseries\upshape} % No bold!

\usepackage{parskip}
%%% END Article customizations

%%% The "real" document content comes below...

\title{Microeconomics }
%\author{The Author}


\begin{document}

\maketitle


\section*{Exercise 1.1: Rational preferences }

\textit{Let $\succsim$ be a rational preference on X(rational and transitive). Show that:}











\begin{enumerate} 
 \item $\succ$ is irreflexive and transitive.

a)
$a \succ b$ and  $b\succ a$
so $b \succsim a$ but $a \succ b$, or $\neg b \succsim a$: contradiction!

b)If  $a>b$ 
and $b>c\rightarrow a \succsim$ b 
and $b \succsim c $
then we have that $ a \succsim c$, which indicates that a $\succ$ c or  a $\sim$ c

$\rightarrow$ $\neg b \succsim a$ and $\neg c \succsim  b$; because if $c\succsim a$ , by transitivity of $\succ$, we have that  $c \succsim b$ : contradiction

\item $\sim$ is reflexive, transitive and symmetric.

a) Immediate


b) If $\begin{cases}
x \succ y \\
y \succ z
\end{cases}$ and  
$\begin{cases} 
y \succ z \\
z \succ y 
\end{cases}
\Rightarrow 
\begin{cases}
x \succ z \\
z \succ x
\end{cases}$

c) If $x \sim y\Rightarrow
\begin{cases} 
x \succ y \\
z \succ y
\end{cases} \Rightarrow y \sim x$  

\item if $x \succ y \succsim z$, then $x \succ z$.

\end{enumerate}

$x>y$ $\succ$ z $\Rightarrow$ x $\succ$ y $\succ$ z $\Rightarrow$  x $\succ$ z by transitivity.

By contradiction, if  z $\succ$ x, by transitivity,  z $\succ$ y. We have by the hypothesis that z $\sim$ y, and by transitivity of $\sim$ we have that $y \sim x$: which is a contradiction. 

\section*{Exercise 1.2: Representation of preferences}

\textit{Let $u:X \rightarrow R$ be utility function which represents the preferences on $\succsim$ on X, such that.
$u(x) \geq u(y) \iff x \succsim y, \forall x,y \in X$}

\textit{Show that for all functions $f:R \rightarrow R$ which are strictly increasing $f \circ u$, also represents $\succsim$. What happens if f is increasing but not strictly?  }

a) $ f(u(x)) \succsim f(u(y)) \iff u(x) \succsim u(y) \iff x \succsim y$

b) In this case, the first equivalence is false:
$u(x) \geq u(y) \rightarrow f(u(x)) \geq f(u(y))$ holds always but the inverse relation does not. 
Example: $f=$constant $, u(z)=z \forall z \in R, x=0, y=1$
$f(u(x))=f(u(y))$ and $u(x)<u(y)$ 

\section*{Exercise 1.3: preferences on a finite set}

\textit{Let X be a finite set and $\succsim$. Show that their exists a utility function $u:X \rightarrow R$ which represents the preferences.}

By induction. Let $M_1 = (x \in X \vert y \succeq  x, \ \forall y \in X) \ne\emptyset $.
We let $u(z)=1 \forall z \in M_1$.
If $M_1 = X$ we are done, 
Otherwise $M_1 \neq X$ and let $ X_1= X \setminus M_1 $.
We let $u(z)=2 \forall z \in M_2= (x \in X \vert y \succeq  x, \ \forall y \in X_1)$
and repeat.

This algorithm takes at most $|X|$ stages and constructs a representative utility function of $\succeq$ with the values of $\mathbb{N} $.

Remark: While X is countable we can represent $ \succeq $ by a utility function u: $x \rightarrow (0,1)$.

\section*{Exercise 1.4: Weak axiom of revealed preferences}

\textit{Let $ X = \{ x,y,z \} $ be an ensemble of alternatives, $ G $ $ = \{\{x,y\},\{x,y,z \} \} $ a sub set of X and let C be a function of choice defined on $ G $ so that $C(\{ x,y \}) = \{ x\}$ . Show that if C satisfies the weak axiom of revealed preferences, then $C(\{x,y,z \})$ is equal to $\{x \} \{z \}, \{x,z \}$}

\textit{Reminder that C verifies the weak axiom of revealed preferences if, when x is revealed to be equally preferred to y, y cannot be revealed to be strictly better than x. Said otherwise, there does not exist an $ A,B \in G $ so that $x, y \in A \cap B, x \in C(A), y \in C(B)$ and $x \notin C(B) $}

Suppose that y $\in C(\{ x,y,z \})$ and let $A = \{x, y \}$ et $B = \{x, y, z \}$

Therefore we have that $y,x \in B \cap A $
$y \in C(B)$
$x \in C(A)$

According to the (WA) this implies that $y \in C(A)$, a contradiction.
We need only verify that C(B) = $\{x \}$ or  $\{x \}$ or $\{x,z \}$ does not contradict the (WA). But this is trivial because $A \cap B = \{x,y \}$ said otherwise, $z \notin A \cap B $ which means that it can't serve as a counterexample to the axiom. 

\section*{Exercise 1.5: Continuity of preferences}

Debreu's Theorem states that $\exists u:X \rightarrow \mathbb{R} $, a continuous function such that: $ u(x) \geq u(y) \Leftrightarrow x \succeq y $. \\

Intermediate value Theorem states that:

$\begin{cases}
f : [0,1] \rightarrow \mathbb{R} \ \textrm{is continous} \\
f(1) \geq t \geq f(0)
\end{cases}
\Rightarrow \exists c \in [0,1] \ \textrm{s.t.} f(c) = t $

Applying to $f(c) = \begin{cases}
[0,1] \rightarrow \mathbb{R}  \textrm{ is continuous} \\
u(cx + (1-c)x)\\
 \textrm{with } t = u(y)
\end{cases}  $, 

We get that there is a $ c$ such that $f(c) = t  $, and then we deduce the existence of $m = cx +(1-c)z \in [z,x]$ such that $u(m) = u(y) $. We conclude that $m \sim y $.

\end{document}
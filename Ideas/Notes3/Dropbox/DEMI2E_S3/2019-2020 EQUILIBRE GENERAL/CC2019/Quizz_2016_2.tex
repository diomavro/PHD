% !TEX TS-program = pdflatex
% !TEX encoding = UTF-8 Unicode

% This is a simple template for a LaTeX document using the "article" class.
% See "book", "report", "letter" for other types of document.

\documentclass[11pt]{article} % use larger type; default would be 10pt

\usepackage[utf8]{inputenc} % set input encoding (not needed with XeLaTeX)

%%% Examples of Article customizations
% These packages are optional, depending whether you want the features they provide.
% See the LaTeX Companion or other references for full information.

%%% PAGE DIMENSIONS
\usepackage{geometry} % to change the page dimensions
\geometry{hmargin=2cm,vmargin=0.7cm}
\geometry{a4paper} % or letterpaper (US) or a5paper or....
% \geometry{margin=2in} % for example, change the margins to 2 inches all round
% \geometry{landscape} % set up the page for landscape
%   read geometry.pdf for detailed page layout information

\usepackage{graphicx} % support the \includegraphics command and options

% \usepackage[parfill]{parskip} % Activate to begin paragraphs with an empty line rather than an indent

%%% PACKAGES
\usepackage{booktabs} % for much better looking tables
\usepackage{array} % for better arrays (eg matrices) in maths
\usepackage{paralist} % very flexible & customisable lists (eg. enumerate/itemize, etc.)
\usepackage{verbatim} % adds environment for commenting out blocks of text & for better verbatim
\usepackage{subfig} % make it possible to include more than one captioned figure/table in a single float
% These packages are all incorporated in the memoir class to one degree or another...
\usepackage{amsmath, amsfonts,amsthm, amssymb,mathrsfs}
%%% HEADERS & FOOTERS
\usepackage{fancyhdr} % This should be set AFTER setting up the page geometry
\pagestyle{fancy} % options: empty , plain , fancy
\renewcommand{\headrulewidth}{0pt} % customise the layout...
\lhead{}\chead{}\rhead{}
\lfoot{}\cfoot{\thepage}\rfoot{}

%%% SECTION TITLE APPEARANCE
\usepackage{sectsty}
\allsectionsfont{\sffamily\mdseries\upshape} % (See the fntguide.pdf for font help)
% (This matches ConTeXt defaults)

%%% ToC (table of contents) APPEARANCE
\usepackage[nottoc,notlof,notlot]{tocbibind} % Put the bibliography in the ToC
\usepackage[titles,subfigure]{tocloft} % Alter the style of the Table of Contents
\renewcommand{\cftsecfont}{\rmfamily\mdseries\upshape}
\renewcommand{\cftsecpagefont}{\rmfamily\mdseries\upshape} % No bold!

%%% END Article customizations

%%% The "real" document content comes below...

\title{Microeconomics 1 / M1: Fall 2016 / Quizz 2}
%\author{The Author}

\date{December 2, 2016} % Activate to display a given date or no date (if empty),
         % otherwise the current date is printed 

\begin{document}

\maketitle

Time : 1 hour 20 minutes

\section*{Exercise 1: Equivalent Variation and Compensated Variation}

The two parts are independent.

\subsection*{Part I {\small (4 points, difficulty *)}}
Miss Moto is keen on ringing the bells of a town church for 10 hours per day. She spends c on the other goods she consumes and spends $x$ hours ringing the bells. Her utility function is :
\begin{equation*}
U (c, x) = c + 3x
\end{equation*}  
for $x<=10$.\\

If $x>10$, she gets extremly painful blisters, and her utility function becomes : 
\begin{equation*}
U (c, x) = c 
\end{equation*} 

Her income is 100 euros and she is allowed by the preacher to ring the bells  for 10 hours. In the following questions, we always compare this initial situation with a new one.
\begin{enumerate}
\item Because of complains in the town, the preacher decides to allow Miss Moto to ring the bells \textit{only} for 5 hours. That is bad news for Miss Moto and is equivalent to a loss in her income. How much is this loss ? (1 point)
\item The preacher makes a new offer : Miss Moto can ring the bells as much as she wants but she has to pay 2 euros per hour. How long will she ring the bells? Which loss in her income would cause to her the same loss in utility as this tax ?  (1 point)
\item The citizens carry on complaining. The preacher increases the price to 4 euros per hour. How long Moto will ring the bells ? Which loss in her income would cause to her the same loss in utility as this tax ?  (1 point)
\item In the previous questions, did you use Equivalent Variation, Compensated Variation or none of them ? Why ? (1 point)
\end{enumerate}



\subsection*{Part II {\small (5 points, difficulty **)}}
Assume a consumer whose the wealth is $m = 200$ euros and who can buy two goods in quantities $x_1$ and $x_2$ respectively. The price of the first good is $p_1 = 1$; that of the second one is $p_2 = 2$. The preferences of the consumer are represented by the utility function :
\begin{equation*}
U (x_1, x_2) = min(2x_1, x_2) 
\end{equation*}

The consumer faces an increase in $p_1$. We note the new price $p'_1 = 4$.
\begin{enumerate}

\item Show that the compensated variation is equal to 120 euros. (2 point)
\item Show that the equivalent variation is equal to 75 euros. (2 point)
\item Represent graphically the equivalent variation and the compensated variation. (1 point)
\end{enumerate} 

\newpage
\section*{Exercise 2: Exchange and Production Economy }
It is advised to do the first part before the second part.\\

\subsection*{Part I {\small (4.5 points, difficulty **)}}
Assume an economy with two consumers $i = A, B$, and two goods $l = 1, 2$. The individual endowments of $A$ and $B$ are $\omega^A = \omega^B = (\frac{1}{2}, \frac{1}{2})$. Good 2 is the numeraire good (i.e. $p_2 = 1$). We note $p_1 = p$. The preferences of the consumer are represented by the utility functions :
\begin{equation*}
u^A(x_1^A, x_2^A) = ln(x_1^A) + ln(x_2^A) \quad  \quad
u^B(x_1^B, x_2^B) = (x_1^B)^{\frac{1}{4}}(x_2^B)^{\frac{3}{4}} 
\end{equation*}


\begin{enumerate}
\item Determine the Walrasian equilibrium (find $p = \frac{3}{5}$ and allocations $((\frac{2}{3},\frac{2}{5}) ; (\frac{1}{3},\frac{3}{5})$). (2.5 points)
\item Check if the Walrasian equilibrium is Pareto-optimal. Which computations should be made to check that the equilibrium is in the core ? (2 points)

\end{enumerate}

\subsection*{Part II {\small (7 points, difficulty ** and ***)}}
We carry on working in the same framework with the same consumers (same preferences and endowments). A firm is created by the consumer B to produce good 2 using good 1 as input. The production function is $y_2 = \sqrt{y_1}$. We note $\pi$ the firm's profit. In the following questions, the firm maximises its profit independently of the consumer B's preferences. The profit is then added to the consumer B's budget.

\begin{enumerate}

\item Determine the demand for good 1 of the firm and the consumers. Prove the price $p$ is equal to $p =\frac{3+\sqrt{59}}{3}$. (2 points) \\
Hint : Look at the Help.
\item The production function becomes $y_2 = y_1$. Determine the demand for good 1 of the firm and the consumers by distinguish 3 cases with respect to the value of $p$. (2.5 points)
\item The production function becomes $y_2 = \frac{y_1}{c}$ (with $c>0$). Determine the values of c such that the the firm is active at equilibrium (i.e. $y_1 > 0$) and the values of $c$ such that the firm is not active (Hint : Show that, for some values of $c$, there is
an excess demand of good 1 when the firm is active). Compare the equilibrium of question I.1 with the equilibrium with the non active firm. (2.5 points)

\end{enumerate}

\section*{Help}
To solve a quadratic equation $ax^2 + bx + c = 0 $, you need to determine the discriminant $\Delta = b^2 - 4 ac $ and to solve $x = \dfrac{-b \pm \sqrt{\Delta}}{2a} $.

 
\end{document}
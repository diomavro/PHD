\documentclass{article}
%%%%%%%%%%%%%%%%%%%%%%%%%%%%%%%%%%%%%%%%%%%%%%%%%%%%%%%%%%%%%%%%%%%%%%%%%%%%%%%%%%%%%%%%%%%%%%%%%%%%%%%%%%%%%%%%%%%%%%%%%%%%%%%%%%%%%%%%%%%%%%%%%%%%%%%%%%%%%%%%%%%%%%%%%%%%%%%%%%%%%%%%%%%%%%%%%%%%%%%%%%%%%%%%%%%%%%%%%%%%%%%%%%%%%%%%%%%%%%%%%%%%%%%%%%%%
\usepackage[utf8]{inputenc}
%\usepackage[frenchb]{babel}
\usepackage{amssymb,amsfonts,amsmath}
%\newcommand\R{\mathbb R}

%\setcounter{MaxMatrixCols}{10}
%%TCIDATA{OutputFilter=LATEX.DLL}
%%TCIDATA{Version=5.00.0.2570}
%%TCIDATA{<META NAME="SaveForMode" CONTENT="1">}
%%TCIDATA{LastRevised=Wednesday, March 25, 2009 15:50:39}
%%TCIDATA{<META NAME="GraphicsSave" CONTENT="32">}
%
%\def \L { {\mathcal L}}
%\setlength{\unitlength}{1cm} \setlength{\textwidth}{17cm}
%\setlength{\oddsidemargin}{0cm} \setlength{\evensidemargin}{0cm}
%\setlength{\topmargin}{-45pt} \setlength{\textheight}{23.5cm}
%\renewcommand{\baselinestretch}{1.3}

%\input{tcilatex}

\begin{document}


\begin{itemize}
\item
\item
\begin{equation}
p_1 = 1, \ p_2 = 2, \ (x^1, y^1) = (\frac{1}{2}, \frac{1}{4}), \ (x^2, y^2) = (1, \frac{1}{2)},...
\end{equation}
\item
\item La fonti�re des utilit�s est telle que la somme des utilit�s est maximale �tant donn�e l'utilit� d'un agent. La fronti�re repr�sente donc l'ensemble des sommes d'utilit�s atteintes aux optima de Pareto.

La fronti�re de production est telle que : 

\begin{align*}
q_1+q_2 = z_1 + \frac{1}{2} z_2 \\
q_1 + q_2 = z_1 + \frac{1}{2}(3 -z_1) \\
q_2 = \frac{3}{2} + \frac{z_1}{2} -q_1 \\
q_2 = \frac{3}{2} -\frac{1}{2}q_1
\end{align*}

Le TMT �tant la d�riv�e (en valeur absolue) de la fronti�re de production, on trouve:

$$ TMT_{2\rightarrow 1} = \frac{1}{2}$$ 


L'�galisation des TMS et du TMT nous donne : 

\begin{equation}
\frac{y^1}{x^1} = \frac{y^2}{x^2} = \frac{1}{2}
\end{equation}

Par ailleurs, on sait que:

\begin{align*}
x^1+x^2 = \bar{q_1} \\
x^2+y^2 = \bar{q_2}
\end{align*}

La courbe des contrats peut s'�crire apr�s arrangements :

$$y^1 = x^1 \frac{3-\bar{q_1}}{q_1} $$

L'�galit� des TMS et du TMT nous donne les valeurs $q_1 = \frac{3}{2} $ et $q_2 = \frac{3}{4} $, les m�mes qu'� l'�quilibre. 

On peut �crire maintenant $ x^1 = 2y^1, \ x^2 = \frac{3}{2}- 2y_1, \ y^2 = \frac{3}{4}-y_1$.

On otient donc $ u^1+u^2 = \sqrt{2}y^1 +\sqrt{2}(\frac{3}{4}-y^2) = \frac{3}{2\sqrt{2}}$

On souhaite maximiser le programme d'optimisation sociale : 

\begin{align*}
\max{x^1, x^2, y^1, y^2} \rho \sqrt{x^1y^1} + (1- \rho) \sqrt{x^2y^2} \textrm{ x^1 + x^2 + 2(y^1+y^2) =3 } \\
\max{x^1, x^2, y^1, y^2} \rho \sqrt{x^1y^1} + (1- \rho) \sqrt{x^2y^2} \textrm{ x^1 + x^2 + 2(y^1+y^2) =3 }
\end{align*}

\end{itemize}


\end{document}
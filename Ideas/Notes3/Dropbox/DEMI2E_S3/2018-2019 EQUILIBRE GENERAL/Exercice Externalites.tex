
\documentclass[11pt]{article}
%%%%%%%%%%%%%%%%%%%%%%%%%%%%%%%%%%%%%%%%%%%%%%%%%%%%%%%%%%%%%%%%%%%%%%%%%%%%%%%%%%%%%%%%%%%%%%%%%%%%%%%%%%%%%%%%%%%%%%%%%%%%%%%%%%%%%%%%%%%%%%%%%%%%%%%%%%%%%%%%%%%%%%%%%%%%%%%%%%%%%%%%%%%%%%%%%%%%%%%%%%%%%%%%%%%%%%%%%%%%%%%%%%%%%%%%%%%%%%%%%%%%%%%%%%%%
\usepackage[applemac]{inputenc}
\usepackage[frenchb]{babel}
\usepackage{amssymb,amsfonts,amsmath}
\newcommand\R{\mathbb R}

\setcounter{MaxMatrixCols}{10}
%TCIDATA{OutputFilter=LATEX.DLL}
%TCIDATA{Version=5.00.0.2570}
%TCIDATA{<META NAME="SaveForMode" CONTENT="1">}
%TCIDATA{LastRevised=Wednesday, March 25, 2009 15:50:39}
%TCIDATA{<META NAME="GraphicsSave" CONTENT="32">}

\def \L { {\mathcal L}}
\setlength{\unitlength}{1cm} \setlength{\textwidth}{17cm}
\setlength{\oddsidemargin}{0cm} \setlength{\evensidemargin}{0cm}
\setlength{\topmargin}{-45pt} \setlength{\textheight}{23.5cm}
\renewcommand{\baselinestretch}{1.3}

%\input{tcilatex}

\begin{document}



\textbf{Externalit\'{e}s avec rendements constants}

Pour d\'{e}terminer l'optimum de Pareto, il faut maximiser la somme des utilit\'{e}s pond\'{e}r\'{e}es des agents. Utiliser un agent repr\'{e}sentatif ne fonctionne pas dans tous les cas comme nous allons le voir ci-dessous.

On r\'{e}soud notamment  $\underset{u^1, \ u^2}{\max} \
 \rho u^1 + (1-\rho) u^2 $ sous contrainte d'apurement des march\'{e}s. 
 
Pour \^{e}tre rigoureux, il faudrait employer un lagrangien avec une contrainte d'apurement des march\'{e}s ET des contraintes positivit\'{e} sur les consommations.

Pour \^{e}tre plus rapide, on peut faire comme le professeur, maximiser un programme en int\'{e}grant directement les contraintes dans la fonction objectif. 

Cela donne : $$\underset{x^A_1, \ x^A_2, \ x^B_1, \ x^B_2, \ y}{\max} \
 \rho u^1 + (1-\rho) u^2  $$ avec $x^A_1+ x^B_1=60-y/2 = 60-\frac{x^A_2+x^B_2}{2}$.
 
Ce qui \'{e}quivaut \`{a} :$$\underset{x^A_1, \ x^A_2, \ x^B_2}{\max} \
 \rho x^A_1 x^A_2 + (1-\rho)(60-\frac{x^A_2+x^B_2}{2}-x^A_1) x^B_2 - (  x^A_2+ x^B_2) $$
 
 On annule les d\'{e}riv\'{e}es par rapport aux trois variables restantes et on obtient :
 
 $x^A_2 = x^B_2 \times \frac{1-\rho}{\rho}, \ x^A_1 = \frac{x^B_2}{2} \times \frac{1-\rho}{\rho} +\frac{1}{\rho}, x^B_2 =-\frac{2}{1-\rho}+ 2x^B_1$
 
 Finalement, on trouve : $$x^B_2 =-\frac{2}{1-\rho}+ 2x^B_1, \ x^A_2 =-\frac{2}{\rho}+ 2x^A_1 \Rightarrow y =- \frac{2}{1-\rho}-\frac{2}{\rho} + 2(x^A_1+x^B_1) \Rightarrow y = 60-\frac{1}{\rho(1-\rho)}   $$
 
 Cependant, le r\'{e}sultat ne tient que si $x^A_2 =-4+ 2x^A_1$ est positif par exemple ($ \Rightarrow x^A_1,x^A_2 \geq 2) $. Si une consommation est au-dessous de 2, il est possible qu'augmenter la consommation de l'agent faiblement pourvu diminue en fait l'utilit\'{e} totale (les  d\'{e}riv\'{e}es partielles sont n\'{e}gatives). Donc, pour les valeurs de $\rho$ proches de $0$ ou $1$, il faut faire ce que l'on a fait en td et faire l'hypoth\`{e}se qu'un seul agent consomme des biens pour maximiser le programme.
 
 
 








\end{document}
\documentclass{beamer}
%\setbeamersize{text margin left=30mm,text margin right=30mm} 
%\setbeamertemplate{frametitle}{\vspace{2cm} \\ \insertframetitle }
%\setbeamertemplate{footline}{\vspace{0.5cm}}
%\addtolength{\headsep}{-16cm}



% The Beamer class comes with a number of default slide themes
% which change the colors and layouts of slides. Below this is a list
% of all the themes, uncomment each in turn to see what they look like.
\mode<presentation> {
\usetheme{Marburg}
\usecolortheme{orchid}
}
%\addtobeamertemplate{beamercolorbox}{}{\vspace{-20em}}

%\usepackage{bbm}


\usepackage{float}
%\usepackage[cmbold]{mathtime}
%\usepackage{mt11p}
\usepackage{placeins}
\usepackage{amsmath}
\usepackage{color}
\usepackage{amssymb}
\usepackage{mathtools}
\usepackage{subfigure}
\usepackage{multirow}
\usepackage{epsfig}
\usepackage{listings}
\usepackage{enumitem}
\usepackage{rotating,tabularx}
%\usepackage[graphicx]{realboxes}
\usepackage{graphicx}
\usepackage{graphics}
\usepackage{epstopdf}
\usepackage{longtable}
%\usepackage[pdftex]{hyperref}
%\usepackage{breakurl}
%\usepackage{epigraph}
%\usepackage{xspace}
\usepackage{amsfonts}
\usepackage{eurosym}
%\usepackage{ulem}
\usepackage{footmisc}
%\usepackage{comment}
\usepackage{setspace}
\usepackage{geometry}
\usepackage{caption}
%\usepackage{pdflscape}
\usepackage{array}
\usepackage[round]{natbib}
\usepackage{booktabs}
\usepackage{dcolumn}
\usepackage{mathrsfs}
%\usepackage[justification=centering]{caption}
%\captionsetup[table]{format=plain,labelformat=simple,labelsep=period,singlelinecheck=true}%

%\bibliographystyle{unsrtnat}
%\bibliographystyle{aea}
\usepackage{enumitem}
\usepackage{tikz}
\usetikzlibrary{decorations.pathreplacing}
\def\checkmark{\tikz\fill[scale=0.4](0,.35) -- (.25,0) -- (1,.7) -- (.25,.15) -- cycle;}
%\usepackage{tikz}
%\usetikzlibrary{snakes}
%\usetikzlibrary{patterns}

%\draftSpacing{1.5}

\usepackage{xcolor}
\hypersetup{
colorlinks,
linkcolor={blue!50!black},
citecolor={blue!50!black},
urlcolor={blue!50!black}}

%\renewcommand{\familydefault}{\sfdefault}
%\usepackage{helvet}
%\setlength{\parindent}{0.4cm}
%\setlength{\parindent}{2em}
%\setlength{\parskip}{1em}

%\normalem

%\doublespacing
%\onehalfspacing
%\singlespacing
%\linespread{1.5}

%\newtheorem{theorem}{Theorem}
%\newtheorem{corollary}[theorem]{Corollary}
\newtheorem{proposition}{Proposition}
%\newtheorem{definition}{Definition}
\newtheorem{axiom}{Axiom}
\newcommand{\ra}[1]{\renewcommand{\arraystretch}{#1}}

\newcommand{\E}{\mathrm{E}}
\newcommand{\Var}{\mathrm{Var}}
\newcommand{\Corr}{\mathrm{Corr}}
\newcommand{\Cov}{\mathrm{Cov}}

\newcolumntype{d}[1]{D{.}{.}{#1}} % "decimal" column type
\renewcommand{\ast}{{}^{\textstyle *}} % for raised "asterisks"

\newtheorem{hyp}{Hypothesis}
\newtheorem{subhyp}{Hypothesis}[hyp]
\renewcommand{\thesubhyp}{\thehyp\alph{subhyp}}

\newcommand{\red}[1]{{\color{red} #1}}
\newcommand{\blue}[1]{{\color{blue} #1}}

%\newcommand*{\qed}{\hfill\ensuremath{\blacksquare}}%

\newcolumntype{L}[1]{>{\raggedright\let\newline\\arraybackslash\hspace{0pt}}m{#1}}
\newcolumntype{C}[1]{>{\centering\let\newline\\arraybackslash\hspace{0pt}}m{#1}}
\newcolumntype{R}[1]{>{\raggedleft\let\newline\\arraybackslash\hspace{0pt}}m{#1}}

%\geometry{left=1.5in,right=1.5in,top=1.5in,bottom=1.5in}
%\geometry{left=1in,right=1in,top=1in,bottom=1in}

\epstopdfsetup{outdir=./}

\newcommand{\elabel}[1]{\label{eq:#1}}
\newcommand{\eref}[1]{Eq.~(\ref{eq:#1})}
\newcommand{\ceref}[2]{(\ref{eq:#1}#2)}
\newcommand{\Eref}[1]{Equation~(\ref{eq:#1})}
\newcommand{\erefs}[2]{Eqs.~(\ref{eq:#1}--\ref{eq:#2})}

\newcommand{\Sref}[1]{Section~\ref{sec:#1}}
\newcommand{\sref}[1]{Sec.~\ref{sec:#1}}

\newcommand{\Pref}[1]{Proposition~\ref{prop:#1}}
\newcommand{\pref}[1]{Prop.~\ref{prop:#1}}
\newcommand{\preflong}[1]{proposition~\ref{prop:#1}}

\newcommand{\Aref}[1]{Axiom~\ref{ax:#1}}

\newcommand{\clabel}[1]{\label{coro:#1}}
\newcommand{\Cref}[1]{Corollary~\ref{coro:#1}}
\newcommand{\cref}[1]{Cor.~\ref{coro:#1}}
\newcommand{\creflong}[1]{corollary~\ref{coro:#1}}

\newcommand{\etal}{{\it et~al.}\xspace}
\newcommand{\ie}{{\it i.e.}\ }
\newcommand{\eg}{{\it e.g.}\ }
\newcommand{\etc}{{\it etc.}\ }
\newcommand{\cf}{{\it c.f.}\ }
\newcommand{\ave}[1]{\left\langle#1 \right\rangle}
\newcommand{\person}[1]{{\it \sc #1}}

\newcommand{\AAA}[1]{\red{{\it AA: #1 AA}}}
\newcommand{\YB}[1]{\blue{{\it YB: #1 YB}}}

\newcommand{\flabel}[1]{\label{fig:#1}}
\newcommand{\fref}[1]{Fig.~\ref{fig:#1}}
\newcommand{\Fref}[1]{Figure~\ref{fig:#1}}

\newcommand{\tlabel}[1]{\label{tab:#1}}
\newcommand{\tref}[1]{Tab.~\ref{tab:#1}}
\newcommand{\Tref}[1]{Table~\ref{tab:#1}}

\newcommand{\be}{\begin{equation}}
\newcommand{\ee}{\end{equation}}
\newcommand{\bea}{\begin{eqnarray}}
\newcommand{\eea}{\end{eqnarray}}

\newcommand{\bi}{\begin{itemize}}
\newcommand{\ei}{\end{itemize}}

\newcommand{\Dt}{\Delta t}
\newcommand{\Dx}{\Delta x}
\newcommand{\Epsilon}{\mathcal{E}}
\newcommand{\etau}{\tau^\text{eqm}}
\newcommand{\wtau}{\widetilde{\tau}}
\newcommand{\xN}{\ave{x}_N}
\newcommand{\Sdata}{S^{\text{data}}}
\newcommand{\Smodel}{S^{\text{model}}}

\newcommand{\del}{D}
\newcommand{\hor}{H}

\setlength{\parindent}{0.0cm}
\setlength{\parskip}{0.4em}

\numberwithin{equation}{section}
\DeclareMathOperator\erf{erf}
%\let\endtitlepage\relax

%\mode<presentation> {

% The Beamer class comes with a number of default slide themes
% which change the colors and layouts of slides. Below this is a list
% of all the themes, uncomment each in turn to see what they look like.

% As well as themes, the Beamer class has a number of color themes
% for any slide theme. Uncomment each of these in turn to see how it
% changes the colors of your current slide theme.

%\usecolortheme{albatross}
%\usecolortheme{beaver}
%\usecolortheme{beetle}
%\usecolortheme{crane}
%\usecolortheme{dolphin}
%\usecolortheme{dove}
%\usecolortheme{fly}
%\usecolortheme{lily}
%\usecolortheme{orchid}
%\usecolortheme{rose}
%\usecolortheme{seagull}
%\usecolortheme{seahorse}
%\usecolortheme{whale}
%\usecolortheme{wolverine}

%\setbeamertemplate{footline} % To remove the footer line in all slides uncomment this line
%\setbeamertemplate{footline}[page number] % To replace the footer line in all slides with a simple slide count uncomment this line


%----------------------------------------------------------------------------------------
%	TITLE PAGE
%----------------------------------------------------------------------------------------

\title[Notes on employee project discounting]{Firm discounting} % The short title appears at the bottom of every slide, the full title is only on the title page

\author{Diomides Mavroyiannis} % Your name
% Your institution as it will appear on the bottom of every slide, may be shorthand to save space

\date{\today} % Date, can be changed to a custom date

\begin{document}

\begin{frame}
\titlepage % Print the title page as the first slide
\end{frame}
%------------------------------------------------

\subsection{The ergodic approach}
\begin{frame}{Discouting in ergodicity}
\begin{itemize}
\item Utility as a growth optimizing function
\item Discounting as a utility function without uncertainty
\end{itemize}
\end{frame}

\subsection{The discounting and ergodicty}
\begin{frame}{Microfoundations of discounting approach: Equalize the growth rate}
\begin{itemize}
\item Utility as a growth optimizing function
\item Discounting as a utility function without uncertainty
\item The results of that paper were that dependend on horizon and dynamics
\item Dynamics are relaivly straightfoward
\item Horizon was left unspecified
\item Technically the discount rate can only exist when the growth rate is the same. 
\end{itemize}
\end{frame}

\begin{frame}{This idea }
\begin{itemize}
\item What are some measurable observavles that can determine your horizon?
\end{itemize}
\end{frame}

\subsection{The model}

\begin{frame}{In brief}
\begin{itemize}
\item The number of employees a company has can affect the optimal discount function
\item Firm has to specialize, to either accept projects of type a or projects of type b.
\item this isn't the only way to get this effect, one could simply imagine a fixed cost for being able to accept a different variety of projects 
\end{itemize}
\end{frame}

\begin{frame}{Setup}
\begin{table}[]
\begin{tabular}{|c|l|l|l|}
\hline
Payout     & Task Length     & Employees required   & Monthly Arrivals \\ \hline
$X_a$ & $t_a$ & $n_a$ & $r_a$                 \\ \hline
$X_b$ & $t_b$ & $n_b$ & $r_b$                 \\ \hline
\end{tabular}
\end{table}
\end{frame}

\begin{frame}{Useful measures}
\begin{itemize}
\item A worker can complete $\frac{T}{t_i}$ projects in T time
\item The required number of workers to accept all projects of type i is: $t_i*r_i*n_i$
\item The maximum number projects of type i than can be accepted: $T*r_i$
\item Payment per unit of time per project $\frac{X_i}{t_i}$
\item Payment per employee per unit of time $\frac{X_i}{t_i*n_i}$
\end{itemize}
\end{frame}

\begin{frame}{General results}
\begin{itemize}
\item So if there is an infinite(or unconstrained) arrival per month, but finite employees maximize payment per employee
\item if there is finite arrival per month, but infinite(or unconstrained) employees maximize payment per unit of time
\end{itemize}
\end{frame}

\begin{frame}{Finite arrival and finite employees}
\begin{itemize}
\item Case 1: The growth rates are equal
\item Case 2: One of the growth rates is more binding
\end{itemize}
\end{frame}

\begin{frame}{Finite arrival and finite employees}
\begin{itemize}
\item Depending on the constraints we might equalize growth rate per project with the growth rate per employee.
\item So the discount rate could take one of the four forms
\item Technically the discount rate can only exist when the growth rate is the same.
\item $\frac{x_a}{x_b}$   
\end{itemize}
\end{frame}


\begin{frame}{Types of discounting}
\begin{table}[]
\begin{tabular}{|c|l|l|}
\hline
1 & Constrained A  & Not constrained A \\ \hline
Constrained B  & $\delta = \frac{t_a n_a}{t_b n_b}$ & $\frac{t_a}{t_b n_b}$                 \\ \hline
Not Constrained B &$ \frac{t_a n_a}{t_b} $ & $\frac{t_a}{t_b}$                \\ \hline
\end{tabular}
\end{table}
\end{frame}

\subsubsection{Example}
\subsection{Motivating example}
\begin{frame}{Motivating example}
\begin{table}[]
\begin{tabular}{|c|l|l|l|}
\hline
Payout     & Task Length     & Employees req   & Monthly Arr \\ \hline
100 & 2 m  & 1 & 3                  \\ \hline
600 & 10 m & 2 & 5                  \\ \hline
\end{tabular}
\end{table}
\end{frame}


\begin{frame}{Useful metrics}
\begin{itemize}
    \item Average projects a worker can complete in a year, long(short): $ \frac{ \text{period under consideration}}{\text{average completion rate}}  = \frac{12}{10}(\frac{12}{2})=1.2(6)$
    \item Required number of employees to accept all projects of type long(short): $10*2*5(1*2*3)=100(6)$
    \item Maximum possible acceptance of projects per year: $5*12=60(3*12=36)$
    \item Cash per month per project: $600/10=60(100/2=50)$
    \item Cash per month per employee: $600/(10*2)=30 (100/(2*1)=50) $
\end{itemize}
\end{frame}

\begin{frame}{Result}
\begin{itemize}
    \item So one has a higher cash amount per project($60>50$)
    \item The other has a higher amount per employee  $30 <50 $
    \item We can compute at how many employees this occurs, here at $n=10$
\end{itemize}
\end{frame}

\begin{frame}{Result}
Thank you for listening
\end{frame}


\end{document}
%\documentclass[AER]{AEA}
\documentclass[12pt]{report}
%\documentclass[12pt]{article}
%\documentclass[12pt,a4paper]{article}

\usepackage[utf8]{inputenc}


\usepackage{mathtools}
\usepackage{amsmath}
\usepackage{amssymb}
\usepackage{amsthm}

\usepackage{float}
%\usepackage[cmbold]{mathtime}
%\usepackage{mt11p}
\usepackage{placeins}
\usepackage{caption}
\usepackage{color}
\usepackage{subfigure}
\usepackage{multirow}
\usepackage{epsfig}
\usepackage{listings}
\usepackage{enumitem}
\usepackage{rotating,tabularx}
%\usepackage[graphicx]{realboxes}
\usepackage{graphicx}
\usepackage{graphics}
\usepackage{epstopdf}
\usepackage{longtable}
\usepackage[pdftex]{hyperref}
%\usepackage{breakurl}
\usepackage{epigraph}
\usepackage{xspace}
\usepackage{amsfonts}
\usepackage{eurosym}
\usepackage{ulem}

\usepackage{tikz}
\usetikzlibrary{spy}

\usepackage{verbatim}



\usepackage{footmisc}
\usepackage{comment}
\usepackage{setspace}
\usepackage{geometry}
\usepackage{caption}
\usepackage{pdflscape}
\usepackage{array}
\usepackage[authoryear]{natbib}
\usepackage{booktabs}
\usepackage{dcolumn}
\usepackage{mathrsfs}
%\usepackage[justification=centering]{caption}
%\captionsetup[table]{format=plain,labelformat=simple,labelsep=period,singlelinecheck=true}%
\bibliographystyle{apalike}
%\bibliographystyle{unsrtnat}



%\bibliographystyle{aea}
\usepackage{enumitem}
\usepackage{tikz}
\usetikzlibrary{positioning}
\usetikzlibrary{arrows}
\usetikzlibrary{shapes.multipart}

\usetikzlibrary{shapes}
\def\checkmark{\tikz\fill[scale=0.4](0,.35) -- (.25,0) -- (1,.7) -- (.25,.15) -- cycle;}
%\usepackage{tikz}
%\usetikzlibrary{snakes}
%\usetikzlibrary{patterns}

%\draftSpacing{1.5}

\usepackage{xcolor}
\hypersetup{
colorlinks,
linkcolor={blue!50!black},
citecolor={blue!50!black},
urlcolor={blue!50!black}}

%\renewcommand{\familydefault}{\sfdefault}
%\usepackage{helvet}
%\setlength{\parindent}{0.4cm}
%\setlength{\parindent}{2em}
%\setlength{\parskip}{1em}

%\normalem

%\doublespacing
\onehalfspacing
%\singlespacing
%\linespread{1.5}

\newtheorem{theorem}{Theorem}
\newtheorem{corollary}[theorem]{Corollary}
\newtheorem{proposition}{Proposition}
\newtheorem{definition}{Definition}
\newtheorem{axiom}{Axiom}
\newtheorem{observation}{Observation}
\newtheorem{assumption}{Assumption}	
\newtheorem{remark}{Remark}
\newtheorem{lemma}{Lemma}
\newtheorem{result}{result}


\newcommand{\ra}[1]{\renewcommand{\arraystretch}{#1}}

\newcommand{\E}{\mathrm{E}}
\newcommand{\Var}{\mathrm{Var}}
\newcommand{\Corr}{\mathrm{Corr}}
\newcommand{\Cov}{\mathrm{Cov}}

\newcolumntype{d}[1]{D{.}{.}{#1}} % "decimal" column type
\renewcommand{\ast}{{}^{\textstyle *}} % for raised "asterisks"

\newtheorem{hyp}{Hypothesis}
\newtheorem{subhyp}{Hypothesis}[hyp]
\renewcommand{\thesubhyp}{\thehyp\alph{subhyp}}

\newcommand{\red}[1]{{\color{red} #1}}
\newcommand{\blue}[1]{{\color{blue} #1}}

%\newcommand*{\qed}{\hfill\ensuremath{\blacksquare}}%

\newcolumntype{L}[1]{>{\raggedright\let\newline\\arraybackslash\hspace{0pt}}m{#1}}
\newcolumntype{C}[1]{>{\centering\let\newline\\arraybackslash\hspace{0pt}}m{#1}}
\newcolumntype{R}[1]{>{\raggedleft\let\newline\\arraybackslash\hspace{0pt}}m{#1}}

%\geometry{left=1.5in,right=1.5in,top=1.5in,bottom=1.5in}
\geometry{left=1in,right=1in,top=1in,bottom=1in}

\epstopdfsetup{outdir=./}

\newcommand{\elabel}[1]{\label{eq:#1}}
\newcommand{\eref}[1]{Eq.~(\ref{eq:#1})}
\newcommand{\ceref}[2]{(\ref{eq:#1}#2)}
\newcommand{\Eref}[1]{Equation~(\ref{eq:#1})}
\newcommand{\erefs}[2]{Eqs.~(\ref{eq:#1}--\ref{eq:#2})}

\newcommand{\Sref}[1]{Section~\ref{sec:#1}}
\newcommand{\sref}[1]{Sec.~\ref{sec:#1}}

\newcommand{\Pref}[1]{Proposition~\ref{prop:#1}}
\newcommand{\pref}[1]{Prop.~\ref{prop:#1}}
\newcommand{\preflong}[1]{proposition~\ref{prop:#1}}

\newcommand{\Aref}[1]{Axiom~\ref{ax:#1}}

\newcommand{\clabel}[1]{\label{coro:#1}}
\newcommand{\Cref}[1]{Corollary~\ref{coro:#1}}
\newcommand{\cref}[1]{Cor.~\ref{coro:#1}}
\newcommand{\creflong}[1]{corollary~\ref{coro:#1}}

\newcommand{\etal}{{\it et~al.}\xspace}
\newcommand{\ie}{{\it i.e.}\ }
\newcommand{\eg}{{\it e.g.}\ }
\newcommand{\etc}{{\it etc.}\ }
\newcommand{\cf}{{\it c.f.}\ }
\newcommand{\ave}[1]{\left\langle#1 \right\rangle}
\newcommand{\person}[1]{{\it \sc #1}}

\newcommand{\AAA}[1]{\red{{\it AA: #1 AA}}}
\newcommand{\YB}[1]{\blue{{\it YB: #1 YB}}}

\newcommand{\flabel}[1]{\label{fig:#1}}
\newcommand{\fref}[1]{Fig.~\ref{fig:#1}}
\newcommand{\Fref}[1]{Figure~\ref{fig:#1}}

\newcommand{\tlabel}[1]{\label{tab:#1}}
\newcommand{\tref}[1]{Tab.~\ref{tab:#1}}
\newcommand{\Tref}[1]{Table~\ref{tab:#1}}

\newcommand{\be}{\begin{equation}}
\newcommand{\ee}{\end{equation}}
\newcommand{\bea}{\begin{eqnarray}}
\newcommand{\eea}{\end{eqnarray}}

\newcommand{\bi}{\begin{itemize}}
\newcommand{\ei}{\end{itemize}}

\newcommand{\Dt}{\Delta t}
\newcommand{\Dx}{\Delta x}
\newcommand{\Epsilon}{\mathcal{E}}
\newcommand{\etau}{\tau^\text{eqm}}
\newcommand{\wtau}{\widetilde{\tau}}
\newcommand{\xN}{\ave{x}_N}
\newcommand{\Sdata}{S^{\text{data}}}
\newcommand{\Smodel}{S^{\text{model}}}

\newcommand{\del}{D}
\newcommand{\hor}{H}



\setlength{\parindent}{0.0cm}
\setlength{\parskip}{0.4em}

\numberwithin{equation}{section}
\DeclareMathOperator\erf{erf}
%\let\endtitlepage\relax
\DeclarePairedDelimiter\floor{\lfloor}{\rfloor}


% https://medium.com/@aerinykim/why-the-normal-gaussian-pdf-looks-the-way-it-does-1cbcef8faf0a

\begin{document}




\section{Agents on a line}

Suppose there are n people holding hands in a line where the first and last person are only holding hands with a single other person. Suppose the first person in the line is infected and every period he coughs and with probability p, his neighbor catches the virus. However, every period there is also some probability q, that the virus vaccine is created and everybody is instantly cured (an absorption state that can be reached from anywhere).

Note that the probability that at least one person is infected at time 1 is 100\% since patient 1 is the original infected person. I am interested in two things, what is the probability that after T periods, EXACTLY N people are infected and what is the probability that at LEAST N people are infected.


\subsection{Probability of cure when n people infected}

\begin{align}
\rho_n = p^{n-1}\frac{q}{q+p}
\end{align}

\subsection{probability of n people infected after t periods}

\subsection{2 people infected after T periods}

\begin{align}
 \sum_{t=1}^{T}p(1-p-q)^{t-1}
\end{align}

\subsection{3 people infected after T periods}
\begin{align}
\sum_{t=2}^{T}p^2(t-1)(1-p-q)^{t-2}
\end{align}

\subsection{4 people infected after t periods}

1 and 2 is 0. 

\begin{align}
\text{After 3 periods}:& p^3 \\
\text{After 4 periods}:& p^3 + p^33(1-p-q) \\
\text{After 5 periods}:& p^3 + p^33(1-p-q) + p^3 6(1-p-q)^2 \\
\text{After 6 periods}:& p^3 + p^33(1-p-q) + p^3 6(1-p-q)^2 + p^3 24(1-p-q)^2 \\
\text{General formula}:& \sum_{t=3}^{T}p^{3}(T-t)!*(1-p-q)^{t-3}
\end{align}

After 3 periods

\subsection{The general case:}

If the probabilities are independent. 

\begin{equation}
\binom{T}{k-1}p^{k-1}(1-q)^T
\end{equation}

If the probabilities are dependent in markov chain style. 

\begin{equation}
\binom{T}{k-1}p^{k-1} (1-p-q)^{T-(k-1)}
\end{equation}

https://math.stackexchange.com/a/3569528/295826

\bibliography{../thesisbib/bibliography}

\section{Complete Graph }

Things to include: 
\begin{align}
\text{individual probability of transmission:}& ~~p \\
\text{probability of vaccine cure:}& ~~v \\
\text{Probability of individual cure:}& ~~c \\
\text{Probability of death:}& ~~d
\end{align}

There is a complete graph, a single person is infected and has independent probability of spreading it. 

\subsection{If 3 people}

The first person is infected, so we are looking for the probabilities conditional on the fact that one person is infected. The transition probabilities if we are in the first persons are:
\begin{align}
\text{Probability that we stay at one:}& ~~(1-p)^2 \\
\text{Probability that we move to two:}&~~ 2p(1-p) \\
\text{Probability of movement to three:}& ~~p^2
\end{align}

If two people infected:

\begin{align} 
\text{Probability that we stay at two:}& ~~(1-q)^2 \\
\text{Probability of movement to three:}& ~~1-(1-q)^2
\end{align}

\subsection{If 4 people}

The first person is infected, so we are looking for the probabilities conditional on the fact that one person is infected. The transition probabilities if we are in the first persons are:
\begin{align}
\text{Probability that we stay at one:}& ~~(1-p)^3 \\
\text{Probability that we move to two:}& ~~3p(1-p)^2 \\
\text{Probability of movement to three:}& ~~3p^2(1-p) \\
\text{Probability of movement to four:}& ~~p^3
\end{align}

If two people infected:
\begin{align}
\text{Probability that stay at two:}&~~ (1-p)^4 \\
\text{Probability of movement to three:}& ~~4p(1-p)+2p^2(1-p)^2 \\
\text{Probability of movement to four:}& ~~p^4 + 4p^3+4p^2
\end{align}

If three people infected:
\begin{align}
\text{Probability of staying at three:}& ~~ (1-q)^3 \\
\text{Probability of movement to four:}& ~~1-(1-q)^3
\end{align}

\section{With probability of death}

$d: $probability of dying if you have disease
$q: $probability of global cure
$p: $probability of transmission
$k_i:$ expectd number of kills if i people infected 

\subsection{Two people}

\begin{align*}
k_1 = q*0+d*1+pk_2 \\
k_2 = q*0+d^2 2 + 2d(1-d)(1+k_1) \\
\end{align*}

% https://twitter.com/nntaleb/status/1239171413342289921?s=20
% https://slideplayer.com/slide/6293982/
% https://twitter.com/HarryDCrane/status/1237446399060475904?s=20
% Crane:
% https://www.researchers.one/article/2020-03-9
% https://www.researchers.one/article/2020-03-8

% https://static1.squarespace.com/static/5b68a4e4a2772c2a206180a1/t/5e737b95403f772d8ce0e04a/1584626591711/CommunityPrevention.pdf

% https://python-intro.quantecon.org/kalman.html

% https://arxiv.org/pdf/1505.00768.pdf

% https://www.cut-the-knot.org/Probability/SeekingHeads.shtml

% https://arxiv.org/pdf/1907.11162.pdf

% https://medium.com/datadriveninvestor/risk-aversion-emerges-evolutionary-in-non-ergodic-systems-401a3e0b37c1

% https://twitter.com/hulme_oliver/status/1242568292197376007?s=19

% https://twitter.com/SimonDeDeo/status/1242562201057136643?s=19

% https://twitter.com/asymmetricinfo/status/1242559022592598022?s=19

% https://twitter.com/paulmromer/status/1242502048865804289?s=19

% http://statisticallyinsignificant.uk/

% https://arxiv.org/pdf/1505.00768.pdf

% https://twitter.com/phl43/status/1240943300879822848?s=20

% https://alhill.shinyapps.io/COVID19seir/

% https://mikesmathpage.wordpress.com/2020/03/16/having-kids-look-at-alison-lynn-hills-amazing-corona-virus-simulation-program/

% https://twitter.com/joshgans/status/1247541002690162689?s=19

% https://mikesmathpage.wordpress.com/2020/03/18/sharing-nassim-talebs-ideas-about-virus-models-with-kids/


% https://www.lemonde.fr/les-decodeurs/article/2020/01/29/coronavirus-zika-ebola-quelles-maladies-sont-les-plus-contagieuses-ou-les-plus-mortelles_6027661_4355770.html


\end{document}

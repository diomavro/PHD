%\documentclass[AER]{AEA}
\documentclass[12pt]{report}
%\documentclass[12pt]{article}
%\documentclass[12pt,a4paper]{article}

\usepackage[utf8]{inputenc}


\usepackage{mathtools}
\usepackage{amsmath}
\usepackage{amssymb}
\usepackage{amsthm}

\usepackage{float}
%\usepackage[cmbold]{mathtime}
%\usepackage{mt11p}
\usepackage{placeins}
\usepackage{caption}
\usepackage{color}
\usepackage{subfigure}
\usepackage{multirow}
\usepackage{epsfig}
\usepackage{listings}
\usepackage{enumitem}
\usepackage{rotating,tabularx}
%\usepackage[graphicx]{realboxes}
\usepackage{graphicx}
\usepackage{graphics}
\usepackage{epstopdf}
\usepackage{longtable}
\usepackage[pdftex]{hyperref}
%\usepackage{breakurl}
\usepackage{epigraph}
\usepackage{xspace}
\usepackage{amsfonts}
\usepackage{eurosym}
\usepackage{ulem}

\usepackage{tikz}
\usetikzlibrary{spy}

\usepackage{verbatim}



\usepackage{footmisc}
\usepackage{comment}
\usepackage{setspace}
\usepackage{geometry}
\usepackage{caption}
\usepackage{pdflscape}
\usepackage{array}
\usepackage[authoryear]{natbib}
\usepackage{booktabs}
\usepackage{dcolumn}
\usepackage{mathrsfs}
%\usepackage[justification=centering]{caption}
%\captionsetup[table]{format=plain,labelformat=simple,labelsep=period,singlelinecheck=true}%
\bibliographystyle{apalike}
%\bibliographystyle{unsrtnat}



%\bibliographystyle{aea}
\usepackage{enumitem}
\usepackage{tikz}
\usetikzlibrary{positioning}
\usetikzlibrary{arrows}
\usetikzlibrary{shapes.multipart}

\usetikzlibrary{shapes}
\def\checkmark{\tikz\fill[scale=0.4](0,.35) -- (.25,0) -- (1,.7) -- (.25,.15) -- cycle;}
%\usepackage{tikz}
%\usetikzlibrary{snakes}
%\usetikzlibrary{patterns}

%\draftSpacing{1.5}

\usepackage{xcolor}
\hypersetup{
colorlinks,
linkcolor={blue!50!black},
citecolor={blue!50!black},
urlcolor={blue!50!black}}

%\renewcommand{\familydefault}{\sfdefault}
%\usepackage{helvet}
%\setlength{\parindent}{0.4cm}
%\setlength{\parindent}{2em}
%\setlength{\parskip}{1em}

%\normalem

%\doublespacing
\onehalfspacing
%\singlespacing
%\linespread{1.5}

\newtheorem{theorem}{Theorem}
\newtheorem{corollary}[theorem]{Corollary}
\newtheorem{proposition}{Proposition}
\newtheorem{definition}{Definition}
\newtheorem{axiom}{Axiom}
\newtheorem{observation}{Observation}
\newtheorem{assumption}{Assumption}	
\newtheorem{remark}{Remark}
\newtheorem{lemma}{Lemma}
\newtheorem{result}{result}


\newcommand{\ra}[1]{\renewcommand{\arraystretch}{#1}}

\newcommand{\E}{\mathrm{E}}
\newcommand{\Var}{\mathrm{Var}}
\newcommand{\Corr}{\mathrm{Corr}}
\newcommand{\Cov}{\mathrm{Cov}}

\newcolumntype{d}[1]{D{.}{.}{#1}} % "decimal" column type
\renewcommand{\ast}{{}^{\textstyle *}} % for raised "asterisks"

\newtheorem{hyp}{Hypothesis}
\newtheorem{subhyp}{Hypothesis}[hyp]
\renewcommand{\thesubhyp}{\thehyp\alph{subhyp}}

\newcommand{\red}[1]{{\color{red} #1}}
\newcommand{\blue}[1]{{\color{blue} #1}}

%\newcommand*{\qed}{\hfill\ensuremath{\blacksquare}}%

\newcolumntype{L}[1]{>{\raggedright\let\newline\\arraybackslash\hspace{0pt}}m{#1}}
\newcolumntype{C}[1]{>{\centering\let\newline\\arraybackslash\hspace{0pt}}m{#1}}
\newcolumntype{R}[1]{>{\raggedleft\let\newline\\arraybackslash\hspace{0pt}}m{#1}}

%\geometry{left=1.5in,right=1.5in,top=1.5in,bottom=1.5in}
\geometry{left=1in,right=1in,top=1in,bottom=1in}

\epstopdfsetup{outdir=./}

\newcommand{\elabel}[1]{\label{eq:#1}}
\newcommand{\eref}[1]{Eq.~(\ref{eq:#1})}
\newcommand{\ceref}[2]{(\ref{eq:#1}#2)}
\newcommand{\Eref}[1]{Equation~(\ref{eq:#1})}
\newcommand{\erefs}[2]{Eqs.~(\ref{eq:#1}--\ref{eq:#2})}

\newcommand{\Sref}[1]{Section~\ref{sec:#1}}
\newcommand{\sref}[1]{Sec.~\ref{sec:#1}}

\newcommand{\Pref}[1]{Proposition~\ref{prop:#1}}
\newcommand{\pref}[1]{Prop.~\ref{prop:#1}}
\newcommand{\preflong}[1]{proposition~\ref{prop:#1}}

\newcommand{\Aref}[1]{Axiom~\ref{ax:#1}}

\newcommand{\clabel}[1]{\label{coro:#1}}
\newcommand{\Cref}[1]{Corollary~\ref{coro:#1}}
\newcommand{\cref}[1]{Cor.~\ref{coro:#1}}
\newcommand{\creflong}[1]{corollary~\ref{coro:#1}}

\newcommand{\etal}{{\it et~al.}\xspace}
\newcommand{\ie}{{\it i.e.}\ }
\newcommand{\eg}{{\it e.g.}\ }
\newcommand{\etc}{{\it etc.}\ }
\newcommand{\cf}{{\it c.f.}\ }
\newcommand{\ave}[1]{\left\langle#1 \right\rangle}
\newcommand{\person}[1]{{\it \sc #1}}

\newcommand{\AAA}[1]{\red{{\it AA: #1 AA}}}
\newcommand{\YB}[1]{\blue{{\it YB: #1 YB}}}

\newcommand{\flabel}[1]{\label{fig:#1}}
\newcommand{\fref}[1]{Fig.~\ref{fig:#1}}
\newcommand{\Fref}[1]{Figure~\ref{fig:#1}}

\newcommand{\tlabel}[1]{\label{tab:#1}}
\newcommand{\tref}[1]{Tab.~\ref{tab:#1}}
\newcommand{\Tref}[1]{Table~\ref{tab:#1}}

\newcommand{\be}{\begin{equation}}
\newcommand{\ee}{\end{equation}}
\newcommand{\bea}{\begin{eqnarray}}
\newcommand{\eea}{\end{eqnarray}}

\newcommand{\bi}{\begin{itemize}}
\newcommand{\ei}{\end{itemize}}

\newcommand{\Dt}{\Delta t}
\newcommand{\Dx}{\Delta x}
\newcommand{\Epsilon}{\mathcal{E}}
\newcommand{\etau}{\tau^\text{eqm}}
\newcommand{\wtau}{\widetilde{\tau}}
\newcommand{\xN}{\ave{x}_N}
\newcommand{\Sdata}{S^{\text{data}}}
\newcommand{\Smodel}{S^{\text{model}}}

\newcommand{\del}{D}
\newcommand{\hor}{H}



\setlength{\parindent}{0.0cm}
\setlength{\parskip}{0.4em}

\numberwithin{equation}{section}
\DeclareMathOperator\erf{erf}
%\let\endtitlepage\relax
\DeclarePairedDelimiter\floor{\lfloor}{\rfloor}


% https://medium.com/@aerinykim/why-the-normal-gaussian-pdf-looks-the-way-it-does-1cbcef8faf0a

\begin{document}

A virus has a tradeoff between infection and lethality. Though the virus is not a rational agent which chooses these two quantities, it is useful to think of what an agent who was designing the virus would choose if their goal was to maximize the number of deaths . It is not strictly possible to say that a virus which selects for this optimal infection/lethality ratio would be favored from an evolutionary point of view since the death of host would entail the eventual death of the virus. 

Let us first stipulate a world. All agents are either healthy, infected, or dead, and there is no cure avaiallable and each period infected agents have some chance of dying, whilst healthy agents have some chance of becoming infected. There is no cure, and the dead cannot infect the healthy. 

The virus optimal behavior depends on the amount of time it has left to kill it's agents. If a virus has unlimited time to continue infecting agents, then the most dangerous virus would, ironically, minimizes lethality, and any non-zero level of infectivity would eventually kill all agents. Maximizing infectivity would not affect the number of hosts to be infected but it would affect the expected time until all agents were dead. So for lethality to be a part of the optimal virus strategy it must be either that the virus agent has a time preference or that there is time pressure to kill quickly. In this paper we choose the latter approach and give the virus time pressure by having an exogenous probability of vaccination\footnote{alternatively one could also just change the optimand of the virus to growth rate}.

From a biological point of view, it may be that a virus would want lethality because it would be correlated with other attributes that help it maximize survival. It could be that within hosts, a virus has competition, and to properly survive the competitive struggle with other mechanisms the virus needs to be lethal\footnote{Alternatively, if the virus was not aiming to maximize the number of deaths but was instead maximizing survival, then it may want to have lethality for a population control function}.

The main ideas explored in this paper is that a virus which wants to maximize the number of deaths will adapt it's strategy to the environment. We will aim to illustrate the example with a simple 2 person illustration and give proofs for the general cases in the appendix. 

The main results is that the expected number of deaths is a monotonic function of transmission rate, that is, a virus will always maximize it's transmission rate. This is an intuitive idea, the virus has very little reason not to maximize the speed of it's infection, both from an evolutionary reasons and as agent maximizing the number of deaths. However, as we will see, if the virus only has a limited amount of resources to spend on either infection or lethality, there is some tradeoff to be made. 

On the other hand a virus will adjust it's lethality based on the vaccination rate. 


\section{Dynamics}

On the other hand, if a virus can adapt, that is, have a strategy for each markovian state we can also produce a similar analysis. Specifically we can show that the virus will increase the death rate 

\section{Model}

We assume there is a complete graph, each node represents an individual potential host whilst the edges represent the fact that these hosts have connections to other hosts. Each host can take on three states, healthy, infected, and dead. A healthy host cannot spread the virus but is vulnerable to being infected. We represent the total number of agents in the graph by $N$, the number of infected by $\alpha$ and the number of dead by $\beta$. As such the number of healthy people is given by $N-\alpha-\beta$. 

An infected host has an independent has an independent probability, $p$ of sprading the virus to it's neighbors, which is every healthy patient. The interpretation of independent probability is that each host can spread to any given agent in such a way that the more infected agents there exists, the higher is the probability that a healthy agent will be infected. 

A dead host is simply removed from the pool agents and will no longer be able to either receive the virus or spread it. Each infected host has an indepdent probability of dying every period, $d$. The number of dead hosts is the optimand of the virus, that is, the it will attempt to maximize the number of agents who have this state. 

Finally at every period, there is an independent probability, $q$ of a vaccine being invented. The vaccine will instantly cure all infected agents and functionall works like a markovian absorbing barrier. Unlike the other probabilities, $q$ does not depend on the number of infected or the number of healthy hosts. 

To summarize the variables: we have, two control variables,the infection probability, $p$ and the death rate, $d$; two state variables, the number of people infected, $\alpha$ and the number of dead people, $\beta$, and two exogenous parameters, the vaccination probability $q$, and the total number of of people, $N$.  

The function that the virus will maximize is the expected number of dead, we will represent this as an expectation: 

\begin{equation*}
E_N[deaths | infected= \alpha, dead = \beta ] = E_N(\alpha ,\beta) = E_N(\alpha ,\beta, p, d) 
\end{equation*}

We begin by describing some properties of this function. 

\begin{assumption}{Death as absorbing barrier}\label{ass1}
\begin{equation}
\alpha = \beta \rightarrow E_N(\alpha,\beta)= \beta
\end{equation}
\end{assumption}

This properly simply say that if all the infected people are dead, the expectation of the number of dead is simply the number of dead. This is simply saying that dead hosts cannot spread the virus. 


\begin{assumption}{Death invariance}\label{ass2}
$E_{N+t}(\alpha +t,\beta +t, p, d)=E_{N}(\alpha , \beta ,p, d) + t$.
\end{assumption}

This property is intuitive, it says that only the absolute number of infected hosts who are alive and the number of healthy hosts matter. So if initially there was a population of $100$ but now $95$ of them have died and there remains a single infected hosts and $4$ healthy hosts, this is fundamentally the same situation as if the graph initially had $10$ hosts and $5$ had died, whilst $1$ infected host remains. The only difference between these two situations is the stock of the dead hosts and a virus which is optimizing at this point in time would have the same strategy in both situation. 

With 1 person infected out of N, this looks like the following: 

\begin{align*}
E_N(1,\beta,p,d) = d + \sum_{i=1}^{N-1}p^i(1-p)^{N-i-1}\binom{N-1}{i}E(i+1,0,p,d ) + q \beta
\end{align*}

Expected number of deaths if there are n infected and $\beta$ dead
\begin{align*}
E_N(\alpha,\beta,p,d ) = \sum_{i = 1+ \beta}^{\alpha} \binom{x}{i}(1-d)^{x-i}d^i E( \alpha , i,p,d ) + \sum_{i=\alpha+1}^{N-\alpha} P(X=i)E(i,y,p,d ) + q \beta
\end{align*}

\begin{align*}
r: &=1-(1-p)^{\alpha} \\
P(X=x)&=P\left(\sum_{i=1}^{N-\alpha}B_i=x-\alpha \right) \\
&=\binom{N- \alpha}{x-\alpha}r^{x-\alpha}(1-r)^{N-x}
\end{align*}

\begin{proposition}{Monotonicity in the transmission rate}\label{prop1}
In all cases, $E_N$ is a strictly increasing 
\end{proposition}

An interesting mathematical property is that if we imagine two viruses, which have the same expected death rate, one of them with a higher death rate and lower transmission rate, and another one with a higher transmission and higher death rate. It seems like if we increase the lethality of the higher transmission virus, this will result in less danger than an increase in the transmission rate of a high lethality virus. 


\section{An example}

We focus on an example with $N=2$ with simply one 


We begin by illustrating the central results with $N=2$ and then generalize. 

We let E be the expected number of deaths. We model this process as a markov chain, where $E_N(1,0)$, indicates that one host is infected, $0$ are dead and there are n total hosts. We illustrate using an $N=2$ case.

\begin{align*}
E(1,0) = &\underbrace{d*1}_\text{probability of 1 death} + 
    \underbrace{pE(2,0) }_\text{transition prob to two infected} +
    \underbrace{(1-d-p-q)E(1,0)}_\text{stay probability} +
    \underbrace{q*0}_\text{probability of all cured} \\
	= &\underbrace{d}_\text{probability of 1 death}+ 
    \underbrace{pE(2,0) }_\text{transition prob to two infected}+ 
    \underbrace{(1-d-p-q)E(1,0)}_\text{stay probability}  \\
E(2,0) = &\underbrace{d^2* 2}_\text{probability of 2 deaths} + 
    \underbrace{2d(1-d)E(2,1) }_\text{probability of 1 death } +
    \underbrace{(1-(1-(1-d)^2)-q)E(2,0) }_\text{stay probability} +
    \underbrace{q*0}_\text{probability of all cured} \\
	   = &\underbrace{d^2* 2}_\text{probability of 2 deaths} + 
    \underbrace{2d(1-d)E(2,1) }_\text{probability of 1 death } +
    \underbrace{(1-d)^2-q)E(2,0) }_\text{stay probability} \\
E(2,1) = &\underbrace{d*2}_\text{probability of 2 deaths} + 
    \underbrace{(1-d-q)E(2,1) }_\text{stay probability} +
    \underbrace{q}_\text{probability of alive cured}
\end{align*}

\begin{align*}
E(1,0) = & \frac{d(d+2p+q)}{(d+q)(d+p+q)} \\
E(2,0) = & \frac{2d}{d+q} \\
E(2,1) = & \frac{2d+q}{d+q} 
\end{align*}

\section{Agents on a line}

Suppose there are n people holding hands in a line where the first and last person are only holding hands with a single other person. Suppose the first person in the line is infected and every period he coughs and with probability p, his neighbor catches the virus. However, every period there is also some probability q, that the virus vaccine is created and everybody is instantly cured (an absorption state that can be reached from anywhere).

Note that the probability that at least one person is infected at time 1 is 100\% since patient 1 is the original infected person. I am interested in two things, what is the probability that after T periods, EXACTLY N people are infected and what is the probability that at LEAST N people are infected.


\subsection{Probability of cure when n people infected}

\begin{align}
\rho_n = p^{n-1}\frac{q}{q+p}
\end{align}

\subsection{probability of n people infected after t periods}

\subsection{2 people infected after T periods}

\begin{align}
 \sum_{t=1}^{T}p(1-p-q)^{t-1}
\end{align}

\subsection{3 people infected after T periods}
\begin{align}
\sum_{t=2}^{T}p^2(t-1)(1-p-q)^{t-2}
\end{align}

\subsection{4 people infected after t periods}

1 and 2 is 0. 

\begin{align}
\text{After 3 periods}:& p^3 \\
\text{After 4 periods}:& p^3 + p^33(1-p-q) \\
\text{After 5 periods}:& p^3 + p^33(1-p-q) + p^3 6(1-p-q)^2 \\
\text{After 6 periods}:& p^3 + p^33(1-p-q) + p^3 6(1-p-q)^2 + p^3 24(1-p-q)^2 \\
\text{General formula}:& \sum_{t=3}^{T}p^{3}(T-t)!*(1-p-q)^{t-3}
\end{align}

After 3 periods

\subsection{The general case:}

If the probabilities are independent. 

\begin{equation}
\binom{T}{k-1}p^{k-1}(1-q)^T
\end{equation}

If the probabilities are dependent in markov chain style. 

\begin{equation}
\binom{T}{k-1}p^{k-1} (1-p-q)^{T-(k-1)}
\end{equation}

https://math.stackexchange.com/a/3569528/295826

\bibliography{../thesisbib/bibliography}

\section{Complete Graph }

Things to include: 
\begin{align}
\text{individual probability of transmission:}& ~~p \\
\text{probability of vaccine cure:}& ~~v \\
\text{Probability of individual cure:}& ~~c \\
\text{Probability of death:}& ~~d
\end{align}

There is a complete graph, a single person is infected and has independent probability of spreading it. 

\subsection{If 3 people}

The first person is infected, so we are looking for the probabilities conditional on the fact that one person is infected. The transition probabilities if we are in the first persons are:
\begin{align}
\text{Probability that we stay at one:}& ~~(1-p)^2 \\
\text{Probability that we move to two:}&~~ 2p(1-p) \\
\text{Probability of movement to three:}& ~~p^2
\end{align}

If two people infected:

\begin{align} 
\text{Probability that we stay at two:}& ~~(1-q)^2 \\
\text{Probability of movement to three:}& ~~1-(1-q)^2
\end{align}

\subsection{If 4 people}

The first person is infected, so we are looking for the probabilities conditional on the fact that one person is infected. The transition probabilities if we are in the first persons are:
\begin{align}
\text{Probability that we stay at one:}& ~~(1-p)^3 \\
\text{Probability that we move to two:}& ~~3p(1-p)^2 \\
\text{Probability of movement to three:}& ~~3p^2(1-p) \\
\text{Probability of movement to four:}& ~~p^3
\end{align}

If two people infected:
\begin{align}
\text{Probability that stay at two:}&~~ (1-p)^4 \\
\text{Probability of movement to three:}& ~~4p(1-p)+2p^2(1-p)^2 \\
\text{Probability of movement to four:}& ~~p^4 + 4p^3+4p^2
\end{align}

If three people infected:
\begin{align}
\text{Probability of staying at three:}& ~~ (1-q)^3 \\
\text{Probability of movement to four:}& ~~1-(1-q)^3
\end{align}

\section{With probability of death}

$d: $probability of dying if you have disease
$q: $probability of global cure
$p: $probability of transmission
$k_i:$ expectd number of kills if i people infected 

\subsection{Two people}

\begin{align*}
k_1 = q*0+d*1+pk_2 \\
k_2 = q*0+d^2 2 + 2d(1-d)(1+k_1) \\
\end{align*}

\subsection{Optimal virus with stay prob, 2 agents}

$E[deaths | infected=x, dead = y ] = E(x,y) $


\begin{align*}
E(1,0) = &\underbrace{d*1}_\text{probability of 1 death} + 
    \underbrace{pE(2,0) }_\text{transition prob to two infected} +
    \underbrace{(1-d-p-q)E(1,0)}_\text{stay probability} +
    \underbrace{q*0}_\text{probability of all cured} \\
	= &\underbrace{d}_\text{probability of 1 death}+ 
    \underbrace{pE(2,0) }_\text{transition prob to two infected}+ 
    \underbrace{(1-d-p-q)E(1,0)}_\text{stay probability}  \\
E(2,0) = &\underbrace{d^2* 2}_\text{probability of 2 deaths} + 
    \underbrace{2d(1-d)E(2,1) }_\text{probability of 1 death } +
    \underbrace{(1-(1-(1-d)^2)-q)E(2,0) }_\text{stay probability} +
    \underbrace{q*0}_\text{probability of all cured} \\
	   = &\underbrace{d^2* 2}_\text{probability of 2 deaths} + 
    \underbrace{2d(1-d)E(2,1) }_\text{probability of 1 death } +
    \underbrace{(1-d)^2-q)E(2,0) }_\text{stay probability} \\
E(2,1) = &\underbrace{d*2}_\text{probability of 2 deaths} + 
    \underbrace{(1-d-q)E(2,1) }_\text{stay probability} +
    \underbrace{q}_\text{probability of alive cured}
\end{align*}

\begin{align*}
E(1,0) = & \frac{d(d+2p+q)}{(d+q)(d+p+q)} \\
E(2,0) = & \frac{2d}{d+q} \\
E(2,1) = & \frac{2d+q}{d+q} 
\end{align*}

With n people: 

\begin{align*}
E_N(1,y) = d + \sum_{i=1}^{N-1}p^i(1-p)^{N-i-1}\binom{N-1}{i}E(i+1,0) + (q y)
\end{align*}

Expected number of deaths if there are n infected and y dead
\begin{align*}
E_N(n,y) = \sum_{i = 1+y}^n \binom{x}{i}(1-d)^{x-i}d^i E(n,i) + \sum_{i=n+1}^{N-n} P(X=i)E(i,y) + q y
\end{align*}

\begin{align*}
D_4(2,0) =  2(1-d)d D_4(2,1) + d^2 2  \\
D_N(n,y) = (N-)(1-d)^{N-} d D_N(n,y+1)
\sum_{i=y+1}^n (1-n) D_N(n,i)
\end{align*}


\begin{align*}
r: &=1-(1-p)^n \\
P(X=x)&=P\left(\sum_{i=1}^{N-n}B_i=x-n\right) \\
&=\binom{N-n}{x-n}r^{x-n}(1-r)^{N-x}
\end{align*}




% \subsection{Optimal virus with stay prob, 3 agents}

% \begin{align*}
% E(1,0) = & d + 
%     2p(1-p) E(2,0)  +
%     p^2 E(3,0)  +
%     ((1-p)^2-q-d)E(1,0)  \\
% %%%%%%%%%%%%%%%%%%%%%%%%%%%%%%%%%%%%%%%%%%%%
% E(2,0) = & 
% d^2*2 + 
% (1-(1-p)^2) E(3,0)  +
% 2d(1-d) E(2,1) +
% ((1-d)^2-q)E(2,0)  \\
% %%%%%%%%%%%%%%%%%%%%%%%%%%%%%%%%%%%%%%%%%%%%
% E(2,1) = &
% d 2 +
% pE(3,1) +   
% q * 1 + 
% (1-d-p-q)E(2,1)  \\
% %%%%%%%%%%%%%%%%%%%%%%%%%%%%%%%%%%%%%%%%%%%%
% E(3,0) = & 
% 3d^2(1-d)E(3,2)  + 
% 3d(1-d)^2E(3,1) +
% d^3 3 +
% ((1-d)^3-q)E(3,0) \\
% %%%%%%%%%%%%%%%%%%%%%%%%%%%%%%%%%%%%%%%%%%%%
% E(3,1) = & 
% d^2 3 +
% 2d(1-d) E(3,2) +
% q*1 +
% ((1-q)^2)-q) E(3,1) \\
% %%%%%%%%%%%%%%%%%%%%%%%%%%%%%%%%%%%%%%%%%%%%
% E(3,2) = & 
% d 3 +
% q*2 +
% (1-d-q) E(3,2) \\
% \end{align*}

% \begin{align*}
% E(3,2) = & 
% \frac{d 3 + q*2}{d+q} \\
% %%%%%%%%%%%%%%%%%%%%%%%%%%%%%%%%%%%%%%%%%%%%
% E(3,1) = & 
% d^2 3 +
% 2d(1-d) \frac{d 3 + q*2}{{d+q}}  +
% q*1 +
% ((1-q)^2-q) E(3,1) \\
% =& \frac{d^2 3 +2d(1-d) \frac{d 3 + q*2}{{d+q}} +q }{2q-q^2} = \frac{3 d^2 (d-q) +2d(1-d) (3d  + q*2) +(d-q)q }{(d-q)(2q-q^2)} \\
% =& \frac{5 d q-3 d^3-d^2 (7 q-6)-q^2}{(2-q) q(d-q)} \\
% %%%%%%%%%%%%%%%%%%%%%%%%%%%%%%%%%%%%%%%%%%%%
% E(2,1) = &
% \frac{2d+q+\frac{p(5 d q-3 d^3-d^2 (7 q-6)-q^2}{(2-q) q(d-q)} )}{p+q+d} = \frac{2d+q}{p+q+d} + \frac{p(5 d q-3 d^3-d^2 (7 q-6)-q^2 )}{(2-q) q(d-q)(p+q+d)}\\
% =& \frac{2d(2-q) q(d-q)+q(2-q) q^2(d-q)+p(5 d q-3 d^3-d^2 (7 q-6)-q^2 )}{(p+q+d)(2-q) q(d-q)} 
% \end{align*}

%%%%%%%%%%%%%%%%%%%%%%%%%%%%%%%%%%%%%%%%%%%%%%%%%%%%%%%%%%%%%%%%%%%%%%%%%%%%%%%%%%%%%%%%%%%%%%%%%%%%%%%%%%%%%%%%%%%%%%%%%%%%%%%%%%%%%%%%%%%%%%%%%%%%%%%%%%%%%%%%%%%%%%%%%%%%%%%%%%%%%%%%%%%%%%%%%%%%%%%%%%%%%%%%%%%%%%%%%%%%%%%%%%%%%%%%%%%%%%%%%%%%%%%%%%%%%%%%%%%%%%%%%%%%%%%%%%%%%%%%%%%%%%%%%%%%%%%%%%%%%%%%%%%%%%%%%%%%%%%%%%%%%%%%%%%%%%%%%%%%%%%%%%%%%%%%%%%%%%%%%%%%%%%%%%%%%%%%%%%%%%%%%%%%%%%%%%%%%%%%%%%%%%%%%%%%%%%%%%%%%%%%%%%%%%%%%%%%%%%%%%%%%%%%%%%%%%%%%%%%%%%%%%%%%%%%%%%%%%%%%%%%%%%%%%%%%%%%%%%%%%%%%%%%%%%%%%%%%%%%%%%%%%%%%%%%%%%%%%%%%%%%%%%%%%%%%%%%%%%%%%%%%%%%%%%%%%%%%%%%%%%%%%%%%%%%%%%%%%%%%%%%%%%%%%%%%%%%%%%%%%%%%%%%%%%%%%%%%%%%%%%%%%%%%%%%%%%%%%%%%%%%%%%%%%%%%%%%%%%%%%%%%%%%%%%%%%%%%%%%%%%%%%%%%%%%%%%%%%%%%%%%%%%%%%%%%%%%%%%%%%%%%%%%%%%%%%%%%%%%%%%%%%%%%%%%%%%%%%%%%%%%%%%%%%%%%%%%%%%%%%%%%%%%%%%%%%%%%%%%%%%%%%%%%%%%%%%%%%%%%%%%%%%%%%%%%%%%%%%%%%%%%%%%%%%%%%%%%%%%%%%%%%%%%%%%%%%%%%%%%%%%%%%%%%%%%%%%%%%%%%%%%%%%%%%%%%%%%%%%%%%%%%%%%%%%%%%%%%%%%%%%%%%%%%%%%%%%%%%%%%%%%%%%%%%%%%%%%%%%%%%%%%%%%%%%%%%%%%%%%%%%%%%%%%%%%%%%%%%%%%%%%%%%%%%%%%%%%%%%%%%%%%%%%%%%%%%%%%%%%%%%%%%%%%%%%%%%%%%%%%%%%%%%%%%%%%%%%%%%%%%%%%%%%%%%%%%%%%%%%%%%%%%%%%%%%%%%%%%%%%%%%%%%%%%%%%%%%%%%%%%%%%%%%%%%%%%%%%%%%%%%%%%%%%%%%%%%%%%%%%%%%%%%%%%%%%%%%%%%%%%%%%%%%%%%%%%%%%%%%%%%%%%%%%%%%%%%%%%%%%%%%%%%%%%%%%%%%%%%%%%%%%%%%%%%%%%%%%%%%%%%%%%%%%%%%%%%%%%%%%%%%%%%%%%%%%%%%%%%%%%%%%%%%%%%%%%%%%%%%%%%%%%%%%%%%%%%%%%%%%%%%%%%%%%%%%%%%%%%%%%%%%%%%%%%%%%%%%%%%%%%%%%%%%%%%%%%%%%%%%%%%%%%%%%%%%%%%%%%%%%%%%%%%%%%%%%%
% \subsection{Optimal virus without stay }


% \begin{align*}
% E[deaths | infected=1, dead = 0 ]=& 
%     \underbrace{d}_\text{probability of 1 death} + 
%     \underbrace{pb}_\text{transition prob to two infected} \\
% E[deaths | infected=2, dead = 0 ]=& 
%     \underbrace{d^2* 2}_\text{probability of 2 deaths} + 
%     \underbrace{2d(1-d)c}_\text{probability of 1 death }\\
% E[deaths | infected=2, dead = 1 ]=& 
%     \underbrace{d2* 2}_\text{probability of 2 deaths} + 
%     \underbrace{q}_\text{probability of alive cured}
% \end{align*}

% \begin{align*}
% E[deaths | infected=1, dead = 0 ]=& d(1-4d^2 p +2dp(3-q)+2p*q) \\
% E[deaths | infected=2, dead = 0 ]=& 2d(q+d(3-q)-2d^2) \\
% E[deaths | infected=2, dead = 1 ]=& 2d+q
% \end{align*}


% https://twitter.com/nntaleb/status/1239171413342289921?s=20
% https://slideplayer.com/slide/6293982/
% https://twitter.com/HarryDCrane/status/1237446399060475904?s=20
% Crane:
% https://www.researchers.one/article/2020-03-9
% https://www.researchers.one/article/2020-03-8

% https://static1.squarespace.com/static/5b68a4e4a2772c2a206180a1/t/5e737b95403f772d8ce0e04a/1584626591711/CommunityPrevention.pdf

% https://python-intro.quantecon.org/kalman.html

% https://arxiv.org/pdf/1505.00768.pdf

% https://www.cut-the-knot.org/Probability/SeekingHeads.shtml

% https://arxiv.org/pdf/1907.11162.pdf

% https://medium.com/datadriveninvestor/risk-aversion-emerges-evolutionary-in-non-ergodic-systems-401a3e0b37c1

% https://twitter.com/hulme_oliver/status/1242568292197376007?s=19

% https://twitter.com/SimonDeDeo/status/1242562201057136643?s=19

% https://twitter.com/asymmetricinfo/status/1242559022592598022?s=19

% https://twitter.com/paulmromer/status/1242502048865804289?s=19

% http://statisticallyinsignificant.uk/

% https://arxiv.org/pdf/1505.00768.pdf

% https://twitter.com/phl43/status/1240943300879822848?s=20

% https://alhill.shinyapps.io/COVID19seir/

% https://mikesmathpage.wordpress.com/2020/03/16/having-kids-look-at-alison-lynn-hills-amazing-corona-virus-simulation-program/

% https://twitter.com/joshgans/status/1247541002690162689?s=19

% https://mikesmathpage.wordpress.com/2020/03/18/sharing-nassim-talebs-ideas-about-virus-models-with-kids/


% https://www.lemonde.fr/les-decodeurs/article/2020/01/29/coronavirus-zika-ebola-quelles-maladies-sont-les-plus-contagieuses-ou-les-plus-mortelles_6027661_4355770.html


\end{document}

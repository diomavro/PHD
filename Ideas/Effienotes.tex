%\documentclass[AER]{AEA}
\documentclass[12pt]{report}
%\documentclass[12pt]{article}
%\documentclass[12pt,a4paper]{article}

\usepackage[utf8]{inputenc}


\usepackage{mathtools}
\usepackage{amsmath}
\usepackage{amssymb}
\usepackage{amsthm}

\usepackage{float}
%\usepackage[cmbold]{mathtime}
%\usepackage{mt11p}
\usepackage{placeins}
\usepackage{caption}
\usepackage{color}
\usepackage{subfigure}
\usepackage{multirow}
\usepackage{epsfig}
\usepackage{listings}
\usepackage{enumitem}
\usepackage{rotating,tabularx}
%\usepackage[graphicx]{realboxes}
\usepackage{graphicx}
\usepackage{graphics}
\usepackage{epstopdf}
\usepackage{longtable}
\usepackage[pdftex]{hyperref}
%\usepackage{breakurl}
\usepackage{epigraph}
\usepackage{xspace}
\usepackage{amsfonts}
\usepackage{eurosym}
\usepackage{ulem}

\usepackage{tikz}
\usetikzlibrary{spy}

\usepackage{verbatim}



\usepackage{footmisc}
\usepackage{comment}
\usepackage{setspace}
\usepackage{geometry}
\usepackage{caption}
\usepackage{pdflscape}
\usepackage{array}
\usepackage[authoryear]{natbib}
\usepackage{booktabs}
\usepackage{dcolumn}
\usepackage{mathrsfs}
%\usepackage[justification=centering]{caption}
%\captionsetup[table]{format=plain,labelformat=simple,labelsep=period,singlelinecheck=true}%
\bibliographystyle{apalike}
%\bibliographystyle{unsrtnat}



%\bibliographystyle{aea}
\usepackage{enumitem}
\usepackage{tikz}
\usetikzlibrary{positioning}
\usetikzlibrary{arrows}
\usetikzlibrary{shapes.multipart}

\usetikzlibrary{shapes}
\def\checkmark{\tikz\fill[scale=0.4](0,.35) -- (.25,0) -- (1,.7) -- (.25,.15) -- cycle;}
%\usepackage{tikz}
%\usetikzlibrary{snakes}
%\usetikzlibrary{patterns}

%\draftSpacing{1.5}

\usepackage{xcolor}
\hypersetup{
colorlinks,
linkcolor={blue!50!black},
citecolor={blue!50!black},
urlcolor={blue!50!black}}

%\renewcommand{\familydefault}{\sfdefault}
%\usepackage{helvet}
%\setlength{\parindent}{0.4cm}
%\setlength{\parindent}{2em}
%\setlength{\parskip}{1em}

%\normalem

%\doublespacing
\onehalfspacing
%\singlespacing
%\linespread{1.5}

\newtheorem{theorem}{Theorem}
\newtheorem{corollary}[theorem]{Corollary}
\newtheorem{proposition}{Proposition}
\newtheorem{definition}{Definition}
\newtheorem{axiom}{Axiom}
\newtheorem{observation}{Observation}
\newtheorem{assumption}{Assumption}	
\newtheorem{remark}{Remark}
\newtheorem{lemma}{Lemma}
\newtheorem{result}{result}


\newcommand{\ra}[1]{\renewcommand{\arraystretch}{#1}}

\newcommand{\E}{\mathrm{E}}
\newcommand{\Var}{\mathrm{Var}}
\newcommand{\Corr}{\mathrm{Corr}}
\newcommand{\Cov}{\mathrm{Cov}}

\newcolumntype{d}[1]{D{.}{.}{#1}} % "decimal" column type
\renewcommand{\ast}{{}^{\textstyle *}} % for raised "asterisks"

\newtheorem{hyp}{Hypothesis}
\newtheorem{subhyp}{Hypothesis}[hyp]
\renewcommand{\thesubhyp}{\thehyp\alph{subhyp}}

\newcommand{\red}[1]{{\color{red} #1}}
\newcommand{\blue}[1]{{\color{blue} #1}}

%\newcommand*{\qed}{\hfill\ensuremath{\blacksquare}}%

\newcolumntype{L}[1]{>{\raggedright\let\newline\\arraybackslash\hspace{0pt}}m{#1}}
\newcolumntype{C}[1]{>{\centering\let\newline\\arraybackslash\hspace{0pt}}m{#1}}
\newcolumntype{R}[1]{>{\raggedleft\let\newline\\arraybackslash\hspace{0pt}}m{#1}}

%\geometry{left=1.5in,right=1.5in,top=1.5in,bottom=1.5in}
\geometry{left=1in,right=1in,top=1in,bottom=1in}

\epstopdfsetup{outdir=./}

\newcommand{\elabel}[1]{\label{eq:#1}}
\newcommand{\eref}[1]{Eq.~(\ref{eq:#1})}
\newcommand{\ceref}[2]{(\ref{eq:#1}#2)}
\newcommand{\Eref}[1]{Equation~(\ref{eq:#1})}
\newcommand{\erefs}[2]{Eqs.~(\ref{eq:#1}--\ref{eq:#2})}

\newcommand{\Sref}[1]{Section~\ref{sec:#1}}
\newcommand{\sref}[1]{Sec.~\ref{sec:#1}}

\newcommand{\Pref}[1]{Proposition~\ref{prop:#1}}
\newcommand{\pref}[1]{Prop.~\ref{prop:#1}}
\newcommand{\preflong}[1]{proposition~\ref{prop:#1}}

\newcommand{\Aref}[1]{Axiom~\ref{ax:#1}}

\newcommand{\clabel}[1]{\label{coro:#1}}
\newcommand{\Cref}[1]{Corollary~\ref{coro:#1}}
\newcommand{\cref}[1]{Cor.~\ref{coro:#1}}
\newcommand{\creflong}[1]{corollary~\ref{coro:#1}}

\newcommand{\etal}{{\it et~al.}\xspace}
\newcommand{\ie}{{\it i.e.}\ }
\newcommand{\eg}{{\it e.g.}\ }
\newcommand{\etc}{{\it etc.}\ }
\newcommand{\cf}{{\it c.f.}\ }
\newcommand{\ave}[1]{\left\langle#1 \right\rangle}
\newcommand{\person}[1]{{\it \sc #1}}

\newcommand{\AAA}[1]{\red{{\it AA: #1 AA}}}
\newcommand{\YB}[1]{\blue{{\it YB: #1 YB}}}

\newcommand{\flabel}[1]{\label{fig:#1}}
\newcommand{\fref}[1]{Fig.~\ref{fig:#1}}
\newcommand{\Fref}[1]{Figure~\ref{fig:#1}}

\newcommand{\tlabel}[1]{\label{tab:#1}}
\newcommand{\tref}[1]{Tab.~\ref{tab:#1}}
\newcommand{\Tref}[1]{Table~\ref{tab:#1}}

\newcommand{\be}{\begin{equation}}
\newcommand{\ee}{\end{equation}}
\newcommand{\bea}{\begin{eqnarray}}
\newcommand{\eea}{\end{eqnarray}}

\newcommand{\bi}{\begin{itemize}}
\newcommand{\ei}{\end{itemize}}

\newcommand{\Dt}{\Delta t}
\newcommand{\Dx}{\Delta x}
\newcommand{\Epsilon}{\mathcal{E}}
\newcommand{\etau}{\tau^\text{eqm}}
\newcommand{\wtau}{\widetilde{\tau}}
\newcommand{\xN}{\ave{x}_N}
\newcommand{\Sdata}{S^{\text{data}}}
\newcommand{\Smodel}{S^{\text{model}}}

\newcommand{\del}{D}
\newcommand{\hor}{H}



\setlength{\parindent}{0.0cm}
\setlength{\parskip}{0.4em}

\numberwithin{equation}{section}
\DeclareMathOperator\erf{erf}
%\let\endtitlepage\relax



% https://medium.com/@aerinykim/why-the-normal-gaussian-pdf-looks-the-way-it-does-1cbcef8faf0a

\begin{document}

\tableofcontents 

\newpage

\section{Production process, subjectivity and free trade}

There are two constraints to EU projects. Eligible costs AND negative NVP. The EU also allows for external funding. 

The lump sum that is given by the EU will be sufficient to bring NPV to zero. 

The problem situation posed by Federalist #10 is that there are factions, and those faction's don't agree. The history of such conflict is that the majority oppresses the minority, conflict is the result. So to prevent conflict there are only two possibilities, either you control the source of factions(the European model) OR the effects of factions. 

To cook up a solution they start with the description that each person has numerous identities, (farmer, cook, chess player, game of thrones fan, ETC). The problem is that quite a few of these identities are rare and you cannot organize around them. For instance maybe the country is too small you are the ONLY one who likes game of thrones. 

First the advantage of a larger amount of people is that there is less chance for corruption, the people who are chosen are not chosen because of their network but because of their broad appeal (for instance in Cyprus a huge amount of our elected are doctors without ANY competency in policy making because they have reputation in the community as doctors). Another argument I would make is that if we imagine that there is some independence between people, even marginal, then larger amount of people will generate more information and if the information spreads properly, you have more accountability overall. 

So this faction size has a second effect, as the faction increases in size, people within it will be able to create sub-factions because they will reach the critical size required to make smaller groups. These smaller groups will be able to reduce the unity of the larger faction and reduce it's hold on society. So you reduce the effectiveness of factions by making them large possible. 

So you need a large republic to create unstable factions, and because the factions are unstable, they won't be able to unify and attack other factions, which means they call co-exist and this creates a stability. 

To summarize:
1) Size is neccesary for low corruption.
2) As size increases it becomes more difficult to unify. 
3) If it is more difficult to unify, there is less feasible oppression. 
4) If there is less feasible opression, the factions can co-exist. 



A republic is the solution. Argument 1) If the republic is large, there is less chance for corruption, more independence more likely someone can call bullshit 2) The larger republic, the larger the faction, and it is harder to unify the faction, which means that you can reduce the power of a faction by increasing it's size. Too many overlapping interests to unify under one goal. So you blunt the faction by making it larger. 3) Without a large republic, you are going to have the tyranny of the majority, so the largness creates stability. 

It is about creating a system that protects the MINORITY. I think the US has been a GREAT success in this, there is no other system where you have nearly as much diversity, indeed if you think of Mormons/Amish/Jehovas etc, you know there is no way those communities could exist in any country in Europe. France has killed off all competing languages by its centralizing policy, there is effectively no way to have a thriving community without having contacts in Paris which will make you rich, there is no concept of diversity, there is only frenchness and how close you are to the french ideal, which is an intellectual ideal. 


47:
Madison Writes under the pseudonym Plubious, inspired by Montesqiu: 
Federalist 47: The accumulation of powers no matter the cause, is the definition of tyranny. What I find extremely funny is that the Europeans are stuck on the moto "Separation of Church and state" but when this motto was made, they actually had different powers. The meaning of this motto SHOULD be that the Church has SOME powers and the state has OTHERS, instead modern people have taken it to mean something like "Don't bring your church stuff into my policy making" but NO initially the Church was supposed to take care of education, poverty, etc. The state took care of Laws and Wars. 

Federalist 30 is less novel but more relevant NOW with #covid19. Basically Hamilton is saying: If it's an emergency, the government should be able to tax internally not just externally(tarrifs), it also basically can't borrow if it can't tax internally. He does say it is a TEMPORARY measure. Most of the paper is responding to critics. 

Critics:
Critics1Can you pass ANY law in order to that TAX, elastic Clause? R: Congress decides if it is neccesary and proper so it's not REALLy federal government.  

C2: What about supremacy clause? You think you can take away the states power to tax! R:Nah don't worry about it, the constitution is the law, that says you can still tax, I can't take away that power. 

C3: Local government knows best how to tax! Response: But the localities will collect them, and the taxes will be proportional so it will not be out of control. 




\bibliography{../thesisbib/bibliography}

\end{document}

\documentclass[12pt]{article}
%\documentclass[12pt]{article}
%\documentclass[12pt]{article}
%\documentclass[12pt,a4paper]{article}

\usepackage[percent]{overpic}
\usepackage{float}
\usepackage{pgfplots}
%\usepackage[cmbold]{mathtime}
%\usepackage{mt11p}
\usepackage{placeins}
\usepackage{amsmath}
\usepackage{amsthm}
\usepackage{color}
\usepackage{amssymb}
\usepackage{mathtools}
\usepackage{subfigure}
\usepackage{multirow}
\usepackage{epsfig}
\usepackage{listings}
\usepackage{enumitem}
\usepackage{rotating,tabularx}
%\usepackage[graphicx]{realboxes}
\usepackage{graphicx}
\usepackage{graphics}
\usepackage{epstopdf}
\usepackage{longtable}
\usepackage[pdftex]{hyperref}
%\usepackage{breakurl}
\usepackage{epigraph}
\usepackage{xspace}
\usepackage{amsfonts}
\usepackage{eurosym}
\usepackage{ulem}
\usepackage{footmisc}
\usepackage{comment}
\usepackage{setspace}
\usepackage{geometry}
\usepackage{caption}
\usepackage{pdflscape}
\usepackage{array}
\usepackage[round]{natbib}
\usepackage{booktabs}
\usepackage{dcolumn}
\usepackage{mathrsfs}
%\usepackage[justification=centering]{caption}
%\captionsetup[table]{format=plain,labelformat=simple,labelsep=period,singlelinecheck=true}%

%\bibliographystyle{unsrtnat}
\bibliographystyle{ecta}
\usepackage{enumitem}
\usepackage{tikz}
\usetikzlibrary{decorations.pathreplacing}
%\def\checkmark{\tikz\fill[scale=0.4](0,.35) -- (.25,0) -- (1,.7) -- (.25,.15) -- cycle;}
%\usepackage{tikz}
%\usetikzlibrary{snakes}
%\usetikzlibrary{patterns}

%\draftSpacing{1.5}

\usepackage{xcolor}
\hypersetup{
colorlinks,
linkcolor={blue!50!black},
citecolor={blue!50!black},
urlcolor={blue!50!black}}

%\renewcommand{\familydefault}{\sfdefault}
%\usepackage{helvet}
%\setlength{\parindent}{0.4cm}
%\setlength{\parindent}{2em}
%\setlength{\parskip}{1em}

%\normalem

%\doublespacing
\onehalfspacing
%\singlespacing
%\linespread{1.5}

\newtheorem{theorem}{Theorem}
\newcommand{\bc}{\begin{center}}
\newcommand{\ec}{\end{center}}
\newtheorem{corollary}[theorem]{Corollary}
\newtheorem{proposition}{Proposition}
\newtheorem{definition}{Definition}
\newtheorem{axiom}{Axiom}
\newcommand{\ra}[1]{\renewcommand{\arraystretch}{#1}}

\newcommand{\E}{\mathrm{E}}
\newcommand{\Var}{\mathrm{Var}}
\newcommand{\Corr}{\mathrm{Corr}}
\newcommand{\Cov}{\mathrm{Cov}}

\newcolumntype{d}[1]{D{.}{.}{#1}} % "decimal" column type
\renewcommand{\ast}{{}^{\textstyle *}} % for raised "asterisks"

\newtheorem{hyp}{Hypothesis}
\newtheorem{subhyp}{Hypothesis}[hyp]
\renewcommand{\thesubhyp}{\thehyp\alph{subhyp}}

\newcommand{\red}[1]{{\color{red} #1}}
\newcommand{\blue}[1]{{\color{blue} #1}}

%\newcommand*{\qed}{\hfill\ensuremath{\blacksquare}}%

\newcolumntype{L}[1]{>{\raggedright\let\newline\\arraybackslash\hspace{0pt}}m{#1}}
\newcolumntype{C}[1]{>{\centering\let\newline\\arraybackslash\hspace{0pt}}m{#1}}
\newcolumntype{R}[1]{>{\raggedleft\let\newline\\arraybackslash\hspace{0pt}}m{#1}}

\geometry{left=1.25in,right=1.25in,top=1.25in,bottom=1.25in}
%\geometry{left=1in,right=1in,top=1in,bottom=1in}

\epstopdfsetup{outdir=./}

\newcommand{\elabel}[1]{\label{eq:#1}}
\newcommand{\eref}[1]{Eq.~(\ref{eq:#1})}
\newcommand{\ceref}[2]{(\ref{eq:#1}#2)}
\newcommand{\Eref}[1]{Equation~(\ref{eq:#1})}
\newcommand{\erefs}[2]{Eqs.~(\ref{eq:#1}--\ref{eq:#2})}

\newcommand{\Sref}[1]{Section~\ref{sec:#1}}
\newcommand{\sref}[1]{Sec.~\ref{sec:#1}}

\newcommand{\Pref}[1]{Proposition~\ref{prop:#1}}
\newcommand{\pref}[1]{Prop.~\ref{prop:#1}}
\newcommand{\preflong}[1]{proposition~\ref{prop:#1}}

\newcommand{\Aref}[1]{Axiom~\ref{ax:#1}}
\newcommand{\Dref}[1]{Definition~\ref{def:#1}}

\newcommand{\clabel}[1]{\label{coro:#1}}
\newcommand{\Cref}[1]{Corollary~\ref{coro:#1}}
\newcommand{\cref}[1]{Cor.~\ref{coro:#1}}
\newcommand{\creflong}[1]{corollary~\ref{coro:#1}}

\newcommand{\etal}{{\it et~al.}\xspace}
\newcommand{\ie}{{\it i.e.}\xspace}
\newcommand{\eg}{{\it e.g.}\xspace}
\newcommand{\etc}{{\it etc.}\xspace}
\newcommand{\cf}{{\it c.f.}\xspace}
\newcommand{\ave}[1]{\left\langle#1 \right\rangle}
\newcommand{\person}[1]{{\it \sc #1}}

\newcommand{\AAA}[1]{\red{{\it AA: #1 AA}}}
\newcommand{\YB}[1]{\blue{{\it YB: #1 YB}}}

\newcommand{\flabel}[1]{\label{fig:#1}}
\newcommand{\fref}[1]{Fig.~\ref{fig:#1}}
\newcommand{\Fref}[1]{Figure~\ref{fig:#1}}

\newcommand{\tlabel}[1]{\label{tab:#1}}
\newcommand{\tref}[1]{Tab.~\ref{tab:#1}}
\newcommand{\Tref}[1]{Table~\ref{tab:#1}}

\newcommand{\be}{\begin{equation}}
\newcommand{\ee}{\end{equation}}
\newcommand{\bea}{\begin{eqnarray}}
\newcommand{\eea}{\end{eqnarray}}

\newcommand{\bi}{\begin{itemize}}
\newcommand{\ei}{\end{itemize}}

\newcommand{\Dt}{\Delta t}
\newcommand{\Dx}{\Delta x}
\newcommand{\Epsilon}{\mathcal{E}}
\newcommand{\etau}{\tau^\text{eqm}}
\newcommand{\wtau}{\widetilde{\tau}}
\newcommand{\xN}{\ave{x}_N}
\newcommand{\Sdata}{S^{\text{data}}}
\newcommand{\Smodel}{S^{\text{model}}}

\newcommand{\del}{D}
\newcommand{\hor}{H}
\newcommand{\subhead}[1]{\mbox{}\newline\textbf{#1}\newline}

\setlength{\parindent}{0.0cm}
\setlength{\parskip}{0.4em}

\numberwithin{equation}{section}
\DeclareMathOperator\erf{erf}
%\let\endtitlepage\relax

\begin{document}

\subsection{Producurement methods and their applications}

Empirical evidence about whther people discount for future generations


\subsubsection{Choice and matching}
Choice: Asking people to choose one thing from two choices. In essence, they give out binary information about their preferences. 

Matching: means that one of the options will have a number left blank, this would allow the subject to give a cardinal number, presumably their indifference point. It allows for a much more precise measurement of the discount factor. Presumably using this method, 4 questions would be sufficient to corroborate each of our 4 setups.

From our perspective, all these questions are incomplete. In ergodicity context this would require that each choice has a different growth rate associated with it. To be able to make such inferences about growth rates it is neccesary to make assumptions about the frequency of the choice. For instance the question 'save 100 lives' needs to be put into context of how often this choice occurs or if this locks in the chooser for a period of time. 

\testb{Example 1}:
Choice A with fixed: Save 100 and choose between A and B again in 6 months

Choice B with fixed: Save 500 in 5 years and choose between A and B again in 6 months

\textb{Example 2}:
Choice A with adaptive: Save 100 and choose between A and B again in 3 months

Choice B with adaptive: Save 500 in 5 years and choose between A and B again in 6 months

\subsubsection{Rating}

Rating: This is in essence a two part question. The first is a simple choice question, the other is an attempt to measure the ratio of badness between the two choices. In our framework this would be very straightfoward, as growth is fully cardinal, you can in fact compare their rating of the badness with respect to the actual growth difference.

In example 1 it is clear that the rating predicted is 1/5. However this is only the case in the case of of additive dynamics. If the dynamics are multiplicative it is more complicated. 

\subsubsection{Total}
Total: Two payments are associated with each option. Say option A has $x_a1$ at $T_a1$ and $x_a2$ at $T_a2$. Once again, our framework allows for a seameless comparison between the two lotteries. However, defining the discount factor would be more difficult in this situation since the growth rates would have numerous variables. 

\subsubsection{Sequence}

Sequence: Here, we are in fact very close to having a question about growth, it compares a program which gets better over time to a program which gets worse over time. In one sense this problem can be interpreted similarly to the Total sense, where the calculuation is plugged into ones own growth rate. Alternatively, one can interpret this directly as a program which has an internal growth rate.

This is actually very important for us, it is proof that people interpret the question in terms of growth rate. 

\subsubsection{Equity}
Equity: This is about keeping the total payout constant but varying its distribution over time. In our case if $r>0$ and we have a multiplicative dynamic, thenm someone would prefer to have it all on the first period. If the dynamic is additive, then one is indifferent between the two.

\subsubsection{Context}

Context: Three options are presented but the questions are pairwise. So A vs B and A vs C. Ergodicity economics has no particular prediction about such framing effects.




\bibliography{../LML_bibliography/bibliography}

\end{document}
\documentclass{article}
\usepackage[utf8]{inputenc}
\usepackage{enumerate}
\usepackage{amsmath}
\DeclareMathOperator*{\argmax}{argmax}
\DeclareMathOperator*{\argmin}{arg\,min}
\usepackage{amsfonts}
\usepackage{dsfont}
\usepackage{bbm}
\usepackage{graphicx}
\usepackage{asymptote}
\usepackage[font=small,skip=0pt]{caption}
\captionsetup[figure]{font=small,skip=0pt}
\usepackage{pstricks}
\usepackage{pst-plot}
\usepackage{pst-plot,pst-math,pstricks-add}
\usepackage{graphicx}
\usepackage{amsmath}
\usepackage{arydshln}
\usepackage{breqn}
\usepackage{amssymb}
\usepackage{amsthm}
\usepackage{geometry}
\usepackage{titlesec}
\usepackage{nth}
\usepackage{enumerate}
%\usepackage{enuitem}
\usepackage{pgfplots}
\usepackage{graphicx}
\usepackage{enumitem}
\usepackage{tikz}
\usetikzlibrary{arrows.meta}
\usepackage[affil-it]{authblk}
\usetikzlibrary{matrix,arrows,decorations.pathmorphing}
\usepgflibrary{arrows}
\usepackage{float}
\pgfplotsset{compat=1.12}
\usepackage{setspace}
\doublespacing 
\newtheorem{theorem}{Theorem}	
\newtheorem{corollary}{Corollary}
\newtheorem{proposition}{Proposition}
\newtheorem{observation}{Observation}
\newtheorem{assumption}{Assumption}	
\newtheorem{definition}{Definition}
\newtheorem{remark}{Remark}
\newtheorem{lemma}{Lemma}
\newtheorem{result}{result}

\begin{document}
\section{Case study: the chain with three firms.}

In this section we study the case of the chain network with three firms. We want to make explicit the optimal strategies played by all firms. For the sake of simplicity we assume throughout that only three firms may potentially enter the market. Each firm decides over (i) entering the market or not and which links to form with its predecessors, then (ii) the royalty it will make one of its successor pay if the later attaches to the former. \\

\indent Given that the game is sequential, we solve using backward induction. Meaning, we start with the decisions taken by firm $3$. This one takes only one decision, which is to enter the market and to maintain a link with firm $2$. We need thence to make sure that these actions of $3$ in the chain constitute a best-response. \\

The chain is labeled as the network $\text{g}_1$ in the graph appendix. 

\indent \chapter{\textbf{Firm 3}}\\
Consider the payoff of firm $3$ in the chain. This firm is the most efficient at producing, and pays a royalty that we denote $r^2_3(\text{g}_1)$ to firm $2$. It follows that: 
\begin{equation*}
    \pi_3(\text{g}_1)= p_3(3,\text{g}_1)-r^2_3(\text{g}_1)-F. 
\end{equation*}

Consider the set of deviations $\mathcal{S}_3$ available to firm $3$. Since only three firms can enter the market, a deviation of $3$ can only consist of a deviation in the link formation and entry strategy of the firm. It follows that: $\mathcal{S}_3=\{(E,1),(E,\emptyset), NE\}$, i.e. if $3$ does not attach to firm $2$, then it may either form a link to firm $1$ ($(E,1)$), to no firm at all $(E,\emptyset)$, or not even enter the market ($NE$). Therefore, the strategy $(E,2)$ is a best-response to firm $3$ to the strategies played by its predecessors if and only if: 
\begin{equation*}
    \pi_3(\text{g}_1)\geq \max\{\pi_3(\text{g}_2), \pi_3(\text{g}_3), 0\},
\end{equation*}
where the first term in the maximum function is the payoff of $3$ in network $\text{g}_2$, the second term is the payoff of the firm in the network $\text{g}_3$ and the zero corresponds to the payoff of $3$ if it does not enter the market. \\
\indent For the sake of comparing the payoffs, we must first determine the royalty $3$ would pay to $1$ in the network $\text{g}_2$. Here, a best-response of firm $1$ in terms of royalty would be to subsidize firm $3$: in fact, if $3$ indeed attaches to $2$, then the market payoff of $1$ decreases compared to as if $3$ was forming a link to $1$. Firm $3$ is indifferent between forming a link to $2$ or to $1$ if $1$ can compensate $3$'s lower market payoff in $\text{g}_2$. And firm $1$ is ready to compensate $3$ up to $p_1(1,\text{g}_2)-p_1(1,\text{g}_1)$. Thence, if $3$ rejects, it must be that $1$'s budget for compensating $3$ is not large enough to deter $3$ from attaching to $2$. It follows that the proposal of $1$ was the maximum level of the subsidy that it was rational for $1$ to offer to $3$:  
\begin{equation*}
   - r^1_3(\text{g}_1)= p_1(1,\text{g}_2)-p_1(1,\text{g}_1)>0. 
\end{equation*}
and the corresponding payoff of firm $3$ in $\text{g}_2$ if it accepts $1$'s offer would  be: 
\begin{equation}
    \pi_3(\text{g}_2)= p_3(2,\text{g}_2) + [p_1(1,\text{g}_2)-p_1(1,\text{g}_1)] -F. 
\end{equation}
Note that the deviation $(E,1)$ of $3$ strictly dominates the deviations $(E,\emptyset)$ and $(NE)$. Thence, we only need make sure that $3$ prefers to form a link to $2$ instead of $1$. \\
\indent Consider the offer made by firm $2$. The later knows that $1$ cannot compete, since it does not have the financial means to compensate $3$ through the subsidy $r^1_3(\text{g}_2)$. Therefore, it is rational for firm $2$ only to set the royalty cost on $3$ to the maximum level that $3$ can accept: 
\begin{equation*}
    r^2_3(\text{g}_1)= p_3(3,\text{g}_1)-F-\pi_3(\text{g}_2), 
\end{equation*}
where $\pi_3(\text{g}_2)$ is the payoff of $3$ if he had attached to $1$ instead. Thence, we find that the equilibrium royalty paid by $3$ to $2$ is: 
\begin{equation}
    r^2_3(\text{g}_1)=[ p_3(3,\text{g}_1)-p_3(2,\text{g}_2)]-[p_1(1,\text{g}_2)-p_1(1,\text{g}_1)]. \label{r23}
\end{equation}


\indent \chapter{\textbf{Firm 2}}\\
We now study the decisions of firm $2$ in the chain. The last action taken by firm $2$ is offering $r^2_3(\text{g}_1)$ to firm $3$. We determined that the level of $r^2_3(\text{g}_1)$ is the value in \eqref{r23}. Therefore, we need to make sure that $2$'s decision of "being more offering" than $1$ for winning the attachment of $3$ is an optimal one. That is: 
\begin{equation*}
    r^2_3(\text{g}_1)\geq p_2(2,\text{g}_2)-p_2(2,\text{g}_1).
\end{equation*}
Therefore, the value of $r^2_3(\text{g}_1)$ in \eqref{r23} is a best-response of firm $2$ if and only if: 
\begin{equation}
   [ p_3(3,\text{g}_1)-p_3(2,\text{g}_2)]-[p_1(1,\text{g}_2)-p_1(1,\text{g}_1)]\geq p_2(2,\text{g}_2)-p_2(2,\text{g}_1) 
\end{equation}

Now we study the other decision taken by $2$ in the chain, which is his link formation. Let $\mathcal{S}_2$ be the set of deviations in terms of link formation strategy available to firm $2$. Firm $2$'s alternate options are either not to form a link to $1$ or not to enter the market at all. Thence, $\mathcal{S}_2=\{(E,\emptyset), (NE)\}$.  \\ 
Let us consider first the deviation $(E,\emptyset)$ of firm $2$. There are two options: either $3$ enters the market or it does not. If $3$ enters the market, it will always form a link with either $1$ or $2$. The reason is fairly simple: $3$ can get a technological level of $2$ for free. The only reason for which $3$ would not enter is that its market payoff in the network $\text{g}_4$ is negative. In conclusion, $2$'s strategy in the chain is a best-response if and only if: 
\begin{equation*}
    \pi_2(\text{g}_1)\geq \max\{0, p_2(1,\text{g}_4)-F\}.
\end{equation*}
We divide the rest of the analysis between two cases when $2$ plays the deviation $(E,\emptyset)$: (i) $3$ enters and attaches to either $1$ or $2$, and (ii) $3$ does not enter. \\

\textit{Case 1. If $2$ plays $s_2=\emptyset$, then $3$ enters and plays $s_3=1$ (or $s_3=2$):}\\
Therefore, it is true that $p_3(2,\text{g}_4)-F\geq 0$ for firm $3$ in the network $\text{g}_4$.\\
It is a best-response for $2$ to form a link to firm $1$ if and only if: 
\begin{equation*}
    \pi_2(\text{g}_1)=p_2(2,\text{g}_1)-r^1_2(\text{g}_1)-F \geq \max\{0, \pi_2(1,\text{g}_4)\}
\end{equation*}
where $\pi_2(1,\text{g}_4)=p_2(1,\text{g}_4)-F$. We need now determine the royalty $r^1_2(\text{g}_1)$ that $1$ charges to firm $2$. Firm $1$ is rational; therefore, it fixes the level of the royalty to the maximum $2$ is ready to pay: 
\begin{equation*}
    r^1_2(\text{g}_1)= p_2(2,\text{g}_1)+r^2_3(\text{g}_1)-F - \max\{0, p_2(1,\text{g}_4)-F\}
\end{equation*}
and $r^2_3(\text{g}_1)$ is the value we obtained in \eqref{r23}. Replacing, we find that the equilibrium value of $r^1_2(\text{g}_1)$ is: 
\begin{equation}
    r^1_2(\text{g}_1)= p_2(1,\text{g}_1)+[ p_3(3,\text{g}_1)-p_3(2,\text{g}_2)]-[p_1(1,\text{g}_2)-p_1(1,\text{g}_1)]-F - \max\{0, p_2(1,\text{g}_4)-F\}. \label{r12}
\end{equation}

\textit{Case 2. If $2$ plays $s_2=\emptyset$, then $3$ does not enter:}\\
Therefore, it is true that: $p_3(2,\text{g}_4)-F\leq 0$. \\
Given that $3$ does not enter if $2$ plays its alternate strategy $s_2=\emptyset$, it is a best-response for firm $2$ to form a link to $1$ if and only if: 
\begin{equation*}
    \pi_2(\text{g}_1)\geq \max\{0, \pi_2(\text{g}_9)\},
\end{equation*}
where $\pi_2(\text{g}_9)=p_2(1,\text{g}_9)-F$. We proceed as in the previous case: we shall get now the level of the royalty $r^1_2(\text{g}_1)$ taht $2$ pays to $1$ in the chain. This is: 
\begin{equation*}
    r^1_2(\text{g}_1)=p_2(2,\text{g}_1) +r^2_3(\text{g}_1)-F- \max\{0, p_2(1,\text{g}_9)-F\}. 
\end{equation*}
Replacing, the equilibrium level of $r^1_2(\text{g}_1)$ for this case is: 
\begin{equation}
    r^1_2(\text{g}_1)=p_2(1,\text{g}_1)+[ p_3(3,\text{g}_1)-p_3(2,\text{g}_2)]-[p_1(1,\text{g}_2)-p_1(1,\text{g}_1)]-F - \max\{0, p_2(1,\text{g}_9)-F\}.\label{r12bis}
\end{equation}

\chapter{\textbf{Firm 1}}\\
We finish with the analysis of the decisions taken by firm $1$ in the chain. The last action the later takes is to decide about the level of the royalty $r^1_2(\text{g}_1)$. We showed that the equilibrium level if $r^1_2(\text{g}_1)$ can only be that in \eqref{r12} or \eqref{r12bis}, depending on the cases. \\
\indent However, it remains to be proved that letting $2$ getting attached to $1$ and pay the consequential royalty is incentive compatible for $1$. If $1$ does not let $2$ getting attached, then there are multiple different networks that may form. These alternate networks are $\text{g}_2, \text{g}_9, \text{g}_{10},\text{g}_6$ and finally $g_{11}$. We investigate the conditions for which each of those networks may form: 
\begin{enumerate}
    \item[\textit{Case 1:}] the network $\text{g}_2$ is realized if both firms $2$ and $3$ can expect a positive payoff. Given that $3$'s payoff in $\text{g}_4$ is always strictly larger than that of $2$ (because $3$ is more efficent at producing, thus $p_3(2,\text{g}_4)>p_2(1,\text{g}_4)$, and $3$ gets its technology for free). Thus if $1$ prevents $2$ from attaching, $\text{g}_2$ is realized if and only if: 
    \begin{equation*}
         p_2(1,\text{g}_4)-F \geq 0.
    \end{equation*}
   \item[\textit{Case 2:}] the network $\text{g}_9$ is realized if $2$ can expect a positive payoff however $3$ cannot and therefore refrains from entering the market. Given that if $3$ enters the market then it always attaches to any one of its predecessors, it must be that $p_3(2,\text{g}_4)-F<0$. The set of conditions that ensures the realization of $\text{g}_9$ is: 
   \begin{align*}
       & p_3(2,\text{g}_4)-F<0, \\
       \mbox{ and : } & p_2(1,\text{g}_9)-F\geq 0.
   \end{align*}
   
   \item[\textit{Case 3:}] the network $\text{g}_{10}$ is realized when $2$ does not enter for the reason that its payoff in $\text{g}_4$ would be negative (recall that if $2$ enters and gets the technology level 1, then if 3 enters it attaches to either 1 or 2). In fact, so long as $3$ enters the market in the network $\text{g}_6$, then $3$ would have entered the market if $2$ had as well. Now, and given that $2$ did not enter the market, firm $3$ must prefer not to form a link to $1$ instead of the opposite. The set of conditions for which $g_{10}$ is realized is then:
   \begin{align*}
      & p_2(1,\text{g}_4)-F<0,\\
      & p_3(2,\text{g}_4)-F\geq 0,\\
       \mbox{ and : } & p_3(1,g_{10})-F \geq \max\{0, p_3(2,g_{6})-r^1_3(\text{g}_6)-F\}.
   \end{align*}
    \item[\textit{Case 4:}] the network $\text{g}_6$ is realized for the same conditions as above, instead that $3$ prefers now to form a link to firm $1$:
     \begin{align*}
      & p_2(1,\text{g}_4)-F<0,\\
      & p_3(2,\text{g}_4)-F\geq 0,\\
       \mbox{ and : } & p_3(1,g_6)-r^1_3(\text{g}_6)-F \geq \max\{0, p_3(2,g_{10})-F\}.
   \end{align*}
  \item[\textit{Case 5:}] the network $\text{g}_{11}$ is realized. Note that the profit of firm $2$ in $\text{g}_9$ must be negative. Otherwise, $3$ would have entered and $\text{g}_4$ would have been formed instead. At last, note that firm $3$'s profit in $\text{g}_6$ must be negative as well. All in all, the full set of conditions that is needed for $\text{g}_{11}$ to be realized when $1$ does not let $2$ attach to her is: 
  \begin{align*}
      & p_3(1,\text{g}_{10})-F=p_2(1,\text{g}_9)-F<0 \mbox{ (a),}\\
      & p_3(2,\text{g}_6)-r^1_3(\text{g}_6)-F <0 \mbox{ (b).}
  \end{align*}
\end{enumerate}

For every and each of these five cases, we need ensure that it is incentive compatible for firm $1$ to make an offer to $2$ in the chain. We proceed case by case. \\

\textit{Case 1: if $1$ does not let $2$ attach to her, then $\text{g}_{4}$ is realized instead.}\\

The payoff of firm $1$ in the chain is: 
\begin{equation*}
    \pi_1(\text{g}_1)=p_1(1,\text{g}_1)+r^1_2(\text{g}_1) -F.
\end{equation*}
If $1$ does not want $2$ to attach to her (it would suffice for $1$ to propose $r^1_2(\text{g}_1)=\infty$), then firm $1$ - given the conditions on the payoffs structure - knows that $\text{g}_4$ will be
realized. Thence $r^1_2(\text{g}_1)$ is a best-response of $1$ if the firm does not prefer to get its payoff in $\text{g}_4$ instead:\\
\begin{equation*}
    p_1(1,\text{g}_1)+r^1_2(\text{g}_1)-F\geq p_1(1,\text{g}_4)-F.
\end{equation*}
that is, $r^1_2(\text{g}_1)\geq p_1(1,\text{g}_4)-p_1(1,\text{g}_1)$. 
(Recall that even if $3$ attaches to $1$ in $\text{g}_4$, the unique stable royalty offer is $r^1_3(\text{g}_4)=0$). Given that $3$'s payoff in $\text{g}_4$ is positive, then the optimal value of $r^1_2(\text{g}_1)$ is the one in \eqref{r12}. Also, $2$'s profit in $\text{g}_4$ is positive (otherwise, $\text{g}_4$ could not have formed). It follows that:
\begin{equation*}
    p_2(2,\text{g}_1)+r^2_3(\text{g}_1)-F -p_2(1,\text{g}_4)-F \geq p_1(1,\text{g}_4)-p_1(1,\text{g}_1).
\end{equation*}
Given the equilibrium value of $r^2_3(\text{g}_1)$ provided in \eqref{r23}, this condition is: 
\begin{equation*}
    p_2(2,\text{g}_1)+[p_3(3,\text{g}_1)-p_3(2,\text{g}_2)]-[p_1(1,\text{g}_2)-p_1(1,\text{g}_1)]-p_2(1,\text{g}_4)\geq p_1(1,\text{g}_4)-p_1(1,\text{g}_1). 
\end{equation*}
which can be rearranged as: 
\begin{equation}
    \sum_{i=1,2,3} p_i(i,\text{g}_1)\geq \sum_{i=1,2,3}p_i(k_i,\text{g}_2)+\sum_{i=1,2,3}p_i(k_i,\text{g}_4)-[p_1(1,\text{g}_1)+p_2(2,\text{g}_2)+p_3(2,\text{g}_4)].
\end{equation}


\textit{Case 2: if $1$ does not let $2$ get attached to her, then $\text{g}_9$ is realized.}\\
\indent Given the payoff that $1$ gets in the chain network $\text{g}_1$, $1$ prefers it over its profit in the network $\text{g}_9$ if and only if:
\begin{equation*}
    p_1(1,\text{g}_1)+r^1_2(\text{g}_1)-F\geq p_1(1,\text{g}_9) -F. 
\end{equation*}
Now, given that $3$ does not enter while $1$ and $2$ both operate with the level of technology $1$, the level of the royalty paid by $2$ to firm $1$ in $\text{g}_1$ is that in \eqref{r12bis}. Also, note that if $2$ enters the market, then it expects a positive payoff. Replacing, we get: 
\begin{equation*}
    p_1(1,\text{g}_1)+p_2(2,\text{g}_1) +r^2_3(\text{g}_1)- p_2(1,\text{g}_9)\geq p_1(1,\text{g}_9)
\end{equation*}
For the equilibrium value of $r^2_3(\text{g}_1)$ given in expression \eqref{r23}, the incentive compatibility constraint of firm $1$ in the chain is here: 
\begin{equation}
    \sum_{i=1,2,3} p_i(i,\text{g}_1)\geq \sum_{i=1,2,3}p_i(k_i,\text{g}_2)+ \sum_{i=1,2} p_i(k_i,\text{g}_9)-[p_1(1,\text{g}_1)+p_2(2,\text{g}_2)]
\end{equation}

\textit{Case 3: if $1$ does not let $2$ get attached to her, then $\text{g}_{10}$ is realized.}\\
Here, a first point to note is that $2$'s payoff in $\text{g}_4$ would be negative, due to $3$'s subsequent entry on the market. In fact, if $3$'s payoff in $\text{g}_4$ were negative, then $2$ would get its payoff in network $\text{g}_9$, which is the same as $3$'s payoff in $\text{g}_{10}$ - and this last payoff is assumed here to be positive. Thus this case implies that $p_2(1,\text{g}_4)-F< 0$ and $p_3(2,\text{g}_4)-F\geq 0$. Thus, $r^1_2(\text{g}_1)$ is given by the expression in $\eqref{r12}$, where the maximum payoff of firm $2$ between its payoff in $\text{g}_4$ and zero is zero. Thence,  
\begin{equation*}
    r^1_2(\text{g}_1)=p_2(2,\text{g}_1)+r^2_3(\text{g}_1)-F. 
\end{equation*}
We can replace the equilibrium value of $r^2_3(\text{g}_1)$ by its expression \eqref{r23}. All in all, the value of $r^1_2(\text{g}_1)$ is a best-response of $1$ if the following relation holds in equilibrium: 
\begin{equation*}
    p_1(1,\text{g}_1)+p_2(2,\text{g}_1)+[ p_3(3,\text{g}_1)-p_3(2,\text{g}_2)]-[p_1(1,\text{g}_2)-p_1(1,\text{g}_1)]-F  \geq p_1(1,\text{g}_{10}),
\end{equation*}
which is equivalent to: 
\begin{equation}
    \sum_{i=1,2,3}p_i(i,\text{g}_1)\geq \sum_{i=1,2,3}p_i(k_i,\text{g}_2)+\sum_{i=1,3}p_i(1,\text{g}_{10})-[p_1(1,\text{g}_1)+p_2(2,\text{g}_2)+p_3(1,\text{g}_{10})]+F.
\end{equation}
\textit{Case 4: if 1 does not let 2 get attached to her, then $\text{g}_6$ is realized.}\\
Once again, firm $2$ does not enter the market because its profit in $\text{g}_4$ would be negative. In fact, the point is that if $3$ enters the market in $\text{g}_6$, then $3$ would always enter if $2$ had entered and get the technology $1$ (i.e. $\pi_3(\text{g}_4)\geq \pi_3(\text{g}_6)$). Thence, the royalty paid by firm $2$ to firm $1$ in $\text{g}_1$ is the value in \eqref{r12}, where $2$'s maximum profit between not entering the market and operating in $\text{g}_4$ would be zero.\\
Given this, $1$'s payoff in the chain is: 
\begin{equation*}
    \pi_1(\text{g}_1)=p_1(1,\text{g}_1)+p_2(2,\text{g}_1)+r^2_3(\text{g}_1)-2F,
\end{equation*}
with $r^2_3(\text{g}_1)$ at its equilibrium value \eqref{r23}. \\

We need to determine now $1$'s payoff in $\text{g}_6$. For this we need first to obtain the value $r^1_3(\text{g}_6)$, that is what $1$ would make $3$ pay in $\text{g}_6$
to let $3$ infringe on her technology. Let us consider this network $\text{g}_6$. If firm $1$ is rational, then it charges to firm $3$ the maximal level of royalty acceptable by $3$. This is the value that makes $3$ indifferent between (i) either not entering the market at all (see network $\text{g}_{11}$), or (ii) entering and producing with the level of technology 1 (see network $\text{g}_{10}$). That is, 
\begin{equation}
    r^1_3(\text{g}_6)= p_3(2,\text{g}_6)-F - \max\{0, p_3(1,\text{g}_{10})-F\}. \label{r13}
\end{equation}
Thus, the level of the royalty in \eqref{r12}
is a best-response of firm $1$ if and only if: 
\begin{equation*}
    p_1(1,\text{g}_1) +p_2(2,\text{g}_1)+ [ p_3(3,\text{g}_1)-p_3(2,\text{g}_2)]-[p_1(1,\text{g}_2)-p_1(1,\text{g}_1)] \geq p_1(1,\text{g}_6)+p_3(2,\text{g}_6) - \max\{0, p_3(1,\text{g}_{10})-F\}
\end{equation*}
Note that firm $2$'s payoff in the chain is always null. This means that $2$ could never operate on the market with any other level of technology. \\
The above expression is the incentive compatibility constraint of firm $1$. It can be rearranged as: 
\begin{equation}
    \sum_{i=1,2,3}p_i(i,\text{g}_1)\geq \sum_{i=1,2,3}p_i(k_i,\text{g}_2)+ \sum_{i=1,3}p_i(k_i, \text{g}_6) -[p_1(1,\text{g}_1)+p_2(2,\text{g}_2)+\max\{0,p_3(1,\text{g}_{10})-F].
\end{equation}

\textit{Case 5: if $1$ does not let $2$ get attached to her, then $1$ is the only firm on the market $(\text{g}_{11})$.}\\
This case is the easiest of all. Note that if $2$ does not enter the market when $1$ does not let her infringe on its technology, then $\max\{p_2(1,\text{g}_9)-F,p_2(1,\text{g}_4)-F\}=p_2(1,\text{g}_9)-F<0$. Also, $3$ cannot operate on a market where the technological network is not a chain, which implies that $\max\{\pi_3(\text{g}_6),\pi_3(\text{g}_{10})\}< 0$. Note that we cannot infer from the previous relations anything about the sign of $\pi_3(\text{g}_4)$, that nonetheless we must know for determining the equilibrium level of $r^1_2(\text{g}_1)$. It turns out that this does not matter when $2$'s payoff in both $\text{g}_4$ and $\text{g}_9$ is always negative. Thence, the equilibrium value of $r^1_2(\text{g}_1)$ is: 
    \begin{equation*}
        r^1_2(\text{g}_1)= p_2(2,\text{g}_1)+r^2_3(\text{g}_1)-F,
    \end{equation*}

Regarding case 4, note that we could add that it was not rational for $1$ to let $3$ enter as in $\text{g}_6$. Meaning, maybe that $3$'s profit in $\text{g}_6$ could have been positive for the value of the royalty $r^1_3(\text{g}_6)$ determined in \eqref{r13}, however $1$ preferred not to have this network been formed. (Note that in this case, it would be true that $\pi_3(\text{g}_4)\geq 0$.) Then $3$'s payoff in $\text{g}_{10}$ must be negative. This implies:
\begin{equation*}
    p_1(1,\text{g}_6)+p_3(2,\text{g}_6)-F - \max\{0, p_3(1,\text{g}_{10})-F\}\leq p_1(1,\text{g}_{11})-F~~\Longleftrightarrow~~p_1(1,\text{g}_6)+p_3(2,\text{g}_6)\leq p_1(1,\text{g}_{11})
\end{equation*}
In the chain, firm $1$ receives the royalty payment $r^1_2(\text{g}_1)$ from firm $2$. This level of royalty is a best-response of firm $1$ if: 
\begin{equation*}
  p_1(1,\text{g}_1)+ p_2(2,\text{g}_1)-F+[ p_3(3,\text{g}_1)-p_3(2,\text{g}_2)]-[p_1(1,\text{g}_2)-p_1(1,\text{g}_1)] \geq p_1(1,\text{g}_{11}).  
\end{equation*}
This incentive compatibility constraint of $1$ in the chain may be re-expressed as: 
\begin{equation}
    \sum_{i=1,2,3} p_i(i,\text{g}_1) \geq \sum_{i=1,2,3}p_i(k_i,\text{g}_2)+p_1(1,\text{g}_{11})-[p_1(1,\text{g}_1)+p_2(2,\text{g}_2)] +F. 
\end{equation}
\indent Finally, $1$'s very first decision was whether to enter the market or not. Thence, it must be that $1$'s payoff in the chain is weakly positive. 
\end{document}

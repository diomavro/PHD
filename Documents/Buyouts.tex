\documentclass{article}
\usepackage{graphicx}
\usepackage{tikz,pgfplots}
\usepackage{preview}	
\usepackage{mathtools}
\usepackage{amsmath}
\usepackage{amssymb}
\usepackage{amsthm}
\usepackage[english]{babel}
\usepackage[utf8]{inputenc}
\usepackage[english]{babel}	
\usepackage{natbib}
\usepackage{color}
\usepackage[a4paper,top=3cm,bottom=3cm, right=2cm, left=2cm]{geometry}
\usepackage[normalem]{ulem}
\usetikzlibrary{math}
\usepackage{blindtext}
\usepackage{natbib}

\usetikzlibrary{decorations.pathreplacing}

\bibliographystyle{agsm}
 
\newtheorem{theorem}{Theorem}	
\newtheorem{corollary}{Corollary}
\newtheorem{proposition}{Proposition}
\newtheorem{observation}{Observation}
\newtheorem{assumption}{Assumption}	

\title{After David}
\pgfplotsset{compat=1.7}
\begin{document}


It is common wisdom that innovation in an industry is generally associated with more buyouts. The logic behind this simple conjecture is straightforward, if there are more firms that are worth buying, more firms will will be bought.

What affects the choice of a firm to buy or not to buy a potential competing firm? 

Does a firm that is already in an industry have more or less incentive to buy a new entrant? A firm that is already in the industry has the risk of being replaced by the new technology. However the existing firm may also have relatively more to gain than other firms, a source of complementarity. Both of these effects increase its affinity to buy. \footnote{The complementarity assumption is much more dubious than the substitutability one. This is purely an empirical matter. It may very well be that firms outside of the industry have a higher level of complementarity than firms inside the industry }

If the market was efficient, it would be the case that the incumbent would be indifferent between buying the competitor at all points in time. What factors can lead to this market not being efficient? Entrepreneur confidence would mean that the firm is better off buying earlier than later. Entrepreneur pessimism might mean that the firm is better off buying later. 

Plausible path, the entry rate of entrepreneurs is not independent of the buyouts. That is if it buys out early, then entrepreneurs actually are more likely to enter. 

Let the firm have a Markov strategy which is its probability of buying. 

Alternative framing, there is some agent who sends innovations to the other player at some cost and gradually changes his view of the other players willingness to pay. 

Result I want: Firms are entering, if you buy too many of them, more will enter until it becomes unsustainable. I you buy too few of them, then the probability of being replaced is too high. 


\textbf{
Suppose that there are two firms that can produce the good at a low price. If competition is Bertrand, then the one with the lower cost will simply set its price equal to the cost of the competitor. Now suppose that a new technology has entered the market and this technology can help the less effective firm more than the more effective firm. 
So initially the profit of the monopolist is $p_m-p_s - c$. After it buys the product, its profit is $p_m-p_s - (c-k)$. Whilst for the other firm, its initial profit is 0. If the innovation is drastic enough then buyout will lead $f$. 
Suppose that the competitor needs to get a certain number of components before he can undercut the current monopolist. However the monopolist can observe the price decreases of the competitor and will attempt to buy out the others. The monopolist gets to decide first which. Bidding for the competitor }

The firm has a prior distribution of a firm innovating. This represents the belief the firm has about the difficulty of entering the industry. 


\section{Newer Formulation}
The relationships between buyouts and innovation are, like most things, is multifaceted. How do buyouts influence the direction of innovation? What is the role of buyouts in incentivizing potential entrepreneurs? 

The naive view of buyouts is simply that buyouts increase the potential payoff from innovation. This is clear enough to see, we need only consider than an entrepreneur is considering the possible payoffs from his investment, ceteris parabis a probability of being bought-out, can only increase the incentive to innovate. 

The naive view is correct in that an extra source of payoff can only increase the upside to the entrepreneur however it may also change the direction of innovation. We need only note that the effect of the buyout on the payoff to the entrepreneur may not be the same along all projects. If some projects for some reason or another cannot be bought out then they will be relatively less worthwhile. 

The naive view is more powerful than it appears. Why would the effect of a buyout not be uniform? If each innovation is associated to a potential profit potential, why would the buyout amount not be proportional to the potential profit? 

From the incumbents point of view either the new innovation is complementary, substitutable or neutral to his own profit. If the innovation is neutral this implies the profit the incumbent would have from purchasing this innovation would be identical to any other purchaser. On the other hand complementary innovations imply that that the incumbent would be willing to pay not only the baseline profit that the innovation can achieve but also an extra profit which emerges from the fusion of the incumbent technology and the innovation. 

On the other hand if the innovation is substitutable with the incumbent technology then then the incumbent by buying out this technology is gaining the profit of the technology plus the leftover profit from his current technology, therefore he should not be willing to pay more than another buyer.

There is however a change when considering a more dynamic framework. In a dynamic framework, if the current innovation's profit potential is what it will become in the future, then external investors will only be willing to pay for what it can earn in the future, on the other hand the current incumbent firm will be willing to pay for not only what it can earn in the future but also what it saves in the present.  

What other factors may lead to the buyout of a product? This paper argues that another factor which may affect the willingness to buy is reducing threatening the margins of a monopolist. 

To clarify, even if a technology has no profit potential, there may still be a willingness to buy it because it forces a monopolist to reduce his price. 


On the one hand it may seem that buyouts occur in an innovative industry. On the other it may be that buyouts only occur when innovation slows down and larger firms consolidate with smaller firms as is seen in the beverage industry. 

Why would a large firm buyout a smaller firm that does research instead of achieve its own research? If economies of scale are not present or there exist fixed costs to setting up innovation facilities this could explain why larger firms would buyout smaller firms. 

Imagine that innovators arrive in the industry with their own technologies. Each innovator has a fixed sum of money that they must use up every period to keep innovating. Innovations differ in two ways, in their probability of moving a stage further, and the jump they can achieve. 

Entrants decide whether they will enter the industry based on their internal characteristics, there is a fixed cost to entering or there is a pecuniary externality. 

The incumbent has a technology which is z gaps from the other two competitors. We can interpret this as the firm not innovating or as the steady state difference between this firm and the other firm. This firm is simply buying out entrants. 

\subsection{Literature review}

Empirically it has been observed that firms which are less innovative are more likely to engage in buyouts. The work of \cite{Gerpott1995} finds that for innovation to be well absorbed by the acquire, the firms size must not differ excessively, or said otherwise, the closer the firms are in size, the more likely they are to merge successfully. \cite{Higgins2006} find that in the pharmaceutical industry  infer that unproductive firms are more likely to engage in acquisition strategies. This is also supported by cross industry studies such as \cite{Zhao2009}. There is also empirical work showing that companies with larger patent portfolios and low research expenditure are more likely to acquire other \cite{Bena2014}.  \textcolor{red}{Note that this is good for another model, buyouts are worthwhile because you get rid of patent royalty cost}. 

\textcolor{red}{An interesting note here is that actually firms that perform less well are more likely to buyout because they have lower negotiating power so they accept worse offers, but the offers payoff in the end, therefore buyouts can be interpreted as a variance strategy}

The basic framework used in this paper borrows from \cite{Cabral2003}. The model involves two firms competing in $R\&D$ and they have an option of either choosng a high variance strategy or a low variance strategy. The general result of the model is that when a firm is lagging behind, it prefers to a high variance strategy, and when it is ahead it chooses a low variance strategy. 

Our results are similar to the literature on firms innovating so that they can escape competition effects,\cite{Aghion2005},\cite{Aghion2001},\cite{Aghion1997}. \cite{Gilbert2016} shows these results only hold in duopolies and not oligopolies. Other work includes \cite{Phillips2012}, where it is argued that large firms avoid engaging in $R\&D$ races. 


\subsection{Model}

The incumbent is a standard Bertrand monopolist who maximizes profit every period. The incumbent has the lowest starting cost $c_i$ and sets the price. The demand function faced by the incumbent is simply $q(p)=A-p$ and the corresponding cost function is simply $q(p)c$. The incumbent sets the Bertrand price, which is simply $min[p^m,c_{ei}]$. 

It it is known that in a Bertrand monopoly, if a buyout is possible it is always profit maximizing to purchase competitors. This is because competitors will achieve no profits if they are not bought out, whilst if they are bought out, the incumbent can increase prices. 

On the other hand the model has an entrant who chooses a technology and attempts to catch up to the incumbent. The initial cost of the entrant, $c_e$ is higher than the monopoly price of the incumbent, and subsequently, does not affect the price. The entrant has the option between two types of technologies, an incremental technology and a radical technology. 

The incremental technology will first allow the entrant to have a cost between $c_{es1} \in [c_i,p]$, which represents the interval where the cost is low enough to bother the incumbent but now low enough for a profit. Every time period, this innovation has probability $\rho$ of occurring. If the first step of the incremental innovation is achieved, the incremental technology will then have another probability, $\rho$, of transitioning to the second state, where  $c_{e2}\in [0,c_i]$. 

\textcolor{orange}{There is room here to separate the two probabilities}

The radical technology on the other hand will transition to the second cost directly. We assume that the cost reduction of the radical technology is the same as the two twp steps of the incremental technology, $c_{e2}=c_r$. So in each time period, with probability q, the firm will transition to $c_r$. 

The timing of the model is as follows. At $t=0$ the entrant makes an irreversible choice about which technology to pursue. \footnote{Irreversibility is important because of the finiteness of the time period, some results may chance if the entrant could change technology. } At $t=1$, the technology is realized, then a buyout can occur, followed by competition. At $t = 2$ the technology is realized once again followed by competition. At $t = 3$ the game ends. 


\begin{tikzpicture}[scale=1]
\node[align=center] at (0,1.5) {$t = 0$};
\draw [thick,->] (0,1) -- (0,0.15);
\node[align=center] at (0,-.7) {Entrant \\chooses\\ technology};
%%%%%%%%%%%%%%%%
\node[align=center] at (2.3,1.5) {$t = 1$};
\draw [thick,->] (2.3,1) -- (2.3,0.15);
\node[align=center] at (2.3,-.7) 
{Technology\\is\\realized };
%%%%%%%%%%%%%%%%
\draw [thick,->] (3.4,-2) -- (3.4,-0.15);
\node[align=center] at (3.4,-2.85) {$t = 1.25$ \\ Buyout \\ occurs\\ here};
%%%%%%%%%%%%%%%%
\node[align=center] at (4.6,1.5) {$t = 1.5$};
\node[align=center] at (4.6,-.7) {Profit\\ /Welfare\\ Realized };
\draw [thick,->] (4.6,1) -- (4.6,0.15);
%%%%%%%%%%%%%%%%
\node[align=center] at (6.5,1.5) {$t = 2$};
\node[align=center] at (6.5,-.9) {Technology \\ is \\realized \\ again};
\draw [thick,->] (6.5,1) -- (6.5,0.15);
%%%%%%%%%%%%%%%%
\node[align=center] at (8.5,1.5) {$t = 2.5$};
\node[align=center] at (8.5,-.9) {Profit\\ /Welfare\\ Realized\\ again};
\draw [thick,->] (8.5,1) -- (8.5,0.15);
%%%%%%%%%%%%%%%%
\node[align=center] at (10,1.5) {$t = 3$};
\node[align=center] at (10,-.9) {World \\ ends};
\draw [thick,->] (10,1) -- (10,0.15);
%%%%%%%%%%%%%%%%
\draw [thick,->] (0,0) -- (11,0);
\end{tikzpicture}

The incumbent in the market sets his monopoly price every period. 

\begin{align*}
\pi_i = (A-p)(p-c_i) \\
\rightarrow p = \frac{A+c}{2} \\
\rightarrow
\pi_i^* = \left(\frac{A-c_i}{2}\right)^2
\end{align*}

So if there is no entrant, the profit over the two periods is simply: 

\begin{equation*}
\pi_i^* = \left(\frac{A-c_i}{2}\right)^2 + \delta \left(\frac{A-c_i}{2}\right)^2
\end{equation*}

The entrant in this model has an option between two technologies. A technology that is incremental, this technology takes two periods to reach a cost that can earn profits, denoted by s(sequential). And a radical innovation which can reach positive profits within a single period. 

Though the incremental technology offers no direct profits if it achieves only one step of innovation it does have the property that it forces the current incumbent to reduce his price. We denote the probability of improving one step, in the incremental innovation as $p$, and the probability of achieving innovation in both states as $p^2$. 

After 1 step of innovation the entrants cost will be $c_{e1}$, a cost that is between, the incumbent cost of production and his monopoly price, $c_i<c_{e1}< \frac{A+c}{2}$. This will occur with probability $p$ in the first period, and with probability $p(1-p)$ in the second period. 

If the second innovation is achieved using the first technology then the firm will have a cost that could undercut the incumbent, $c_{e2}$. This cost is below the incumbent current cost and can therefore undercut the incumbent and reap profits. 

Since the radical innovation has the potential for profit in both periods, this means that the firm will significant prefer to go with the radical path even if its probability of arriving at the second state is lower than the first. We denote the probability of successfully innovating in the first period $q$. The probability that it succeeds on the second period is $q(1-q)$. Therefore the total probability of innovating with the radical technology is $q+q(1-q)$. 

The default profit for the incumbent if the entrant successfully moves to $c_{i1}$ in the first period will be:

\begin{align*}
\pi_{i1}(1) =(A-c_{e1})(c_{e1}-c_i) + \delta(A-c_{e1})(c_{e1}-c_i) \\
\text{Therefore the willingness to pay for the new firm in period 1 is:} \\
\pi_i(1)-\pi_{i1}(1) =\left(\frac{A-c_i}{2}\right)^2 + \delta \left(\frac{A-c_i}{2}\right)^2-(A-c_{e1})(c_{e1}-c_i) - \delta(A-c_{e1})(c_{e1}-c_i) \\
\rightarrow \frac{1}{4}(A+c_i-2c_{e1})^2 + \frac{\delta}{4}(A+c_i-2c_{e1})^2  \\
\pi_i(1)-\pi_{i1}(1) = \frac{(1+\delta)}{4}\left(A+c_i-2c_{e1} \right)^2 \\
\text{The profit potential if it is discovered in the second period is similarly:} \\
\pi_i(2) = \left(\frac{A-c_i}{2}\right)^2 \\
\pi_{i1}(2) = (A-c_{e1})(c_{e1}-c_i) \\
\pi_i(2) - \pi_{i1}(2) = \frac{(A+c_i-2c_{e1})^2}{4}
\end{align*}

\textcolor{orange}{easy generalization to multiple periods}

From the entrants point of view, his discount factor is $\gamma$ if he goes for the incremental innovation and there is no buyout :

\begin{equation*}
\pi_{ens} = \gamma \rho^2 (A-c_i)(c_i-c_e) 
\end{equation*}

\textcolor{red}{This assumes that the gap is not big enough to create a new monopoly price}

If there is a buyout then:

\begin{align*}
\pi_{ebs} = \frac{\rho (1+\delta)}{4}\left(A+c_i-2c_e \right)^2  \\
\end{align*}

\textcolor{red}{This assumes entrant has all the negotiating power}

if the entrant chooses the radical innovation he gets:
\begin{align*}
\pi_{er} =
q((A-c_i)(c_i-c_e)+\gamma (A-c_i)(c_i-c_e)) 
+ (1-q)q \gamma (A-c_i)(c_i-c_e) \\
= (A-c_i)(c_i-c_e)(q + q \gamma + (1-q)q \gamma) \\
= q(A-c_i)(c_i-c_e)(1 +  2 \gamma - q \gamma) \\
= q(A-c_i)(c_i-c_e)(1 +  \gamma(2  - q))
\end{align*}

\begin{proposition}
If $\frac{4q(1+i\gamma(2-q))(A-c_i)(c_i-c_e)}{(1+\delta)(A+c_i-2c_e)^2}<p<\sqrt{\frac{q(1+\gamma(2-q))}{\gamma}}$ then allowing buyouts intermediate innovations changes the choice of the innovators. 
\end{proposition}

\begin{proof}
For the lower bound we use, $\pi_{er}<\pi_{ebs}$ and solve for p. For the upper bound we use, $\pi_{er}>\pi_{ens}$. This gives the above condition. 

If p is above the upper bound, then the firm will always choose the incremental innovation anyway, and if p is below the lower bound, the firm will always select the radical innovation. 
\end{proof}

Consumer surplus by default is:

\begin{align*}
CS=\frac{(A-p)(A-p)}{2}+\delta\frac{(A-p)(A-p)}{2} \\
=\frac{(A-p)^2}{2}+\delta\frac{(A-p)^2}{2} \\
=\frac{(A-c_i)^2}{8}+\delta\frac{(A-c_i)^2}{8} \\
= \frac{(A-c_i)^2}{8}(1+\delta)
\end{align*}


Consumer surplus if firm prefer to incremental but there is no buyout:

\begin{equation*}
CS_{ns} = 
\rho \left(
\frac{(A-c_{s1})^2}{2} 
+
\delta
\left(
\rho \frac{(A-c_{i})^2}{2}
+(1-\rho) \frac{(A-c_{s1})^2}{2}
\right)
\right)
+(1-\rho) \left(
\frac{(A-c_i)^2}{8}+ \delta
\left( 
\rho \frac{(A-c_{s1})^2}{2} 
+(1-\rho) \frac{(A-c_i)^2}{8}
\right)
\right)
\end{equation*}

If there is a with the first technology being pursued then consumer surplus is simply the default, then $CS_{bs}=CS$. CS if there is a radical innovation: 

\begin{align*}
CS_r=(1-q)\left(
\frac{(A-c_i)^2}{8}
+q\frac{(A-c_i)^2}{2}
+(1-q)\frac{(A-c_i)^2}{8}
\right)
+
q\frac{(A-c_i)^2}{2}(1+\delta) \\
=\frac{(A-c_i)^2}{8}(1+4(2+\delta)q-5q^2)
\end{align*}

Suppose the incremental technology is not stochastically dominated by the first technology, so that $\rho>q$. Then we have the following: 

A trivial result of the model is that if $q \geq p$ then the radical innovation is clearly better for the social surplus. Since this implies that the probability of advancing two steps is higher than the probability of advancing one step. 

On the other if $p>q$ then which technology is preferred depends on the consumer discount factor, s we have the following theorem.
\begin{theorem}
Consumer surplus is maximized with the radical innovation if:
\begin{equation*}
\delta>\frac{A^2 (3 \rho+q (5 q-8))+2 A (c (\rho+(8-5 q) q)-4 k \rho)-c^2 (\rho+(8-5 q) q)+4 k^2 \rho}{A^2 \left(3 \rho^2-6 \rho+4 q-1\right)+2 A \left(c \left(\rho^2+2 \rho-4 q+1\right)-4 k
   (\rho-1) \rho\right)-c^2 \left(\rho^2+2 \rho-4 q+1\right)+4 k^2 (\rho-1) \rho}
\end{equation*}

\begin{proof}
First note that if $\delta=0$ and $p>q$
\end{proof}

\end{theorem}

\begin{align*}
CS>CS_r \\
\end{align*}

\section{Infinite time}

\subsection{Discreet time}

The default profit is 

\begin{equation*}
\pi=\sum^{\infty}_{i=0} \delta^i \left(\frac{(A-c_i)}{2} \right)^2
\end{equation*}

\subsection*{Entrant chooses technology 1}

\subsubsection{If buyout is allowed}

\begin{equation*}
\pi_{ibs}=\sum^{\infty}_{i=0} \delta^i \left(\frac{(A-c_i)}{2} \right)^2 - \delta^{\frac{1}{p}} \beta 
\end{equation*}

\begin{equation*}
\pi_{ebs} = \delta^{\frac{1}{p}} \beta 
\end{equation*}

\subsubsection{If buyout is not allowed}

\begin{equation*}
\pi_{ins} =\sum^{\frac{1}{p}}_{i=0} \delta^i \left(\frac{(A-c_i)}{2} \right)^2 
+
\sum^{\frac{2}{p}}_{i=\frac{1}{p}} \delta^i (A-c_e)(c_e-c_i) 
\end{equation*}

\begin{equation*}
\pi_{ens} = \sum^{\infty}_{i=\frac{2}{p}} \delta^i (A-c_i)(c_i-c_{e2}) 
\end{equation*}

\subsection*{Entrant chooses technology 2}

\begin{equation*}
\pi_{ir} = \sum^{\frac{1}{q}}_{i=0} \delta^i \left(\frac{(A-c_i)}{2} \right)^2 
\end{equation*}

\begin{equation*}
\pi_{er} = \sum^{\infty}_{i=\frac{1}{q}} \delta^i (A-c_i)(c_i-c_{e2}) 
\end{equation*}

\subsection{Continuous time}

The default profit is:

\begin{equation*}
\pi = \int^{\infty}_0 \left(\frac{A-c_i}{2} \right)^2 e^{-rt}dt
\end{equation*}

\subsection*{Entrant chooses technology 1}

\subsubsection{If buyout is allowed}

\begin{equation*}
\pi_{ibs} = \int^{\infty}_0 \left(\frac{A-c_i}{2} \right)^2 e^{-\delta t}dt - e^{-\frac{\delta}{p}} \beta
\end{equation*}

\begin{equation*}
\pi_{ebs} = e^{-\frac{\gamma}{p}} \beta
\end{equation*}

\subsubsection{If buyout is not allowed}

\begin{align*}
\pi_{ins}= \int^{\frac{1}{p}}_0 \left(\frac{A-c_i}{2} \right)^2 e^{-\delta t}dt 
+ \int^{\frac{2}{p}}_{\frac{1}{p}} (A-c_e)(c_e-c_i)  e^{-\delta t}dt
+ \int^{\infty}_{\frac{2}{p}} 0 dt \\
\int^{\frac{1}{p}}_0 \left(\frac{A-c_i}{2} \right)^2 e^{-\delta t}dt 
+ \int^{\frac{2}{p}}_{\frac{1}{p}} (A-c_e)(c_e-c_i) e^{-\delta t}dt
\end{align*}

\begin{equation*}
\pi_{ebs} = \int^{\infty}_{\frac{2}{p}} (A-c_i)(c_i-c_{e2})  e^{-\delta t}dt
\end{equation*}

\subsection*{Entrant chooses technology 2}

\begin{align*}
\pi_{ir} = \int^{\frac{1}{q}}_0 \left(\frac{A-c_i}{2} \right)^2 e^{-\delta t}dt 
+ \int^{\infty}_{\frac{1}{q}} 0 dt 
\\
=\int^{\frac{1}{q}}_0 \left(\frac{A-c_i}{2} \right)^2 e^{-\delta t}dt 
\end{align*}

\begin{equation*}
\pi_{ebs} = \int^{\infty}_{\frac{1}{q}} (A-c_i)(c_i-c_{e2})  e^{-\delta t}dt
\end{equation*}

\bibliographystyle{cell}
\bibliography{Acquire.bib}


\end{document}

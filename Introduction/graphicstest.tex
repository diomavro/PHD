% Venn diagram with magnifier
% Author: Dennis Heidsiek
\documentclass{minimal}

\usepackage{tikz}
\usetikzlibrary{spy}

\usepackage{verbatim}
\usepackage[active,tightpage]{preview}
\PreviewEnvironment{tikzpicture}

\begin{document}
\pagestyle{empty}
% First, we define three circles:
\def\firstcircle{(-0,0) circle (4)}
\def\secondcircle{(-1.6,1) circle (2)}
\pgfmathparse{-(2.4^2-2^2)^0.5} % by pythagoras
\let\h\pgfmathresult % shortcut for further use
\def\thirdcircle{(1,1) circle (1.5)}

\begin{tikzpicture}[ spy using outlines={circle,
  magnification=1, size=1cm, connect spies}]
 % Let's draw the scene (to magnify):
  \begin{scope}[spy using outlines=
      {magnification=16, size=8cm, connect spies, rounded corners}]
    
    % the boarder:
    \draw[thick, rounded corners] (-5,-4) rectangle (5,4);
    \draw (0,3.3) node[scale=2] {Asset Universe};
    
    % The transparency:
    \begin{scope}[fill opacity=0.5]
      \fill[yellow] \firstcircle;
      \fill[blue] \secondcircle;
      \fill[red] \thirdcircle;
    \end{scope}
    
    % letterings and missing pieces:
    \draw (0,-2) node {Possible uses of an asset};
    \draw[align=center] \firstcircle node {};
    \draw[align=center] \secondcircle node {Uses that\\do not affect\\physical\\characteristics\\of other assets};
    \draw[align=center] \thirdcircle node {Uses on which\\privilege is \\granted};
    \fill (-1.5,-0.5) circle (0.000)
        node[scale=0.04, align=center] {Intellectual\\ Property\\is possible};
    \fill (-1.5,-1.5) circle (0.000)
        node[scale=0.04, align=center] {Intellectual\\ Property\\is possible};
    
    % now we can draw the magnifying glass:
    \spy [circle,cyan, size=1.9cm] on (-1.5,-0.5) in node [left] at (5,0);
    \spy [circle,cyan, size=1.9cm] on (-1.5,-1.5) in node [left] at (5,-3);
  \end{scope}
\end{tikzpicture}

\end{document}
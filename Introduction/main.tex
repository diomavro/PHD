%\documentclass[AER]{AEA}
\documentclass[12pt]{article}
%\documentclass[12pt]{article}
%\documentclass[12pt,a4paper]{article}

\usepackage{float}
%\usepackage[cmbold]{mathtime}
%\usepackage{mt11p}
\usepackage{placeins}
\usepackage{amsmath}
\usepackage{color}
\usepackage{amssymb}
\usepackage{mathtools}
\usepackage{subfigure}
\usepackage{multirow}
\usepackage{epsfig}
\usepackage{listings}
\usepackage{enumitem}
\usepackage{rotating,tabularx}
%\usepackage[graphicx]{realboxes}
\usepackage{graphicx}
\usepackage{graphics}
\usepackage{epstopdf}
\usepackage{longtable}
\usepackage[pdftex]{hyperref}
%\usepackage{breakurl}
\usepackage{epigraph}
\usepackage{xspace}
\usepackage{amsfonts}
\usepackage{eurosym}
\usepackage{ulem}
\usepackage{footmisc}
\usepackage{comment}
\usepackage{setspace}
\usepackage{geometry}
\usepackage{caption}
\usepackage{pdflscape}
\usepackage{array}
\usepackage[authoryear]{natbib}
\usepackage{booktabs}
\usepackage{dcolumn}
\usepackage{mathrsfs}
%\usepackage[justification=centering]{caption}
%\captionsetup[table]{format=plain,labelformat=simple,labelsep=period,singlelinecheck=true}%
\bibliographystyle{apalike}
%\bibliographystyle{unsrtnat}

%\bibliographystyle{aea}
\usepackage{enumitem}
\usepackage{tikz}
\def\checkmark{\tikz\fill[scale=0.4](0,.35) -- (.25,0) -- (1,.7) -- (.25,.15) -- cycle;}
%\usepackage{tikz}
%\usetikzlibrary{snakes}
%\usetikzlibrary{patterns}

%\draftSpacing{1.5}

\usepackage{xcolor}
\hypersetup{
colorlinks,
linkcolor={blue!50!black},
citecolor={blue!50!black},
urlcolor={blue!50!black}}

%\renewcommand{\familydefault}{\sfdefault}
%\usepackage{helvet}
%\setlength{\parindent}{0.4cm}
%\setlength{\parindent}{2em}
%\setlength{\parskip}{1em}

%\normalem

%\doublespacing
\onehalfspacing
%\singlespacing
%\linespread{1.5}

\newtheorem{theorem}{Theorem}
\newtheorem{corollary}[theorem]{Corollary}
\newtheorem{proposition}{Proposition}
\newtheorem{definition}{Definition}
\newtheorem{axiom}{Axiom}
\newcommand{\ra}[1]{\renewcommand{\arraystretch}{#1}}

\newcommand{\E}{\mathrm{E}}
\newcommand{\Var}{\mathrm{Var}}
\newcommand{\Corr}{\mathrm{Corr}}
\newcommand{\Cov}{\mathrm{Cov}}

\newcolumntype{d}[1]{D{.}{.}{#1}} % "decimal" column type
\renewcommand{\ast}{{}^{\textstyle *}} % for raised "asterisks"

\newtheorem{hyp}{Hypothesis}
\newtheorem{subhyp}{Hypothesis}[hyp]
\renewcommand{\thesubhyp}{\thehyp\alph{subhyp}}

\newcommand{\red}[1]{{\color{red} #1}}
\newcommand{\blue}[1]{{\color{blue} #1}}

\newcommand*{\qed}{\hfill\ensuremath{\blacksquare}}%

\newcolumntype{L}[1]{>{\raggedright\let\newline\\arraybackslash\hspace{0pt}}m{#1}}
\newcolumntype{C}[1]{>{\centering\let\newline\\arraybackslash\hspace{0pt}}m{#1}}
\newcolumntype{R}[1]{>{\raggedleft\let\newline\\arraybackslash\hspace{0pt}}m{#1}}

%\geometry{left=1.5in,right=1.5in,top=1.5in,bottom=1.5in}
\geometry{left=1in,right=1in,top=1in,bottom=1in}

\epstopdfsetup{outdir=./}

\newcommand{\elabel}[1]{\label{eq:#1}}
\newcommand{\eref}[1]{Eq.~(\ref{eq:#1})}
\newcommand{\ceref}[2]{(\ref{eq:#1}#2)}
\newcommand{\Eref}[1]{Equation~(\ref{eq:#1})}
\newcommand{\erefs}[2]{Eqs.~(\ref{eq:#1}--\ref{eq:#2})}

\newcommand{\Sref}[1]{Section~\ref{sec:#1}}
\newcommand{\sref}[1]{Sec.~\ref{sec:#1}}

\newcommand{\Pref}[1]{Proposition~\ref{prop:#1}}
\newcommand{\pref}[1]{Prop.~\ref{prop:#1}}
\newcommand{\preflong}[1]{proposition~\ref{prop:#1}}

\newcommand{\Aref}[1]{Axiom~\ref{ax:#1}}

\newcommand{\clabel}[1]{\label{coro:#1}}
\newcommand{\Cref}[1]{Corollary~\ref{coro:#1}}
\newcommand{\cref}[1]{Cor.~\ref{coro:#1}}
\newcommand{\creflong}[1]{corollary~\ref{coro:#1}}

\newcommand{\etal}{{\it et~al.}\xspace}
\newcommand{\ie}{{\it i.e.}\ }
\newcommand{\eg}{{\it e.g.}\ }
\newcommand{\etc}{{\it etc.}\ }
\newcommand{\cf}{{\it c.f.}\ }
\newcommand{\ave}[1]{\left\langle#1 \right\rangle}
\newcommand{\person}[1]{{\it \sc #1}}

\newcommand{\AAA}[1]{\red{{\it AA: #1 AA}}}
\newcommand{\YB}[1]{\blue{{\it YB: #1 YB}}}

\newcommand{\flabel}[1]{\label{fig:#1}}
\newcommand{\fref}[1]{Fig.~\ref{fig:#1}}
\newcommand{\Fref}[1]{Figure~\ref{fig:#1}}

\newcommand{\tlabel}[1]{\label{tab:#1}}
\newcommand{\tref}[1]{Tab.~\ref{tab:#1}}
\newcommand{\Tref}[1]{Table~\ref{tab:#1}}

\newcommand{\be}{\begin{equation}}
\newcommand{\ee}{\end{equation}}
\newcommand{\bea}{\begin{eqnarray}}
\newcommand{\eea}{\end{eqnarray}}

\newcommand{\bi}{\begin{itemize}}
\newcommand{\ei}{\end{itemize}}

\newcommand{\Dt}{\Delta t}
\newcommand{\Dx}{\Delta x}
\newcommand{\Epsilon}{\mathcal{E}}
\newcommand{\etau}{\tau^\text{eqm}}
\newcommand{\wtau}{\widetilde{\tau}}
\newcommand{\xN}{\ave{x}_N}
\newcommand{\Sdata}{S^{\text{data}}}
\newcommand{\Smodel}{S^{\text{model}}}

\newcommand{\del}{D}
\newcommand{\hor}{H}

\setlength{\parindent}{0.0cm}
\setlength{\parskip}{0.4em}

\numberwithin{equation}{section}
\DeclareMathOperator\erf{erf}
%\let\endtitlepage\relax

\begin{document}


The goal of this introduction will be to give the reader a general introduction to how economics treats property rights and how this relates to intellectual property. Much of the dicussion is aimed summarizing the literature but some commentary is original. The first section aims to give a brief introduction to the roots of the debate and it's modern framing. The second section aims to introduce the reader to basic economic notions in a static context and to familiarize the reader with how the Coase theorem is used to discuss property rights. In a third section we will discuss the dynamic aspects of property rights and some literature around incomplete contracts. In a fourth section we will discuss some mechanism design literature that relates to intellectual property rights. 

We will then give a brief introduction to each article and attempt to situate them within the notions introduced in the introduction. 

The first article aims to show that even if it is desirable to give a monopoly on a good, this does not entail that property rights should also be used against consumers. 

The second article uses the coase theorem but shows that the ability to negotiate for the buyout of innovations causes firms to pursue innovations that increase externalities. 

The third article is a general model of discounting, the aim of this paper is to show that it is possible to define discounting as something other than a preference, under this interpretation, the choices entrepreneurs make is not a function of their preferences but a function of their environment. 

Finally the fourth model presents a general model of sequential innovation in a network in an aim to show that royalties are in fact subject to Bertrand like competition which.

\section{The origins of the debate about private property}

Plato's most famous work, the Republic, is a treatise on an idealized society, one that has managed to halt to a minimum it's own deterioration from the perfect form. Plato's view of property rights is purely instrumental in that it is something that will help maintain the ideal society from deteriorating. Plato views ownership as an important source of corruption that creates clannish self interest and considerw the pannacea of this influence to be the abolition of private property. 
In "Politics", Aristotle takes a stand against his mentor and defends private property. Aristotle reasons that without private property people would interfere in each others affairs without being motivated by love. Indeed Aristotle viewed the act of waiving your rights to property against an individual a way to be virtuous, and a limitation of this right would limit the ability to be virtuous. The debate between Aristotle and Plato has echoed for millennia. With various philosophers taking their sides on this debate. For instance Hegel defended property rights based on his theory of person-hood, stating that people are defined by their will and the only way to manifest their will is through physical objects. 

Perhaps the most influential modern non-economic view of property is John Locke's theory of homesteading. \footnote{\cite{locke2014second}} Locke's view of property rights is as a method of linking a person who is adding value to that value. This is done by mixing one's labor with the object or land which makes the physical object inseparable from its founder. In other words this is a theory about the creation of property rights an originalism of sort. 

Economics has always focused not on the origins of propery but on its effects. Using this lense, perhaps the the most famous critic of the Lockean theory of property was Karl Marx who claimed the opposite, that private property is the means by which workers become alienated from their labor. The logic behind this is rather simple, if an employee adds a number of hours worth of labor, he will necessarily be compensated less than that number of hours worth by the property owner otherwise there would be no way of making profit, hence exploitation. This is one of the first views of property which focused on the dynamics of property, specifically here, the dynamics on wealth inequality. However these kind of interpretations have been superceded as value has been associated not with inputs but by the tastes of agents and the relative scarcity of resources. Similarly profit could be entirely explained by other factors such as the the relative advantage firms have in information, whether it be an edge in production, taste, impulses of consumers etc. This does not entail that property is disconnected from value, merely that value is not simply related to labor. 

Perhaps the first fully prescritive system of property was articulate by Henry George \cite{progress}, which aims to reduce some of the dynamics described by Marx. Henry George devised a system where property is temporarily allocated to the highest bidder. What is ingenious about the modern version of the Georgist scheme is that it aims to eliminate land rends by making tenants bid for their own rents. This creates a sytem where people will only earn their labor rent and not the land rent of value. Perhaps the most known response to this view is the view of Hayek \footnote{\cite{Fatal}}. In this view function of property is not homogenous across individuals and making ownership temporary is prescriptive in not only the system of property but also in what agents should pursue. For instance an agent may wish to pursue non-monetary goals and the Georgist scheme cannot accomodate such a structure of production.


\newpage


\subsection{What is a property right?}

What is a property right? \footnote{The presentation borrows from \cite{Munzer1990}}
Some possible answers to this question are: Property is simply the default contract, that is, if people do not agree on a contract, property is what is taken as the baseline. 'Property as the default' view is simple enough: Person A can contract with person B that person B will not touch or use item z without A's permission. This in fact requires no property right at all. What does require a property right is that all other people will also not be able to do with z as they please. If agents could all simultaneously consent or if there were but two agents who could contract, there would be no need for property rights. Indeed property rights rely on the inability to contract. 



\subsection{The language of property rights}

Discussion of property rights often suffer from precision. To clarify our definition we use some legal language \footnote{see \cite{Hohfeld}}. 

It is convenient to speak of two categories of rights. First order rights and second order rights. First order rights are what is often called, the right to use or the right to exclude. The right to use is often called a "privilege" and the right to exclude is called a "claim". 

Second order rights are rights are rights about first order rights. For instance when one talks of "power" this is in reference to the right to transfer, waive or annul "claim" and "privilege" rights. For instance, the right to transfer property to someone else is a second order right. One can also speak of "immunity", which means that one has the right for his "claim" or "privilege" to something to not be affected by others. 

Notice that if an agent has full second order rights over an object, this entails the ability to have first order rights. Both first order and second order rights may be under negotiation in contracts, the possibility space is automatically much narrower without second order rights. The idea of second order rights entirely entails the right to destroy and to abandon, if someone does not have second order rights this entails the agent has no right to destroy or abandon. \footnote{for an interesting analysis of the right to destroy/abandon see, \cite{Strahilevitz2005}, \cite{Strahilevitz2009}} 

To clarify ideas it useful to know how this taxonomy matches with traditional economic ideas. For instance clearly if there is a law that requires property owners to allow access, this is a limitation of the first order right of "privilege". A regulation stating that one may not consume or use one's land in a specific way is similarly a reduction in "claim" rights. A price control is a limitation on what price one can sell their good for, as such it is a "power" limitation. A constitutional constraint is an example of a second order right of "immunity" being increased. 

Similarly, the types of arrangements possible(corporations, partnerships, non-profits, licenses, bailments, non-voting common stock, trusts, agencies, employee-employer relationships marriages etc) entirely depend on the above taxonomy. With this in mind we clarify how some property right paradigms fit into this conceptual framework. 

\textit{Private property} grants all of the above rights. Private property gives the right to use, the right to exclude, the right to transfer/waive the exclusion and immunity from someone others altering ones claim. \textit{Private property} has been defined as the ability to use ones property in any way one wishes as long as the \textit{physical} characteristics of others private property is not affected \footnote{\cite{Alchian1965}}. This definition of private property has the advantage of setting an objective standard and requires no reference to subjective clauses. To see an example of the issues with defining private property rights as "the right to do with your property as you wish as long as nobody else is harmed by it", one need only notice how ambiguous this standard is if the question is about the right of someone to build a skyscraper given that it blocks the view of a neighbor. 

\textit{Communal/Public/ property} on the other hand gives all agents in society the right to use but not the right to exclude. \footnote{what is deemed the right of the absentee owner see \cite{Alchian1973}}. Communal property gives second order rights to the state. In practice, the distinction between communal and public ownership is whether the state exercises its second order rights. If it exercises it for setting up things such as a military reservations, then we we call it "public". Communal property is technically a first come first served type of property. It has been said that agents with communal rights tend to ignore the cost of use \footnote{\cite{Alchian1973} mentions how the Canadian government in 1970 set an upper limit to the number of seals to be clubbed which caused speed of hunting to be the competitive trait}. 

\textit{Georgian system of property}. The Georgian system of property is a system which temporarily grants first order rights to individuals. The system does not have discretion like public property has, however, second order rights like power and immunity are still in the hands of the state, as such it is merely about how the state decides to use, which is simply to create an auction for temporary rental.  

The language presented is especially useful for the analysis of intellectual property. Whilst each physical property can be seen as a list of rights and the matching of those rights to individuals or groups of individuals the concept of intellectual property is qualitatively different. The notion of intellectual property is a limitation of the first order right of "privilege" on physical property. That is, if one has intellectual property on the concept of a wooden chair, this is in fact the limitation on the use rights of all owners of wood. Alternatively, there are instances where the law permits copying but not commercialization, in these instances the first order rights are not affected but the second order rights of "power" are limited conditional on the use of the asset.  

Is the change in allocation of rights substantive? In other words is this whole exercise just a redefinition but does not imply any changes in resulting actions? Well the nice thing about this taxonomy is that the proposition that both systems work identically is the proposition that second order rights do not matter. In theory, public and private property can both pursue the same kind of goals, profits/charity, in practice, once the incentives of the agents are taken into account the theory of property rights becomes descriptive. For example, a price control or a "power" limitation, as it restricts the conditions of transfer an owner can make, this can lead to the agents choosing differently. For instance, an agent renting out an apartment may prefer childless/petless adults to avoid noise or damage to his property, in other words, leading to both investment and allocation differences. 

\newpage
\section{The static economics of property rights}

Much of economics treats law as merely instruments to utility maximization. Where each branch of law, tort, contract, is aiming to create rules that increase utility and each branch captures new elements. Discussion of property rights can be broken down into four distinct questions. 1) What are the assets that property rights protects? 2) From who is the property protected? 3) What is the content of property protection. 4) What is the enforcement mechanism by which property is to be protected. The economics way of answering these questions generally leans on two kinds of efficiency: allocative efficiency and investment efficiency. Allocative efficiency means either to give it to the agent who values it the most or the agent who has the lowest cost to operate it. Both of these dynamic notions require the static concepts of value. Value in economics is usually broken down into two components, market value and subjective value. These notions are important in that the whole framework of analyzing property rights leans on their interaction. We quickly analyze some cases of how these notions interact. One of the achievement of economics is the deduction of the market value of an asset by the description of the subjective value. Note that a positive market value does not imply that exchange occurs, indeed subjective value is key to whole framework of the optimal allocation of property rights. When discussing numerous independent assets the above logic holds, however when the utility of assets is not independent then additional notions enter into the framework. Independence of assets simply means that each asset is to be valued separately and does not depend on wether other assets are aqcuired with them. 


\subsubsection{Identical and strictly positive subjective value for all agents}

If an asset has an identical positive value for all agents then it has a positive market value but no exchange occurs. 

\subsubsection{Identical and null value for all agents}

Suppose an asset has zero subjective value for all agents, then it also has zero market value because nobody is willing to buy it. 

\subsubsection{Identical and negative value for all agents}

Then the market value is negative and the allocation of property rights just means,  "who will be targeted to receive this asset". In such a case, there is a demand for abandonment or destruction. The decision whether to force the ownership of the asset to occur should depend on whether the asset is best left abandoned or destroyed, if the optimal use of the asset is its destruction then ownership should be forced, if its optimal use is abandonment then no property right is necessary. Of course there may also be a situation when one requires someone to own something without giving that person the right to abandon or destroy(keys). 

\subsubsection{Variable and weakly positive value for all agents}

Suppose now that we introduce variance into the mix. If agents have differential positive value for the asset, then a positive market value exists and an exchange occurs unless the highest value user is one who is allocated the property.

\subsubsection{Variable and weakly negative value for all agents}

Hot potato: Similarly if all agents have a differential negative value of the good then there is still a value market value to the good unless the highest value user owns it. This is because if anyone other than the highest value user owns the good, they would be willing to pay the highest value user to own it. In this case they would consider the lowest cost alternative between subsidizing the highest value user to own it versus destroying it or abandoning it. 

\subsubsection{Variable, positive and negative value}

Some difficulty arises when we mix the cases, for instance if the distribution of subjective value include both positive and negative values, then clearly if transaction costs are zero then there will occur a trade if the good is given to anyone but the highest value user. 

\subsection{Coasian paradigm}

A \textit{transaction cost} can be defined as the cost of accessing the market value. So by definition, if an agent owns an item at equilibrium and has a lower valuation of it than the market value of the object, this must must be because of the transaction cost. In other words the broad category of transaction cost can include, psychological, institutional, physical factors etc, anything that prevents an entailment of the form "If this individual owns it then this individual has the highest value". From the point of view of efficiency(to be defined in the next paragraph), the question of making destruction or abandonment illegal becomes relatively more important as transaction costs increase due to the risk of over-destruction or over-abandonment. A liquidity constraint(also called a pecuniary externality) is also a sort of transaction cost, if agents cannot buy a good whose market value is lower than their subjective value then we have a reason for allocational inefficiency. Similarly, if an agent does not know of the market price or say is ethically against using the market mechanism, these are both types of transaction costs. There are many things in society which are either naturally or legally inalienable(kidneys, votes, future labor, historically important, etc), and to the extent that inalianable endowments exist these can be interpreted as exorbitantly high transaction costs. From the framework examined above, a transaction cost is usually a function of a lack of second order rights, one can only transact on the rights they have. 

The notion of efficiency in economics has a static and a dynamic dimension. Static efficiency is usually termed \textit{allocationally} efficient, this means simply that the set of actions which maximize the total payoffs is taken. When the question being posed is related to ownership of an asset, allocational efficiency simply means that an asset is owned by its highest subjective value user. When discussing assets allocational efficiency generally does not make reference to pareto efficiency but to Kaldor Hick's, which says say that enough value has been created that the sum of values is greater than before. 

The dynamic notion of efficiency used in economics is the notion of \textit{investment} efficiency, this notion of efficiency brings attention to growth. The idea behind investment efficiency is that the allocation that results will lead to the highest amount of growth and hence eventually, the highest long run payoff eventually. The two notions are sometimes in conflict in that static efficiency is not neccesarily good for growth. The interaction between these two ultimately depends on the discount rates of agents, when the agents don't discount the future, the two are perfectly compatible. 

The Coase theorem is fundamentally about static efficiency. The theorem states that if transaction costs are zero, the result of the market interactions is allocationally efficient. This can also be interpreted from an action stand point to say that the actions which maximize total payoffs are undertaken. If on the other hand there are non-zero transaction costs we can only discuss constrainted efficiency in the sense "of those who entered the market, the highest subjective value will receive it". 

The Coase theorem is of direct relevance to most analysis of externalities. Externality is often a poorly defined concept \footnote{for details about why it is a poorly defined concept see \cite{Cheung1970}}, one temptation is to define it as effects on non-consenting parties, however this is too large of a conception since competition is all about negative externalities between firms. Instead externalities are best defined as effects on non-consenting parties which do not pass through the market mechanism. The theorem was initially framed with externalities in mind, perhaps it's most counterintuitive result is that it implies that externalities become internalized if there are sufficiently low transaction costs. 

The theorem also describes the kind of effects the legal system can have. Take a problem situation where there is an infringer and the owner of the property that is being infringed. If owner has veto capacity on his property and others can only use it with his permission, this is called a property rule. If on the other hand there is fixed or court determined cost associated with infringment this is called liability rule. The theorem states that when transaction costs are sufficiently low, both liability and property rules will result in an allocationally efficient outcome. This has sprouted a rich literature on choosing the legal rules as a function of transaction costs.\footnote{Theoretical: \cite{calabresi1972property} Empirical: \cite{kaplow1995property}} For instance, the liability rule may be prefered due to: the holdout problem; free riders; accident situations; if the infringer is better informed; if the infringer has less liquidity; etc etc. Alternatively, if transaction costs are deemed to be sufficiently low, the legal rules can be chosen for other criteria than allocational efficiency, for example, distributional considerations. 

\newpage

\section{Thje Dynamic creation of property rights}

Once we introduce time into the picture a few things become more complicated. Time may create new property in one of two ways, either because the actual material goods have increased or because new information has lead to an increase property, for example, the discovery of existing assets. New property creates questions about how to allocate property that previously had no owner. In other words, time gives rise to questions about property allocation before it exists, \textit{ex-ante}, and whether property is allocated after it exists, \text{ex-post}. 

There are ex-ante rules one could adopt that solves property allocation problems.  For instance if all surface area is fully allocated, then new physical property will just be allocated to whoever owns the surface on which it is discovered. Full geographical rights in this manner give rise to questions of volume rights, such as air or underground, one has to decide if land property expands into the sky via cone shape or just a tower like rectangle, these kind of questions can determine the operating costs of underground facilies or the cost of flying overhead due to air rights.  Ex ante fully allocating surface area rights is  difficult, mainly because agents are often not interested in allocating property before it has a value, instead property rights emerge naturally as the value of assets increases, at some poiint there will be demand to create rights \footnote{for details about the emergence of property rights see \cite{Alchian1973}}. 

An ex-ante regime of property can apply to both physical and biological property. For instance if a piece of land is found to contain oil, it would go to its owner. Similarly for organisms, if a pet is owned, one usually owns its offspring.

Consider an asset that creates new assets and is owned ex ante. If the production of future assets is independent of useage then the owner need only consider the demand side of the market. If on the other hand the production of future assets increases with use the tendency will be for useae to be maximized. If on the other hand the generation of future assets decreases with use, then the the optimal extraction rate will depend on the discount rate of the owner. In these cases the concepts of allocational and investment efficiency are unclear concepts, mostly because there is no clear question as whose discount rate matters. Nevertheless if it happens that the discount rate of the owner is somehow identical to some broader notion of discount rate, the owner will have the incentive to harvest at optimal rate. For the case of fisheries this just means the owner will tend to calculate the optimal rate of fishing per period, if the demand for fish is more or less constant per period, this harvest rate will correspond to the long run maximum number of fish rule.  

The cost of ex-ante allocation is an important factor in determining the regime that will be adopted. For animals there are times when ex ante allocation can be cheap(branding, collars, microchips, etc) and times when it can be expensive(fish, birds, etc). If it is difficult to create ex ante allocation then there will be difficulties which depend on the ex post regime adopted. 

The basic problem of dynamic property rights is conditionality. That is, property that is only allocated conditional on some effort. A potential normative role for the economist is to judge if the effort in question is desirable. It seems clear that if the effort is investment in some socially desirable good, then the conditionality is positive. However conditionality can also cause negative effects. 

Consider the case where animals move between properties. The ex ante ownership of the animals would result in Coasian barganing, in this case investment efficiency may look like an oasis or fences.

If it is costly create an ex ante allocation on animals a number of ex-post conditional property regimes may arise each with its own effects. If the animals are only owned conditionally on being on the land, this creates incentives for fencing as long as the the wild animal has higher subjective or market value than zero. If the animal is only owned if killed, then this creates an incentive to kill it. If land is lost(re-possessed) conditionally on having deer on it, then this creates an incentive to kill/abandon or sell the deer. In other words, the conditionality of property rights can have a plethora of effects. Notice that the fence may emerge in both the ex-ante ownership and the conditional on land ownership, however in the latter case, the presence of the fence is not neccesarily efficient(in the Kaldor Hick's sense). 

In the case of public property rights often the conditionality is on geography. For instance if some property's fruits are shared based on some geographical specification, this incentivizes entering the geographical area in question. In a sense the only way to sell one's share in the property is to move. This often has the effect of involuntary dillution of one's share due to new entrants.

In the case of private property a similar dilution may occur in the case of stock ownership, but it is usually for an associated sum with the idea of increasing the value of the shares held by investors by more than their dilution. In other words, when production plays a role, property is best attributed more directly to the people who are responsible for the production, this could be because they have knowledge of how to use it, or because they have some characteristic, such as risk bearing ability which would create higher productivity.  

A specific case of this conditionality are conditions on labor. Firms decide to reward employees based on their production, based on the assumption that agents put in effort as a function of the compensation conditional on that effort. For instance if there is a set of agents and a set of assets, and each agent can only work on a single asset, then it is simple to show that more production will be achieved in the case where agents own a higher fraction of the assets they work on than if their ownership was more distributed. This basic logic has lead to the development of the modern theory of the firm due to Hart and Moore. 


Conditionality can shed new lights on the normative theories of property rights. For instance Locke's theory that something is owned conditionally on mixing ones labor with it, whilst a moral theory, from the point of view of economics has descriptive content in the sense that such a property rights paradigm incentivizes people to put their labor into things. \footnote{Nozick, Anarchy State and Utopia}. From the economic point of view this is not neccesarily efficient relative to ex-ante ownership because this creates an over-incentive for labor instead of output. However if we consider that there is uncertainty about the creation of investments and a general unforeseeability, yet we also think that labor has positive upside, then the regime of labor mixing may be superior for output. 

The idea that property is granted conditional on some actions is a simple way to frame numerous concepts in economics. The questions of allocation and investment efficiency are both dependent on the conditions under which new property is distributed. For instance, take an example of the tragedy of the commons, overfishing. The issue with fishing is generally that all agents have the privilege of fishing the fish without actually owning them. The specific conditionality is that the fish is only owned once fished out of the water, which creates an incentive to overfish. The tragedy of the commons arises when first order rights and second right order rights are granted conditionally. Of course if there are no conditional property rights at all, even conditionally, then the only action the economic agent will undertake is direct pleasure and survival. 

The tragedy of the commons disappears with a number property right regimes, what they have in common is elimination of some sort of conditionality. For instance if fishing does not give you the right to sell(power) the fish but gives you the right to use it(privilege), then the level of fishing depends entirely on the cost of fishing and the demand for fishing, the problem then becomes purely a problem of population, clearly as the population increases it causes increases in fishing. Note however that this outcome may not be efficient in the Kaldor Hick's sense since some agents may have a lower cost of fishing. 

In general we may say that if there are full unconditional property rights, and the transaction costs are low, the allocational and efficient outcomes are achieved. On the other hand conditional property rights give rise to overuse(relative to the unconditional case) and no property rights gives rise to underusage. 

When numerous effort levels are possible it is also possible to talk of the strength of the conditionality. That is, the more effort, the more rights, this gives rise to analysis of the connection between effort and distribution. In the incomplete contracting approach the new property is created conditional on some effort, but distributed as a function of ex-post barganing power. While in the staic Coasian view, the ex post distribution does not affect decisions, in an incomplete contract world, the distribution matter. This basic tension motivates some general results on private property. 

The general motivation for property rights in the incomplete contracting literature is that property rights allow for investments to be undertaken before a contract is signed. Why is there a need to give negotiating power ex-post? Because ex-post, the other the other party has no reason to compensate for more than the value added to the transaction. In other words, other parties have no need to compensate agents for their fixed costs which were undertaken before the contract was signed. 

Why can't contracting be done ex-ante? There are two commonly given reasons, either contracting a priori is not profititable or because the future states cannot be described\cite{Hart1999}. The foundations of incomplete contracting have often been criticized because firms can just contract on outcomes instead of states and this can be equivalent to the first best contracts.\footnote{see \cite{Maskin2002} and \cite{maskin1999unforeseen}}. This is part of a setup for a larger problem in economics, the \textit{hold-up} problem, which says that if agents cannot use their sunk costs in the first period to negotiate in the second period, they will always underinvest. 

The justifications often go very far to explain something that can have quite a common sense foundation. Why can't agents contract ex-ante? Because they are not agents ex-ante. For instance if we imagine an individual throughout their life, some of their choices will be decided by those around them, either because of the cultural atmosphere or because they are not capable of making decisions. In a family structure, a parent may wish to invest for their child but they cannot contract long term on the child's behalf(this would be a form of slavery). Instead the parents can optimize ex-ante investments for their children without commiting them to long term contracts. 

Incomplete contracts imply a number of things about the theory of the firm. The theory of the firm is often framed as being about whether outsource or insource production. The problem with insourcing is that agents will be less motivated to put in effort, the problem with outsourcing is that agent ex-ante investments cannot be recuperated later on. This reasoning has been used by economists to explain why innovation often occurs in small firms, in large firms workers have an incentive effort that is observable to the owner. \cite{Holmstrom1989}. 

The mechanics of the model are simple. In a first period firms can undertake ex-ante unobservable investments, in the later period they can try to sign a contract with some other party that owns the asset that makes the investment useful. The problem is that the investment has already been undertaken by the time they are re-negotiating with the other party and the other party has no reason to compensate them based on their past investment, this will then lead to firms underinvesting. If there is only one firm that can undertake the investment, then the solution is simple, the firm need only ex-ante buy the asset it needs from the other party and then invests and reaps the profit on its own. 

However if both firms can undertake investments, there is larger issue. The party that does not own the asset will underinvest. There is no notion of equal ownership possible because the asset is indivisible. Fortunately there is a solution to this, however, it requires more time periods. First one of the firms owns the asset and invests on the asset, it then sells the asset to the second firm which then undertakes its own investment, this can be setup ex-ante by giving second firm an option contract to buy the asset. Since both firms will fully reap the benefits of their investments this will not lead to underinvestment. This same logic can be interpreted through re-negotiation, instead of buying the asset that has been worked on, the parties just bargain after the investment, but all the bargaining power has to be given to the party that invested. This sequential logic has limitations if there is uncertainty about what the optimum is \footnote{for designing the option contract see, \cite{Noldeke1998}, for sequential investments with complementary assets see, \cite{Zhang2014},\cite{bessen2009sequential} , for the breakdown of conditional contracts see, \cite{Maskin1999}} 

Note that the ownership of the asset itself is not the causal factor. What is is important is that a firm that is not needed has veto power. The framework is interesting because it does not identify the source of the value of a transaction but instead merely states the neccesary conditions for the value to be created. 

More generally, the model has a few conclusions: $1)$ If only a single agent can make asset specific investments, then allocative efficiency says that that agent should own all the assets; $2)$ All assets should be controlled by a single person at a time; $3)$ No more than a single agent should have veto power over an asset; An explicit assumption of this model is that assets only have value on the ultimate coalition that ends up using them. While this may be true in physical property, it is probably false for intellectual property. 

Conceptually we can imagine the effort as flowing towards three different components. The three components are, either the effort flow directly into the agent putting in the effort(human capitalish), either into an asset(physical capital), or into another party. The question of private property has to do with how many parties should have veto power over the use of the asset, and how should the veto power be distributed. In general the presence of the veto power is a disruptive force so it is best to give it to the party whose non-participation would already be most disruptive or whose participation is already neccesary. \footnote{the orignal model was intended for human capital only, created by \cite{Hart1990} which builds on the work of \cite{Grossman1986} }

Specifically if the effort(s) flows directly into either the agent(s) putting in the effort or directly into the asset, it is best to minimize veto power. When the effort flows directly into the it's own agent, then it is best if no veto power exists at all. If on the other hand the effort flows directly into the asset, then it is best if the asset is given to a single agent, the agent who is most productive with the asset. \footnote{For efforts flowing to assets, see \cite{schmitz2013investments}, \citet{gattai2016investment}, \cite{schmitz2017incomplete}. }

On the other hand if effort flows into other agent's(perhaps we can imagine agent's funding each others education) then we no longer minimize veto power. If only one agent exerts the effort then he should own the asset. If both agent's exert effort that flow into the other, then both agents should have veto power. \footnote{see \cite{hamada2011incentive}}

Some additional results from the incomplete contracts literature are highlighted below: Agents can also endogenously decide between them who will own the asset, this will depend on their relative marginal contributions to the asset and their ex-post barganing power, or if there are liquidity constraints, they may prefer a third party to own the asset; The framework can also be used to discuss the narrow incentives of the firm that will potentially integrate as opposed to the broader incentives of the firm being bought over in a scientific vs commercial payof context; In the context of innovation, the incomplete contracts framework implies that for exante contracts to be less restrictive, a larger amount of liability is required, to weed out bad inventors; If there is also assymetric information between the two parties, it can also be shown that joint ownership with veto power is optimal, this induces parties to share their information; \footnote{see respectively, \cite{Aghion1994},\citep{Lerner2005},\citep{Anton1994}, \citep{Rosenkranz1999}}

The setup of veto power given to non-necesary members is especially suited to analyzing intellectual property. The kind of situation decribed where, where a party is in a coalition for the sole purpose that they have an asset is, in fact, the norm in intellectual property regimes. The contracts framework is interesting because the value created between parties in the original work is not the price of the good but the value of the transaction. To render this point clear, suppose there is only one coalition using intellectual capital, now a different coalition without any intersecting members may adopt this good without decreasing the value of the first coalition. 

Notice that the notion of value being used here is not profits but subjective value. That is, while it is true that being the second firm to use an intellectual asset may mean a firm reaps less profits, in the subjective value sense this is not correct. That is, it is not because one agent figured out how to use his assets better first that the second agent will be less happy about discovering the same method. The framework of \cite{Anton1994} use the profit notion of value and not the subjective notion of value because they assume that as knowledge leaks from the intellectual asset occur, eventually the asset becomes worthless. 

Depreciation in use is the most natural way to conceptualize the quantitative differences between physical and intellectual capital. In a model one can say that assets are used sequentially and that transactions which use physical assets lose value the later they are in the sequence, whilst transactions that use intellectual capital do not lose value with use. 
you could just assume that as the asset is used, its knowledge leaks out into other users and eventually becomes worthless. The value of intellectual property only decreases because of copies or leakage. 

%%%%%%%%%%%%%%%%%%%%%%%%%%%%%%%%%%

To take an example, suppose that some person can produce k units if he were to work on a specific asset. Suppose there are 10 assets and 20 citizens. The citizens own assets equally, so each citizen owns $5 \%$ . If a citizen can work on a specific asset to produce 20 units of a good, then he will only receive 1 of those goods in the end. Which means his willingness to expend effort is 1. If on the other hand, the assets are redistributed in such a way that every citizen still owns $5 \%$ but the the citizens own higher shares of the assets they work on, if for example that same citizen had $10\%$ share of the asset he worked on then, his willingness to expend effort had doubled. In other words, we can say that the distribution of ownership does NOT matter if every citizen works on every asset and their productivity of every citizen is the same. 



To make ownership of an asset illegal is to make its market value zero or negative. 

The Coasian view of property rights claims that allocative efficiency is in fact irrelevant as long as there are no transaction costs and barganing is possible. 

The question of what assets property rights protects can be quite complex. Some possible categories to consider, are tangible vs intangible; the ratio of subjective value to market value. One could also have alienable vs inalienable. Or one could define assets as having market value, in this formulation an asset only acquires property rights when it has accumulated sufficient components to acquire either subjective or market value. The market value component could also be used to justify the scope of protection. 

Note that positive market value exists only if at least one non-possessor of the good has a subjective value for it. If the only person with positive subjective value of the good acquires it, then by definition the good no longer has market value. It is also possible for a good to have negative market value or subjective value. If the possessor of the good has a negative subjective value of the good AND no property rights to it, then the person is likely to either destroy it or abandon it. 

If a good only has value for a single person, then it has no market value. If it has value for more than a single person then it does have market value. ue 

\subsection{Externalities}

To define the Coase theorem we need the concept of externalities. Externalities are usually defined on a set of actions, it is when an action of A, has an effect on anybody else but A. Contracts can be framed as trying to create incentives for positive externalities, in other words to increase the payoff of an agent by A by taking a specific action.  

If an agent tries to 

\subsubsection{Coasian}




The coasian view of property rights aims to explain or justify property through two categories. Transaction costs and negative externalities. The Coasian approach has a simple statement of the form: externalities will only matter if transaction costs are large. 

If ownership and possession are not identical, this creates the possibility of temporary transfer of assets to higher value users or to lower cost producers. Similarly there is the investment efficiency from stable ownership, this is asset related skills.

The value of assets is not only direct, it can be expected to increase the value of other assets that are complementary to the asset itself.  

There is also of course a subjective value from ownership. For instance, suppose an owner who does not intend to use the asset but intends to rent it forever. If this owner discounts, the asset merely represents a series of cash flows, for which he should able to exchange a lump sum for. If we suppose that someone prefers to own rather than rent, liquidity is only constraint for transfer to occur. 

Economies of scope between monitoring and apprehension. 

\bibliography{../thesisbib/bibliography}

\end{document}

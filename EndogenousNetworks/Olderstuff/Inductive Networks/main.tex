\documentclass{article}
\usepackage{graphicx}
\usepackage{tikz,pgfplots}
\usepackage{preview}	
\usepackage{mathtools}
\usepackage{amsmath}
\usepackage{amssymb}
\usepackage{amsthm}
\usepackage[english]{babel}
\usepackage[utf8]{inputenc}
\usepackage[english]{babel}	
\usepackage{natbib}
\usepackage{color}
\usepackage[a4paper,top=3cm,bottom=3cm, right=2cm, left=2cm]{geometry}
\usepackage[normalem]{ulem}
\usepackage{blindtext}
\usepackage{xcolor}
\usepackage[colorinlistoftodos]{todonotes}
\usepackage[colorlinks=true, allcolors=blue]{hyperref}
\usepackage{cleveref} %label the theorem

\usetikzlibrary{math}

\bibliographystyle{agsm}

\newtheorem{theorem}{Theorem}	
\newtheorem{corollary}{Corollary}
\newtheorem{proposition}{Proposition}
\newtheorem{observation}{Observation}
\newtheorem{assumption}{Assumption}	
\newtheorem{definition}{Definition}
\newtheorem{remark}{Remark}
\newtheorem{lemma}{Lemma}
\newtheorem{result}{result}

\Crefname{assumption}{Assumption}{Assumptions}
\Crefname{assump}{Assumption}{Assumptions}

\begin{document}
\section{The model}

\indent Consider a set of firms that each decides sequentially to enter the market of a given good. Let $i$ be any typical firm in the former set, let $N$ be the set of firms that decide to enter the market and let $n=|N|$ the total number of firms which operate on the market. If $i\in N$, then $i$ pays a fixed cost $0\leq F< \infty$. This fixed cost is paid by all in $N$, irrespective of their production decisions. If on the contrary $i\notin N$, then $i$'s payoff is zero. A firm is characterized by its production technology. This technology is simply the firm's marginal cost of production. We assume that every firm is endowed with the same marginal cost, denoted as $c_1=c_2=...=c$ ex ante. A firm $i$ that enters the market can upgrade its original technology $c$ by improving upon some of the technologies already used in the industry, that is $c_1,\ldots, c_{i-1}$. For this, we assume that a firm $i$ may form directed links with some of its predecessors $j$ on the market. A link will capture the transmission of $j$'s technology to $i$. This process of accessing to the technologies in the industry (via a firm's investment in links) and innovating upon them enables the firms to reduce their marginal cost of production. \\
\indent We make clear how the innovation process is modeled. After taking the decision to enter the market, any firm $i\in N$ chooses with which of its predecessors' technologies it wants its technology to be correlated. By correlated, we mean improvement upon some of the technologies that already existed in the market prior to $i$'s entry. For example, if $i$ chooses to improve upon firm $j$'s technology, for $j\in \{1,...,(i-1)\}$, we say that $i$'s technology is correlated with that of $j$'s with probability one. We then say that $i$ has a link with $j$, and this link is represented graphically as $i~\rightarrow ~j$ and denoted as $ij$ throughout. Any link is always directed ; it allows the transfer of $j$'s technology to its successor $i$ but not vice-versa. A strategy of link formation $s_i$ for any firm $i\in N$ is an element of the class of all subsets of $\{1,\ldots , i-1\}$.  The set of all strategies of $i$ is $\mathcal{S}_i$; it is the set of all firms with whom can possibly form links\footnote{A firm cannot form a link with any of its successors. Therefore $i$ can only choose from the set of its predecessors for forming links.}, and it follows that $|\mathcal{S}_i|=2^{i-1}$. All of the firms' decisions in link formation map a technological network. Generally, a network \text{g}$=(V,E)$ is defined on its set of vertices $V$ (i.e. the nodes of the network, thus here $V=N$) and its set of edges (or links). Thence if $j\in s_i$ and $i$ is a vertex of some network \text{g}, $i$'s technology is correlated with that of $j$, and the link $i\rightarrow j$ exists in the technological network \text{g}. \\
\indent We make the strong assumption that the firms choose which technologies they want to infringe on. That is, we consider that a firm may choose to innovate upon some existing technologies - orienting purposefully its R\&D efforts on improving the technologies of the firms the former has links with. Assume that some firm $i$ belongs to some technological network \text{g}. The set $T_i$ refers to the set of firms that have their technologies infringed on by $i$'s. As we mentioned in the precious paragraph, a link $ij$ implies that $i$'s technology infringes on that of $j$. Thus $s_i\subseteq T_i$. But note that if $j$'s technology is correlated with the technology of some firm $k\in \{1,\ldots,j-1\}$ then $i$'s technology also infringes on that of $k$'s. Belong to $T_i$ all firms $j\neq i$ such that there exists a directed path from $i$ to $j$ in \text{g}. The set $T_i$ is the set of all predecessors of $i$ such that $i$ has its technology correlated with theirs. Here, a path from $i$ to $j$ is a sequence of links which connect a sequence of vertices (firms). Note that if $j\in T_i$ and $i$ does not have a link with $j$, then there must exist $k\in T_i$ such that $j\in s_k$, for all distinct firms $j,k\in T_i$. Now, the longest path that starts from firm $i$ (denoted just after as $i_1$) in \text{g} is the longest sequence of directed links $ i_1i_2,\ldots , i_mj$ in \text{g} which first link is a link formed by $i$. The length of this longest path will be referred to as $\rho_i$ (for the previous sequence $\rho_i=m$). Given the network \text{g}, firm $i$'s ex-post technology $\tilde{c}_i$ is given by the expression:
\begin{equation}
\tilde{c}_i=\beta \rho_ic
\end{equation}
for $\beta \in [0,1]$. \\
\indent It follows that the existence of a link entitles the firm which initiates it to some technological benefit, i.e. a reduction in its marginal cost of production. The discrepancy $\beta \rho_i$ in the marginal cost of firm $i$ depends on the length of the longest path that starts at firm $i$, that is on the largest number of different technologies $i$'s technology infringes on.\\
\indent However improving upon some technologies does not come at no cost for the firms. We assume that all firms that do produce must pay royalties when their technology infringes on some other technologies. For our typical firm $i$, her technology $\tilde{c}_i$ costs her $rt_i$, for $r$ the fixed royalty paid to any firm in $T_i$ and $t_i$ the cardinal of the later set. The correlations between $i$'s technology and all the technologies of the firms in $T_i$ are graphically captured by paths from $i$ to any firm in $T_i$. We consider that a firm's expenditure in royalties is linear in the number of the firms to which it is connected in the technological network. By the same principle, firm $i$ may receive royalties paid by some of its successors which technology infringes on $i$'s (there exists a path from some firm $j>i$ to firm $i$ in \text{g}, i.e. $i\in T_j$). Let $M_i$ be the set of these firms $j$ which technology infringes on that of $i$ in some network \text{g}, and let $m_i$ the cardinal of this set.\\

\indent Once the technological network is mapped and the correlations (infringements) between the technologies of the firms revealed, the firms play a classic Cournot game. We take a linear inverse demand function of the form $P(q_1,...,q_n)=\alpha - Q$ for $Q$ the total output supplied on the market. For the sake of clarity, let $q$ be the (row) vector of all of the firms' outputs such that $q\times 1_{n,1}=Q$. Note that a firm that did decide to enter the market in the first stage may not necessarily produce in the second stage of the game. Such a firm does not pay royalties at all, but it can receive some royalties revenue (as long as at least one firm in the corresponding set $M$ does produce a strictly positive amount of output). Therefore, firm $i$'s total expenditure in link formation is given by the expression $r t_i$ if and only if $i$ does offer a strictly positive quantity on the market ; otherwise, $i$ does not pay anything at all. Also, $i$ receives a royalty revenue from some firm $j>i$ if and only if $j$ chooses to produce.\\

The payoff of any firm $i$ that entered the market in the first stage and which produces is given by the following expression: 

%\begin{equation}
%\pi_i(q, (s_i,s_{-i}),r) = \Bigg(\alpha - %\sum_{j=1}^n q_j - \tilde{c}_i \Bigg)q_i + r\Bigg( %\sum_{j\in M_i}\mathbbm{1}_{q_j>0} - %\mathbbm{1}_{q_i>0}~ t_i\Bigg)  - F \label{payoff},
%\end{equation}  
where the first term into bracket is $i$'s net revenue on every unit of good that it sells on the market, and the second term into bracket is $i$'s net revenue from royalties. \\

\indent The next section of the paper is devoted to solving for the subgame perfect Nash equilibria of the game. We will use backward induction in order first to get the best-response functions for the firms that decide to enter the market, then the optimal strategies in link formation, and finally solve for the entrance decision. We will consider that the firms are foresighted, which entails that they do consider in their expected profit the revenue from royalties that might be generated by the arrival of successor firms on the market. 

\section{SPNEs of the game }

\indent We solve for the SPNE of this sequential game using backward induction. We first clarify the stages of the game, and the actions taken by the representative firm $i$ in each and every of these stages. \\

\textit{First stage: entry and link formation }\\

\indent In the first stage, a firm $i\in \{1,2,...\}$ decides whether or not it enters the market. If $i$ decides not to enter the market, the game ends for this firm and its payoff is zero. If $i$ enters instead, therefore expecting a positive profit from doing so, it chooses its innovation level $\tilde{c}_i$ that depends on $i$'s strategy of link formation $s_i$. This strategy of link formation determines which technologies are correlated with $i$'s as well. Note that the decision to enter and the choice of the links that the firms maintain can be broken down into two stages. \\

\textit{Second stage: Cournot game }\\

\indent All of the firms in $N$ choose a quantity to offer on the market. By the beginning of stage two, the technological network \text{g} is realized. If $i$ chooses not to produce, then this firm is free not to pay any royalty - since $i$ does not use its technology. However, $i$ may receive some royalty revenue from some firms in $M_i$. If a firm chooses not to produce, then this firm is referred to as a \textit{rent seeker}. If $i$ does produce, then we refer to $i$ as a \textit{producer}. Note that $i$'s net revenue from royalties never depends on the quantity it produces when this quantity is strictly positive. A producer $i$ is always more efficient at producing than any producer in $T_i$ (i.e. $\tilde{c}_i<\tilde{c}_j$ with $j\in T_i$), and always less efficient at producing than any producer in $M_i$ (i.e. $\tilde{c}_i>\tilde{c}_j$ with $j\in M_i$).  \\

\textit{Third stage: the payoffs are realized}\\

See equation \ref{payoff}. \\

\indent We solve for the second stage of the game. Assume that the technological network \text{g} has been mapped after the first stage. Now all firms in $N$ must decide whether they want to produce, and if so which quantity. Let $q_{-i}$ be the total production supplied on the market when $i$ does not produce. The production strategy $q_i$ of firm $i$ is said to be a best-response to $q_{-i}$ if given some fixed value $r$ for the royalty : 
\begin{equation}
\pi_i((q_i,q_{-i}), (s_i,s_{-i}),r)\geq \pi_i((q'_i,q_{-i}), (s_i,s_{-i}), r),~~\text{for all}~~q_i\in \mathbb{R}~~\text{and}~~q_{-i}\in \mathbb{R}^{n-1}.
\end{equation}

The set of all of firm $i$'s best-responses (in production) is denoted $BR_i(q_{-i})$. Furthermore, the total output $Q$ supplied on the market is a SPNE if $q_i\in BR_i(q_{-i})$ is the strategy played by each firm $i\in N$. We now present the analytic form of the best-responses of all firms in this second stage.\\

\textbf{Proposition 1.} \textit{Let $i$ be any firm in the set $N$. Let $\Omega\subseteq N$ be the set of firms which produce on the market, and let $\omega$ the number of these firms. The best-response of firm $i$ to the vector $q_{-i}$ of its competitors' strategies is given by the expression: }
\begin{equation}
BR_i(q_{-i}) = \dfrac{1}{\omega+1}\Big(\alpha - (\omega+1)\tilde{c}_i + \sum \limits_{ j\in \Omega}\tilde{c}_j \Big), \label{BR}
\end{equation}

\textit{for any $i \in \{1,...,n\} $. For some $i\in N$, $i$'s ex-post payoff is :}

%\begin{multline} 
%\pi_i(q, (s_i,s_{-i}),r)=\\
%\max 
%\begin{cases}
%\dfrac{1}{(\omega+1)^2}\Big(\alpha - %(\omega+1)\tilde{c}_i + \sum \limits_{ j\in %\Omega}\tilde{c}_j \Big) ^2 + r\Bigg( \sum_{j\in %M_i}\mathbbm{1}_{q_j>0} - \mathbbm{1}_{q_i>0}~ %t_i\Bigg)  - F 
% \\
 %r  \sum_{j\in M_i}\mathbbm{1}_{q_j>0}-F
%\end{cases}\\
%\end{multline}

\indent Although a firm's best-response may prescribe to produce a strictly positive amount of output, it might be more profitable for this firm to adopt a rent seeking behavior. This is true whenever the Cournot profit of a firm is less than its expenditure in links formation (or royalties expenditure). \\

\indent We turn to the equilibrium strategies of the firms for the first stage of the game. First, we make explicit the optimal strategies in terms of link formation of the firms that decide to enter the market. We show in what follows that forming a single link in the network strictly dominates any strategy that consists of maintaining strictly more than one link, and this for all the firms in $N$. \\

\textbf{Proposition 2.} \textit{If $r>0$ then any firm $i\in \Omega$ always forms either one link or none in equilibrium. }\\

\indent \textbf{Proof.} This is a direct proof. Take any firm $i\in \Omega$. Let $s_i\in \mathcal{S}_i$ be any strategy for firm $i$ that consists of maintaining at least two links in the industry. Let $j$ be the firm that has the lowest marginal cost in the set $s_i$. Now consider the alternate strategy $s_i'\subset s_i$  for firm $i$ that consists of forming one single link to this firm $j$ just defined. We show that $s'_i$ always strictly dominates $s_i$. First the technological benefits of $s_i$ and $s'_i$, respectively. By hypothesis, only the longest path that starts at vertex $i$ matters for gauging the technological benefit from $i$'s connections. Thus $\tilde{c}_i=\tilde{c}'_i$, for $\tilde{c}_i$ the cost achieved by $i$ when it plays $s_i$ and $\tilde{c}'_i$ when $i$ plays $s'_i$. And $T'_i\subset T_i$ since $s'_i\subset s_i$, for $T'_i$ the set of all firms that have their technology correlated with $i$'s technology when the later plays $s'_i$, and $T_i$ the same set but when $i$ plays $s_i$. Let $t'_i$ the cardinal of $T'_i$ and $t_i$ the cardinal of $T_i$. Since $i\in \Omega$, then $i$ produces. Its expenditures in links is $rt_i$ if $i$ plays $s_i$, and $rt'_i$ if $i$ plays $s'_i$. By the previous point, $rt_i>rt'_i$. Note that $i$ playing $s_i$ over $s'_i$ has no influence on the number of the successors of $i$ which will eventually connect to $i$ (since $i$'s technology is the same for both strategies). The result follows. $ \qed$\\  

\textbf{Remark 1.} For all $i\in \Omega$ we have that $\rho_i=t_i$ in equilibrium, and the path from $i$ to any $j\in T_i$ is unique in \text{g} if \text{g} is a Nash network. \\

\indent \textbf{Proof.} This result is implied by proposition 2. To see this, consider the set $s_i$ for any firm $i\in \Omega$. The statement needs only be proven when $s_i\neq \emptyset$. Therefore $|s_i|=1$ by proposition 2. Consider all paths that start at vertex $i$ in the technological network $i$ belongs to. The path from $i$ to a firm $j$ that is a predecessor of $i$ and that has a worse technology than $\tilde{c}_i$ exists if and only if $j\in T_i$. And this path is unique since all of $i$'s predecessors $j$ with $\tilde{c}_j>\tilde{c}_i$ have either no link or one link. Thus if there is $\tilde{c}_k=\tilde{c}_j>\tilde{c}_i$ then if a path from $i$ to $j$ exists then no path from $i$ to $k$ exist by proposition 2. Therefore : $j\in T_i~~\Leftrightarrow$ a unique path from $i$ to $j$ exists. It follows that if $i\in \Omega$, $i$ must pay royalties to all firms in $T_i$ provided that all of these firms belong to the longest path that starts at $i$. This path is unique and is the path that connects $i$ to the firm in $s_i$ that has the worst technology. Hence proved. $\qed$\\

\textbf{Remark 2.} \textit{The last firm which enters the market always produces in equilibrium.}\\

\indent \textbf{Proof.} If $n$ is the last firm to enter the market, then $M_n=\emptyset$. Thus if $n$ does not produce, its payoff is always $-F$. But then no entry would have been preferable over entry and not producing for $n$. Therefore a contradiction. \\
\indent \textit{One then notes that the Cournot profit of the last firm that enters must exceed the sum of its royalties expenditures and the fixed cost. }
\section{Two firms}

\begin{align*}
    \pi_m -F \\
    \pi_s -F \\ 
    \pi_l -F -r \\
    \pi_h -F +r 
\end{align*}

\textcolor{orange}{Where $\pi_m>\pi_l>\pi_s>\pi_h$ }

The first is the monopoly profit, the second is the being dominant profit, the third is the symmetric profit, and the fourth is the dominated profit. 

\subsection{When will firms enter?}

\subsubsection*{ Cost is: $F> \pi_m>\pi_l>\pi_s>\pi_h$}
No entry by anybody.

\subsubsection*{Cost is: $\pi_m>F>\pi_l>\pi_s>\pi_h$}
If F is higher than three of them then there is entry by the first firm and not the second.

\subsubsection*{Cost is : $\pi_m>\pi_l>F>\pi_s>\pi_h$}
If this is the case then outcome depends on r. 
If $\pi_h -F+r>0$ then the first firm will enter and be followed by the second firm and will attach itself to the first firm. 

Otherwise no entry by anybody because the first firm will anticipate that it will not earn enough profits and not enter.

\subsubsection*{Cost is: $\pi_m>\pi_l>\pi_s>F>\pi_h$}
No matter what happens there will be entry by both firms. The specific network depends on r:

If $ \pi_h -F+r>\pi_s -F \rightarrow \pi_h+r>\pi_s $
Then there will be a hierarchy. 

$\pi_h -F+r<\pi_s -F \rightarrow \pi_h+r<\pi_s$. Then there will be no hierarchy. 

\subsubsection*{Cost is: $\pi_m>\pi_l>\pi_s>\pi_h>F$}
Identical to the previous case

\section{Three firms}
Payoffs
\begin{align*}
    \pi_{m} -F ~~\text{if only a monopoly} \\
    \pi_{2} -F>\pi_{3} -F  ~~\text{if only a symmetry}\\ 
    \pi_{1 \overline{2}}-r-F  ~~\text{if low tree, low payoff is} \\
    \pi_{\overline{1}2}+2r-F  ~~\text{if low tree, high payoff} \\
    \pi_{2 \overline{1}} -r-F  ~~\text{if 1 low 2 high, low payoff} \\
    \pi_{\overline{2} 1 }+\frac{r}{2}-F  ~~\text{if 1 low 2 high, high payoff} \\
    \pi_{\overline{1}11}  +2 r-F  ~~\text{if chain, high} \\
    \pi_{1 \overline{1}1} + r-r-F=\pi_{1 \overline{1}1} -F  ~~\text{if chain, middle} \\
    \pi_{11 \overline{1}} -2r-F  ~~\text{if chain, low}
\end{align*}

Some unambiguous relationships: 

$\pi_{11 \overline{1}}>\pi_{1 \overline{1}1}>\pi_{\overline{1}11}$

$\pi_{1 \overline{2}}>\pi_{\overline{1}2}$

$\pi_{2 \overline{1}}>\pi_{\overline{2}1}$

$\pi_{2}>\pi_{3}$

\textcolor{orange}{$\pi_{11 \overline{1}}>\pi_{2 \overline{1}}>\pi_{1 \overline{2}}$ unclear if this assumption should be made}

Plan now is, make a condition for there being a chain, and then make a sub-condition for if it is possible that the first and third produce but not the middle. 

\subsection{When is there a chain?}

\subsubsection{Conditions on third player}
A minimum for this to be feasible condition is that:
\begin{equation*}
        \pi_{11 \overline{1}}-2r-F>0
\end{equation*}
Third agent must prefer the chain over a lower tree:
\begin{align*}
    \pi_{11 \overline{1}}-2r-F>\pi_{1 \overline{2}}-r-F \rightarrow \pi_{11 \overline{1}}-r>\pi_{1 \overline{2}} 
\end{align*}

Third agent must also prefer the chain to independent technologies. 
\begin{align*}
    \pi_{11 \overline{1}}-2r-F>\pi_{h21}-F \rightarrow \pi_{11 \overline{1}}-2r>\pi_{h21} 
\end{align*}

\subsubsection{Condition on second player}

The second player has the most strategic position. The payoff if the second player attaches himself to the first player if he thinks this will get him to the outcome with the third player making the optimal choice for him. 

If he attaches himself to the first player, then if the conditions above are met, the payoffs in the end will be:

\begin{equation*}
\pi_{1 \overline{1}1}-F
\end{equation*}

If he creates an independent technology, then the payoff depends on the third players choices. Either the last player will also choose an independent technology, or will attach himself to one of the two players, with equal probability or will choose none of the two because it will be a negative payoff. 

So if the highest payoff of the third player is the the symmetric outcome then the payoff of the second player's payoff is simply: 

\begin{equation*}
\pi_{\overline{3}}-F
\end{equation*}

If the highest payoff of the third player is a downstream technology, then the second players payoff is:

\begin{equation*}
\pi_{\overline{2}1} - \frac{1}{2}r-F
\end{equation*}

If the highest payoff of the third player is not to enter then the second players payoff is:

\begin{equation*}
\pi_{\overline{2}}-F
\end{equation*}

Note that only one of these outcomes will be available of at a time, so we need only make sure that the available outcome is lower the outcome above. 

\subsubsection{Condition on first player}

Now the first player has only two choices, enter or not enter. If it turns out that the second player will lead the equilibria toward the symmetric equilibria, the outcome will be:

\begin{equation*}
\pi_{\overline{3}} -F 
\end{equation*}

Or if its not profitable for the third firm to enter:

\begin{equation*}
\pi_{\overline{2}} -F 
\end{equation*}

Otherwise if the second player will strategically choose to go towards the chain. Then the first player will simply enter if 

\begin{equation*}
\pi_{\overline{1}11}+2r-F
\end{equation*}

\subsection{When there is a chain, is everyone producing?}
The last person is always producing because he has no other source of revenue and that revenue is must be larger than both the royalties and fixed cost.

The second to last person must also be producing because he has no revenue from royalties so to overcome the fixed cost he must produce. So at the very LEAST half the people MUST be producing? This generalizes

\begin{proposition}
If someone has as many people he has to pay as people he has to pay, then he MUST produce. 
\end{proposition}

\begin{corollary}
If the above agent HAS to produce, then all agents further down the chain ALSO have to produce. This is because they have less royalty revenue, and more royalty costs, so the only way to overcome these costs is to have a revenue from the market. 
\end{corollary}

So at the minimum, the person in the middle of the chain has to produce. Now the question becomes, who ABOVE the middle person produces? 

A similar reasoning can be applied, to the first firm, the first firm will produce as long as the marginal cost is lower than the price. 

We know that the effect on the price of the more productive firms will be. So we can use this to get an upper bound on the price.  

\section{Four firms}

\subsection{Chain}

\subsubsection{Last firm}

The last firm has a single strategic consideration. The trouble is that the strategic consideration might affect the production of any subset of firms. If the last firm extends the chain it might make other firms not produce. We know a priori that it can't make the royalty negative firms not produce. So it can only make the royalty positive firms not produce.

Within the royalty positive firms we can look at ONLY the firms who receive sufficient revenue from the royalties to overcome the fixed cost. Which firms receive sufficient royalty revenue to overcome the fixed cost? The earlier firms. In other words we have a guarantee that the firms closer to the royalty neutral firms will produce. That is if $F>r$ that means the closest firm to the revenue neutral HAS to produce. 

From the relationship between $F$ and $r$. We can know which firms will produce surely. 


If the choice of the fourth firm will not change the production decisions of other firms, then what it needs to consider is simply: 


\begin{equation*}
\pi_{111\overline{1}}-3r-F
\end{equation*}

This must be greater than 
\begin{equation*}
max\{\pi_{11\overline{2}}-2r-F,\pi_{1\overline{2}1}-r-F,\pi_{\overline{2}11}-F,0\}    
\end{equation*}



\subsubsection{Third firm}

If the fourth firm does not enter because 0 is its greatest elements then there are no strategic considerations and will choose:

\begin{equation*}
max \{\pi_{11 \overline{1}}-2r-F,\pi_{1\overline{2}}-r-F,\pi_{\overline{2}1}-F,0 \}
\end{equation*}

If the fourth firm will enter then there are strategic considerations. The only non trivial case is if profits are concave with respect to innovation. For the chain to emerge, it must be that the highest payoff is: 

\begin{equation*}
\pi_{11 \overline{1}1}-r-F
\end{equation*}

If the last firm will extend the chain no matter what the choice of the third firm then the appropriate maximization vector to compare this payoff to is: 

\begin{equation*}
max \{ \pi_{1 \overline{2} 1}-F, \pi_{ \overline{2}1 1}-F, 0 \}
\end{equation*}

\begin{proposition}
If a firm has first mover advantage, and we ignore the strategic considerations of other firms, then it always prefers being the median firm than being unattached. 
\end{proposition}

\begin{proof}
Note that the payoff of being unattached is $\pi-F$ and the payoff of being the intermediate firm is $\pi '-F$. By assumption the profit is greater in the intermediate case. 
\end{proof}

The assumption that the fourth firm will extend the vector no matter what the choice of the last firm is simply: 

\begin{align*}
\pi_{1 2 \overline{1}}-2r-F> max\{ \pi_{1 \overline{3} }-r-F, \pi_{ \overline{2}2 } -F  \} \\
\pi_{ 2 1 \overline{1}}-2r-F> max\{ \pi_{2 \overline{2} }-r-F, \pi_{ \overline{3} 1 }-F  \} \\
\pi_{ 3 \overline{1}}-r-F > max\{ \pi_{ \overline{4} }-F  \} \\
\end{align*}

This assumption is neither sufficient nor necessary for the emergence of the chain. Can we get a sufficient or a necessary condition. It is only sufficient as a condition on this player. If this was true for all agents, there would ONLY be a chain by definition. 

If this assumption is not verified then we may still be able to have the chain emerging. It must only be that the fourth firm prefers the chain whenever the third firm also prefers the chain. 

If the third firm prefers the non chain but ONLY if the latter firm takes a specific action AND the other firm cannot be induced to take that other specific action then this payoff does not need to be compared to the chain payoff. So if we refine the set to have only the payoffs that meet this criteria then we can guarantee the chain. 

\textcolor{red}{I use the assumption that the follow up firms will automatically attach themselves only to the first firm to start the chain(first mover advantage)}

\textcolor{red}{If we say that the effect of another element in the chain has a level effect this may be an easier way to solve the problem}


\subsubsection{Second firm}


The second firms payoff 

\subsubsection{First firm}

So in this case we know that the last two firms will produce. We are looking for the equilibrium where the first firm will also produce but not the second firm. 

The potential payoffs from attaching to the first firm is:

\begin{equation*}
\{ 
\pi_{1 \overline{1}11}+r-F,
\pi_{1 \overline{1}2} + r -F ,
\pi_{1 \overline{2}1} - F , 
\pi_{1 \overline{3} -r -F },
\pi_{2 \overline{1}1} -F ,
\pi_{2 \overline{2}} -r - F, 
\pi_{3 \overline{1}} -r -F, \}
\end{equation*}

Potential payoffs if does not attach:
\begin{equation*}
\{ 
\pi_{ \overline{2}11}-F,
\pi_{\overline{2}2} -F ,
\pi_{ \overline{3}1} - F , 
\pi_{1 \overline{3} -r -F }
\}
\end{equation*}

The decision to not produce means that the following condition must hold. 
\begin{equation*}
2r-F>\pi_{1\overline{1}11}+r-F
\end{equation*}

If the equilibrium will not be a chain then:

\begin{equation*}
2r-F>\pi_{1\overline{1}2}+r-F
\end{equation*}

Note that the payoff is higher if the chain is less long. 

\begin{equation*}
\pi_{1\overline{1}11}+r- F> \pi_{1\overline{1}2}
\end{equation*}

\subsubsection{First firm}

In that case the revenue of the first firm will be:

\begin{equation*}
\pi_{\overline{1}011}+2r-F
\end{equation*}

OR

\begin{equation*}
\pi_{\overline{1}02}+2r-F
\end{equation*}

\subsection{End tree}

\subsubsection{Fourth firm}

The last firm has no strategic considerations, if the chain equilibrium is to emerge then it must have the following conditions: 

\begin{equation*}
\pi_{11\overline{2}}-2r-F
\end{equation*}

This must be greater than 
\begin{equation*}
max\{\pi_{111\overline{1}}-3r-F,\pi_{1\overline{2}1}-r-F,\pi_{\overline{2}11}-F,0\}    
\end{equation*}

If we have the paradigm that two other firms are attached to first person. 

\subsubsection{Third firm}

\subsubsection{Second firm}

\subsubsection{First firm}

\section{Can we get an externality formulation? }

Suppose that the presence of each firm has an externality. That is identical on other firms. The externality will depend on how many firms are present at each level. 

So if the externality of a firm further down the line is quite large, then it might affect the decision to produce of firms further up the line. In other words, the pure rent seeking choice is more likely to happen in a chain than a tree.  

\begin{equation*}
E()
\end{equation*}


\subsection{generalization}

Let us represent the users who are below half the chain:

\begin{align*}
R(n) = \pi(n)-nr-F \\
\text{If people further down the chain earn more then we have a chain} \\
\frac{\partial R(n)}{\partial n} = \pi'(n)-r  
\end{align*}

If the profit is decreasing, or concave then eventually it will not be optimal to go further down the chain. 

If eventually going down the chain is negative then, we know that the firm must choose somewhere else on the chain to go to. 

Since if a firm chooses to make a tree and then the next firm chooses to to make a chain, this will dominate the the tree choice. 

Expanding the chain has a higher marginal effect on the market profits of others than making a tree. So the chain may prevent the next person from attaching when the tree would not and never the other way around. So the number of firms in equilibrium is minimized if there is a chain. 

\section{Suppose the first firm makes zero profits and its cost is too high to produce}

The profit of the first firm is then:

\begin{assumption}
The first firm in the chain makes 0 profits. 
\end{assumption}

\begin{align*}
rn-F=0
\end{align*}

This gives us the exact number of firms that must emerge in equilibrium. 

\begin{equation*}
n = \frac{F}{r}
\end{equation*}

\begin{proposition}
IF a firm does not produce, then it earns 0 profits. 
\end{proposition}

\begin{proof}
We need only note that the firm with the highest amount of royalties earns 0 profits. The second firm to enter will also pay F and will receive a royalty for each downstream firm producing. Since the firm is lower than the first firm, it cannot receive more royalties than the first firm. If it receives any less, then it will not enter. 
\end{proof}

Suppose that n=1, this means that there are only two types of revenues. 

Suppose that n=2, this means that there are at least 3 types of revenue. 

From the producing firms however we may get either a chain or a tree. Firms have strategic considerations that will help them. Suppose that n-1 firms have already been placed on a tree. The last firm may find it optimal to place itself at either the chain or the tree.  

Things to change,: The cost function, and perhaps make the royalty per unit

\section{Cournot with geometric formulation}

A firm with a technology k, will have marginal cost $c b^k$. Let the number of firms producing be n. 

\begin{align*}
P= A-q_i+q_{-i}=A-Q \\
\Pi = P q_i - q_i c b^k \\
\end{align*}

If there is a chain then the minimum cost condition is:

\begin{equation*}
c<
\frac{A}{n-\Sigma_{i=1}^n b^i}
=\frac{A}{n-\frac{b(1-b^{n-1})}{1-b}}
=\frac{A(1-b)}{n(1-b)+b^{n}-b}
\end{equation*}

\begin{equation*}
A-Q = \frac{1}{1+n} \left(
A+ \Sigma^n_{i=0} b^ic
\right) 
= \frac{1}{n+1} \left(
A+ \frac{(1-b^n)c}{1-b}
\right) 
\end{equation*}

If the first firm is not producing the second firm will produce if : 
\begin{equation*}
c< \frac{A}{(n-\Sigma^{n-1}_{i=1}b^i)b} 
= \frac{A}{(n-\frac{b(1-b^{n-1})}{1-b})b}
=\frac{A(1-b)}{b(n(1-b)+b^{n}-b)}
\end{equation*}

The price in this case is:

\begin{equation*}
A-Q= \frac{1}{n+1} \left( 
A+\Sigma^n_{i=1}b^i c
\right) 
= \frac{1}{n+1}\left(A+\frac{(b-b^{n})c}{1-b} \right)
\end{equation*}

The price if n firms are producing in a chain without the first k firms producing the price is given by:

\begin{equation*}
p(n,k)=\frac{1}{n+1}\left(
A+ c \left(
\frac{b^k-b^n}{1-b}
\right)
\right)
\end{equation*}

Similarly if n firms are producing without the first k firms producing the condition for the k+1 firm to produce is given by:

\begin{equation*}
c(n,k)< \frac{A(1-b)}{b^k(n(1-b)+b^n-b)}
\end{equation*}

Let the n firms be producing, let their technology level be denoted by i. Then when i=0, the marginal cost is $cb^0$, if i=1 it is $cb^1$ etc. 

Then if all the n firms in a chain are producing, then the ith firms profits are given by:

\begin{equation*}
q(i,n) = \frac{1}{n+1} \left( 
A+c\left(\frac{1-b^n}{1-b}-(n+1)b^i\right)
\right)
\end{equation*}

If the first k firms are NOT producing: 

\begin{equation*}
q(i,n,k) = \frac{1}{n+1} \left( 
A+c\left(\frac{1-b^{n+k}}{1-b}-(n+1)b^i\right)
\right)
\end{equation*}

The royalty revenues on a chain are similarly: 

\begin{equation*}
r(n-2i+1)
\end{equation*}


\section{3 firm case fully worked out}

\subsection{3rd firm}

If there are no attachments existing, the third firm to enter will only not attach itself if:

\begin{align*}
\text{max} 
\{ 0,
\left(\frac{A-c}{4}\right)^2, 
 \left( \frac{1}{4}(A+c(2-3b))\right)^2-r
\}
\\
\left(\frac{A-c}{4}\right)^2> \left(\frac{1}{4}(A+c(2-3b))\right)^2-r \\
\rightarrow 
r> r_{sym} = \frac{3c}{16}(1-b)(2A+c-3bc)
\end{align*}

This is a necessary condition for non-attachment and the upper RHS must be higher than 0.

\begin{equation*}
\text{max} \{
0,
\frac{1}{4}(A+(b-2)c), \frac{1}{4} (A+c(1-2b))-r, \frac{1}{4}(A+c(1+b-3b^2))-2r \}
\end{equation*}

Will prefer no attachment to attachment 1st firm iff:

\begin{equation*}
r> \frac{3}{16} c \left(2 A (1-b)-c(1-b^2 ) \right) = r_1
\end{equation*}

Will prefer no attachment to attachment 2nd firm iff:

\begin{equation*}
r > \frac{3}{32} \left(1-b^2\right) c \left(2 A-c(1-2b+3 b^2) \right) = r_2
\end{equation*}


Will prefer attach to 1st over 2nd iff:

\begin{equation*}
r > \frac{3}{16} (1 - b) b c (2 A + (2 - b - 3 b^2) c) = r_3
\end{equation*}

The third firm can only attach itself to the middle if:

\begin{align*}
\frac{3}{16} (1 - b) b c (2 A + (2 - b - 3 b^2) c)<\frac{3}{16} c \left(2 A (1-b)+c(b^2 -1) \right) \\ 
\Leftrightarrow 
2A>c(1+b(4+3b))
\end{align*}


Let the three r's be denoted by, $\underline{r},\overline{r}$

\subsection{2nd firm}

\subsubsection{Unconditional expansion } The third firm always expands the chain. 

So the relevant vector the second firm will be looking at is: 

\begin{equation*}
\text{max} \{ 
0, \left(\frac{1}{4}(A+ c(b-2)) \right)^2+\frac{1}{2}r, \left(\frac{1}{4}(A+c(1-3b+b^2)) \right)^2
\}
\end{equation*}

\begin{equation*}
r>r_{unc}=\frac{1}{8} \left(b^2-4 b+3\right) c \left(2 a+\left(b^2-2 b-1\right) c\right)
\end{equation*}

\subsubsection{Moderate Innovator} Then the third firm always expands to the second tier but never to the the third tier. 

\begin{equation*}
\text{max} \{ 
0, \left(\frac{1}{4}(A+ c(b-2)) \right)^2+\frac{1}{2}r, \left( \frac{1}{4}(A+c(1-2b)) \right)^2
\}
\end{equation*}

\begin{equation*}
r>r_{mod}= \frac{3}{8} c \left(2 a (1-b)-c(1-b^2)\right)    
\end{equation*}

\subsubsection{Luddite}

The third firm will never attach itself. 

Will prefer no attachment if: 

\begin{equation*}
\text{max} \{ 
0, \frac{A-C}{4}, \frac{1}{4}(A+c(2-3b))-r
\}
\end{equation*}

\begin{equation*}
r> r_{lud} = \frac{3}{16} (1-b) c (2 A-3 b c+c)
\end{equation*}

\subsubsection{Double or nothing}

If it is the case that the third firm will either make a chain or be a singleton then: 

\begin{equation*}
\text{max} \{ 
0, \left(\frac{A-c}{4} \right)^2, \left(\frac{1}{4}(A+c(1-3b+b^2)) \right)^2
\}
\end{equation*}

In this case the firm will always choose the chain setup. 

\subsubsection{Copycat}

If you don't attach, he also will not attach, if you attach, he will attach to same person. 

\begin{equation*}
\text{max} \{ 
0,  \left(\frac{A-C}{4} \right)^2, \left( \frac{1}{4}(A+c(1-2b)) \right)^2-r
\}
\end{equation*}

\begin{equation*}
r> r_{copy}=\frac{1}{4} (1-b) c (A-b c)
\end{equation*}

\subsubsection{Entrepreneur with an ego}

\begin{equation*}
\text{max} \{ 
0, \left(\frac{1}{4}(A+ c(b-2)) \right)^2+\frac{1}{2}r, \left( \frac{1}{4} (A+c(2-3b) )\right)^2-r
\}
\end{equation*}

\begin{equation*}
r> r_{ego} = \frac{1}{3} (1-b) c (A-b c)
\end{equation*}

\subsection{First firm}

The first firm only chooses whether to enter or not to enter. 

Chain
\begin{equation*}
\text{max} \{ 
0, \frac{1}{16} \left(A+c(b^2 +b-3) \right)^2+2 r
\}
\end{equation*}

2 Suckers
\begin{equation*}
\text{max} \{ 
0, \frac{1}{16} (A+2 b c-3 c)^2 + 2r
\}
\end{equation*}

One possible sucker

\begin{equation*}
\text{max} \{ 
0, \left(\frac{1}{4}(A+ c(b-2)) \right)^2+\frac{1}{2}r
\}
\end{equation*}

Equality:

\begin{equation*}
\text{max} \{ 
0, \left(\frac{A-c}{4} \right)^2
\}
\end{equation*}

\subsection{Inter firm Results}


\begin{proposition}
If the third firm is double or nothing $\Rightarrow$ the second firm attaches itself
\end{proposition}

\begin{proof}
Need only note that profit is superior being the middle firm than being part of three unproductive firms.
\end{proof}

\begin{proposition} \label{symislud}
If the third firm is a luddite $\Rightarrow$ the second firm does not attach
\end{proposition}

\begin{proof}
Need only note that the r for making the decision is the same in both cases. ($r_{sym}=r_{lud}$)
\end{proof}




\subsection{Intra firm results}

\begin{proposition}
The copycat strategy never exists
\end{proposition}

\begin{proof}
The definition of copycat implies when the 2 does not attach then 3 also does not attach(symmetric outcome). $r > \frac{3 c(1-b)}{16}(2A+c-3bc)=r_{sym}$. When firm 2 does attach to the first firm, then the third firm also attaches to the first firm. $r< \frac{3}{16} c \left(2 A (1-b)-c(1-b^2 ) \right)=r_{1}$ and $r > \frac{3}{16} (1 - b) b c (2 A + (2 - b - 3 b^2) c)=r_{3} $ Therefore the copycat strategy is Nash iff $r \in [max\{r_{sym}, r_{3} \}, r_{1}]$. We need only note that $r_{sym}$ is strictly greater than $r_{1}$, therefore the set is empty and the result follows
\end{proof}

\begin{proposition}
$r_{copy}<r_{ego}$. This implies that if 2 attaches when 3 played ego, then 2 always attaches when 3 plays copycat. 
\end{proposition}

\begin{proof}
trivial
\end{proof}



\begin{proposition}
$r_{unc}<r_{mod}$. This implies that if 2 attaches when 3 played mod, then 2 always attaches when 3 plays unc.  
\end{proposition}


\begin{proof}
trivial
\end{proof}

\begin{proposition}
$r_{ego}<r_{mod}$. This implies that if 2 attaches when 3 plays mod, then it always attaches when 3 plays ego.   
\end{proposition}

\begin{proof}
trivial
\end{proof}

\begin{corollary}
By transitivity we have that $r_{copy}<r_{ego}<r_{mod}$, this implies that if 2 attaches when 3 plays moderate innovator, then 2 always attaches when 3 plays copycat. 
\end{corollary}


\begin{proposition}
$r_{ego}<r_{lud}$. This implies that if 2 attaches when 3 plays luddite, then it always attaches when 3 plays ego.   
\end{proposition}

\begin{proof}
trivial
\end{proof}

\begin{corollary}
By transitivity we have that $r_{copy}<r_{ego}<r_{luddite}$, this implies that if 2 attaches when 3 plays luddite, then 2 always attaches when 3 plays copycat.
\end{corollary}

\begin{corollary}\label{minr}
The minimum over all the bounds for r is given by $min \{ r_{unc},r_{copy}  \}$. So if r is lower than both of these, then 2 always attaches irrespective of 3s strategy.
\end{corollary}

\begin{proposition}
IF r is lower than this minimum then we have either a tree or a chain
\end{proposition}

\begin{corollary} \label{maxr}
The maximum over all the bounds for r is given by $max \{ r_{mod},r_{lud}  \}$. So if r is larger both of these, then 2 never attaches irrespective of 3s strategy.
\end{corollary}

\begin{proposition} \label{Symmetric}
If $r>$ max$\{ r_{mod},r_{luddite} \}$, then the only Nash equilibrium is the symmetric outcome. 
\end{proposition}

\begin{proof}
First note that max$\{ r_{mod},r_{luddite} \}$ is the maximum over all the bounds. So if r is larger than these two, it is larger than all the bounds. Which implies that firm 2 never attaches itself, regardless of 3's strategy. 

Firm 3 will only attach to either firm 1 or 2 iff $r \leq r_{sym}$. But by \ref{symislud} $r_{sym}=r_{lud}$; and by hypothesis $r > r_{lud}$. Thus firm 3 never attaches. 
\end{proof}

\begin{corollary}
If $r>$ max$\{ r_{mod},r_{luddite} \}$ then firm 3 only plays either luddite or entrepreneur with ego in equilibrium. 
\end{corollary}

\begin{proposition}
If  $r< min\{ r_{unc}, r_{copy} \}$ then a Nash equilibrium is either a chain or a tree.
\end{proposition}

\begin{proof}
If $r$ is lower than $ min\{ r_{unc}, r_{copy} \}$ then by \ref{minr} firm 2 always attaches to firm 1. Now, firm 3 prefers to attach to firm 1 over not getting attach at all if $r<r_1$. We need only note that the relation $r_1<r_{unc}$ is always verified.  
\end{proof}


\begin{corollary}
If $r< min\{ r_{unc}, r_{copy} \}$ then 3 plays either unc expansion, copycat, OR moderate innovator in equilibriuim.
\end{corollary}



\subsection{General results}

\begin{proposition}
In a chain, the lowest payoff is either the first or the last firms'. 
\end{proposition}

Why would the last firm choose to go to spot n instead of of n-1? The only reason is because by going on n-1 it will deter any future entrants. 

If it can deter future entrants but it is NOT the last firm under this setup, then what does it do? It will simply compare the outcome if it takes the deterrence action to the outcome if it doesn't take the deterrence action. If his action is necessary to deter the other person, but it is NOT 

\begin{proposition}
It must be Pareto improving to make a chain for the chain to be a Nash equilibrium. 
\end{proposition}


\textcolor{red}{Notes to attempt to rule out a singleton}

If the first firm does not produce, there are no singletons. 

If the middle firm has the highest profit in equilibrium. Then a firm can either position itself to have some probability of also being the middle firm or it can position itself after the middle firm. This only occurs if the innovation per extension is low.(I think). This will occur until the probability of being selected is low enough that its better to just extend the chain. 

General, when will firms prefer to lay in wait, and when will they prefer to extend. This problem only exists at or after the single crossing point. 

To show, the chain is single crossing. 
To show, the chain is symmetric around the single crossing point. 

Suppose the single crossing point is in the middle of the final chain. That is, there is symmetry about the payoff function. That is, the extra market revenue of the j+1 firm is equal to the market loss relative j-1 firm. Then a singleton can only exist if it arrives after the last firm on the chain OR if it arrived before the first firm to be attaching itself. If it arrives after the last firm on the chain attached itself, it has the lowest profit AND everybody produces. 

Suppose that there are already k firms that have entered. If the k+1 firm does not make a link then it is a singleton. Suppose that the longest chain upon entering is j, where j $\leq$ k. Then if in equilibrium there is a chain longer than j, all firms lower in the chain than j+1 will have a lower profit than the singleton. The singleton will also have a lower profit than all the firms on the j chain because those firms could have just been singleton but chose not to. So if a singleton exists, either there will ONLY be singletons, OR it will have emerged at the single crossing point OR after the jth firm has entered the chain. 

\textcolor{green}{This is true only if the chain is monotonic, OR if we have already passed the single crossing}

If firm k+1 is a singleton, that means it will have higher profit than any firms other strategy. Singleton actually has the lowest deterence effect of all strategies. So it can only increase the stock of firms entering. If a firm before firm k+1 has not attached itself then 


Since the cost decrease is proportional this implies that as we approach some small number, the competitive edge between the newest competitors in fact decreases. So if n is large enough, we should eventually start seeing a decrease again. 

So if r is very high we have that payoffs are strictly decreasing in a chain. If we can show that being lower on the chain has lower extra profit the lower down we go, or put another way, the extra profit of being lower down the chain is decreasing. So if the royalty is also decreasing down the chain, then we can just have a sufficient condition. 

If r is 0, payoffs are strictly increasing in the chain. This creates the longest chain possible IFF all firms in the chain have positive profit.  

If is r intermediate, we have a single crossing property. Why? because being lower down the chain always has linear cost increase, whilst it has a concave market profit increase, due to proportionality. 

\section{Bertrand}

\textcolor{green}{In the setup of the model, since only firms that are producing have to pay royalties, if the competition was purely Bertrand competition, the chain would expand maximally until the cost of production was at the minimum, there are only two conditions, the first firm wants to enter, and the royalty fees}

\begin{proposition}
If $F=0$, then only the first firm enters
\end{proposition}

\begin{proof}
In Bertrand, cournot profit is 0 for producers. 
\end{proof}

\begin{proposition}
If we are in Bertrand competition then there is never a tree as long as $F>0$
\end{proposition}

\begin{proof}
To be made
\end{proof}



\begin{proposition}
Suppose the maximal profit possible is $\overline{\pi}$, then the innovation is expanded to its maximum iff: $\overline{\pi}-n r-F>0$ or $\frac{\overline{\pi}-F}{n}$
\end{proposition}

\begin{proof}
To be done
\end{proof}


\begin{corollary}
If the maximum is not pursued then, the optimum number of n is given by:
\begin{equation*}
max_n\{\overline{\pi}(n)-n r-F  \}
\end{equation*}
\end{corollary}

\subsection{Reduced form notes}

Let the market profit of firm i in an n firm chain be given by: $\pi(i,n)$. 

If this is strictly increasing in i, then we trivially have something that is Nash stable. What can we say about profit as a function of n? 

If it is strictly decreasing, it may still be stable, depending on how profitable it is to be a singleton. 

We may be able to combine these two insights for the single crossing notes. 


\end{document}

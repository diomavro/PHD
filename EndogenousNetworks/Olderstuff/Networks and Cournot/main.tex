\documentclass[11pt]{article}

\usepackage{enumerate}
\usepackage{geometry}
\usepackage[dvips]{graphicx}
\usepackage[T1]{fontenc}
\usepackage{amsmath}
\DeclareMathOperator*{\argmax}{argmax}
\DeclareMathOperator*{\argmin}{argmin}
\usepackage{amsfonts}
\usepackage{bbm}
\usepackage{graphicx}
\usepackage{lmodern}
\usepackage{asymptote}
\usepackage[font=small,skip=0pt]{caption}
\captionsetup[figure]{font=small,skip=0pt}
\usepackage{pstricks}
\usepackage{pst-plot}
\usepackage{pst-plot,pst-math,pstricks-add}
\usepackage{graphicx}
\usepackage{float}
\usepackage{pdflscape}
\usepackage{setspace}
\doublespacing
\usepackage{amsmath}
\usepackage{breqn}
\usepackage{amssymb}
\usepackage{amsthm}
\usepackage{indentfirst}
\usepackage{geometry}
\usepackage{titlesec}
\usepackage{nth}
\usepackage{enumerate}
%\usepackage{enuitem}
\usepackage{pgfplots}
\usepackage{graphicx}
\usepackage{enumitem}
\usepackage{tikz}
\usepgflibrary{arrows}
\usepackage{multirow}
\usepackage{booktabs}
\usepackage{fancyhdr}
\usepackage{url}
\usepackage{imakeidx}
\usepackage{float}
\pgfplotsset{compat=1.12}

\overfullrule=2cm
\geometry{left=3.5cm,right=3.5cm,top=2.5cm,bottom=2.5cm}
\renewcommand{\headrulewidth}{0pt}
\begin{document}

Royalty stacking is the phenomenon that when a firm enters a market it must pay numerous royalties because its product builds upon numerous previous innovations. This occurs because there is no legal obligation that the total royalty fees must remain below some threshold(such as a fixed monetary amount or a proportion of the cost of the product). Royalty stacking is similar to the the term "Patent stacking" except that the latter implies a single owner whilst a former is indifferent to the ownership structure of the stack. 

The empirical evidence is that the cost of royalty stacking can be quite significant. In smartphones, the cost of royalties has been said to be higher than the cost of components. 

Royalties come to represent such a significant portion because of innovation is sequential. When each innovation builds on the previous innovations it leads to firms a chain of innovation. If strong intellectual property rights are available then this is equivalent to saying that every predecessor in the chain must consent to the new product being built. One interpretation of patent length is how far back in the chain one must pay royalties. 

The hold up problem is not the only kind of issue that can arise. In the usual analysis, the more fragmented is the ownership structure of the chain, the more it will be hard to create a new product. However the hold up problem only a sort of worse case scenario. In practice even if it is assumed that royalties are fixed and there is no hold up problem, there are structural problems that can arise due to royalty stacking. 

Courts are said to recognize that royalty fees must reward with respect to the value added of the innovation and not using the entire value of the product, nevertheless, in practice this does not occur. However in practice the value added of a given royalty is not an observable quantity. In practice, what is especially difficult to discern is the value added of non-patented innovations relative to patented innovations. For a full discussion about why royalties end up in practice being a larger share than their value added, see "Patent holdup and Royalty Stacking"

The main result we have in this paper is to show that in Cournot competition, if there are n firms entering sequentially, for the n firms to form an chain of length n, it must be that the last firm to enter has a higher profit than the first firm to enter. 

\section{The model}

\indent Consider an infinite number of firms indexed by $\{1,2,...\}$ that decide sequentially to enter the market for a given good. Let $i$ be a representative firm that takes the decision to enter the market (or not) at time $t_i=i$. If $i$ decides to enter the market, then $i$ has to pay a fixed cost $0\leq F< \infty$, and does not otherwise. Let $N$ be the set of firms that choose to enter the market, and let $n \geq 0$ be the number of these firms. Briefly, note that only the firms which expected profit is positive enter; the profit function of a firm $i$ will be given later on. The technology used by firm $i$ is captured by the parameter $c_i$, which is $i$'s marginal cost of production. For the sake of simplicity, we will assume throughout the paper that any firm $i$ has its marginal cost equal to its average cost of production. We also assume that every firm is endowed with the same marginal cost $c_1=c_2=...=c$ ex ante. A firm that enters the market can upgrade its original technology $c$ by improving upon some of the technologies used in the industry. For this, we assume that a firm $i$ can access the technologies of other firms by forming links with them - and improve upon them. This process of accessing to the technologies in the industry (via a firm's investment in links) and innovating upon them enables the firms to reduce their marginal cost of production. We describe below two ways through which a firm $i$ can improve its technology $c$.\\

\indent  The first way is referred to as the \textit{intentional correlation in technologies}: firm $i$ chooses with which of its predecessors' technologies it wants its technology to be correlated. By correlated, we mean improvement upon some of the technologies that already existed in the market prior to $i$'s entry. For example, if $i$ chooses to improve upon firm $j$'s technology, for $j\in \{1,...,(i-1)\}$, we say that $i$'s technology is correlated with that of $j$'s with probability one. This is translated by a direct link from firm $i$ to firm $j$, and denoted by $i~\rightarrow ~j$. In general, the set of available actions to any firm $i$ at this stage is denoted by $\mathcal{G}_i$; it is the set of all possible links that $i$ can form with any of the $(i-1)$ firms that entered the market prior to $t_i=i$. A typical element $g_i$ of $\mathcal{G}_i$ is an $[1\times (i-1)]$ dimensional vector which elements are either zeros or ones. The $j$th element of $g_i$, denoted by $g_{ij}$, is equal to one if $i$ forms a direct link to firm $j$, and equals zero otherwise. All decisions in link formation of the firms that did decide to enter the market form a directed network $\text{g}$. \\
\indent The second way is referred to as the \textit{random correlation in technologies}: $i$ creates a technology that will reveal itself ex-post to be correlated with that of any predecessor $j$ with probability $p_i\in [0,1]$, for all firms $j\in \{1,...,i-1\}$. A firm $i$ decides first on the level of $p_i$; once the innovation process is finished, $i$ observes the realization of $p_i$, i.e. to which technologies $i$'s technology is directly built on. The only difference from with the latter case is that $i$ does not choose ex ante the technologies it wants to improve upon, but rather discovers ex post, once the technology operational, which of its predecessors' technologies its is correlated to. Again, the realized correlation of $i$'s technology with that of some firm $j$ is drawn as $i ~\rightarrow~j$. A directed network $\text{g}$ is then drawn from the realization of the decisions in $p_i$, for all $i\in N$.  \\
 
\indent The reasoning behind allowing for these two innovation processes is a philosophical one. We consider that a firm may choose to innovate deliberately upon some existing technologies - orienting purposefully its R\&D efforts on improving certain technologies, or it might target a certain degree of correlation for its technology, and this technology created reveals itself ex post to infringe on some already existing ones (or not, if firm $i$ chooses $p_i=0$). This degree of correlation to all technologies existing in the industry prior to $t_i$ is captured by $i$'s decision in the level of $p_i$. In both cases, the choices in link formation (whether it is deterministic or random) of the firms that enter the market draw, \textit{ex-post}, a directed network. Let \text{g} be any such particular network. The set $N_i(\text{g})$ refers to the set of firms with which $i$'s technology is correlated to. Therefore, for all firms that entered the market, the links from every of these firms $j$ to all firms in the sets $N_j(\text{g})$ exist in the network \text{g}. The longest path from firm $i$ to firm $j$ in \text{g} is the longest sequence of directed links $g_{i l_1}=...=g_{l_k j}=1$ from firm $i$ to firm $j$, and the length of this longest path will be referred to as $\rho_i(\text{g})=k+1$. We will refer to as an indirect connection from $i$ to $j$ all cases for which there exist a path of strictly more than one link that separates $i$ from $j$. Given the network \text{g}, firm $i$'s ex-post technology $c_i(\text{g})$ is given by the expression:
\begin{equation}
c_i(\text{g})=c-\beta \rho_i(\text{g})
\end{equation}
for $\beta \geq 0$. \\
\indent It follows that the existence of a direct link entitles the firm which initiates the link to some technological benefit, i.e. a reduction in its marginal cost of production. The discrepancy $\beta \rho_i(\text{g})$ in the marginal cost of firm $i$ depends on the length of the longest path that starts at firm $i$; therefore on the largest number of different technologies $i$ has been improving upon. Since a firm cannot have more than one marginal cost, it convenes to consider only the lowest marginal cost achievable by the firm - since it entails larger profits for the firm in fine. \\
\indent A link is nonetheless costly to the firm which forms it. We assume that for all the firms such that a directed path exists from firm $i$ to them in the network \text{g} (these firms have an index $j<i$), the former may have to pay royalties to all of these latter firms. Let $S_i(\text{g})$ be the set of these firms to which $i$ is directly or indirectly connected in \text{g}, and $s_i(\text{g})$ its cardinal. This captures the idea that if firm $i$ improves upon firm $j$'s technology, and $j$'s was already an improvement upon firm $k$'s technology, then that of $i$ \textit{indirectly} infringes on the technologies of both firms $j$ and $k$ - and therefore $i$ must pay royalties to both of them. These correlations in technologies are captured by the path $i~\rightarrow~j~\rightarrow~k$, and $i$'s expenditure in royalties is assumed to be linear in the number of the firms to which it is directly (here $j$) and indirectly (here $k$) connected. By the same principle, firm $i$ may receive some revenue from the royalties paid by some of its successors which technologies infringes on that of firm $i$ (there exists a path from some firm $j>i$ to firm $i$ in \text{g}, i.e. $i\in S_j(\text{g})$). Let $T_i(\text{g})$ be the set of these firms $j$ which technology infringes on that of $i$.\\

\indent Once the network has been realized and the correlations / infringements between the technologies of the firms revealed, the firms play a classic Cournot game.  We take a linear inverse demand function of the form $P(q_1,...q_n)=\alpha - \sum \limits_{j=1} \limits^K q_j$, for $n$ the number of firms that produce on the market. Note that a firm that did decide to enter the market at the previous stage may not necessarily produce in this second stage of the game. In such a case, the firm in question does not have to pay its expenditure in royalties at all, however it is still entitled to receive some revenues in royalties (as long as at least one firm in the corresponding set $T_.(\text{g})$ does produce a strictly positive amount of output). \\
\indent Therefore, firm $i$'s total expenditure in link formation is given by the expression $r s_i(\text{g})$ if and only if $i$ does offer a strictly positive quantity on the market, where $r\geq 0$ is the royalty that $i$ pays to any of its predecessors in the set $S_i(\text{g})$; otherwise, $i$ does not pay anything at all. Also, let $t_i(\text{g})$ be the number of firms in the set $T_i(\text{g})$ that do produce a strictly positive quantity. It follows that $i$'s revenue in royalties is given by the expression $rt_i(\text{g})$.

The payoff of any firm $i$ that entered the market in the first stage is given by the following expression: 

\begin{equation}
\pi_i(q_i, c_i(\text{g}), r, \text{g}) = \Bigg(\alpha - \sum_{j=1}^n q_j - c_i(\text{g})\Bigg)q_i + r\Bigg( t_i(\text{g}) - \mathbbm{1}_{q_i>0}~ s_i(\text{g})\Bigg)  - F \label{payoff},
\end{equation}  
where the first term into bracket is $i$'s net revenue on every unit of good that it sells on the market, and the second term into bracket is $i$'s net revenue from royalties. \\

\indent The next section of the paper is devoted to solving for the subgame perfect Nash equilibria of the game. We will use backward induction in order first to get the best-response functions for the firms that decide to enter the market, then the optimal strategies in link formation (whether they are vectors of links or probabilities upon the existence of some links), and finally solve for the entrance decision that happens in the first place. For this we will consider that the firms are myopic, which entails that they do not consider the expected revenue from royalties that might be generated by the arrival of successor firms on the market. We extend this analysis of the game later on by considering foresighted firms. 

\section{SPNEs of the game }
\subsection{The firms are myopic}

\indent As mentioned in the previous paragraph, we solve for the SPNE of this sequential game using backward induction. We first clarify the stages of the game, and the actions taken by the representative firm $i$ in each and every of these stages. In the first stage, a firm $i\in \{1,2,...\}$ decides whether or not it enters the market at $t_i=i$. If $i$ decides not to enter the market, the game ends for this firm and its payoff is zero. If $i$ enters instead, therefore expecting a positive profit from doing so, it chooses its innovation level by determining its strategy in terms of link formation. In the first version of the model, we assume that $i$ chooses directly which of its predecessors' technologies it wants its technology to be correlated to, for any firm $i\in N$. This determines the vector of links maintained by $i$, $g_i$. In the second version of the model, $i$ decides on the degree of correlation of its technology with the already existing state of the technology in the industry. This corresponds to $i$'s choice of $p_i\in [0,1]$, where $p_i$ is the probability that $i$'s technology is directly correlated to that of its predecessor $j\in \{1,...,i-1\}$, for any $i\in N$. Note that the decision to enter and the choice in link formation can be broken down into two stages, or else they can be considered as one single stage. For the sake of simplicity, we assume that these two decisions constitute one single stage of this game. \\

\indent Let us consider now the second and final stage of the game, in which the firms in the set $N$ choose a quantity to offer on the market. By the beginning of stage two, the network \text{g} is realized. If $i$ chooses not to produce, then it is free not to pay any royalty - since $i$ does not use its technology. However, $i$ may receive some royalty revenue from the firms that constitute the set $T_i(\text{g})$. Take any of these firms $j\in T_i(\text{g})$. Since there exists a path from $j$ to $i$ in \text{g}, it follows that $c_j(\text{g})<c_i(\text{g})$, for all $j\in T_i(\text{g})$. It follows that if $i$ does find it profitable to supply a strictly positive amount of output, so do all of these firms in the set $T_i(\text{g})$. Therefore, any firm $i\in N$ solves the following maximization program: 
\begin{equation}
\begin{aligned}
& \max_{q_i} && \pi_i(q_i, c_i(\text{g}), r, \text{g})= \Bigg(\alpha - \sum_{\substack{j\in N \\ j\neq i}} q_j - q_i - c_i(\text{g})\Bigg) q_i + r(t_i(\text{g})-s_i(\text{g}))-F\\
& \text{subject to} && q_i> 0, \;  i=1, \ldots, n.
\end{aligned}\label{program}
\end{equation}

\indent Here, $i$'s net revenue from royalties never depends on the exact level of $q_i$ when $q_i>0$ (the net revenue from royalties only depends on whether $q_i>0$ or $q_i=0$). Therefore, if $q_i=0$ then the firm is only rent seeking on the market. We will refer to $q_i^*(q_{-i})$, for $q_{-i}=\{q_1,...,q_{i-1}, q_{i+1},...,q_n\}$, as the best-response function of any firm $i\in N$ derived from the program in \eqref{program}, for a given the vector $(c_i(\text{g}), r, \text{g})$. \\

\textbf{Claim 1.} \textit{Let $i$ be any firm in the set $N$. The best-response of firm $i$ to the vector $q_{-i}$ of its competitors' strategies is given by the expression: }
\begin{equation}
q_i^*(q_{-i}) = \dfrac{\alpha - (n+1)c_i(\text{g}) + \sum \limits_{ j\in N}c_j(\text{g})}{n+1}, \label{BR}
\end{equation}
\textit{for any $i \in \{1,...,n\} $, and $i$'s ex-post payoff is: for any value $q_i^*(q^*_{-i})\geq 0$,}
\begin{multline} 
\pi_i(q_i^*(q^*_{-i}), c_i(\text{g}), r, \text{g})=\\
\max 
\begin{cases}
 \Big(\alpha - \sum \limits_{j\in N} q_j^*(q^*_{-j})- c_i(\text{g})\Big)q_i^*(q^*_{-i}) +r(t_i(\text{g})-s_i(\text{g})) -F
 \\
 rt_i(\text{g}) -F
\end{cases}
\end{multline}
\indent Thence, although a firm's best-response may prescribes it to produce a strictly positive amount of output, it is nonetheless possible to be strictly more profitable for this firm not to do so and adopt a rent seeking behavior. This is true whenever the Cournot profit of the firm is not large enough to compensate for its expenditure in links formation, i.e. when $[P(q^*_i,q^*_{-i})-c_i(\text{g})]q^*_i< rs_i({\text{g}})$. This might happen when $i$'s technology is not competitive enough compared to that of its successors: these firms produce substantially large quantities, which impacts the level of the price negatively, and $i$ ends up earning a strictly larger profit by not producing at all. It seems reasonable to conclude from this case that only the firms which technology is good enough (low values for their marginal cost) take care of the supply in the market. Therefore, there might exist a threshold in the industry for the value of the marginal cost $c_.(\text{g})$ above which a firm's best response is to not produce at all. This is made explicit further in our analysis of the first stage of the game when focusing on the firms' decisions in links formation.  \\

\indent We turn to the strategies of the firms for the first stage of the game. First, we make explicit the optimal strategies in terms of link formation of the firms that decide to enter the market. We start with the case of the \textit{intentional correlation in technologies} process. We show in what follows that forming a single link in the network strictly dominates any strategy that consists of maintaining strictly more than one link, and this for all the firms in $N$. \\

\textbf{Corollary 1.} \textit{In the version of the model where the firms choose directly to which technologies they want their own to be correlated, and $r>0$, any firm $i$ in the set $N$ always forms either exactly one link or none in equilibrium. }\\

\indent \textsc{Proof.} This is a direct proof. Take any firm $i\in N$. Let $g'_i\in \mathcal{G}_i$ be any strategy for firm $i$ that consists of maintaining at least two links in the industry formed by the firms $\{1, \ldots , i-1\}$. Let $N_i(\text{g'})$ be the set of players to which $i$ gets connected when the latter plays $g_i'$. Let $j$ be the firm that has the lowest marginal cost in the set $N_i(\text{g'})$; let this cost be written $c-\beta d$, for some $d\in \{0,1, \ldots , i-2\}$. Now consider the alternative strategy $g_i\subset g'_i$ that consists for firm $i$ of forming one single link to this firm $j$ just defined. We show that $g_i$ always strictly dominates $g'_i$. \\
\indent We first investigate the technological benefits of $g_i$ and $g'_i$, respectively. For all paths generated by $i$'s link creation into the already existing network formed by the firms $1,\ldots, i-1$, only the longest one benefits $i$ in terms of cost reduction. Therefore all paths that start by any link $i~\rightarrow~ k$ does not benefit firm $i$ at all, for $k\neq j$, $k\in N_i(\text{g'})$. Only $i$'s link to $j$ matters in determining the level of $i$'s technology; its marginal cost of production is $c-\beta(d+1)$ whether the latter chooses $g_i$ or $g'_i$. However it is always true that $g'_i$ entails a strictly greater expenditure in links than does $g_i$ - if, of course, firm $i$ does choose to produce $q_i>0$ in the second stage of the game. Note also that choosing $g_i$ over $g'_i$ has no impact on the level of output supplied by firm $i$ in equilibrium, nor on its revenue from royalties (since $i$'s marginal cost is the same under both strategies). It follows that, on expectation, firm $i$'s profit is strictly lower by playing $g'_i$ than by playing $g_i$. Therefore $g'_i$ cannot be an equilibrium strategy. \\
\indent For any strategy vector $g'_i$ with at least two links and its associated set $N_i(\text{g'})$, it is always possible to isolate the firm $j$ in $N_i(\text{g'})$ that has the lowest marginal cost of production. Therefore it always exists a strategy $g_i \subset g'_i$ that consists of the link $i~\rightarrow j$ only. Thus, for any $g'_i$, it is always possible to find a strategy $g_i$ that strictly dominates it. It follows that a firm that enters the market never maintains strictly more than one link in the network. \\

\indent The Nash network architectures in the version of the model with \textit{intentional correlation in technologies} are ones in which firms have either no link at all or just one. For \text{g} a given Nash network, a firm $i\in N$ that maintains no link at all forms an isolated singleton. Therefore, a Nash network can be divided into two subgraphs $\text{g}_1$ and $g_2$ with $\text{g}_2$ being the set of all such isolated singletons of the network \text{g} - these firms $j$ in $N$ which Nash strategies in link formation is $g_j=[0]_{1\times (j-1)}$. Of course the case where $\text{g}_2= \emptyset$ can occur in equilibrium. If not, the firms in $\text{g}_2$ have the same ex-post payoff as that of firm $1$, and their technology is $c_j(\text{g})=c_1(\text{g})=c$. \\
\indent  The next lemma clarifies the structure of the subgraph $\text{g}_1$ of any Nash network \text{g}, by providing with a restriction on the layout of the links that constitute this subgraph. By the result of our first corollary, one already knows that the firms in $\text{g}_1$ maintain one link each. Also, $\text{g}_1$ is non-empty if $n\geq 2$ and if firm $2$'s optimal strategy consists of forming a direct link with firm $1$. These two conditions are implicitly assumed to hold for the results of lemma $1$ to be true. \\

\textbf{Lemma 1.} \textit{Assume that the parameters values for $F, r$ and $c$ allow for the link $1~ \leftarrow ~2$ to exist in equilibrium. For $N$ the set of firms that choose to enter the market with $n\geq 3$, and $i$ a typical firm in this set, there exists one index $i^*$ for a firm such that:}
\begin{itemize}
\item[-] \textit{for all $i\leq i^*$, $i$'s optimal strategy in link formation consists of maintaining a single link to $i-1$, }
\item[-] \textit{for all $n \geq i > i^*$, $i$'s optimal strategy in link formation consists of (i) either a single link to some agent $k<i^*$ or (ii) no link at all. }
\end{itemize}
 

\indent \textsc{Proof.} This is a direct proof. Assume that the firms $1, \ldots , i$ have entered the market, and that $j \rightarrow j-1$ is $j$'s optimal strategy in links, for any $j\in \{1,...,i\}$. Therefore, by time $t_{i}$, the 'chain'  $1 \leftarrow 2 \leftarrow \ldots i-1 \leftarrow i$ exists, and $j$'s technology is $c-\beta (j-1)$. Now consider the case of firm $i+1$. Assume that this firm optimally chooses not to get connected to firm $i$. We show that this further implies that no $j\in N$ with $j\geq i+1$ forms a link to $i$ in equilibrium. \\
\indent We go back to $(i+1)$'s optimal choice of not forming a link to $i$. Let $g_{i+1}$ be this optimal strategy, and let $g'_{i+1}$ be the alternative strategy that consists for $(i+1)$ of forming one single link with $i$. Since the firms are supposed to be myopic, they are assumed neither to count their expected revenue from royalties, nor any further entry of competitors on the market when calculating their expected profit. Given $i$'s optimal strategy in links, it is true that the latter's expected profit from strategy $g'_{i+1}$ (left hand side of the above inequality) is strictly lower than that of playing $g_{i+1}$ (given by its right hand side):
\begin{multline*}
\mathbb{E}(\pi_{i+1}(g'_{i+1}, g_{-(i+1)}))=[P(q'_1, \ldots , q'_{i+1})- (c-\beta i)]q'_{i+1} -r i  <\\
 \mathbb{E}(\pi_{i+1}(g_{i+1}, g_{-(i+1)}))=
\max \limits_{l\in \{0,...,(i-1)\}} [P(q_1, \ldots, q_{i+1})- (c-\beta l)]q_{i+1} - r l
\end{multline*}
where: (i) $q'_j$ is $j$'s best-response to all the production strategies of the firms $1,\ldots, j-1, j+1, \ldots,  i+1$ when any firm $k$'s technology is $c-\beta (k-1)$, for any $j\in \{1,\ldots, i+1\}$ and $k\neq j$; and (ii) $q_j$ is $j$'s best-response to all its expected competitors' $1,\ldots j-1, j+1, \ldots i+1$ production strategies, given competitor $k$'s technology $c-\beta(k-1)$ for $k<i+1$, and $(i+1)$'s determined by $g_{i+1}$, for any $j\in \{1,\ldots , i+1\}$. \\
\indent Now consider the strategy $g'_{i+2}=g'_{i+1}$ for firm $(i+2)$. This gives to the latter an expected payoff that is strictly lower than that the value of $(i+1)$'s expected payoff of playing $g'_{i+1}$ at $t_{i+1}$(see left hand side of the above inequality). This is due to the fact that $(i)$ while the implied expenditures in links are the same, $(ii)$ $(i+2)$'s expected profit from production is always lower than the above value of $(i+1)$'s (since both the price and a firm's optimal quantity is decreasing in the number of firms present on the market). Thus: 
\begin{equation*}
\mathbb{E} [\pi_{i+2}(g'_{i+2}, g_{-(i+2)}) ] < \mathbb{E}[\pi_{i+1}(g'_{i+1}, g_{-(i+1)}) ]
\end{equation*}
\indent We continue with the strategy $g_{i+2}=g_{i+1}$, i.e. $(i+2)$ mimics $(i+1)$'s strategy.  



Plan: 

\begin{itemize}

\item[2.] diff in profits from connecting to the last player in the chain vs profit from getting connected to anyone else is single crossing (reversal in inequality happens once for one specific i, so that for all j<i, it is +; otherwise, -) (step 1.2)
\item[3.] the last firm to enter the market always produces a strictly positive output (step 1.1)
\end{itemize}

\indent 





\end{document}
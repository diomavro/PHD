\documentclass{article}
\usepackage[utf8]{inputenc}
\usepackage{enumerate}
\usepackage{amsmath}
\DeclareMathOperator*{\argmax}{argmax}
\DeclareMathOperator*{\argmin}{arg\,min}
\usepackage{amsfonts}
\usepackage{dsfont}
\usepackage{bbm}
\usepackage{graphicx}
\usepackage{asymptote}
\usepackage[font=small,skip=0pt]{caption}
\captionsetup[figure]{font=small,skip=0pt}
\usepackage{pstricks}
\usepackage{pst-plot}
\usepackage{pst-plot,pst-math,pstricks-add}
\usepackage{graphicx}
\usepackage{amsmath}
\usepackage{arydshln}
\usepackage{breqn}
\usepackage{amssymb}
\usepackage{amsthm}
\usepackage{geometry}
\usepackage{titlesec}
\usepackage{nth}
\usepackage{enumerate}
%\usepackage{enuitem}
\usepackage{pgfplots}
\usepackage{graphicx}
\usepackage{enumitem}
\usepackage{tikz}
\usetikzlibrary{arrows.meta}
\usepackage[affil-it]{authblk}
\usetikzlibrary{matrix,arrows,decorations.pathmorphing}
\usepgflibrary{arrows}
\usepackage{float}
\pgfplotsset{compat=1.12}
\usepackage{setspace}
\doublespacing 
\newtheorem{theorem}{Theorem}	
\newtheorem{corollary}{Corollary}
\newtheorem{proposition}{Proposition}
\newtheorem{observation}{Observation}
\newtheorem{assumption}{Assumption}	
\newtheorem{definition}{Definition}
\newtheorem{remark}{Remark}
\newtheorem{lemma}{Lemma}
\newtheorem{result}{result}

\usepackage{natbib}
\usepackage{color}
\bibliographystyle{agsm}

\begin{document}


\section{Introduction}

Royalty stacking is the phenomenon that when a firm enters a market it must pay numerous royalties because its product builds upon numerous previous innovations. This occurs because there is no legal obligation that the total royalty fees must remain below some threshold (such as a fixed monetary amount or a proportion of the cost of the product). Royalty stacking is similar to the the term "Patent stacking", except that the latter implies a single owner whilst the former is indifferent to the ownership structure of the stack. 

The question this phenomenon raises is under what conditions would a firm prefer to use the latest technology and pay a higher royalty cost rather than use a less cutting edge technology and diminish its royalty cost? To answer this question one must look at the incentives of both the firm seeking permission and the firm which is giving permission. If giving permission has no cost associated with it\footnote{the cost does not have to be direct, it could be a cost on the value of its license, this is explored in \cite{Katz1986}}, then the entrant will always infringe upon the best technology it finds upon entering the industry. Necessarily for any other structure but a chain of increasingly better innovations to occur either a cost must be present for the permission giving firm or some behavioral assumption must be assumed. 

The empirical evidence, though limited, is that the cost of royalty stacking can be quite significant. In smartphones, the cost of royalties has been estimated to be higher than the cost of components. \cite{Armstrong2014}. Royalties come to represent such a significant portion of the cost because innovation is sequential. When each innovation builds on the previous innovations, this leads to firms forming a chain of innovation. If strong intellectual property rights are available then this is equivalent to saying that an entrant must find at least one predecessor who has property rights to consent to the entry. 

The usual analysis of the fragmented ownership is through the hold up problem. In the usual analysis, the more fragmented is the ownership structure of the chain, the more difficult it will be to create a new product. However the hold up problem is only a sort of worst case scenario. In practice, even if it is assumed that royalties are fixed and there is no hold up problem, there are structural issues that can arise due to royalty stacking. This paper aims to highlight the conditions under which a new innovation is preferred with sequential innovation with endogenous royalties. 

Why does patent stacking occur? Though formally courts are meant to recognize that royalty fees must be proportional to the value added of the innovation and not using the entire value of the product, there are practical difficulties that prevents this from occurring. In practice the value added of a given royalty is not an observable quantity. This means that the value must be inferred and the metric usually employed is the difference in price between non-patented and patented innovations. Non-patented products have stronger competition: their costs of components are generally cheaper, which implies that the difference between the two prices is not only the value added of the patented technology but also the difference between industry efficiency. For a full discussion about why royalties end up in practice being a larger share than their value added, see \cite{Elhauge2008}

Royalty stacking implies that costs associated to royalties gets more important as innovations increase. There is a nuance here between royalty related \textit{total costs} and royalties \textit{paid}. We find that the later a firm enters in the industry, the higher the \textit{total royalty costs}, however we also find that later entry implies a lower royalty \textit{paid}. This is because the two costs are not independent: the royalties paid will be related to future royalty revenue; hence if a firm enters later, the willingness to accept to pay a royalty because of future royalties is decreased. Thenceforth the royalty paid depends on how much money the patent infringement permission is ultimately worth. 

The reason for our result is that each patent owner can extract the surplus of all future patent holders by charging the appropriate price. This means that patentees can only earn as a function of other market opportunities they had available at the time. Intuitively, if there is only one firm in the market, an entrant can either earn the infringement profits or the non-infringement profits. As the number of firms rises, the infringement options increase and this makes previous firms that have entered compete to sell their respective licenses. However the ability to license one's technology is a foreseeable, so the price paid to develop it will include the future royalties it will bring. 

If we assume the firms are operating on the same market, the effects of competition change the willingness to license. A firm may get a license from another firm that is currently producing or from one that has stopped producing. Ceteris Paribus, a firm that is not producing should charge less for a patent than one that is currently producing; this is because the presence of a competitor imposes a direct cost on the profit of a producing firm, while it imposes only the indirect cost of reducing patent value if the said firm is not producing. 

The framework presented here allows for a structural interpretation of patent length. That is the disagreement payoff of an entrant is increasing as the patent length decreases. However our model shows that unless there are market complementarities, the patent length does not matter. That is, it may be the case that the firms that are not on the cutting edge cannot charge for their license because, for the simple reason that they have less to offer than cutting edge firms. This means that whether the patent length is long or short does not matter since the disagreement payoff of the entrant is not affected. 

The structure of the paper will be as follows. In the second section we will present the model and clarify our three main assumptions of market symmetry, foresightedness and monotonicity in technology.  The third section will describe the equilibrium results that do not depend on the technological network firms do form. The fourth section highlights the relations between the network topology and the payoffs in the industry. In the fifth section, we show how the sort of market competition can strongly affect the firms' infringement decisions. 

%Some suggestions from me. The results are mainly about how network formation under very general conditions imply a decreasing amount of royalties paid. The intuition is that the ONLY way to have increasing royalties paid, as we observe empirically is that firms are not foresighted. 

\section{The model}
\indent Consider a set of firms $\Omega=(\omega_1,\omega_2,\ldots)$ that each decides sequentially to enter the market of a given good. 
Consider that $N=(1,2,\ldots,n)$ is the set of firms which enter the market, ordered by the time at which they made their decision. The game immediately ends for these in $\Omega\setminus N$; they all get a null payoff. Those in $N$ pay a fixed cost $F\in \mathbb{R}_+$ upon entry, irrespective of their production decisions. Once in the industry and before producing, every firm has the possibility to access and improve upon the pre-existing production technologies owned by the firms that entered at an earlier date. We denote firm $i$'s technology level by $k_i$ ; $k_i$ measures the efficiency of firm $i$ at producing the good, and it is an integer between 1 and $i$, for any firm $i\in N$. This process of accessing the technologies in the pre-existing industry and innovating upon them is made possible via the formation of directed links, from a predecessor firm to the entrant, and captures the technology transfer from the former to the later. \\
\indent To capture the concept of sequentially improving innovation we will represent firms with the most advanced technology to have a technology that builds upon the greatest number of its predecessors'. We model the technology decisions of the firms in $N$ as a game of endogenous network formation. The transmission of technology from a firm $h$ to a successor, say firm $i$, occurs if and only if $i$ has a directed link $h\rightarrow i$ to its said predecessor, $h$. The link means that $k_i>k_h$, i.e. a firm always successfully improves upon a technology it has access to. A strategy of link formation is denoted $s_i$ for any $i\in N$, and $s_i$ is the set of all of $i$'s predecessors with which $i$ forms links. All $n$ firms' decisions in link formation map to a technological network \text{g}. When $i\in N$ has infringed upon many technologies, thus $|s_i|>1$, we assume that $i$ produces using its most efficient one: i.e. we set $k_i=\max_{\forall j\in s_i} k_j+1$. The level of technology $k_i$ of firm $i$ is equal to the length of the \textbf{longest directed path} that starts at $i$ in \text{g} plus 1\footnote{Note that the longest path from $i$ always has a length between zero and $i-1$; if it is equal to zero, then $k_i=1$.}. If no path from $i$ exist in \text{g}, then $k_i=1$. Note that firm $1$'s technology level is $k_1=1$ always. \\
\indent Improving upon someone's technology comes at a cost. Prior to entry, every potential entrant is given take it or leave offers from all firms already in the industry. The offers consist of access to their technology and a transfer which is the royalty. The royalty $r^{h}_i$ is the transfer made by firm $i$ to $h$ for that $i$ accesses and builds upon firm $h$'s technology. One can see $r^{h}_i$ as the cost on $i$ for the link $h\rightarrow i$. Note that every firm can access the technology level of 1 for free.\\ 
\indent The strategy of a firm consists in which of the offers to accept. \footnote{In this paper, we will consider only pure strategies}. Therefore, if $\mathcal{S}_i$ is the set of all pure strategies of $i$ then $|\mathcal{S}_i|=2^{i-1}$; also, we call $\mathcal{S}=\mathcal{S}_1\times\ldots \mathcal{S}_n$ the space of all firms' pure strategies.  \\

\begin{definition}
A technological network \text{g} is a directed acyclic graph (DAG). Abusing notation, we denote $g=(s_1,\dots, s_n)$ the network that is formed by some vector of strategies $(s_1,\ldots, s_n)\in \mathcal{S}$ played by the firms in $N$. 
\end{definition}
\begin{proof}
The result that \text{g} is a DAG follows from the fact that no firm can form a link to any of its successors. 
\end{proof}

%The payoffs of the firms are made of two components, their technology level and network structure. 


\indent The firms all have two kinds of payoffs, a royalty payoff stemming from the directed edges and a competitive market payoff. The total payoff ultimately depends on the technological level of a firm and the network structure. The payoff of firm $i$ is denoted by $\pi_i(k_i,\text{g})$, and $\pi_i$ has the two following components: (i) $p_i(k_i,\text{g})$ is $i$'s \textit{market payoff}, to be interpreted as how much profit the firm can achieve by freely competing. We assume that the larger $k_i$, the more efficient firm $i$ at producing the good; thus the larger its market profit $p_i(k_i,.)$ (see assumption 1 for the full statement). And (ii) $r_i(\text{g})=r_i^+(\text{g})-r_i^-(\text{g})$ is $i$'s \textit{royalty net revenue}. This is all royalty payments that $i$ receives from its successors that have a link to $i$, minus $i$'s royalty expenditures the later pays to all of its predecessors in $s_i$. The expressions are $r^+_i(\text{g})=\sum_{h:~ i\in s_h}r^i_i=h$ and $r^-_i(\text{g})=\sum_{h\in s_i} r^{h}_i$ if we follow the notations introduced in the former paragraph. The total payoff of firm $i$, given the technological network \text{g} and $i$'s technology $k_i$ is: 
\begin{equation}
    \pi_i(k_i,\text{g})=p_i(k_i,\text{g})+r_i(\text{g}) -F. 
\end{equation}

\indent The next assumption imposes a direct relation between a firm's relative location in the network and its market payoff.\\  

\begin{assumption}{Technology and payoffs} \label{ass1}\\
Firms with higher technology have a higher market payoff:
\begin{equation*}
     k_i\geq k_j~~ \Leftrightarrow ~~ p_j(k_i,.)\geq  p_i(k_j,.),
\end{equation*}
where the equality holds if and only if $k_i=k_j$.
\end{assumption}

Note that this assumption does not mean that two firms that have the same technology have the same total payoff since their royalty revenues may differ. The rest of the assumptions are commonly used in any market environment. \\

\begin{assumption}{Farsightedness} \label{ass2}

Firms are farsighted: they anticipate the arrival of successor firms on the market, and take decisions accordingly. 
\end{assumption}

Since there is no risk or uncertainty in the model, this assumption implies that all firms must have positive payoffs. 

\begin{assumption}{Market power}\label{ass3}\\
Consider any technological network \text{g}; and take any subgraph $\text{g}'\subseteq \text{g}$ of the former network. Then: $p_i(k_i,\text{g}')\geq p_i(k_i,\text{g})$ for all $i\in \text{g}'$. For a same technology level, a firm has a higher market payoff the less competition it faces on the market. 
\end{assumption}

In other words, firms may face competition from both upstream (firms with higher technologies) and downstream (firms that have a lower technology) firms. \\

\indent Finally, we make explicit the timing of the game:       

\begin{enumerate}
    \item[] \textsc{1) Entry, network formation, royalty payment.}
\begin{itemize}
\item  \textit{At $t=t_i$, firm $\omega_i$ decides whether it enters the industry or not; if it enters, $\omega_i$ chooses simultaneously its strategy of link formation and pays the associated royalty cost. }
\item \textit{At every date $t$, for $t_n \geq t>t_i$ with $t_n$ the date at which the last firm $n$ enters the industry, firm $\omega_i$ offers a take it or leave it contract to the firm which enters the industry at date $t$ if ever a firm enters at this date, and does not do anything otherwise. }
  \end{itemize}
\textit{By the end of this stage, the technological network $\text{g}=(s_1,\ldots,s_n)$ is formed. Royalty payments for all firms are cleared and settled.}
    \item[] \textsc{2)} \textit{Competition/Payoffs are realized.}

\end{enumerate}  

\section{Subgame Perfect Nash networks and equilibrium royalty payments}
Since we did not specify the market payoffs of the firms in $N$, we solve for the Nash equilibria of the network formation stage as well as for the equilibrium royalty payments. A firm's decisions regarding entry, its choice of technology and the payment it makes to acquire it all happen simultaneously. Later on in the game, if a successor wants to form a link to a predecessor, the predecessor will then make a take it or leave it royalty offer for its successor. In other words, upon entering a firm will pay the royalties immediately to the firm it will connect to, and will only receive royalty payments when its successors enter.  \\ 
\indent We first present the results about the royalty payments between the different firms in the industry. These results will help to derive some of the properties of the subgame perfect Nash networks featured later on in section 4. These results hold regardless of the network structure. The specific relations between royalty payments and network structure are explored in the next section. We start by setting clear some terminologies. 
\begin{definition}\label{netrandp}
\begin{itemize}
\item[]
    \item Consider some network $\text{g}\in \mathcal{S}$. Firm $i\in N$ is a net receiver if and only if: 
    \begin{equation*}
        r_i^+(\text{g})> r_i^-(\text{g}),
    \end{equation*}
    and is a net payer otherwise. 
    \item Firm $i$ is said to be a rent seeker if and only if:
    \begin{equation*}
        \pi_i(k_i,\text{g})=r_i(\text{g})-F.
    \end{equation*}
    In equilibrium, all rent seekers are net receivers. 
    \item A firm that is not a rent seeker is called a producer: the firm makes a strictly positive market profit. In equilibrium, all net payers are producers.
    \item In equilibrium, there is at least 1 producer, which is firm $n$. 
\end{itemize}
\end{definition}
\begin{proof}
The second and third bullet points are implied by our assumption that firms are farsighted. For the last statement, note that $n$ does not have any successor: thus $r_n(\text{g})\leq 0$ for any $\text{g}\in \mathcal{S}$. Thence $n$ must produce. 
\end{proof}

The first proposition shows that when two firms acquire the same technology, then any of their common successor can improve upon their technology for free. \\ 

\begin{proposition}\label{prop:zerorevenue}
In equilibrium, two firms $h$ and $i$ that have the same technology level $\Tilde{k}=k_h=k_i$ transmit $\Tilde{k}$ to their common successors for free. Meaning, if $j>h,i$ is a common successor of both $h$ and $i$ then $r^{h}_j=r^{i}_j=0$. 
\end{proposition}
\begin{proof}
If firm $j$ wants to acquire technology $\Tilde{k}+1$, then $j$ rationally pays for only one link to either $h$ or $i$ if the royalty is strictly positive. Firms $h$ and $i$ are in competition for $j$'s attachment to them; here, $h$ and $i$ play a prisonner's dilemma game which Nash equilibrium is $r^{h}_j=r^{i}_j=0$.
\end{proof}

The proposition implies, among other things, that two firms which are one behind the other in $\Omega$ and that decide to compete on the market with the same technology both sacrifice any prospect of royalty revenue. In the next proposition, we show that a firm cannot subsidize the technology of any of its successors. \\

\begin{proposition}\label{prop:positiveroyality}
In equilibrium, a firm that infringes on another one's technology always pays a positive royalty cost for doing so. In other words, there is no case where a firm subsidizes the technology of its successors in equilibrium.
\end{proposition}
\begin{proof}
Consider the following scenario: there are $i-1$ firms which entered the market, and the highest technology level used so far is denoted $\Bar{k}$. Consider firm $i$. Assume that a predecessor of $i$, that we shall call firm $f$, offers to subsidize a link from $i$ to herself. For this, firm $f$ must have a technology $k_f$ that is strictly worse than $\Bar{k}$. This offer is rational if $f$ intends here to deter $i$ from further innovating strictly beyond the technology level $k_f+1$. Assume that $i$ accepts the offer: it forms a link to $f$ and gets paid for it. But then $i$ could use this money to pay for a strictly better technology, in which case $i$ uses the technology it paid for since it is better than $k_f+1$. Thus $f$ subsidized $i$ for nothing. Therefore offering a subsidy was irrational for $f$. Now, if $i$'s negotiations with all other firms were unsuccessful, then $f$ would have been better off by offering a positive price for the link from $i$ to her. 
\end{proof}

The following corollary sheds light on the implications of our two first propositions. \\

\begin{corollary}\label{firmswithsametech}
In equilibrium, (i) there are at most 2 firms that have the same technology, and (ii) both of them are producers. Also, (iii) all of their common successors have strictly superior technologies to theirs. 
\end{corollary}
\begin{proof}
(i) Assume not; $i,j$ and $m$ are three firms in $N$ with $i<j<m$, and $k_i=k_j=k_m=\tilde{k}$ in the network \text{g}. Consider firm $m$; the later could have gotten the technology $\tilde{k}+1$ for free and earned more market profit by assumption \ref{ass1}. In \text{g}, $m$ has no royalty revenue by proposition \ref{prop:zerorevenue} and the fact that $m>i,j$. If $m$ had formed a link with either $i$ or $j$ instead, $m$ could have earned a positive royalty revenue. The result follows. (ii) All common successors of some firm $j$ which acquire the technology $\tilde{k}=k_j+1=k_i+1$ with $i<j$ do not trigger any royalty revenue to either $i$ or $j$ by proposition \ref{prop:zerorevenue}; therefore $j$ must produce in equilibrium by definition \ref{netrandp}. Note that if $i$ was not producing, thus $p_i(\tilde{k},\text{g})\leq 0$, then $j$ would not produce either as $k_i=k_j$. A contradiction that $j$ is farsighted. Hence both $i$ and $j$ must produce. Finally, (iii) follows from (i). 
\end{proof}

A firm that does not improve upon the highest technology available in the industry at the date it enters never gets any royalty revenue. We conclude this section by setting clear the relation between the set $\Omega$ of all firms which face the decision of entering the industry and the set $N$ of these which decide to enter. 

\begin{proposition}
In equilibrium, if firm $\omega_i\in \Omega$ decides to enter the industry, then so did all firms $\omega_1,\ldots, \omega_{i-1}$. Put differently, $\omega_i=i$ in equilibrium for all $i\in N$. 
\end{proposition}
\begin{proof}
Assume that $\omega_1,\ldots, \omega_{i-1}$ entered the market, $\omega_i$ decided not to, and $\omega_{i+1}$ does. Hence $\omega_{i+1}=i$ with $i\in N$. Let $\pi_i(k_i,\text{g})$ the payoff of $\omega_{i+1}$; since the later firm is farsighted and rational, then $\pi_i(k_i,\text{g})\geq 0$. But then $i$ could have earned the same payoff by entering the industry, payoff that is weakly larger than $i$'s of 0. 
\end{proof}

\section{Network topology and its relation with royalty payments}
 \subsection{Subgame Perfect Nash networks}
We solve for the optimal decisions of the firms in terms of their choice of links. We first set clear the definition of a subgame perfect Nash network. \\

\begin{definition}
The vector $(s_1,\ldots,s)$ of all firms' strategies is a SPNE of the game if and only if there is no alternate strategy $t_i$ for each $i\in N$ that gives firm $i$ a strictly larger payoff, for any $t_i\neq s_i$ with $t_i\in \mathcal{S}_i$. A network $\text{g}=(s_1,\ldots, s_n)$ is said to be a subgame perfect Nash network if and only if $(s_1,\ldots, s_n)$ is a SPNE. 
\end{definition}

We first reveal the common features of the subgame perfect Nash networks.\\

\begin{proposition}\label{prop:onelink}
Maintaining at most 1 link is a weakly dominant strategy for every firm in $N$. 
\end{proposition}
\begin{proof}
Let $s_i\in \mathcal{S}_i$ be any strategy for firm $i$ that consists in maintaining at least 2 links in the technological network $\text{g}=(s_i,s_{-i})$. Let $h\in s_i$ be the firm that has the largest technological level among all firms in $s_i$. Consider the alternate strategy $h=s_i'\subset s_i$  that consists for firm $i$ in forming one single link to $h$; and let us call $\text{g}'=(s'_i,s_{-i})$ the resulting network. We show that $s'_i$ always weakly dominates $s_i$. By assumption only the longest path that starts at $i$ determines $i$'s production technology. The longest path that starts at $i$ has the same length whether $i$ plays $s_i$ or $s'_i$. Therefore $s'_i$ gives the same market payoff to $i$ than $s_i$ does.
Now, the royalty revenue of firm $i$. If $i$ maintains a single link to $h$, then $i$ receives revenues from its successors which infringe upon its technological level $k_i=k_h+1$. If $i$ maintains another link with firm $e\neq h$ in \text{g}, then given that $k_e<k_h$, there is at least one predecessor of firm $i$ which production technology is $k_e+1$. By proposition \ref{prop:zerorevenue}, $i$ owning the technology $k_e+1$ does not trigger any royalty revenue to $i$. \\
Since the only difference between $s_i$ and $s'_i$ is the number of links maintained, and since a link is never subsidized by proposition \ref{prop:positiveroyality}, then $r_i^-(\text{g}) = {r}_i^{-}(\text{g}')$ if and only if the extra links are costless to $i$, and $r_i^-(\text{g}) > {r}_i^{-}(\text{g}')$ if not. So we have the result. 
\end{proof}
Already quite a lot can be said about the topology of the subgame perfect Nash networks of the game. What should be recalled so far is that the first firms to enter the market maintain a link to their direct predecessor so as to form a chain $1\rightarrow 2 \ldots j-1 \rightarrow j$, for some threshold firm $1\leq j\leq n$. Each of these firms (except for $j$) earns a royalty revenue paid by its direct successor, guaranteeing to themselves some income in case their technology is not good enough to earn a positive market profit once the full network realized. For the rest of the firms from $j+1$ to $n$, each may improve upon the highest technology they find when they enter the industry (thus extending the chain). Or they get a cheaper technology used already by one of their predecessors (by corollary 1 only two firms produce with the same technology in equilibrium), strategy that dooms them to produce in equilibrium.\\
\indent We now give a result that ties the connectedness properties of the technological network to the equilibrium production decisions of the firms. \\
 
 \begin{corollary}\label{connectednessproperties}
 \begin{itemize}
     \item[]
     \item[(i)] If \text{g} is a subgame perfect Nash network and \text{g} is not weakly connected, meaning that another firm than 1 operates with the technology level $1$, then all firms in the industry produce. 
     \item[(ii)] If \text{g} is a subgame perfect Nash network and \text{g} is weakly connected, then all firms in the industry produce if firm 1 produces. 
\end{itemize}
 \end{corollary}
 \begin{proof}
 (i) Consider firm $i>1$ in \text{g} such that $k_i=1$. By proposition 1, $r_i(\text{g})=0$. Since $i$ is farsighted, it follows that $i$ must produce in equilibrium and $p_i(1,\text{g})\geq F$. Therefore $p_j(k_j,\text{g})\geq F$ for all firms $j$ with technology $k_j\geq 1$, that is for all firms in $N$. (ii) If firm 1 produces, and given that firm $1$ has the worst technology in the industry, then $0< p_1(1,\text{g})\leq p_i(k_i,\text{g})$ for any $i\neq 1$. The result follows.    
 \end{proof}
 

 \subsection{Payoffs and royalty payments in Subgame Perfect Nash networks}
 
In this section we clarify the relations between the royalty payments in the industry, the payoffs of the firms and the network that is formed in equilibrium. The first result deals with these firms that all have a same production technology. It has been showed earlier on that at most two firms produce with the same technology in a subgame perfect Nash network. We now highlight the relation between their profits and royalty payments. \\

\begin{remark}
Consider some technological network $\text{g}$. Assume that firms $i$ and $j$, with $i$ a predecessor of $j$, have the same technology $\Tilde{k}$; further, assume that all firms that are successors of $i$ and predecessors of $j$ have each a technology that is strictly worse than $\tilde{k}$. Then $\pi_i(k_i,\text{g})\leq \pi_j(k_j,\text{g})$ in equilibrium.  
\end{remark}
\begin{proof}
We first show that $r^{-}_i(\text{g})\geq r^{-}_j(\text{g})$. If there are two firms, both predecessors of $i$, that have the technology $\Tilde{k}-1$, then neither $i$ nor $j$ have paid anything for theirs. Note that $r^+_i(\text{g})=r^+_j(\text{g})=0$ by proposition \ref{prop:positiveroyality} and the fact that no predecessor of $j$ has attached to $i$ by the time $j$ enters the industry; and finally $p_i(\tilde{k},\text{g})=p_j(\tilde{k},\text{g})$ by assumption \ref{ass1}. Let us continue with the case where only 1 firm, say firm $f$, owns $\Tilde{k}-1$ when $i$ enters the industry. At the time $i$ negotiates, firm $f$ vies with the other $i-2$ predecessors of $i$ for $i$'s attachment; while when $j$ negotiates, $f$ faces the additional competition of these successor firms of $i$ and predecessor firms of $j$. More competition can only drive the cost of a link to $f$ down, i.e. $r^{i}_{f}\geq r^{j}_{f}$. Since by proposition \ref{prop:onelink} a firm pays a strictly positive royalty to at most one of its predecessors, it follows that $r^{-}_i(\text{g})\geq r^{-}_j(\text{g})$.
Also, $r^{+}_i(\text{g})=r^{+}_j(\text{g})=0$ for the same reason as the one aforementioned. Finally, both firms produce by corollary \ref{firmswithsametech} point (ii) and  $p_i(\Tilde{k},\text{g})=p_j(\Tilde{k},\text{g})$ by assumption \ref{ass1}. 
\end{proof}

The remark above says that there is a malus associated with being the first firm to get some technology level when another firm will acquire the same later on. The idea is that the firm which is the second to acquire the technology pays less for it, as there is more competition for winning this second firm's attachment than for the first firm. Also, if between $i$ and $j$ entry dates no firm has attached to $i$, then $i$ has no longer any prospect of earning a royalty revenue - since it will necessarily vie against $j$ for winning any successor firm's attachment. Consequently, both of these two firms have zero royalty revenue; also they earn the same market payoff as their production technologies are identical. \\

\indent \textit{Discussion:} let us focus on the specifics of the bargaining procedure through which an entrant acquires the technology of one of its predecessors. Consider a firm which enters the market and that does not build upon the current highest technology level - understand here that the entrant does not extend the "chain" of the network he finds upon entering. For the sake of clarity, assume that by the time entrant $i$ enters, the highest technology $\Bar{k}=k_h$ is owned by firm $h<i$. We see two possible explanations for why $i$ does not infringe on $\Bar{k}$: 
\begin{enumerate}
    \item[(i)] firm $i$ could not afford the cost of a connection to firm $h$. Meaning that given that $h$ charges a royalty that is at least equal to the \textit{negative externality} of having $i$ producing with the technology $\Bar{k}+1$ instead of some other less efficient one; and that the maximum $h$ can charge to $i$ is the differential in $i$'s payoff when the later produces with technology $\Bar{k}+1$ instead of with some other less effective production technology; the second term is less than the first one. (Firm $i$ can never afford to fully compensate the negative externality it imposes on firm $h$),
    \item[(ii)] firm $h$ strategically prevents $i$ from being more cost effective at producing than it itself is. Here, $i$ could afford to compensate the negative externality imposed on $h$ from having $i$ producing with $\Bar{k}+1$; however $h$ demands an extravagantly high royalty for the sake of deterring $i$ from acquiring the superior technology $\Bar{k}+1$. 
\end{enumerate}

\indent In the next proposition, we show that a predecessor strategically prevents an entrant from accessing its technology only if the said predecessor manages to secure a monopoly on its technology until the end of the network formation process. \\

\begin{proposition}
Consider any three firms $h,i$ and $j$ in $N$ such that $h<i<j$. If $k_h>k_i$ and $k_j>k_h$ in equilibrium, then $i$ could not afford technology $k_h+1$. ($h$ did not prevent $i$ strategically from acquiring $k_h+1$.)
\end{proposition}
\begin{proof}
Assume not. Firm $h$ predecessor of $i$ strategically prevents $i$ from forming the link $h\rightarrow i$; and $j$, successor of $i$, produces with technology $k_j\geq k_h+1$ in equilibrium. Let \text{g} be the technological network. For the sake of simplicity, assume that $k_j=k_h+1$ and $j$ owns a link to $h$ in \text{g}. If $h$ strategically prevents $i$ from getting the technology $k_h+1$, then it cannot be that firm $j>i$ acquires this technology level in the end. To see why: first, because if $i$ gets a lower technology than $k_h+1$, $h$ faces an increased competition for $j$'s attachment, which consequently reduces $h$'s expected royalty revenue (if ever the link $h\rightarrow j$ exists in \text{g}; otherwise $h$'s royalty revenue is null). Second, because having $i$ acquiring some inferior technology leaves the door open for the successors of firm $i$ to enter and access some technology either for free or for a lower price than what they would have paid if $i$ had connected to $h$. Thus firm $h$ would have faced less competition on the market; also, $h$ would have earned a larger market profit by assumption \ref{ass3}. Finally, the negative externality on the predecessors of $h$ imposed by the strategy of the later is larger than if $h$ had let $i$ get the technology $k_h+1$. First because in the end $j$ manages to reach $k_j=k_h+1$; second, because of the consequent increased competition just mentioned.   
\end{proof}

A firm that chooses not to improve upon the current best technology lowers by its action the cost paid by its successors for their own technology compared to what they would have paid for the same technology otherwise. This is because a firm that chooses the aforementioned link strategy leaves the possibility for its successors to access some technology for free, which therefore increases the competition between the predecessors for winning an entrant's attachment. Other things being equal, the royalty paid by an entrant for some given technology level depends positively on the technology acquired by its direct predecessor. \\
\indent The next proposition compares the royalty revenues of the firms in the case they all sequentially improve on the current best technology they find in the industry.\\

\begin{proposition}\label{monotonicity in royalty in the chain}
Consider some technological subgame perfect Nash network $\text{g}=(s_1,\ldots, s_n)$ where $i\in s_{i+1}$ for all $1\leq i\leq n-1$. In equilibrium, the royalty paid by a firm is decreasing in its index: $r^-_i(\text{g})$ is decreasing in $i$, for all $i\in N$.  
\end{proposition}  
\begin{proof}
In \text{g}, all firms in $\{1,2,\ldots, n\}$ could afford to pay for the \textit{negative externality} they are imposing on their direct predecessor (we mean by \textit{negative externality} what we have been defining as so in point (i) in the discussion paragraph). Because the network \text{g} is subgame perfect Nash and royalties paid are in equilibrium, both individual rationality and incentive compatibility constraints are satisfied for every firm. Given firm $i$'s linking strategy $s_i$ such that $i-1\in s_i$, the royalty paid by $i$ to $i-1$, $r^-_i(\text{g})=r^{i-1}_{i}(\text{g}) $, to reach the technology level $k_i=i$ is in equilibrium if and only if: 
\begin{align*}
& (\text{IR}_i) ~~ \pi_i(i, \text{g})\geq 0~~\Leftrightarrow~~ p_i(i,\text{g})-r^{i-1}_i(\text{g})+r^{i}_{i+1}(\text{g})-F \geq 0\\
&  (\text{IC}_i) ~~ \pi_i(i, \text{g})\geq \max\{\pi_i(\tilde{k}_i,\tilde{\text{g}})~,~0\} ~~\Leftrightarrow~~ p_i(i,\text{g})-r^{i-1}_i(\text{g})+r^{i}_{i+1}(\text{g})-F \geq \max\{p_i(\tilde{k}_i,\tilde{\text{g}})-r^{h}_i(\tilde{\text{g}})-F~,~0\}
\end{align*}
where: $\pi_i(\tilde{k}_i,\tilde{\text{g}})$ is the maximum payoff firm $i$ would get by attaching to some firm $h<i-1$ instead of to $i-1$, and $\tilde{\text{g}}$ the network that would result if $i$ was playing the linking strategy $\tilde{s}_i$ with $i-1\notin \tilde{s}_i$ and $h\in \tilde{s}_i$. Note that any other link that $i$ may have with other firm(s) than $i-1$ in \text{g} or $\tilde{g}$ is free by proposition \ref{prop:onelink}: $r^-_i(\text{g})=r_i^{i-1}(\text{g})\geq 0$ and  $\tilde{\text{g}}$: $r^-_i(\tilde{g})=r^{h}_i(\tilde{\text{g}})\geq 0$. Also, $r^+_i(\tilde{\text{g}})=0$ by proposition \ref{prop:zerorevenue} since $i$ shares the same technology as its predecessor, firm $h+1$, in the alternate network $\tilde{g}$. \\

\indent \textsc{Claim.} \textit{If \text{g} is subgame perfect Nash and the royalty payments are in equilibrium, then the IC constraints of all firms in $N$ are binding.}\\
\indent \textit{Proof.} Otherwise, $i$'s predecessor, firm $i-1$, could increase the royalty it claims against $i$'s attachment by $\epsilon$ sufficiently small so that $i-1$ still prefers to maintain its link to $i-1$ over deleting it and forming a link to $h$ instead, for any $2\leq i\leq n$. $\qed$ \\

Let us rewrite the $IC$ constraint of any firm $i$ in equilibrium as:
\begin{equation*}
(\text{IC}_i)~~ r^{i-1}_i(\text{g})= \max\{\Delta p_i+ r^-_i(\tilde{\text{g}})~,~ p_i(i,\text{g})-F\}+  r^{i}_{i+1}(\text{g})
\end{equation*}
with $\Delta p_i=p_i(i,\text{g})-p_i(\tilde{k}_i,\tilde{\text{g}})>0$. Replacing the last term $r^{i}_{i+1}(\text{g})$ by firm $i+1$'s IC constraint and applying the same principle recursively gives: 
\begin{equation*}
r^{i-1}_i(\text{g})= \sum_{l=i}^{n} ~\max\{\Delta p_l + r^-_l(\tilde{g}) ~,~ p_l(l,\text{g})-F\},
\end{equation*}
where every term of the sum is positive (otherwise $l$ would be better off not entering the industry). It follows that $r^{i-1}_i(\text{g})$ is a decreasing function of the index $i$ of a firm. Therefore the result. 
\end{proof}

The result introduced in the last proposition has strong implications onto the relative royalty revenues and production decisions of the firms in a chain network. \\

\begin{corollary}
Assume that $i=k_i$ in \text{g} where $\text{g}$ is a subgame perfect Nash network. In equilibrium, all firms but firm 1 are net payers; and they are producers.  
\end{corollary}
\begin{proof}
Note that $i-1\in s_i$ is verified for any firm $2\leq i\leq n$. Applying our result from proposition \ref{monotonicity in royalty in the chain}, we have $r_i(\text{g})=r^{i}_{i+1}(\text{g})-r^{i-1}_i(\text{g})\leq 0$ for any $i\neq 1$. It follows from definition \ref{netrandp} that all $i\geq 2$ are net payers - thus producers. Firm $1$ is a net receiver; therefore firm 1 is a producer if and only if $p_1(1,\text{g})>0$; otherwise firm 1 is a rent seeker. 
\end{proof}

The last proposition and remark essentially say that in a network that is a technological chain (i.e. $i=k_i$ for all $i$), every firm gets \textit{expropriated}, through the royalty it pays, from the surplus it realizes by linking to its direct predecessor instead of by forming a link to any other firm. This expropriation process goes as follows: at the time firm 2 enters and forms a link to firm 1, the later makes the former pay for the total surplus all subsequent firms $2,3,\ldots,n$ make in the chain. Firm 2, which ends up paying its connection to firm 1 for more than what it individually gains from it, gets its compensation by charging a royalty to firm 3 that is the sum of all of the surpluses other than its own (that it yet had to pay to firm 1); that is, the surpluses that $3,4,\ldots ,n$ realize by producing with the best technology they can get. This process continues recursively, so that the cheapest technology that is acquired by a firm is also the most efficient one. 

\section{Competition}
This section will hopefully help the reader to develop an overall understanding of the intricacies of our game. The first subsection shows how the type of competition considered may strongly imply the architecture of the subgame Nash network of the game. For the case of the competition a la Bertrand, at most two firms enter the market and only the last one produces. Next, we discuss the central ideas and findings of the model, and talk about several adaptations of our game to other assumptions. 
    
\subsection{The case of Bertrand competition}
In this subsection we assume that the firms in $N$, once the network formation and royalty payment stage finished, compete a la Bertrand. It is immediate that there are either 1 or 2 producers, and these firms are these with the best technology in the industry (recall from corollary \ref{firmswithsametech} that at most 2 firms can have the same technology). We directly move on to the first stage of the game. We find that there are only two architectures possible for the subgame perfect Nash networks. This result is presented in the remark and proposition below. The result about the royalty payments is featured in the second proposition. \\

\begin{remark}\label{bertrand is a chain}
If $\text{g}=(s_1,\ldots, s_n)$ is a subgame perfect Nash network, then $k_i=i$ (i.e. $i-1\in s_i$) for all firms $i\in N$.
\end{remark} 
\begin{proof}
Assume that all firms $h\in \{1,\ldots i-1\}$ have $h=k_h$ however $k_i<i$. Let $\bar{k}$ the highest technology level in the network $\text{g}$. Since $k_i<i$, $i$ shares the same technology as that of one of its predecessors; hence by corollary \ref{firmswithsametech} $i$ is a net payer and $i$ must produce in equilibrium. Thus it must be that $i=n$; and $k_n=k_{n-1}=\bar{k}$ since (i) firms compete a la Bertrand thus only the most cost efficient firm produces, (ii) the last firm to enter, firm $n$, always produces hence $k_n=\bar{k}$ and (ii) both firms $n$ and $n-1$ are farsighted. Consider firm $\omega_{n+1}\in \Omega$. If the later enters and forms a link to either $n$ or $n-1$, it gets the superior technology $\bar{k}+1$ for free. Therefore a contradiction that $n$ is farsighted and the last firm to enter the industry.    
\end{proof}

We show in the proposition below that a subgame perfect Nash network can have only 2 architectures. \\

\begin{proposition}
Given that the firms in $N$ compete a la Bertrand, then: 
\begin{enumerate}
\item either there is only 1 firm in the industry (i.e. $N=1$) and we denote this network $\text{g}_1$,
\item or $\text{g}_2=1\rightarrow2$ with $n=2$; meaning, there are 2 firms in the industry, firms 1 and 2 with $k_1=1$ and $k_2=2$. Only firm 2 produces and firm 1 is a rent seeker.  
\end{enumerate}
\end{proposition}
\begin{proof}
By remark \ref{bertrand is a chain}, there is a subgraph of \text{g} that is a chain $\mathcal{C}$ on all $n$ firms; hence if the conclusion of the proposition is false, then $n>2$. Along $\mathcal{C}$, $r_i(\text{g})\leq 0$ for any firm $2\leq i\leq n$ by proposition \ref{monotonicity in royalty in the chain}. Since only firm $n$ produces due to the type of competition considered, it must be that $r_i(\text{g})=0$ for any $2\leq i\leq n-1$; thus $\pi_i(i,\text{g})=-F<0$ for these firms. A contradiction that $i$ is farsighted. Thus $n=2$. If $\text{g}=\text{g}_2$ (note that $s_2=1$ in $\text{g}_2$), then 1 is a rent seeker since firm 2's technological is superior to firm 1's. Firm 2 is a producer and net payer since it is the last one to enter the market. 
\end{proof}

At last we determine conditions on the value of the entry cost $F$ for which either one of these two architectures is a subgame perfect Nash network. \\

\begin{proposition}\label{fullbertrand}
Consider the 2 networks $\text{g}_1$ and $\text{g}_2$ featured in the last proposition. Then:
\begin{itemize}
\item[] (i) $\text{g}_1$ is the unique subgame perfect Nash network of the game if and only if $p_1(1,\text{g}_1)-F> \max\{p_2(2,\text{g}_2-2F~,~0)\}$,
\item[] (ii) $\text{g}_2$ is the unique subgame perfect Nash network of the game if and only if $p_2(2,\text{g}_2)-2F\geq \max\{p_1(1,\text{g}_1)-F~,~0\}$, 
\item[] (iii) there is never any firm in the industry if $\max\{p_1(1,\text{g}_1)-F~,~ p_2(2,\text{g}_2)-2F\}<0$. 
\end{itemize}
\end{proposition} 
\begin{proof}
Consider first the network $\text{g}_2$. If firm 1 decides to let firm $2$ enter the market and form a link, then the incentive compatibility constraint of firm 2 is saturated by proposition \ref{monotonicity in royalty in the chain}. If firm $2$ does not enter, its payoff would be zero; and if it enters without forming a link to firm 1, then 2 would get a negative payoff. Therefore: $r^1_2(\text{g}_2)=p_2(2,\text{g}_2)-F$. Since firm 1 is a rent seeker in the network $\text{g}_2$, it follows that $\pi_1(1,\text{g}_2)=p_2(2,\text{g}_2)-2F$. If firm 1 does not let firm 2 enter, then the network that is realized is $\text{g}_1$ in which firm 1's payoff is $\pi_1(1,\text{g}_1)-F=p_1(1,\text{g}_1)-F$. The inequalities in (i), (ii) and (iii) follow immediately.  
\end{proof}

\subsection{Discussion}
Consider first the general result that when technology is differentiated vertically, as is the case in our model, there are never more than two firms who have a single level of technology (see corollary \ref{firmswithsametech}). The intuition is straightforward: once a firm is competing for the licensors, the rent extraction of its technology instantly falls down to zero. This is because if there is a single competitor when no horizontal or vertical differentiation in productive capacity exists, this competitor's technology is a perfect substitute. Note that this result holds independently on what kind of competition will occur in the end. \\
\indent One of the drivers that cause firms to only ever make a single link is the fact that firms cannot control what links other firms make (see proposition \ref{prop:onelink}). For instance, suppose a predecessor does not wish for the current entrant to create a new technology because the negative externality on the said predecessor is too high. It would intuitively be appealing if the entrant received a royalty payment from the predecessor as an incitement not to create the externality, a Coasian bribe to minimize damage. For this to be credible however there must be a notion of an exclusive contract; without such a notion, the entrant will happily receive the royalty and also deviate to attach to some superior technology. This reasoning holds true even if the entrant were providing any sort of positive externality by entering. In this case, the game would turn into a standard public goods problem where firms will not subsidize (in this case, a reverse royalty) as each of these firms would prefer that other firms (unless of course the externality is high enough that a single firm can subsidize). A reverse royalty can then occur in one of two scenarios: exclusive contracts or positive externalities. \\
\indent An intuitive question that a technological chain network raises is the following: under what market and network structures does the maximum innovation occur? The relevant conditions to analyze are those under which a chain emerges. A chain represents a situation where every firm improves upon the technology of all of its predecessors. Consider for a moment what would in the chain if there was no intellectual property, that is innovation was free, $r=0$. If the fixed cost was too high, no firms enter because the resulting chain would cause for the first firm to be at a loss. If the fixed cost approaches zero, the maximum possible chain on all firms in the set $\Omega$ would emerge. For this case, consider the effect that an extra entry has on the payoffs of all existing firms, and let us call this the \textit{competition effect}. Consider also the effect of having a relatively higher payoff than other firms because it has a more advanced technology, and let us call this the \textit{technology effect}. With our terminology straight, we can now clarify under what conditions no intellectual property still results in a chain. If the competitive effect has a discontinous effect in that the entry of an additional firm causes all firms, including the entrant, to have a negative payoff (i.e. the market payoff of all firms would be less than the fixed entry cost); and absent this entry all firms would have had a positive payoff, then a chain would have emerged anyway. Recall that this condition is hard to satisfy because without intellectual property the latest firm that enters gets the highest payoff. In other words if the competitive effect is strong relative to the technology effect then the chain emerges. As a special case consider what occurs if each entry has $\epsilon$ more payoff than its predecessor. It would be that the chain always exists because causing the first firm to have a negative payoff will also cause the latest firm to have a negative payoff. \\
\indent Our Bertrand case can be interpreted more broadly as what occurs if competition is of the form "winner takes all". The problem that emerges here is a commitment one: once a firm incurs a fixed cost and enters, its predecessor has no obligation to compensate the fixed cost whatsoever. This results in having no intermediaries, i.e. firms that do not produce and that are not either the first or the last firm. This is because the intermediaries will get charged by the first first firm its whole future royalty revenue; also they would not have a market payoff and they will not be able to recoup their fixed cost, this even for an $\epsilon$ fixed cost. This means that the only network that can emerge is either a singleton or two firms that form a chain. If the fixed cost was incurred before the negotiation on the royalty payment occurring, the two firm subgame perfect Nash network ($\text{g}_2$ in proposition \ref{fullbertrand}) would also be ruled out and only the singleton network $\text{g}_1$ would exist. 

\bibliography{../thesisbib/bibliography}


\end{document}


For the sake of simplicity, we assume that a firm takes its entry and technology decisions simultaneously.

There are two possible values of the royalty paid by $j$ for its acquisition of $k_j$: (i) either all firms that enter after $i$ and before $j$ have sequentially built on $k_i$: meaning, if $(i,i_1,\ldots, i_m,j)$ is the set of firms ordered by the time they enter the industry, $k_{i_1}=k_i+1,\ldots, k_{i_m}=k_i+m$ and $k_j=k_i+m+1=k_h+1$ and all of these firms got their technology for free by proposition \ref{twotechgiveszeroroyal} and corollary \ref{firmswithsametech}. Also, note that $h$ cannot hope that have any royalty revenue in this configuration by proposition \ref{twotechgiveszeroroyal} since firms $h$ and $i_m$ have the same technology. Therefore $h$'s royalty revenue is less than if it had let $i$ form a link to it.  